\chapter{Conclusion} \label{chap:9}

The ancient Greek dialects, it should have become clear by now, were eagerly studied in early modern Western Europe, ever since they were put on the scholarly agenda by humanists in the second half of the fifteenth century. The main motivation to do so was philological, since without mastering the different literary varieties of Ancient Greek it was impossible to get a firm grip of the ancient Greek world, its literature, culture, and history. This was also why scholars believed it necessary to devote entire handbooks to the Greek dialects. Knowledge of them was, in other words, largely auxiliary and instrumental, and they were never studied in and for themselves. After the initial stage of philological focus on the great pagan classics of Greek literature, the dialects were soon introduced into related fields of interest, most importantly biblical philology as well as \isi{historiography} with a strong antiquarian touch. In particular, awareness of the Greek dialects urged them to investigate two additional matters, especially from the seventeenth century onward: the peculiar nature of Biblical Greek, on the one hand, and the language of Greek inscriptions which were being uncovered in large numbers all over the Mediterranean, on the other. In a more stereotypical fashion, the Greek dialects were also introduced in historiographical and ethno-geographical accounts of ancient Greece. Outside of Greek studies, the dialects proved to be a welcome orientation point for grammarians and philologists interested either in describing and codifying other languages or in gaining insight into language history, change, and diversity. This was, in very general terms, the context in which early modern scholars, exclusively men, studied the Greek dialects. I have tried to demonstrate how in this setting knowledge about the Greek dialects evolved from the end of the fifteenth until the end of the eighteenth century and how it related to ancient and medieval ideas. In doing so, I have focused on two aspects: contents and valorization. On the one hand, I have charted actual early modern ideas on the Greek dialects, their origins and development, and their successes as much as their failures. I have, on the other hand, also laid out in which ways scholars put their knowledge of the Greek dialects to use in dealing with linguistic themes and problems outside the scope of Greek philology. These most often related to the \is{standardization!codification}\isi{codification} and description of Oriental and especially \isi{vernacular} tongues, for the \isi{standardization} of which the greatly admired Greek language was often hailed as a welcome reference point.

Early modern scholars devoted considerable effort to untangling the mystery of the Greek dialects, a focus of interest gradually conceived as a separate subfield within philology, as I have shown in Chapter \ref{chap:3}. Hellenists attempted to develop accurate classifications of the complex linguistic situation of ancient Greece. These were not only more diverse but also finer-grained than those of their ancient and medieval predecessors, on whose ideas they as a rule elaborated. A major innovation of early modern times was the division between principal and less principal dialects, a distinction based on the language-external criterion of \isi{literary usage}. The principal dialects were used in writing, the minor ones were not. This bipartition was so widely known and popular that some \isi{vernacular} grammarians with a Hellenist background transposed it to their native context, even if the criterion of \isi{literary usage} was not so easily applicable to it. A major setback was the invention of a poetical Greek dialect by early modern Hellenists, who often included it in their classifications; this resulted from the semantic ambivalence of the term \textit{dialect(us)}, which could mean not only ‘regional form of a language’ but also more generically ‘manner of expression’. The concept of \textsc{poetic} \textsc{dialect} was, however, far from unsuccessful in the early modern landscape of linguistic thought, since \isi{vernacular} grammarians adopted it to describe the particularities and liberties of language found in their native poetical traditions. A notable success of early modern scholarship was the clear separation of the Koine from the other dialects. Many Hellenists recognized its particular position as a \isi{common language} transcending regional diversity, an insight likely fostered by the emergence of similar common languages in their times, for which the Koine was often cited as a model. The establishment of the peculiar nature of the Koine did not, however, mean that its history and its emergence conditions were adequately understood. Apart from the Koine, Hellenists experienced great difficulties also in understanding the place of Homeric as well as Biblical Greek within the Greek language and its history. These were usually interpreted as an artificial mix of the canonical Greek dialects. Some Hellenists in the seventeenth and eighteenth century had, however, flashes of remarkable insight, proposing solutions approximating those of modern philologists. These resulted from a better appreciation of the historical conditions under which both Homeric and Biblical Greek emerged. Homeric Greek was identified as representing earlier stages of Greek, whereas scholars like Saumaise recognized that Biblical Greek was a \isi{vernacular} variety of the Koine. These solutions were part of a wider tendency to describe the evolution of the Greek language in detail, a tendency formalized in the new genre of linguistic histories of Greek, which emerged in the seventeenth century. Early modern scholars tried to fit the different dialects of Greek, their historical development, and their interrelationships into this historical puzzle. Heavy with the burden of tradition and Strabo’s authority, they perceived a close link between Aeolic and Doric by \isi{analogy} with the strong bond Attic and Ionic shared, a misconception definitively corrected only in modern times. Scholars were similarly misled in their claim that Latin was closely associated with Greek and especially two of its dialects: first Aeolic and later also Doric. This faulty idea was likewise only abandoned in modern scholarship, even though pioneering linguists like Rasmus \citeauthor{Rask2013} (1787–1832) still believed in this age-old link (\citeyear{Rask2013}: 152–153).

When outlining the precise differences among the Greek dialects, Hellenists recognized in the wake of their ancient and medieval predecessors certain regularities in this variation, noticing at the same time that these were not so much stable linguistic laws as they were fickle letter changes. However, they did not blindly focus on the level of the letter and looked at differences in terms of accent, lexicon, syntax, and style as well. The source material from which they worked was initially, like that of their Greek predecessors, restricted to literary texts, usually the pagan classics and from the sixteenth century also the Greek Bible. When, however, Greek inscriptions became more widely known to Hellenists in the seventeenth and eighteenth century, dialect specialists also broadened their perspective and introduced this new type of evidence into their manuals. This was a foundational step toward the development of a modern \isi{dialectology} of Ancient Greek in the early nineteenth century, usually associated – too narrowly – with Heinrich Ludolf Ahrens, even though the contributions of other pioneers like Albert Giese (1803–1834) also deserve further study \citep{Giese1837}. These nineteenth-century dialectologists, moreover, recognized that they were indebted to the efforts of their early modern predecessors, in particular those of Michael Maittaire.

The dialects were, finally, considered to offer a window on the ancient Greek world. They were not isolated linguistic media, but embedded forms of speech that allowed early modern Hellenists to construe a more lively image of this distant society. The close connection between the dialects, on the one hand, and Greek texts and tribes, on the other, tempted them to continue the long-standing tradition of projecting properties of the latter onto the former. Doric, for example, was a rustic and harsh dialect, because it was the dialect of \isi{bucolic poetry} and the rough Dorians. The existence of dialects in Greek was moreover believed to betray the versatility and maliciousness of the Greek people as a whole and to reflect also the diversified geographical and political landscape of ancient Greece.

Early modern Hellenists were greatly indebted to the work and ideas of their predecessors from ancient and Byzantine Greece and, to a much lesser extent, the ancient Roman world. Symptomatic of this fact is that in each chapter I have had to outline earlier conceptions before tackling early modern views. Scholars in the Renaissance and later were not mere parrots, however, and produced many original contributions of their own. They did not blindly adopt ancient Greek and Byzantine ideas when they recognized that they were inadequate, but rather tried to formulate more consistent solutions, grounded not only in the authority of ancient authors but also in an assessment of the Greek language itself, its different varieties, and their history as well as other pieces of evidence. In other words, they systematized the ideas of ancient and medieval grammarians, while at the same time surpassing them. They saw a wider application for the knowledge of the dialects, which their predecessors had limited more strictly to grammar and philology and which they extended, first and foremost, to Bible studies and antiquarian investigations. They designed, in addition, more transparent methods of presenting Greek dialectal features. In contrast to their predecessors, who principally discussed Greek diversity per dialect, many early modern Hellenists arranged their treatment of the matter per grammatical category, often making use of elucidating tables, intended to facilitate memorization. This was, however, a more superficial innovation, since descriptions of dialectal features remained fairly traditional throughout the early modern period: Hellenists usually operated with the traditional frameworks of \isi{pathology} and letter permutations, while taking over information found in ancient and medieval treatises. Yet gradually other linguistic domains such as accent, the lexicon, morphology, and syntax were introduced into handbooks for the Greek dialects. This development was based primarily on a more thorough reading of ancient grammatical works, where Hellenists encountered scattered comments on various features of the Greek dialects. A handful of seasoned philologists went beyond the dialectological accounts contained in earlier treatises to make an actual contribution of their own. The sixteenth-century Hellenist prodigy Henri Estienne was one of the most intriguing exceptions in this regard, as he systematically relied on his own reading of Greek literary texts in order to cast doubt on the dialectal features transmitted by earlier scholarship.

The Greek dialects were a subject that appealed to scholars across Western Europe, and it is difficult, if not impossible, to discern regional traditions or focuses. The dialects belonged to the transnational Republic of Letters (cf. \citealt{Bots1997}). This does not mean, of course, that no hotbeds of Hellenism can be pinpointed. What were the main centers of interest in the Greek dialects throughout the early modern period? The Greek turn was initiated in the north of the Italian peninsula in cities like Florence and Venice. The latter played a major role in the early modern history of Ancient Greek \isi{dialectology}, since it was there that in 1496 Aldus Manutius issued for the first time three Greek treatises on the dialects that were to be read enthusiastically for decades to come, before being superseded by new handbooks by Western Hellenists. These were published in different parts of Western Europe, from Italy and Spain to Denmark and from England to the Holy Roman Empire. It is not easy to identify true centers of reflection on the Greek dialects, because of the nature of the evidence. The manuals appeared scattered across Europe and often catered to local didactic needs, their composition being encouraged by a lack of available textbooks rather than by a scientific interest in the matter. I believe that it is nevertheless possible to label some cities as major centers, including not least of all Paris and Wittenberg, both cities where the teaching of Greek was well established and outstanding throughout most of the early modern period. Several handbooks on the Greek dialects were published in these two cities.\footnote{See e.g. \citet{Schmidt1604}, published in Wittenberg, and \citet{Merigon1621}, printed in Paris.} In the \isi{Protestant} stronghold of Wittenberg, Hellenists did not limit themselves to didactic publications, as was largely the case in Paris, but also ventured original investigations into the Greek dialects, their nature, and their history. This often occurred in the form of academic disputations (see e.g. \citealt{Thryllitsch1709}), as Greek had been incorporated into the university curriculum ever since Philipp Melanchthon’s appointment in 1518. More scientific concerns were also behind the publication of Claude de \citegen{Saumaise1643a} commentary on the Greek language, triggered by Daniel Heinsius’s positing of a Hellenistic dialect, which Saumaise keenly contested. This academic dispute took place in the city of Leiden, the main hub of Hellenism in the Dutch Republic. Given the didactic goals of most manuals on the Greek dialects, direct debate was limited in them, and their main concern was either to accumulate all known dialect features or to select the most important properties, depending on their aims and intended student readership. In philological and historiographic dissertations, on the other hand, conceptions of the Greek dialects were sometimes heavily discussed and could even stir up fierce controversies, as in the case of Heinsius and Saumaise; their rivalry was, however, spurred as much by personal hostilities as it was by intellectual disagreement.

One of the main arguments I have tried to deploy in the present book is that knowledge and awareness of the ancient Greek dialects were frequently valorized by grammarians concerned with a wide range of different languages. This usually happened to stress either the similarities or the differences between Greek and \isi{vernacular} dialects. In the former case, Greek was a model context; its variation encouraged scholars to grammatically codify their native tongues in spite of the enormous dialectal diversity they encountered in them. In the latter case, \isi{vernacular} grammarians stressed the particularity of the Greek situation, where the dialects were canonized for literary reasons; this was considered impossible or undesirable for \isi{vernacular} tongues, which they were trying to mold into a uniform whole. The Greek dialects were taken as a point of comparison not only for language-ideological purposes. The close kinship among them was also a useful descriptive reference point for many philologists, in particular those engaged in charting the interrelationships of the so-called Oriental tongues and in describing various aspects of non-Greek languages. In contrast, the Greek dialects were often in need of explanation themselves, especially in a didactic context. Teachers broaching the topic of Greek linguistic diversity frequently felt compelled to draw their students’ attention to their native dialects in order to clarify what the Greek dialects were. These cross-linguistic comparisons often involved the suppression of differences between Greek dialects and dialects in other languages, usually in sociolinguistic terms, as the Greek dialects were rather exceptional in having reached literary status.

Given the frequency of the comparison of the Greek dialects with other contexts of language-internal diversity, I would like to venture two final thoughts on the broader impact of the Renaissance rehabilitation of the Greek language and its dialects on the early modern study of linguistic diversity. Firstly, it does not seem inconceivable that the intense early modern confrontation with the Greek dialects greatly stimulated the comparative study of languages as it started to emerge in the latter half of the sixteenth century.\footnote{This is often dubbed “precomparativism”. See e.g. \citet{Considine2010b}, with further references.} In fact, in the early sixteenth century, Greek language teachers introduced a new presentation of dialect data that likely contributed to triggering a comparative reflex among philologists. At that time, Hellenists started to put the different dialectal realizations of one and the same form next to each other in large tables printed in their manuals. This made it possible to assess at a single glance the similarities and differences between related word forms. It is moreover not farfetched to assume that the cross-dialectal letter variations philologists perceived in Greek increased their awareness of similar variations cross-linguistically. This conclusion becomes very tangible indeed when one thinks of the frequent early modern paralleling of sigma–tau variation in Greek with s–t alternation in Germanic. It is further corroborated, I believe, by the inclusion of Latin in the Greek linguistic sphere; the language of the ancient Romans was frequently claimed to be in a close relationship with especially Aeolic and Doric, which some scholars even attempted to prove by means of concrete linguistic correspondences. It led, in other words, to an active comparison of Latin with varieties of Greek. The comparative reflex of early modern philologists was, in short, partly fostered by their rediscovery of Greek dialectal diversity.

Secondly, the confrontation with the Greek dialects contributed to exciting awareness of, and curiosity about, language-internal diversity of other tongues. From the second half of the seventeenth century onward, non-Greek dialects likewise received book-length treatments. Apart from the popular format of dialect wordlists and lexica, numerous descriptions of \isi{vernacular} dialects and their peculiarities were published, usually written out of \isi{patriotic sentiment} or antiquarian interest.\footnote{For \isi{patriotic sentiment}, see e.g. \citet{Meisner1705} on Silesian. For antiquarianism, see e.g. \citet{Oberlin1775}. For curiosity-driven dialect wordlists, see the excellent account of \citet{Considine2017}.} Several of the authors of these works were trained as Hellenists and actively put their knowledge of the Greek dialects to use in charting the unique features of individual dialects. These included, most notably, Michael Richey (1678–1761), compiler of a dialect lexicon of Hamburg speech \citep{Richey1743} and author of a \textit{Dialectologia Hamburgensis} \citep{Richey1755}, and Sven Hof, who published in 1772 a monograph on the Västergötland dialect in Sweden, which stands out for its frequent usage of Greek terminology and examples. Early sixteenth-century humanists such as Juan Luis Vives were, one might conclude, getting lost only in the vast labyrinth of the Greek dialects, but by the end of the eighteenth century scholars were wandering in a much larger one, that of dialect diversity \textit{tout court}. For the first time in history, the universality of the phenomenon was widely recognized, a feat which may count as a fundamental achievement of early modern dialect studies.

\begin{center}
\large⁂
\end{center}

\noindent In this book, I have only provided a first exploration of the history of Greek \isi{dialectology}. Much work remains to be done. More case studies are needed to deepen our understanding of the aspects I have highlighted here and to adjust my conclusions wherever necessary. It would, for instance, be interesting to analyze more closely the use of Greek in treatises on \isi{vernacular} dialects. Closer studies of individual Greek dialect manuals and their unique book-historical aspects are needed in order to understand how these books were actually used by early modern readers and students of Greek. This includes investigating the annotations contained in many copies of these handbooks. It would moreover be fruitful to find out to what extent the insights I have formulated here can be extrapolated to ideas scholars have expressed in the large body of extant manuscript material, which I have included only very marginally in my discussion because it was not feasible to survey it all in this book. Finally, I have pointed out that the so-called pioneers of Ancient Greek \isi{dialectology} were partly indebted to some of their eighteenth-century predecessors, a remarkable continuity which has hitherto remained under the radar of historians of linguistics and which provides an intriguing example that counters the nineteenth-century trend to neglect early modern insights.


% \begin{itemize}
% \item 
% historical persons
% 
% \item 
% languages \& dialects
% 
% \item 
% (subjects, if desired, but the book is already structured thematically)
% 
% \end{itemize}

 
