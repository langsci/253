\addchap{Editorial choices}
\hypertarget{Toc19704800}{}
In order to facilitate reading, I have opted to offer only English translations of quotes and titles in the main text. The original text can be found either in the footnotes in the case of quotes or in the bibliography in the case of titles. Unless otherwise indicated, translations are mine. I have transcribed Greek keywords quoted in the main text into the Latin alphabet (with the original between round brackets), but in order to avoid overloading the footnotes I have refrained from doing the same for Greek citations appearing there. I have regularized Latin orthography, opting for ⟨u⟩ and ⟨i⟩ spellings, but I have preserved the original orthography of early modern vernacular texts, standardizing only ⟨u⟩/⟨v⟩ and ⟨i⟩/⟨j⟩ alternations in accordance with modern practice. For both Latin and vernacular quotes, I have regularized capitalization and punctuation marks to current practices. Errors in the source texts are marked with “[\textit{sic}]”. Names of Greek, Latin, and early modern authors have been anglicized whenever this is common in secondary literature. Otherwise, I have opted for the most common form. Life dates are provided in the main text when an author is first introduced. Finally, I refer to early modern dissertations by mentioning the name of the chairman – the \textit{praeses} – as well as the student presenting the dissertation – the \textit{respondens} – unless there are sound reasons to suppose that one of both persons should be considered the sole author of the dissertation.\footnote{On the problem of authorship in early modern dissertations, see e.g. \citet{Considine2008b}.}
