\chapter{Old, older, oldest: Writing the linguistic history of Greek}\label{chap:5}

An event of great importance in the history of Greek studies occurred in 1518, when Philipp Melanchthon was appointed as the first professor of the language at the university of Wittenberg. His teaching there laid the groundwork for the strong Protestant tradition in this discipline. In the century or so after the installment of the Greek chair in Luther’s city, countless Hellenists were educated in the humanist spirit. A major goal they pursued was to arrive at a fuller understanding of the New Testament in its original language, as I have pointed out in the previous chapter. An exponent of Protestant Hellenism was Lorenz Rhodoman (1546–1606). A student of several of Melanchthon’s pupils, Rhodoman later became professor at the Wittenberg academy himself. He was a prolific scholar and poet, who showed off his mastery of Greek more than once in his compositions. When in 1604 one of his promising students left the city, he delivered a lengthy oration on the Greek language and its historical development, which was printed in Strasbourg the next year \citep{Rhodomanus1605}. Part praise, part history, the text constituted a precursor to later histories of the Greek language, a genre flourishing particularly in the Holy Roman Empire. This can be regarded as a symptom of the wider interest in the historical development of languages during the early modern period. A major achievement of humanist scholars in this regard was the formulation of the idea that many European and Asian languages, including Greek, were related and, in fact, descendants of a lost original language, often dubbed “Scythian”. This so-called Scythian hypothesis foreshadowed to some extent the later modern concept of \textsc{Proto-Indo-European}.\footnote{On the Scythian hypothesis, see e.g. \citet[34--39]{Metcalf2013}; \citet{Droixhe1980}; \citet[]{Swiggers1984, Swiggers1998}; \citet{Villani2003}; \citet{Considine2010}; \citeauthor{VanHal2010b} (\citeyear{VanHal2010b}; \citeyear[esp. 335–401, 473–475]{VanHal2010a}).} The increasing interest in language history forced humanists to think about the place of the Greek language and its dialects in it. Yet before moving to early modern ideas, I have to briefly consider the very few earlier remarks on the matter that are extant. How did ancient Greek and Byzantine scholars picture the history of the Greek language and its dialects?

\section{The linguistic history of Greek in ancient and medieval scholarship}\label{sec:5.1}

In Chapter \ref{chap:2}, I showed that Strabo, likely inspired by Alexandrian scholarship, proposed a classification into four dialects – Ionic, Attic, Doric, and Aeolic – and that he saw a close kinship between Ionic and Attic, on the one hand, and Doric and Aeolic, on the other. Strabo did more than merely suggesting kinship, however, as he framed the Greek dialects into a historical scheme. He claimed that initially old Attic was the same as Ionic, and that Doric was identical to Aeolic, suggesting that there were initially only two dialects (\textit{Geographica} 8.1.2). In the Byzantine period, the Homer commentator Eustathius of Thessalonica (ca. 1115–1195/1196) took over Strabo’s language-historical scheme (\textit{Commentarii ad Homeri Iliadem} 1.14). Strabo’s brief account is the most extensive consideration of the historical position of the Greek dialects found in ancient and medieval texts, which indicates that scholars of these eras were barely interested in this question. It is moreover a little surprising that Strabo did not go further back. One might have expected him to point out that the two branches, Attic–Ionic and Doric–Aeolic, were also originally one language, as they went back to one and the same mythological ancestor, Hellen, and Greeks were aware that they spoke in essence a single tongue (see \citealt{Morpurgo1987}). The idea of a Greek protolanguage was, however, usually not made explicit by Greek authors, perhaps because they regarded it as obvious. An exception is the early Byzantine scholar John Philoponus, who assumed that Greek was originally unitary and believed that a process of geographical dispersion was responsible for ethnic and linguistic diversification. Philoponus argued this in the following anacoluthic sentence:

\begin{quote}
For when [the children of Hellen] were dispersed toward multiple places and no longer preserved the same speech, but changed along with their migration also their speech, it happened that they were called dialects.\footnote{John the Grammarian (\citealt{Manutius1496Thesaurus}: 236\textsc{\textsuperscript{v}}): “διασπαρέντων γὰρ τoύτων, εἰς πλείoνας τόπoυς, καὶ τὴν αὐτὴν φωνὴν, oὐκ ἔτι φυλαξάντων\text{\textgreek{;}} ἀλλὰ τῇ τῶν τoύτων μεταβoλῇ άμα [sic] καὶ τὴν φωνὴν μεταβαλλόντων, συνέβη διαλέκτoυς λέγεσθαι”.}
\end{quote}

Strabo distinguished between old and new forms of a dialect, especially with reference to Attic. Other Greek scholars did so, too. Around the same time, the literary critic Dionysius of Halicarnassus (ca. 60–after 8/7 \textsc{bc}) expressed the view that Plato and Thucydides wrote in an older variety of Attic (\textit{De Lysia} 2). The philosopher Sextus Empiricus (\textit{fl.} ca. \textsc{ad} 190–210) also made a distinction between old and current Athenian speech (\textit{Aduersus mathematicos} 1.228). A Byzantine commentator on the Hellenistic poet Theocritus’s work distinguished between the old, harsh Doric of older poets and the new, mellower Doric of Theocritus. He seemingly suggested that the latter was influenced by other dialects, most importantly the allegedly effeminate dialect of the Ionians.\footnote{\textit{Scholia in Theocritum (scholia uetera)} \textsc{f} a.–d. For this attitude toward Doric and Ionic, cf. Chapter 7, \sectref{sec:7.2}.}

Strabo, Eustathius, and John the Grammarian put the four Greek dialects on the same chronological scale. Attic, Ionic, Doric, and Aeolic derived from the legendary ancestor Hellen. This happened through an unspecified process of change, and as a result different stages of dialects could be distinguished. Other Greek thinkers, however, preferred to ignore the traditional ethno-mythological scheme and projected one particular dialect as the oldest, thus introducing an imbalance into the chronological relationship between the dialects. The mysterious philosopher Pythagoras (\textit{fl.} 6th/5th cent. \textsc{bc}) went so far as to claim that his preferred medium of communication, Doric, was not only the most harmonious but also the oldest Greek dialect, at least if one is to believe his biographer Iamblichus (ca. \textsc{ad} 240–325; \textit{De uita Pythagorica} 34.242–243). The Early Christian author Epiphanius of Salamis (ca. 310/320–403) seems to have reserved this honor for Ionic, which he associated with the biblical figure of Javan, a son of Japheth and grandson of Noah whom he identified with Ion, the mythological forefather of the Ionians (\citealt{VanRooy2013}: 44 n.43).

Greek scholarship on the dialects was largely Hellenocentric; other languages were not invoked in treatments of this theme. In Roman times, however, a clearly distinct dimension to the historization of the Greek dialects manifested itself. Roman authors acknowledged that their culture was greatly indebted to the Greek world, a realization that made them consider the idea that this was perhaps also the case in terms of language. In the first centuries \textsc{bc} and \textsc{ad}, several scholars, including Dionysius of Halicarnassus, Varro, and Quintilian, claimed that Latin descended – at least partly – from Aeolic, an idea which some modern scholars have dubbed “Aeolism”, even though it was hardly the full-fledged theory this term might suggest it was.\footnote{E.g. Dionysius of Halicarnassus, \textit{Antiquitates Romanae} 1.90.1. On Aeolism, see especially \citet[]{Stevens2006}. For Quintilian, see \citet[149]{Fogen2000}. Cf. also \citet[117--119]{Schopsdau1992}.}

Scarce though language-historical ideas in Greek scholarship may be, early modern scholars gratefully took them as their starting point, quickly going beyond them. Not only did they systematize earlier thought, but – more importantly – they also greatly contributed to a better historical understanding of the history of the Greek language and the place of the dialects in it. They looked farther back in time and asked themselves: “How do the dialects relate to earlier stages of Greek?” They also looked into later developments: “What happened with the dialects after antiquity?” They zoomed out even further still to connect the Greek language in various ways to other tongues, often by means of mechanisms involving either a specific Greek dialect or the concept of \textsc{dialect}. In the remainder of this chapter, I will demonstrate the main contributions of early modern scholars to the better historical understanding of the Greek tongue and its variability.

\section{In Strabo’s wake}\label{sec:5.2}

Strabo’s idea of an original binary division between Attic–Ionic and Doric–Aeolic was an influential one; it was the classical answer humanists offered when treating the question of the historical relationships between the Greek dialects. The Dutch polymath and experienced Hellenist Hugo Grotius (1583–1645) formulated it as follows in a 1622 letter to one of his French contacts:

\begin{quote}
The most ancient division of the Greeks is into Ionians and Dorians, whence a variety of dialects spread itself into several branches, but all of them are to be reduced to these stocks. Just like the Attic dialect is part of the Ionic, but separated from the commonality with Ionic in certain properties, in like manner Aeolic pertains to Doric.\footnote{\citet[143]{Grotius1648}: “Graecorum antiquissima diuisio est, in Iones et Dores; unde dialectorum uarietas in plures se ramos fudit; sed qui omnes ad illas stirpes deferendi sunt. Sicut Attica dialectus pars est Ionicae, sed a communitate Ionicae certis proprietatibus distincta, ita Aeolica ad Doricam pertinet”.}
\end{quote}

This view, truly ubiquitous throughout the entire early modern period, was only discarded after linguistics emerged as a separate field of research in the nineteenth century.\footnote{See e.g. \citet[64\textsc{\textsuperscript{r}}]{Sabellicus1490}; \citet[235]{Estienne1573}; \citet[563]{Lancelot1655}; \citet[\textsc{c.2}\textsc{\textsuperscript{r}}–\textsc{c.2}\textsc{\textsuperscript{v}}]{Schwartz1702}; \citet[i]{Maittaire1706}; \citet[82]{Vitringa1712}; \citet[\textsc{xv}]{Castelli1769}; \citet[\textsc{a.2}\textsc{\textsuperscript{r}}]{Hauptmann1776}.} One author held that there were originally four dialects, but that through mixture Doric and Aeolic eventually merged, resulting in three main dialects, thus historically reversing Strabo’s Doric–Aeolic unity \citep[20]{Gedike1782}.

The authority of Strabo eclipsed actual empirical evidence. Indeed, unlike Ionic–Attic unity, Doric–Aeolic identity could not be convincingly corroborated by linguistic facts, even though a few early modern scholars tried to do exactly that. Most notably, Henri Estienne attempted to substantiate such claims of kinship by pointing to a number of alleged linguistic similarities between Aeolic and Doric. Estienne remarked, among other things, that the nominative plural of nouns like \textit{hippeús} (ἱππεύς), ‘knight’, was the same in both dialects: \textit{hippêis} (ἱππῇς), as opposed to Koine \textit{hippeîs} (ἱππεῖς).\footnote{\citet[25--26]{Estienne1581}: “Sicut enim ἱππῇς et ἱερῇς et βασιλῇς pro ἱππεῖς et ἱερεῖς et βασιλεῖς dicunt, ita dialectus Aeolica necnon Dorica in infinitiuis hac mutatione utuntur”. Cf. also \citet[179]{Trendelenburg1782}.} The assumption of an originally binary division implied for many scholars, as it had done for Strabo, that there were two forms of each dialect: an older, rougher and a newer, more elegant form (see e.g. \citealt{Mazzocchi1754}: 119). This was emphasized especially often for Attic and Ionic.\footnote{See e.g. \citet[18]{Hauptmann1737}; \citet[137]{Walch1772}; \citet[\textsc{iv–v}]{Facius1782}.} Some scholars even associated specific linguistic and alphabetic properties with the different diachronic stages of a dialect. The usage of the letter xi ⟨ξ⟩ instead of the sigma ⟨σ⟩, the absence of the letter upsilon ⟨υ⟩ in long vowels, and the epigraphic usage of capital eta ⟨H⟩ to denote aspiration were associated with Old Attic, claimed to be identical to Ionic and to “degenerate not very much from ancient Hellenic”, conceived as a kind of ancestral Greek language.\footnote{\citet[4--5]{Munthe1748}: “nec ualde degenerans a prisca Hellenica”.}

\section{The dialects between Greek and biblical genealogy}\label{sec:5.3}

Like their ancient and medieval predecessors, early modern scholars tried to answer the question as to how the dialects and the tribes speaking them fitted into traditional genealogical schemes. As I have repeatedly pointed out, the Greek dialects were usually linked closely – for etiological reasons – with the history of the four main Greek tribes and their mythological forbears. This connection persisted in the early modern period, even though a number of eighteenth-century scholars rationalized the issue and rejected the link with the mythological figures but not the association with the Greek tribes.\footnote{See e.g. \citet[3--4]{Walper1589}; \citet[166--167]{Labbe1639}; \citet[73]{Vitringa1689}; \citet[\textsc{xxiiii–xxvi}]{Harles1778}. The mythological link was rejected by \citet[\textsc{c.4}\textsc{\textsuperscript{v}}\textsc{–d.1}\textsc{\textsuperscript{r}}]{Thryllitsch1709} and \citet[108--110]{Hemsterhuis2015}, who stressed that dialectal diversification requires a large time span (see \citealt{Gerretzen1940}: 151–152).} In the very same attempt at demythologizing the Greek dialects, the diversification of the Greek dialects was sometimes related to colonization movements of the Greek tribes. For instance, the Enlightenment pedagogue Friedrich \citet[12]{Gedike1782} pointed out that the dialects could greatly contribute to elucidating the initial stages of the Greek states and their colonies and vice versa.

Scholars also tried to fit the history of the Greek dialects into the genealogical framework of the Bible. This endeavor was still rare in the Greek tradition, the most significant exception being the Early Christian scholar Epiphanius of Salamis, who identified Javan with Ion and claimed that Ionic was the oldest dialect (see \sectref{sec:5.1} above). In the early modern period, attempts at framing the Greek dialects into biblical history were more intensive.\footnote{\citet[\textsc{c.1}\textsc{\textsuperscript{v}}]{Schwartz1702} were aware of such attempts at reconciliation.} Let me demonstrate this by means of a striking example. In his annotations on a Greek inscription and its peculiar dialect, the Oxford chronologer Thomas Lydiat (1572–1646) initially tried to prove the Hebrew origin of the “founders” (\textit{auctores}) of the three Greek dialects, Dorus, Aeolus, and Xuthus (Ion’s father; Lydiat in \citealt{Prideaux1676}: \textsc{ii}.21). In his later notes, however, Lydiat offered an account that was more in agreement with Greek tradition and seemed to be an expanded version of Strabo’s scheme. He now claimed that there was only one dialect at first, as long as the Greeks lived in Thessaly. This original dialect subsequently broke up into Aeolic and Ionic. Ionic then disintegrated into Attic and Ionic, whereas Aeolic developed into Aeolic, Boeotian, and Doric (Lydiat in \citealt{Prideaux1676}: \textsc{ii.134}, \textsc{ii.}155). Lydiat failed to see that he was offering contradictory outlines of Greek linguistic history, grounded in different traditions.

\section{The early stages of the Greek language}\label{sec:5.4}

Greek scholars were not very concerned over the origin and early stages of their language, which they usually approached from a static and synchronic perspective. Humanists, however, developed a broad interest in the diachronic development of language and linguistic diversity, from which scholarship on the Greek tongue also benefitted. How did early modern authors sketch the early stages of the Greek language? And what was the place of the dialects in them?

A frequently proposed solution consisted in propagating a specific dialect as the oldest form of Greek. Following biblical genealogy, early modern scholars often claimed Ionic primacy, as Epiphanius had done in late antiquity (e.g. \citealt{Alsted1630}: 2019; \citealt{Von1705}: 17). Due to the assumed close kinship between Attic and Ionic, Ionic primacy came to be equated with Ionic–Attic primacy by some Hellenists (see e.g. \citealt{Schmidt1604}: 5–7). Occasionally, Attic was claimed to be the pristine dialect, from which Ionic, and later on Doric and Aeolic, originated (\citealt{Baile1588}: 4\textsc{\textsuperscript{r}}–5\textsc{\textsuperscript{r}}); this view was possibly motivated by the common idea that Attic was the most elegant dialect, used by the most valued prose authors. Several scholars, often inspired by Iamblichus’s biography of Pythagoras, proposed Doric as the oldest Greek dialect.\footnote{See e.g. \citet[860]{Goropius1569}; \citet[29]{Burton1657}; \citet[118]{Mazzocchi1754}; \citet[\textsc{iv}]{Facius1782}; \citet[21]{Gedike1782}.} Aeolic was only rarely suggested to be the oldest dialect. A French orientalist did so in 1697 while oddly stating that this dialect stemmed from Elisa, the son of Javan, from whom he thought the Ionic dialect to have originated \citep[110]{Thomassin1697}. He apparently did not realize that this idea obviously compromised the chronology of biblical genealogy.

In short, scholars often relied on ancient or biblical authorities to propagate one dialect or another as the oldest form of Greek. These proposals were usually not motivated by any linguistic evidence. A major exception to this tendency was, however, the case of Doric primacy. The antiquity of this dialect was often allegedly proved by, among other things, the prevalence of monosyllabic words and the low frequency of double consonants claimed to be inherent to it.\footnote{For the former alleged piece of evidence, see \citet[17]{Munthe1748}. For the latter, see \citet[\textsc{xxvi}]{Harles1778}.} Here, the common early modern idea that monosyllabicity indicated antiquity was applied to the Greek language (for this idea, see \citealt{Jansen1995}: 297–300). One German author cited the intrinsic ruggedness of the Doric dialect as evidence for its antiquity, probably presupposing that linguistic cultivation and polishing was a time-consuming process \citep[21]{Gedike1782}.

In the seventeenth and eighteenth century, a number of scholars were not content with simply positing a specific dialect as the most ancient form of the Greek tongue. Instead, they suggested that there was some kind of prehistoric, unitary Greek before the emergence of dialectal diversification. They were in other words gaining deeper insight into the early stages of this language. The French classical scholar Claude de Saumaise, in his 1643 monograph on the Greek language, its origin, and its dialects \citep{Saumaise1643a}, posited an original Hellenic tribe speaking an ancient variety of Greek he dubbed “Hellenic”. This tongue, Saumaise claimed, first came to be divided into the different Greek dialects before evolving into the Greek Koine. In Saumaise’s wake, a dissertation presented in 1702 at the Wittenberg academy argued that there was a now extinct ancestral Greek language, which grammarians have been able to distill out of the common features of the different dialects. In other words, the Koine was a grammatical reconstruction of the Greek protolanguage, according to this text (see Chapter 2, \sectref{sec:2.9}). Other eighteenth-century scholars likewise presupposed a now lost Greek protolanguage, termed “Pelasgian” and closely associated with Ionic. Let me look at two instructive examples. Firstly, according to the Lutheran theologian Valentin Ernst Löscher (1673–1749), original Pelasgian Greek may be lost, but it partly lives on in the dialects descending from it. These have preserved the original Pelasgian roots to varying degrees of accuracy, with Ionic safeguarding them best. Indeed, out of this dialect, the roots can be reconstructed, Löscher explicitly stated.\footnote{\citet[24–25, 84–85]{Loscher1705}, where the Latin verb \textit{restituere} is used to express the notion “to reconstruct”.} Secondly, the Dutch orientalist Albert Schultens (1686–1750) frequently compared Oriental (Semitic) with ancient Greek linguistic diversity and emphasized that both have four dialects deriving from one – now lost – common ancestor, called “Pelasgian” or “Ionic” in the case of Greek.\footnote{\citet[\textsc{lxxv–lxxvi,} \textsc{xcii–xciv,} \textsc{civ}]{Schultens1748}. For Schultens’s concept of a \textsc{Semitic} \textsc{protolanguage}, see e.g. \citet[esp. 84--86]{Eskhult2015}. See also Chapter 8, \sectref{sec:8.3.1} of this book.} Schultens formulated a key methodological principle in this context: comparing related dialects helps to penetrate into the nature of the extinct tongue.\footnote{See Schultens in \citet[ §§\textsc{cxv–cxx}]{Eskhult_albert_nodate} [ca. 1748–1750]. Cf. also \citet[19--20]{Schultens1738a}; Schultens in \citet[ §§\textsc{xc–xcii}]{Eskhult_albert_nodate} [ca. 1748–1750] where also a Proto-Germanic tongue is suggested).} There were several other eighteenth-century scholars who assumed a prehistoric, extinct Greek language.\footnote{See e.g. \citet[1--2]{Munthe1748}; \citet[104--106]{Hemsterhuis2015}; \citet[15]{Wise1758}.} Considering all the evidence cited here, it seems safe to conclude that the concept of \textsc{Proto-Greek} was an achievement of early modern rather than modern language studies, even if no straightforward terminology was coined to express it and no rigorous comparative method was designed to prove it by means of Greek dialect data.

\section{The later fate of the Greek dialects: Extinction and vestiges}\label{sec:5.5}

How did early modern scholars picture the fate of the ancient Greek dialects in late antiquity and beyond? As with the Greek protolanguage, the idea of extinction was central to discussions of this question, since several authors argued that the ancient Greek dialects had perished in late antiquity. This idea was explored most influentially by Claude de Saumaise, who described the Greek language situation at the time of Justinian’s reign (reigned 527–565) as follows:

\begin{quote}
all varieties of the dialects were abolished. […] Finally, it came through a progression of time to the point that all differences of dialects were done away with among the Greeks and a uniform shape of the Greek language spread over the whole of Greece, and an extremely corrupt one.\footnote{\citet[446--447]{Saumaise1643a}: “omnes dialectorum uarietates abolefactae sunt. […] Eo postremo deuentum est temporis progressu, ut tollerentur omnes dialectorum differentiae apud Graecos et uniformis facies Graeci sermonis per uniuersam Graeciam diffunderetur eaque corruptissima”.}
\end{quote}

It would be in vain, \citet[447–449]{Saumaise1643a} added, to retrieve the dialectal variation of Ancient Greek in the vernacular tongue. As a consequence of the extinction in late antiquity, Byzantine grammarians were not in a position to assign the Greek dialects to the different regions of Greece, as they were no longer spoken. Instead, they linked them to the names of literary authors. This, \citet[450, 453–455]{Saumaise1643a} ingenuously asserted, is also why Byzantine scholars such as John the Grammarian and Gregory of Corinth changed the Greek word \textit{tópos} (τόπoς) into \textit{túpos} (τύπoς) in the traditional definition “A dialect is speech showing the particular character of a region (\textit{tópos})/model (\textit{túpos})”. Even though Saumaise may well have been right on this point, it is difficult to back up this conjecture with actual evidence; a thorough study of the complex transmission of these texts could clarify the matter (see \citealt{VanRooy2016d}: 264 n.47). In Saumaise’s tracks, several scholars suggested similar evolutions for the dialects in late antiquity. For example, in one dissertation, it was claimed that “the common language resembles the Attic dialect most closely, the Attic dialect has entirely obscured the remaining dialects, and, finally, the common language has abolished all” after Alexander the Great’s conquests. The dialects were all “absorbed into the common language”.\footnote{\citet[\textsc{a.5}\textsc{\textsuperscript{r}}]{Schorling1678}: “Lingua communis proxime ad Atticam accedit, Attica plane obscurauit reliquas dialectos, communis tandem omnes aboleuit. […] Ita fuere absorptae dialecti sub imperio Seleucidarum in Syria et Ptolomaeorum in Aegypto et in linguam communem redactae”.}

The French philologist Charles Du Cange (1610–1688) refuted Saumaise’s claim that the dialects had perished entirely. Du Cange did so by quoting the account of the Greek scholar Symeon Cabasilas (1546–after 1605), according to whom Vernacular Greek contained remnants of the four ancient dialects.\footnote{ \citet[viii]{Du1688}, referring to Cabasilas in \citet[462]{Crusius1584}.} Inspired by ideas such as Cabasilas’s, a few early modern scholars tried to trace properties of contemporary Vernacular Greek to the ancient dialects. The German theologian and grammarian Johann Tribbechow (1677–1712) set out to prove that the “vulgar Greek language” (\textit{lingua Graeca uulgaris}) had taken elements from all the ancient dialects. Tribbechow did so in a dissertation on the origin and nature of Vernacular Greek, prefixed to his grammar of this tongue.\footnote{\citet[a.3\textsc{\textsuperscript{r}}]{Tribbechow1705}. The Greek grammarian Romanos Nikiforos, writing ca. 1650, likewise relied on the traditional ancient Greek dialects to account for vernacular forms (see e.g. \citealt{Nikiforos1908}: 40, 45).} In contrast to Cabasilas and Du Cange, he took great pains to support his hypothesis by means of empirical – but largely faulty – linguistic evidence. For example, Vernacular Greek allegedly followed Attic in supplementing certain verbs – e.g. \textit{aréskō} (ἀρέσκω) – with the accusative instead of the dative case. Ionic influence was allegedly visible in the accusative and nominative feminine plural forms of the Vernacular Greek definitive article; instead of ancient \textit{taîs} (ταῖς, actually dative case) and \textit{hai} (αἱ), Greeks now wrote in Ionic fashion \textit{têis} (τῇς) and \textit{hē} (ἡ), respectively – forms likely pronounced [tis] and [i] as in Modern Greek. The high frequency of the letter alpha, in turn, was supposedly inherited from Doric – e.g. in vernacular Epirotic verbal endings such as \textit{epígaman} (ἐπήγαμαν), ‘we went away’, instead of more usual \textit{epígamen} (ἐπήγαμεν). The addition of the particle -\textit{ske} (-σκε) was likewise Doric, Tribbechow claimed. The Aeolic dialect allegedly surfaced in the accusative and nominative feminine plural adjective \textit{kalaîs} (καλαῖς), ‘good’ – i.e. Modern Greek \textit{kalés} (καλές) – replacing ancient \textit{kalás} (καλάς) and \textit{kalaí} (καλαί). Tribbechow attributed the absence of aspiration at the beginning of words in Vernacular Greek – so-called psilosis – likewise to this dialect. Uncovering the dialectal origin of a vernacular form often demanded great effort, he emphasized (\citealt{Tribbechow1705}: a.3\textsc{\textsuperscript{v}}). Tribbechow’s precise reasons for putting forward this hypothesis are unclear, but it might have been a strategy to elevate the status of Vernacular Greek by narrowing the gap with its ancient counterpart. Demonstrating continuity between both forms of Greek would, in this scenario, accord prestige to the vernacular variant he was describing in his grammar. In the nineteenth century, Greek scholars entertained the idea of ancient–vernacular continuity in an adapted form, known as Aeolodorism, the Romantic hypothesis that Vernacular Greek derived from the ancient Aeolic and Doric dialects rather than from Medieval Greek. This idea, put forward in a number of Greek grammars of the time, was definitively refuted by the Greek linguist Georgios Hatzidakis (1843–1941; see \citealt{Argyropoulos2009}: 289; \citealt{Mackridge2009}: 264–265).

Before Du Cange and Tribbechow, other scholars had already suggested continuity between ancient and vernacular Greek dialects, but in a different fashion. Most significantly, the Tsakonian tongue was correctly accorded a privileged relationship with Ancient Greek. The German theologian Stephan Gerlach (1546–1612), a friend of Martin Crusius (1526–1607), was the first to elaborate on its particular status, even though he wrongly labeled speakers of Tsakonian “Ionians” rather than “Dorians”, as one would expect:

\begin{quote}
And all [Greeks], whatever areas they are from, understand each other, with the exception of the Ionians who, inhabiting fourteen villages in the Peloponnese between Nafplio and Monemvasia, use the ancient language, which, however, violates grammar in many respects. They understand a grammatical speaker, but a speaker of the vulgar language only very poorly. These are commonly called Tsakonians.\footnote{Gerlach in \citet[489]{Crusius1584} “Et omnes, quorumcumque locorum, se mutuo intelligunt, exceptis Ionibus, qui in Peloponneso inter Naupliam et Monembasiam, 14. pagos inhabitantes, antiqua lingua, sed multifariam in grammaticam peccante, utuntur, qui grammatice loquentem intelligunt, uulgarem uero linguam minime. Hi Zacones uulgo dicuntur”.}
\end{quote}

This brief remark remained the main source of information on Tsakonian for the remainder of the early modern period, until its rediscovery at the end of the eighteenth century.\footnote{Cf. \citet[44]{Howell1650a} and \citet[vii]{Du1688}. For its rediscovery, see \citet{Famerie2007}.} Still other scholars seem to have downplayed the differences between Ancient and Vernacular Greek by intuitively comparing the diachronic variation existing among them to dialect-level differences; Vernacular Greek was, in other words, a dialect of the language just like Attic and Doric were.\footnote{See e.g. \citet[47--48]{Castillo1678}; \citet[\textsc{i.}184, 4th sequence of pagination]{Chambers1728}; \citet[]{Freret1809}.} This idea was especially common among Greek scholars active in the late eighteenth century, who posited Vernacular Greek as an additional dialect next to the four or five traditional ones (\citealt{Mackridge2009}: 264, \citealt{Mackridge2014}: 138--139) .

However, the bulk of early modern scholars assumed, usually silently, that there was great \textit{dis}continuity between Ancient and Vernacular Greek. The fact that Vernacular Greek was often characterized as “vulgar” or “barbarous” can be taken to imply that its dialects – in contrast to their ancient counterparts – were also regarded as defective speech forms. Was this view indeed advanced in early modern times? This occurred to a certain extent. For example, it became customary to contrast Attic, the ancient literary variety par excellence, to Vernacular Athenian, conceived as the most barbarous and ridiculous form of Vernacular Greek. This idea, linked to the decline of Athens, which had become a small provincial town, was first expressed by three acquaintances of the German Philhellene Martin Crusius: the two Greek scholars Theodosius Zygomalas (1544–1607) and Symeon Cabasilas, and the German theologian Stephan Gerlach.\footnote{For Zygomalas, see \citet[99, 216]{Crusius1584}. For Cabasilas, see \citet[461]{Crusius1584}. See also \citet[91]{Rotolo1973}; \citet[185, 189–190]{Rhoby2002}. For Gerlach, see \citet[489]{Crusius1584}. Cf. \citet[194]{Ben-tov2013}, quoting Michael Neander.} It was repeated several times throughout the early modern period, primarily in German-speaking territory, where Crusius’s \textit{Turcograecia} (\citeyear{Crusius1584}), a book of miscellanea containing the relevant texts, was best-known.\footnote{See e.g. \citet[215]{Becman1673}; \citet[\textsc{a.3}\textsc{\textsuperscript{v}}]{Rodigast1685}; \citet[\textsc{ii}.824]{Hofmann1698}; \citet[1135]{[frisch]1730};\ia{Frisch, Johann Leonhard@Frisch, Johann Leonhard} \citet[9]{Gedike1782}.} This does not mean that all vernacular varieties were considered defective. Scholars promoted several different dialects as the best variety of the contemporary tongue. Most commonly, the speech of the Ottoman capital, Constantinople, was granted this status.\footnote{See e.g. Gerlach in \citet[489]{Crusius1584}; \citet[215]{Becman1673}; \citet[74]{Blount1680}; \citet[vii]{Du1688}.} \citet[a.4\textsc{\textsuperscript{v}}, a.7\textsc{\textsuperscript{r}}]{Tribbechow1705} mentioned the speech of Ioannina, at that time a flourishing intellectual center in Epirus (north-western Greece), alongside the Constantinopolitan dialect, contrasting both of them to the inferior Greek spoken in the Ceraunian Mountains (modern-day south-western Albania). The Swiss doctor and language cataloguer Conrad \citet[47\textsc{\textsuperscript{r}}]{Gessner1555}, however, believed the speech of the Peloponnese to be the purest, without providing any further specifications.

In short, several scholars seem to have agreed that the ancient dialects perished as native forms of speech in late antiquity. There was, however, discord about the degree of continuity between Ancient and Vernacular Greek. Some intellectuals argued that there were no traces whatsoever of the ancient dialects in contemporary Greek, whereas others, most importantly Johann Tribbechow, tried to prove that there were clear vestiges of the literary dialects in the vernacular tongue. In the latter case, ennobling Vernacular Greek by associating it more closely with its illustrious ancient predecessor may have been an underlying incentive.

\section{Aeolism and its early modern transformations}\label{sec:5.6}

Already during antiquity, Latin had been incorporated into the history of the Greek language and its dialects thanks to the idea known as Aeolism. As I have mentioned above, certain ancient scholars assumed, without relying on much linguistic evidence, if any at all, that Latin descended at least in part from the Aeolic dialect. Early modern scholars usually relied on ancient authorities such as Dionysius of Halicarnassus when claiming that Latin was principally or entirely derived from the Aeolic dialect, generally without adducing any additional proof.\footnote{See e.g. \citet[84]{Crinesius1629}; \citet[\textsc{xvii}]{Bentley1726}; \citet[76, 106]{Hemsterhuis2015}; \citet[30]{Munthe1748}; \citet[215--216]{Simonis1752}. On Renaissance views on the relationship of Latin to Greek, see \citet{Tavoni1986}.} The Dutch jurist Hugo \citet[144--146]{Grotius1648} was exceptional in trying to demonstrate by means of an extensive array of linguistic arguments that Latin derived from Aeolic. Grotius referred, for instance, to the short [a] sound allegedly present in both Aeolic and Latin words such as \textit{pháma} (φάμα) and \textit{fama}, ‘rumor, reputation’, which contrasted to the Doric long alpha and the Attic–Ionic long eta. Grotius also pointed out that the letter digamma ⟨F⟩ is present in the alphabets of both Aeolic and Latin, and that these varieties have similar morphological properties in certain verbal endings. The presenters of an early eighteenth-century dissertation, defended in Wittenberg, went even further than Grotius by systematically comparing and tracing back Latin pronunciation to that of Aeolic.\footnote{\citet{Thryllitsch1709}, which deserves further study.}

Interestingly, a number of authors preferred to adapt Greek history rather than to provide linguistic evidence in order to corroborate Aeolism. The grammarian of Greek Georg Heinrich Ursin (1647–1707), for instance, asked himself, “Where was the Aeolic dialect in usage?”, a question he answered as follows:

\begin{quote}
At first among the Aeolians, a Greek tribe, who left their fatherland, crossed to Asia, and, after establishing a settlement there and founding the region of Aeolis, instituted this dialect. Yet just like the Doric dialect, which was cognate to it, Aeolic thrived in that part of Italy which is called Magna Graecia. This is also why the Aeolic tongue is, above all, the mother of Latin.\footnote{\citet[509]{Ursin1691}: “\textit{Aeolica dialectus ubi in usu fuit?} Apud Aeoles Graeciae gentem primum, qui relicta patria in Asiam traiecere sedibusque ibi captis et Aeolide regione condita dialectum hanc instituerunt; quae tamen, ut et Dorica, ei cognata, in Italiae parte illa, quae magna Graecia dicitur, uiguit, unde et Aeolica lingua Latinae potissimum mater est”.}
\end{quote}

In other words, Ursin felt compelled to locate the Aeolians on the Italian peninsula in order to account for the alleged derivation of Latin from Aeolic, even though there were no Aeolic colonies in this area historically.\footnote{Cf. \citet[289]{[schulze]1711};\ia{Schulze, Johann Heinrich@Schulze, Johann Heinrich} \citet[\textsc{i}.69]{Ten1723}; \citet[30]{Munthe1748}; \citet[89]{Facius1782}; \citet[199]{Ries1786} for similar suggestions.} The idea that Aeolians migrated to Italy was no doubt inspired by the ancient myth about the settlement of Arcadians, who allegedly spoke Aeolic Greek, on the peninsula (see \citealt{Lamers2019}: 30, with further references).

Aeolism was not the only solution put forward to account for the supposedly Greek origin of Latin. Some scholars transformed it into what could be dubbed Graecism, the idea that Latin derived from Greek as a whole (see \citealt{Tavoni1986}: 214, 218; \citealt{Lamers2015}: 173–180, 190–192; \citeyear{Lamers2019}). In 1493, the Byzantine émigré Janus Lascaris (ca. 1445–1535) argued exactly this in his \textit{Florentine oration} by means of an elaborate etymological method, citing Doric words – e.g. \textit{pháma} (φάμα) as the equivalent of Latin \textit{fama} – while also relying on the Aeolic hypothesis \citep[179]{Lamers2015}. The popular \textit{pháma}–\textit{fama} example was probably taken from Priscian (\citealt{Lamers2019}: 36 n.34). Other authors proposed a different dialectal connection, which did make sense geographically: Doric, varieties of which were spoken in Magna Graecia in the south of Italy. The Protestant Hellenist Philipp Melanchthon, for instance, described Doric as follows, also making use of the \textit{pháma}–\textit{fama} example:

\begin{quote}
Doric is Sicilian, very close to Italy, familiar to Theocritus. It changes eta into alpha, e.g. \textit{phḗmē} (φήμη) \textit{pháma} (φάμα), something which we Latins are also accustomed to when making use of Greek words, as we are neighbors of the Dorians.\footnote{\citet[a.i\textsc{\textsuperscript{v}}]{Melanchthon1518}: “Dorica Sicula est, Italiae proxima, Theocrito familiaris, η in α mutat, φήμη φάμα, id quod et Latini solemus cum Graeca usurpamus, quippe Doribus uicini”.}
\end{quote}

Melanchthon, however, seemingly invoked geographical vicinity and language contact rather than genealogical derivation to explain the parallels between Doric and Latin, even though the term \textit{uicinus} was often used in the Renaissance to express not only closeness in location but also genealogical kinship. Later scholars posited a Doric origin for Latin in a more straightforward fashion, some of whom invoked particular linguistic features to support their claims.\footnote{See e.g. \citet[10]{Sylvius1531}; \citet[\textsc{i.}14]{Estienne1572}; \citet[208]{Merula1605}; \citet[1191]{[frisch]1730};\ia{Frisch, Johann Leonhard@Frisch, Johann Leonhard} \citet[20--21]{Gedike1782}. Both Frisch and Gedike adduced linguistic evidence.} Apart from Graecism and Dorism, a number of scholars proposed dialectally mixed solutions for the Greek origin of Latin. This usually consisted in claiming a Doric–Aeolic foundation for the language of ancient Rome.\footnote{E.g. \citet[11--12]{Anon.1613}. Cf. \citet[304--305]{Verwey1684}; \citet[a.2\textsc{\textsuperscript{v}}, 159]{Maittaire1706}; \citet[161]{Gesner1774}. \citet[270, 379]{Casaubon1650} suggested Aeolic and Sicilian Greek as the origins of Latin.} Some added to Aeolic and Doric other source languages such as Pelasgian or Etruscan and Umbrian.\footnote{For Pelasgian, see \citet[a.4\textsc{\textsuperscript{r}}]{Canini1555}. For Etruscan and Umbrian, see \citet[39]{Rudiger1782}.} One author posited a shared Aeolic–Celtic origin.\footnote{\citet[13]{Nicolson1715}, who did so on the authority of Wolfgang Lazius (1514–1565).}

Not every early modern scholar assumed that there was a close relationship between Latin and Greek. The renowned Hellenist Joseph Justus \citet{Scaliger1610} clearly separated both tongues in his classification of the languages of Europe and believed that they did not show any kinship at all. The orientalist Albert \citet[109]{Schultens1738b}, in turn, actively refuted the idea that Latin was a dialect of Greek when reflecting on the interrelationships of the so-called Oriental tongues. Still, the predominant early modern view remained that Latin was somehow connected to the Greek language and especially to one or more of its dialects, usually Aeolic and Doric. This was without a doubt connected to ideas about the alleged close kinship of these two Greek dialects and to the ancient presence of Doric varieties in southern regions of the Italian peninsula.

\section{The Greek dialects in relation to other tongues}\label{sec:5.7}

Languages other than Latin were also claimed to have a special connection to certain ancient Greek dialects. Alleged Doric broadness was occasionally used as an argument for positing a special relationship with Syriac, allegedly broad itself (\citealt{Saumaise1643a}: 415–417). The Greek dialects were retraced to the Egyptian language as well, and Attic and especially Doric were even proclaimed to be dialects of it by Lord Monboddo (1714–1799), who accorded a pivotal role to Egyptian civilization in the evolution of human language (\citealt{Monboddo1774}: e.g. 637, 655). The great polymath Gottfried Wilhelm (von) Leibniz (1646–1716), who was also an enthusiastic language scholar, stipulated a particular link between Laconian and German, which he tried to corroborate by pointing to their predilection for ending nouns with the so-called dog’s letter ⟨r⟩ (\citealt{Leibniz1991}: 253). Philologists moreover conducted a debate on the status of certain undocumented or poorly known tongues: should Lycaonian and Phrygian be considered Greek dialects or separate languages?\footnote{For Phrygian, compare \citet[465]{Rijcke1684} to \citet[16]{Jablonski1714}. For Lycaonian, see \citet[2]{Jablonski1714}. See also \citet[]{VanRooyFcd}.} Such discussions, however, usually revolved around interpretations of ancient texts rather than actual linguistic evidence and were fueled by different presuppositions about what constituted a dialect. Finally, as a result of the emerging early modern interest in the origin and diversity of language, Greek was frequently framed within larger language families. In this context, scholars sometimes pictured it as a dialect of a protolanguage, often termed “Scythian”, and claimed that it differed only in dialectal terms from such tongues as Saxon, Gothic, and Celtic.\footnote{For Greek as a Scythian dialect, see e.g. \citet[xxxiv]{Court1778}. For the Scythian hypothesis, see the introduction to this chapter. For Saxon, see \citet[139; 190]{Casaubon1650}. For Gothic, see \citet[*.3\textsc{\textsuperscript{v}}]{Junius1665}. For Celtic, see \citet[\textsc{i}.44]{Martin1727}.} These suggestions were, however, usually not grounded in thorough and systematic linguistic research but rather in an intuitive and sporadic comparison of words and letter changes.

\section{By way of conclusion: Linguistic histories of Greek}\label{sec:5.8}

All in all, attempts at writing the linguistic history of the Greek language and especially that of its dialects remained relatively futile in the early modern period. It is true that there was a great variety of ideas on the Greek dialects and their historical interrelationships as well as their connection with other tongues, especially Latin. However, few Hellenists tried to offer a full and detailed history of the Greek language from prehistoric to contemporary times. A rare and late counterexample is the German Hellenist Daniel Christoph Ries (1741–1825), who proposed an elaborate periodization of the Greek language in his unusually encyclopedic school grammar, published in Mainz. \citet[199--202]{Ries1786} organized the history of Greek into three eras. At first, there was one Greek language, which gradually developed into different dialects following political diversification. Subsequent political unification went hand in hand with the elaboration of a common language and linguistic change for the worse, the four main dialects being nevertheless preserved in literary works. Finally, barbarian, principally Ottoman, invasions further corrupted the Greek language. Unlike his predecessors, Ries succeeded in providing his readers with a comprehensive account, but as was common in the early modern period, it focused on language-external circumstances. In fact, the interrelationships of the Greek dialects were usually not corroborated in the first place by an independent study of linguistic data, but by the authority of ancient scholars, especially Strabo. As a result, the geographer’s ill-informed suggestion that Aeolic and Doric were closely related lived on for several centuries.

Histories of the Greek tongue other than Ries’s were usually incomplete, in that they omitted several episodes, often those regarding the later fate of the language.\footnote{Cf. \citet[267--464]{Saumaise1643a}, \citet{Burton1657}, \citet{Lagerloof1685}, \citet{Rodigast1685}, \citet{Eling1691}, \citet{Florinus1707}, \citet{Reinhard1724}, \citet{Munthe1748}, and \citet{Harles1778}.} Still, the authors of these texts often reflected at length on the place of the ancient Greek dialects in the development of the language. The earliest example of an entirely freestanding history of the Greek language seems to be William Burton’s (1609–1657) \textit{History of the Greek language}, an oration published in London in 1657 but already held twenty-six years earlier in Oxford.\footnote{The \textit{Historia linguae Graecae methodica} by Stephanus (Étienne) \citet{Simon1615} is nothing more than a grammar of Ancient Greek.} In his history, \citet[27]{Burton1657} suggested a mixed dialectal origin for the Ancient Greek language on the authority of Strabo. A second important but somewhat peculiar example of a history of the Greek language is Claude de Saumaise’s \textit{Commentary on the Hellenistic tongue}. As I have mentioned in the previous chapter, Saumaise tried in this lengthy work to disprove the existence of a so-called Hellenistic dialect. To be able to do so, an elaborate description of the history of the Greek language and its dialects was indispensable to Saumaise’s mind. Most of these histories still require a more thorough and systematic investigation, going beyond the place they accord to the Greek dialects, yet this does not lie within the scope of the present book.

