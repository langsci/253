\chapter{Order in chaos? Classifications of Greek dialects}\label{chap:2}

“In the Greek language, there are great labyrinths and enormously vast recesses, not only in the various dialects, but in every one of them”.\footnote{See the Preface for the original quote.} Juan Luis Vives’ judgement on the immense diversity within the ancient Greek language was crystal clear; it posed an enormous challenge to him and his early modern colleagues. Even though not all scholars were excited about tackling this thorny issue, their fascination with the dialectally diverse Greek literature was strong enough to make them reflect at length on the questions: what forms of Greek are there? And how can they be distinguished? Their solutions to these matters constitute the main subject of this chapter, as it is impossible to understand other key aspects of early modern scholarship on the Greek dialects without gaining insight into this sometimes complex matter. Yet before treating early modern classifications of the Greek dialects and the problems central to them, it is indispensable that I elaborate first upon ancient and medieval scholarship, on which their Renaissance successors relied.\footnote{This and the following chapter are an extended, updated, and more detailed version of \citet{VanRooy2016a}, integrating information from \citet{VanRooy2018b}.}

\section{Between mythology and dialectology}\label{sec:2.1} 

How did ancient writers try to map out and categorize Greek linguistic diversity? Attempts at classifying the Greek dialects appeared relatively early and exhibited a striking mythological-etiological dimension, which I have to discuss at some length here, as it shaped early modern views to a certain extent. From a modern perspective it must seem rather odd to associate dialect with mythology, and it requires some explanation why this was the case in ancient Greece. To understand this connection, I should quote a seminal text fragment of the poet Hesiod (\textit{fl.} late 8th cent. \textsc{bc}) that usually heads outlines of Greek scholarship on the dialects:

\begin{quote}
From Hellen, the warlike king, were born  Dorus, Xuthus, and Aeolus the chariot-fighter.\footnote{\textit{Fragmenta} 9: “Ἕλληνoς δ’ ἐγένoντo φιλoπτoλέμoυ βασιλῆoς {\textbar} Δῶρός τε Ξoῦθός τε καὶ Aἴoλoς ἱππιoχάρμης”. See also \citet[64]{Hainsworth1967}.}
\end{quote}

Hellen, the son of Deucalion – or Zeus in a different tradition – and Pyrrha, was the mythological primogenitor of the Greeks, who were divided into four principal tribes, descended from Hellen’s sons Dorus, Xuthus, and Aeolus. Xuthus produced two distinct tribes through his sons Ion and Achaeus, from whom the Ionians and Achaeans derived. This somewhat complex mythological genealogy of the four Greek tribes, which oddly blended different generations of one and the same family, is visualized in \figref{fig:2.1} below.

\begin{figure}
\caption{The genealogy of Deucalion’s children (source \citealt[208]{VanRooy2016b}).\label{fig:2.1}}

%%[Warning: Draw object ignored]
\begin{forest} for tree={forked edge', grow=south}
[,phantom [Deucalion ⚭ Pyrrha [\textbf{Hellen} ⚭ Orseis,child anchor=150
    [\textbf{Dorus}] [Xuthus ⚭ Creusa,child anchor=150 [\textbf{Ion}] [Achaeus]] [\textbf{Aeolus}] ] [Amphictyon  ⚭  \textbf{Atthis},child anchor=150,name=atthis]] [Cranaus,name=cranaus]]
\draw let \p1=(cranaus),\p2=(atthis.north) in (cranaus) -- (\x1,\y2);
\end{forest}
\end{figure}

As J. B. \citet[64--65]{Hainsworth1967} has rightly remarked, the four tribe model was projected onto the dialect groups of the Greek language. The myth was in other words of an etiological nature, as it explained the existence not only of the different Greek tribes and their names, but also the dialects they spoke. The earliest extant testimony of a dialect classification inspired by the mythological model is a fragment of the geographer Heraclides Criticus’ (\textit{fl.} 3rd cent. \textsc{bc}) \textit{Description of Greece} (fr. 3.2), which was incorrectly attributed for a long time to his colleague Dicaearchus (\textit{fl.} 4th cent. \textsc{bc}; see \citealt{Brodersen2015}). It suggested a division of the Greek language into Attic, Doric, Aeolic, Ionic, and – atypically – Hellenic, the variety of Hellas, for Heraclides apparently a region in Thessaly.\footnote{On Heraclides’ interpretation of Hellas and its place in his discourse, see McInerney (2012: 257–260).} This early classification, however, occupied a somewhat peculiar position, wielding barely any influence on later thought. Instead, two other dialect classifications dominated Greek scholarship.

\section[Four or five dialects?]{Four or five dialects? The two major dialect classifications in Greek scholarship}\label{sec:2.2}

The notable geographer Strabo (ca. 62 \textsc{bc–}ca. \textsc{ad} 24) seems to have been the earliest scholar to propose a classification into four dialects: Ionic, Attic, Doric, and Aeolic (\textit{Geography} 8.1.2). He saw a close kinship between Ionic and Attic, on the one hand, and Doric and Aeolic, on the other. The former connection, corroborated by modern linguistics, might have been inspired by an intuitive comparison of actual linguistic data; the claim that Doric and Aeolic are closely related cannot, however, be backed by dialectal evidence and was probably maintained solely for the sake of analogy with the Ionic–Attic group. Another scholar proposing the fourfold classification was the grammarian Apollonius Dyscolus (1st half 2nd cent. \textsc{ad}), who according to the Byzantine encyclopedia known as the \textit{Suda} (at α.3422) composed a now lost work on the four Greek dialects Doric, Ionic, Aeolic, and Attic. The origin of this influential classification is unclear, but it seems to have been an achievement of Hellenistic scholarship, flourishing especially in Alexandria, Egypt, with its famous library. Strabo passed by the city on his many travels, and Apollonius lived and worked there.

Soon after Apollonius, however, an alternative classification into five dialects appears to have taken root, adding the koine to Attic, Ionic, Doric, and Aeolic. It was presented as a common Greek opinion by the Early Christian author Clement of Alexandria (ca. \textsc{ad} 140/150–before 215/221) in his miscellaneous work entitled \textit{Patchwork} (\textit{Stromata} 1.21.142.4). This widespread classification was later adopted by, among many others, the Byzantine scholar-emperor Constantine \textsc{vii} Porphyrogennetos (905–959; see \textit{De thematibus} 17). The classification into five dialects also prevailed in the Byzantine treatises on the Greek dialects by John the Grammarian and Gregory of Corinth, who clearly struggled with the status of the koine and its relationship to the four other dialects.

\section{The koine in Greek scholarship}\label{sec:2.3}

The koine dialect must indeed have been a major problem for Greek scholars working on the dialects, not only in Byzantium but already during antiquity, even though there are no direct sources available proving this. It is, however, hard to believe that ancient scholars were not aware that the koine, a widely used lingua franca, had a status different from the other literary dialects \citep{Consani2000}. As a matter of fact, the early Byzantine author John the Grammarian reported different opinions on the koine, likely reflecting a debate held in earlier works on the matter that are now lost. By means of John’s account and other evidence, often fragmentary, I have been able to reconstruct the most important Greek attitudes toward the koine, five in total:

\begin{itemize}
\item 
The four other dialects derived from the koine.\footnote{This overview is an adapted and augmented version of the list in \citet[209]{VanRooy2016b}. Cf. also \citet[614--617]{Consani2000}.}

\item 
The koine was the mother of the four other dialects, was formed by mixing them, and therefore consisted of them. In other words, the koine was the variety comprising all the other dialects, since it contained elements of each of them. It embraced, as it were, the dialects, as was befitting for a mother. It can be noted that despite the usage of the mother image, the koine was clearly seen as being posterior in time to the dialects in this view.

\item 
The koine was the subject of grammar and characterized by rules, whereas the other dialects deviated from those rules (e.g. \textit{Scholia in Pindarum (scholia uetera)} \textsc{o} 3.81c; John the Grammarian in Manutius \textit{et al.} 1496: 236\textsc{\textsuperscript{v}}).

\item 
The koine was used commonly by all, which implies that this did not hold for the other dialects (e.g. John the Grammarian in Manutius \textit{et al.} 1496: 236\textsc{\textsuperscript{v}}).

\item 
As a final attitude I can add the usually unexplained addition of the koine as the fifth dialect (e.g. Clement, \textit{Stromata} 1.21.142.4).

\end{itemize}

None of these solutions became generally accepted in Greek scholarship. From a modern perspective, one might regard the third view, according to which the koine was the normative variety from which the other dialects were deviations, as making the most sense. Yet even though there was a tradition of normative thinking separating correct from incorrect forms of Greek, the position of the dialects in this dichotomy remained unclear, to say the least (\citealt{Versteegh1986}; \citealt{Dickey2007}: 235). Overall, the Greek linguistic ideal of \textit{Hellēnismós} (Ἑλληνισμός) usually encompassed the canonical literary dialects other than the koine, too, as James \citet{Clackson2015} has convincingly pointed out. This was especially true of the Attic dialect, relaunched as the best form of Greek during the cultural and literary movement known as the Second Sophistic in the first centuries \textsc{ad} (\citealt{Whitmarsh2005}: esp. 41–56).

An important reason why Greek scholars perceived the koine to be problematic was likely the fact that this form of their language could not be linked to a specific Greek tribe or region, two main parameters they put forward in their definitions of the term \textit{diálektos} (διάλεκτoς; see \citealt{VanRooy2016d}), and that the koine did not seem to have features clearly distinguishing it from the other dialects. The latter was no doubt also the reason why treatises on the dialects usually did not include koine features in their discussion. An additional reason was perhaps that readers of these treatises were expected to already command the koine, the first variety of Greek to be learned in class by a Byzantine student (cf. \citealt{VanRooy2016b}).

\section{Zooming in: Below the level of dialect}\label{sec:2.4}

Greek scholars did not limit themselves to listing the principal Greek dialects; they were aware that they could be further divided into what modern linguists would perhaps call \textit{subdialects}. In the Byzantine period, varieties of individual Greek dialects were from time to time mentioned in grammatical and philological works, a practice with roots in ancient scholarship (\citealt{Hainsworth1967}: 70–72). The Byzantine theologian and grammarian Gregory of Corinth discussed several “local subdivisions” (\textit{hypodiairéseis topikaí}/ὑπoδιαιρέσεις τoπικαί) of Doric in his treatise \textit{On the dialects} (at 3.111). A similar concept was expressed in different terms by scholiasts of the grammar attributed to Dionysius Thrax (170–90 \textsc{bc}; see \citealt{Lambert2009}: 21–22; \citealt{VanRooy2016d}: 261–263). The multiplicity of individual Greek dialects was occasionally alluded to by ancient authors as well, for instance by Sextus Empiricus (\textit{fl.} ca. \textsc{ad} 190–210), who drew attention to the multitude of Attic and Doric varieties.\footnote{\textit{Aduersus mathematicos} 1.89. Cf. current discourse on so-called \textit{Englishes}.} Lower-level varieties of dialects were generally closely connected to the practice of glossography, the collection of rare, often local words, of which Hesychius’ (?5th/6th cent. \textsc{ad}) \textit{Lexicon} is the best known example. Yet the precise relationship between the main dialects, on the one hand, and their “local subdivisions”, on the other, is a matter on which Greek scholars failed to comment.

\begin{center}
\Large⁂
\end{center}

What were, in a nutshell, the main insights of Greek scholarship on the dialects? Firstly, inspired by a mythological tradition, authors devised different classifications of the Greek dialects which, though certainly not perfect, were still partly accurate and in keeping with actual linguistic evidence. The division into Attic, Ionic, Doric, and Aeolic, either with or without the koine, was overwhelmingly predominant in Greek scholarship and, as I will demonstrate, left an indelible mark on later thought. Secondly, whereas Greek scholars were aware that some of the dialects could be further divided into numerous other varieties, they had great difficulty in adequately determining the exact position of the koine vis-à-vis the other dialects.

\section{The Greek dialects in the ancient and medieval Latin world}\label{sec:2.5}

Greek scholarship, rediscovered by humanists during the Renaissance, was no doubt the most importance source for early modern authors interested in the dialects. It was, however, not the only source, as they could also read relevant observations in a number of Latin works from antiquity and the Middle Ages. The five-way classification seems to have been widely known to Latin authors. In fact, the Roman orator Quintilian (ca. 35–ca. 100 \textsc{ad}), though not referring to the Greek dialects by name, knew that there were five Greek “differences in speech”.\footnote{{\textit{Institutio oratoria}} {11.2.50: “quinque Graeci sermonis differentias”.}} He mentioned this when recounting that Publius Crassus Mucianus (ca. 180–130 \textsc{bc}), Roman proconsul in Asia Minor, could speak in all five of them. Before Quintilian, this anecdote had also been related by Valerius Maximus (\textit{fl.} 14–37 \textsc{ad}) in his \textit{Nine books of memorable deeds and sayings} (\textit{Facta et dicta memorabilia} 8.7.6), in which the Latin term \textit{genus}, ‘kind’ or ‘species’, was employed to refer to the Greek dialects. Remarkable is that the testimonies of Valerius and Quintilian predate the appearance of the five-way classification in extant Greek sources. It is, however, not unlikely that Valerius and Quintilian took this anecdote from a common source now lost, perhaps a Greek one, which itself might have been related to Hellenistic scholarship on the dialects.

Quintilian also discussed a fallacy known as \textit{Sardismós} (Σαρδισμός), consisting of dialect mixture and named after Sardis, a city in Asia Minor (modern-day Turkey) which supposedly had a dialectally mixed population (see \citealt{Gitner2019}). In this context, Quintilian mentioned the four canonical Greek “tongues” (\textit{linguae}):

\begin{quote}
Also, Sardism is the name of a certain speech mixed from a diverging range of tongues, for instance, in case one would confound Doric, Ionic, or even Aeolic words with Attic ones. Yet we have a similar vice in cases where someone mixes lofty with lowly words, old with new ones, and poetic with vulgar ones – that is indeed such a monstrosity, as Horace writes in the first part of his book on the art of poetry: if a painter would want to join a horse’s neck to a human head – and would place other things of different natures under it.\footnote{{ \textit{Institutio oratoria}} {8.3.59: “Σαρδισμός quoque appellatur quaedam mixta ex uaria ratione linguarum oratio, ut si Atticis Dorica, Ionica, Aeolica etiam dicta confundas. Cui simile uitium est apud nos, si quis sublimia humilibus, uetera nouis, poetica uulgaribus misceat – id enim tale monstrum, quale Horatius in prima parte libri de arte poetica fingit: Humano capiti ceruicem pictor equinam iungere si uelit – et cetera ex diuersis naturis subiciat”. See \citet[46]{Carruthers2009} for the relevance of this passage to grasp the ancient and medieval concept of} {\textit{uarietas}}.}
\end{quote}

Whereas the Greek example referred to regional varieties that had been elevated to literary dialects, the Latin situation did not concern regional linguistic diversity, but mainly register-based variation and – to a certain extent – differences in terms of time and social class. Quintilian’s case is somewhat problematic, in that in the Crassus anecdote he referred to five dialects, whereas in his discussion of \textit{Sardismós} he suggested that there were only four. The solution to this question is probably that Quintilian himself was not very well-versed in the Greek dialects and that the discrepancy in his work is due to the fact that he was drawing on diverging sources.

Several late antique Latin authors too mentioned the division into five dialects. Let me limit myself to the most puzzling example, revealing that knowledge of the dialects was often indirect and incomplete in the Latin world. The Early Christian bishop Isidore of Seville (ca. 560–636), author of an encyclopedic work entitled \textit{Etymologies}, spoke of the fivefold “variety” (\textit{uarietas}) of Greek. The bishop adhered to the view that the koine was the mixed common language, but his remarks on the use of Attic and Doric are rather unusual and betray a clear lack of competence in the Greek language. The former is said to have been used by all literary authors of Greece, whereas the latter is oddly claimed to have been current in Egypt and Syria.\footnote{\textit{Etymologiarum siue Originum libri} \textsc{xx} 9.1.4, on which see \citet[227–229]{Denecker2017}}

Three Roman and early medieval classifications of the Greek tribes and their dialects are somewhat peculiar, which is why they deserve a specific mention here. Firstly, the famous orator Cicero (106–43 \textsc{bc}) asserted that there were three Greek tribes, which later scholars interpreted as referring to a tripartite linguistic classification into Athenian, Aeolic, and Doric (see \textit{Pro L. Valerio Flacco oratio} 64). Secondly, the grammarian Diomedes (\textit{fl.} ca. 370–380) associated each of the five Greek “tongues” (\textit{linguae}) with certain linguistic-rhetoric usages and fallacies (\textit{Ars grammatica} 2, ed. \citealt{Keil1855}–1880: \textsc{i}.440). This shows the artificial solutions on which some grammarians relied to account for variation in Greek and betrays a transfer of the Latin concept of vices (\textit{uitia}) to the Greek dialects. Ionians, Diomedes claimed, were well-versed in figurative speech – tropes in his terminology – whereas Attic displayed solecisms and Doric was characterized by barbarisms; Aeolic was considered excessive. In the koine, presumably because of its status as common variety, all these elements were allegedly present (\citealt{Consani1991}: 32–33). Thirdly, Pseudacro (\textit{fl.} 7th/8th cent.) offered a peculiar classification of the Greek dialects, claiming that there were “five characters of tongues” of the Greeks: “Attic, Asian, Aeolic, common, Rhodian”, with “Asian” no doubt referring to Ionic and “Rhodian” to Doric.\footnote{{See} {\textit{Scholia in Horatium: Glossarum “gamma” appendix}} {4 (i.e. \citealt{Pseudacro1902}–1904:} {\textsc{ii.}}{385): “}\textstylehigh{quinque} autem sunt \textstylehigh{caracteres sermonum:} Atticus, Asianus, Aeolius, communis, Rhodius{”.}} Pseudacro’s alternative glottonymic designations were inspired by the names of three well-known ancient rhetoric trends: exuberant Asianism, traditional Atticism, and the intermediate Rhodian style.

In conclusion, some ancient and early medieval Latin authors were superficially acquainted with the traditional five literary dialects of Greek. Most remarks were of a very general nature, however, with the exception of the Latin grammarian Priscian, who, working in Byzantium around 500 \textsc{ad}, expressed great interest in the Greek dialects in as far as he was able to tie them to Latin (see Conduché fc.). During the greater part of the Middle Ages, Greek was barely studied in the West, as this language was considered either heretic or simply irrelevant (see e.g. \citealt{Boulhol2014}). When copyists encountered Greek words or phrases, they usually had to confess that they were unable to read it: “It is Greek, it is not read” was an often recurring note.\footnote{ “Graecum est, non legitur”. See e.g. \citet[\textsc{i.}246–275]{Bischoff1961, Bischoff1981}; {\citet[esp. 3ff]{Weiss1977}; \citet[36]{Saladin2000}.}} This lack of knowledge was related to the fact that at this time no adequate grammar of Greek composed in Latin existed \citep[215]{Bischoff1961}. The language nevertheless excited considerable practical interest, evidenced by, among other things, the compilation of a number of lexica (\citealt{Bischoff1961}: 217–219; \citealt{Dahan1995}: 267–269). Given the rarity of competence in Greek, it is not surprising that knowledge of the canonical dialects too was highly limited. Even awareness of their existence was rare. The theologian Hugh of Saint Victor (ca. 1096–1141), for instance, was only able to repeat the ill-informed statement of Isidore of Seville in his work on grammar \citep[79]{Hugh1966}. The twelfth-century \textit{Vatican Mythographer}, in turn, made an oddly placed, completely isolated remark about the canonical five “Greek tongues” (\textit{Graecae linguae}), whereas Eberhard of Béthune (\textit{fl.} ca. 1212) likewise mentioned the division into five “idioms” (\textit{idiomata}), remarkably substituting, however, the koine with Boeotian.\footnote{\textit{Mythographus Vaticanus} 1.192; Eberhard of Béthune, \textit{Graecismus} 8.1–2.}

The only exception seems to have been the English polymath Roger Bacon (ca. 1214/1220–ca. 1292), who accorded generous attention to the dialects in his Greek grammar, which he composed in Latin around 1268. Even though early modern scholars were unable to make use of Bacon’s work – the only edition of Bacon’s grammar appeared in 1902 – his remarkable views deserve to be briefly treated in a history of premodern scholarship on the Greek dialects. How did Bacon classify the Greek dialects? He stated that “there were five and six [\textit{sic}] idioms of the Greek language”.\footnote{\citet[26]{Bacon1902}: “5 et 6 fuerunt idiomata Graecae linguae”.} This phrase is revealing in two ways. It indicates, on the one hand, that Bacon was apparently aware that the Greek dialects were no longer spoken, as he used the perfect indicative form \textit{fuerunt} of the Latin verb \textit{sum}, ‘to be’. On the other hand, he added an additional dialect to the traditional fivefold classification: Boeotian \citep[27]{Bacon1902}. The clumsy formulation “five and six” may suggest that Bacon was hesitant about including it. The koine was clearly perceived as somehow distinct from the other Greek dialects. He regarded it as the variety consisting of what was common to all Greek tribes and which was used for communication by all. It was, Bacon suggested, the core nature and substance of the Greek language, on which the other idioms were mere variations.

In conclusion, Western scholars were usually ill-informed about Greek linguistic diversity. The late medieval polymath Roger Bacon, who expressed a unique interest in the Greek dialects, was the proverbial exception. This state of affairs changed profoundly in the Renaissance, to which I now turn.

\section{Tradition and innovation: Old classifications and a new principle}\label{sec:2.6}

As I have pointed out in \chapref{chap:1}, \sectref{sec:1.2}, the renowned printer Aldus Manutius was responsible for a key turn in the history of Greek dialect studies. This coincided with his issuing an impressive collection of ancient Greek and Byzantine grammatical treatises, intended for the experienced Hellenist. Explaining the range of this reference work in his preface, Manutius boldly stated that

\begin{quote}
it moreover treats the Attic, Ionic, Aeolic, Doric, Boeotian, Cretan, Cypriot, Macedonian, Thessalian, Rhegian, Sicilian, Tarentine, Chalcidian, Argive, Laconian, Syracusan, Pamphylian, and Athenian tongues. These the Greek poets, and Homer in particular, are found to have used. Due to these tongues and their various inflections they have an astonishing liberty. They add, subtract, transmute, invert. What don’t they do? In short, they use words like wax.\footnote{Manutius (1496: *.ii\textsc{\textsuperscript{v}}): “Linguarum praeterea meminit Atticae, Ionicae, Aeolicae, Doricae, Boeticae, Cretensis, Cypriae, Macedonicae, Tessalae [\textit{sic}], Rheginae, Siculae, Tarentinae, Chalcidicae, Argiuae, Laconicae, Syracusanae, Pamphyliae, Atheniensis, quibus usi Graeci poetae inueniuntur, et Homerus praecipue. His linguis ac figuris uariis habent illi miram licentiam. Addunt, detrahunt, transmutant, inuertunt. Quid non faciunt? Denique utuntur dictionibus ut cera”. My translation is inspired, but only very loosely, by the rather free and inadequate rendering of \citet[12]{Bean1958}} 
\end{quote}

Manutius here already mentioned most of the varieties included in the widely accepted modern classification of the ancient Greek dialects into Aeolic, Arcado-Cypriot, Attic–Ionic, Doric, Northwest Greek, and Pamphylian (cf. \chapref{chap:1}, \sectref{sec:1.1}), even though he did not offer much more than a mere listing. Manutius listed the dialects Attic, Ionic, Aeolic, and Doric first. This is neither a coincidence nor a surprise; as I have mentioned, these were the four canonized literary dialects of ancient Greek. Manutius did not refer to the koine here, but he was no doubt aware of its existence from the treatises in the collection, which he later translated into Latin. It is immediately apparent from the above passage that Manutius associated the dialects closely with poetry. This reveals the primary reason why humanists studied the dialects; much like their Greek predecessors, they wanted to master them in order to better understand Greek poems. For humanists in the Latin West, however, studying the dialects was initially only a second-degree auxiliary tool. They wanted to know the dialects because they wanted to be able to read Greek literature, which in turn served as a means to gaining deeper insight into Latin literature, as it was modeled on Greek examples. This likely explains why Renaissance Hellenists were content, initially at least, with the two traditional classifications into four or five dialects; they were pursuing philological goals similar to their predecessors.

An early scholar propounding the fourfold classification into Attic, Ionic, Doric, and Aeolic was, for instance, Johann Reuchlin – if, at least, I may presume that he backed the ideas contained in the Byzantine treatise he tried to pass off as his own (see \citealt{VanRooy2014}: 510–515). Others adhered to the classification including the koine. An early example is Nicolaus Clenardus (1493/1494/1495–1542), a humanist from Diest in modern-day Belgium, whose manual for Greek was so popular in the early modern period that the name \textit{Clenardus} even became synonymous with Greek grammar. From this handbook, first published in 1530 in the university city of Leuven, the student of Greek could gather that “there are five principal tongues among the Greeks: common, Attic, Ionic, Doric, Aeolic”.\footnote{\citet[7 (misprint for 6)]{Clenardus1530}: “Quinque Graecorum linguae praecipuae, Communis, Attica, Ionica, Dorica, Aeolica”.} The latter fivefold classification was best-known in early modern linguistic scholarship, most likely because it was the one that predominated in the Byzantine treatises by John the Grammarian and Gregory of Corinth; these works were definitely known to Hellenists of the time, as they were published together by Manutius in 1496 and subsequently in many other handbooks. Gradually, however, scholars felt the need to alter, correct, and supplement traditional Greek dialect classifications. What alternatives did they propose and why? Did they employ the same classificatory principles grounded in mythological and philological assumptions as Greek philologists had done? Or did they break away from Greek tradition?

The insight that some of the traditional four dialects could be further divided into different speech forms, only marginally present in Greek thought, was further developed by early modern philologists. They introduced a distinction between “principal” and “less principal” dialects, to which Clenardus already alluded in his grammar. Principal dialects were those relevant to the study of literature, whereas the less principal dialects were those for which scholars only had fragmentary or indirect evidence from ancient and Byzantine sources and which were not of direct concern for philologists. What was the origin of this new bipartition? It dates without a doubt from the beginning of the Cinquecento and had its roots on the Italian peninsula. The earliest testimony I have been able to trace thus far can be found in a grammatical commentary of 1509, published in Ferrara and authored by the humanist professor Ludovico da Ponte (Ponticus Virunius; ca. 1460–1520), whose contribution to Greek studies merits a closer study.\footnote{For biographical information on Da Ponte, see \citet{Ricciardi1986}, with many further references.} Da Ponte maintained that “even though there are seventeen tongues of the Greeks, there are nevertheless five principal tongues”.\footnote{\citet[20\textsc{\textsuperscript{v}}–21\textsc{\textsuperscript{r}}]{Da1509}: “cum \textsc{xvii} sint linguae Graecorum, tamen principales sunt quinque linguae”.} Da Ponte did not clarify, however, whether the former were subsumed under the five principal ones or stood next to them on the same level; nor did he mention all seventeen dialects by name.\footnote{For similar wordings (esp. the adjective \textit{principalis}) see \citet[51--52]{Oecolampadius1518}. Cf. also \citet[12, a.3\textsc{\textsuperscript{v}}]{Canini1554, Canini1555}, using the terms \textit{generalis}, \textit{princeps}, and \textit{superior}; \citet[2]{Walper1589}, speaking of \textit{dialecti primariae}.} Another early testimony, this time from north of the Alps, is Adrien Amerot’s Greek grammar of 1520, in which one reads that “there are almost as many tongues of the Greeks as there are tribes, among which nevertheless five are principally employed”.\footnote{\citet[\textsc{q}.i\textsc{\textsuperscript{v}}]{Amerot1520}: “Graecorum linguae tot paene sunt, quot nationes, ex his tamen praecipue quinque celebrantur”.} The phrase also occurred in Amerot’s popular booklet on the Greek dialects, a separately published excerpt from his grammar which first appeared in 1530 and enjoyed countless reprints during the entire early modern period.\footnote{See \citet[5--19]{Hoven1985} for an extensive list, which can even be augmented by digital searches. See also Chapter 1, \sectref{sec:1.2}.} This greatly contributed to the spread of the idea that there were “principal” and “less principal” Greek dialects. For instance, Amerot certainly inspired the information on the dialects in Michael Neander’s (1525–1595) popular Greek grammar, and he is also likely to have influenced the statement of Nicolaus Clenardus on the Greek dialects.\footnote{See \citet[187]{Neander1553}. For Amerot’s possible influence on Clenardus’ grammar, see \citet{VanRooyNoDate}. For the adjective \textit{praecipuus}, see also \citet[42]{Mosellanus1527}, who claimed that there were about 24 dialects in total.}

Early humanist Hellenists did normally not explain why they made this division into “principal” and “less principal” dialects. There are a number of exceptions, however, which are worth a closer look. The earliest justification of the innovation occurred as a passing remark in the Greek grammar of Georg Simler (ca. 1477–1536), a German humanist who was the teacher of, among others, Philipp Melanchthon: “We have called them principal, for they are used by poets, especially Homer”.\footnote{\citet[\textsc{aa.}i\textsc{\textsuperscript{r}}]{Simler1512}: “Principales diximus, sunt enim quibus utuntur poetae, praesertim Homerus”.} For a more extensive motivation, we have to wait until the middle of the sixteenth century. The French Hellenist Pierre Davantès (Petrus Antesignanus; ca. 1525–1561), an influential commentator of Clenardus’ Greek grammar, explained that the traditional dialects were dubbed “principal”, because these were the varieties of Greek mainly used by literary authors. Other dialects such as Boeotian and Thessalian were labeled “less principal”, since there were no literary works extant which were entirely composed in them. These dialects would have been lost to the ages if Greek authors had not introduced some elements of them into their works.\footnote{\citet[11]{Antesignanus1554}, on which see \citet[129--130]{VanRooy2016c}.} In other words, the mere existence and survival of literary works was employed as a classificatory principle to distinguish between the “principal” and “less principal” dialects of the Greek language. This criterion, proving once again that philology was the primary motivation to study the Greek dialects in the Renaissance, became highly popular and was adopted by numerous early modern Hellenists.\footnote{See e.g. \citet[2--3]{Walper1589}; \citet[7--8]{Schmidt1604}; \citet[3--4]{Merigon1621}; \citet[83--84]{Rhenius1626}; \citet[66]{Busby1696};  \citet[1132–1133]{[frisch]1730}.}

Briefly, the traditional classifications were still vastly important in the early modern era, even though scholars introduced a finer-grained distinction based on philological criteria by opposing “principal” to “less principal” dialects. In this case, the latter were frequently viewed as varieties subsumed under the former.\footnote{See e.g. \citet[a.3\textsc{\textsuperscript{v}}]{Canini1555}; \citet[439]{Saumaise1643a}; \citet[2–3, 7–9]{Munthe1748}; \citet[490--491]{Valckenaer1790}.} The early modern discourse on “principal dialects” came to be extrapolated to other languages too. A notable example is Alexander Gill (1565–1635). This English schoolmaster who taught Greek to John Milton was the author of a grammar of his native tongue, in which he claimed that there were six main dialects in English:

\begin{quote}
There are six principal dialects: Common, Northerners’, Southerners’, Easterners’, Westerners’, Poetic. I neither know nor have heard all their particularities. Yet I will describe as far as I can those I remember.\footnote{\citet[15]{Gill1619}: “Dialecti praecipuae sunt sex: Communis, Borealium, Australium, Orientalium, Occidentalium, Poetica. Omnia earum idiomata nec noui, nec audiui; quae tamen memini, ut potero dicam”. See \citet{Kokeritz1938}. See e.g. also \citet[liv]{Thomassin1697} on the three principal (\textit{principes}) dialects of Chaldean; \citet[\textsc{xciii}]{Schultens1748} on the principal (\textit{principes}) dialects of the primeval language.}
\end{quote}

Several elements in the above quote suggest influence from the tradition of early modern grammars of ancient Greek, with which Gill, being a distinguished Hellenist, must have been acquainted. He used the designation “common dialect”, reminding of the Greek koine, and included a poetical dialect among his English dialects \citep[18]{Gill1619}, a concept first developed within Greek grammar, as I will demonstrate later (see \sectref{sec:2.7} below). There are, however, also differences. The names of the English dialects were more strictly geographical than their Greek counterparts, which had a link with Greek mythology and tribal history. The English dialects were, moreover, by no means literary varieties. In fact, the Western English dialect, especially in Somerset, was so barbarous that it barely deserved the name “English”. \citet[17]{Gill1619} did grant, however, that it preserved some notable archaic features. He moreover acknowledged that it was useful to know the dialects, since English poets occasionally used dialect elements \citep[18]{Gill1619}. This suggests that his appreciation of English dialects was not unequivocally negative, an attitude for which he might have found support in the prestige of the Greek dialects.\footnote{Cf. also Chapter 8, \sectref{sec:8.1.2}, on the model status of the Greek dialect context.}

In the present section, I might have created the impression that the two traditional classifications were the only ones proposed by early modern thinkers and that their only contribution was to introduce the distinction between “principal” and “less principal” dialects. Was this really the case? Or did scholars also innovate and design alternative classifications? If so, how and why did they do so?

\section{The invention of a poetical dialect}\label{sec:2.7}

In his \textit{Booklet on the Greek dialects} of 1569, a poorly known French Hellenist by the Latin name of Robertus Vuidius, originating from Tonnerre in the center of northern France, explained that he intended “to treat the five idioms or dialects, i.e. Attic, Ionic, Aeolic, Doric, and Poetic”.\footnote{\citet[137\textsc{\textsuperscript{v}}]{Vuidius1569}: “Quinque autem idiomata siue διαλέκτoυς tractare sumus ingressi Atticum uidelicet, Ionicum, Aeolicum, Doricum et Poeticum”.} In so doing, Vuidius heralded a new era in early modern classifications of the Greek dialects, during which a set of new varieties was added to the traditional four or five. Before him, influential scholars like Petrus Ramus (Pierre de la Ramée; 1515–1572) and Joseph Justus Scaliger (1540–1609) had already spoken in passing of a Greek “poetical dialect” (\citealt{Ramus1560}: 18–19; \citealt{Scaliger1594}: 56). Scaliger even claimed to have composed a grammar of this dialect when he was about twenty years old, of which no traces remain today, however. What is more, linguistic particularities proper to poetry, often simply explained as “poetical license”, had been noticed well before Ramus, Scaliger, and Vuidius by earlier Greek scholars. The ancient grammarian Tryphon, perhaps the founding father of Greek dialect studies, had already associated procedures such as metathesis with poetry.\footnote{See e.g. Tryphon’s Περὶ παθῶν 3.18. Cf. \citet[78\textsc{\textsuperscript{v}}]{Da1509}; \citet[209, 230, 235]{Vergara1537}.} The idea that Greek poetical language was dominated by a far-ranging license, however, received prominence only in Renaissance thought, even if ancient authorities were invoked to back it. A case in point is Manutius’ preface to his collection of Greek grammatical treatises of 1496. He supported his observation that Greeks, especially their poets, used words like wax by the authority of the ancient Roman poet Martial (ca. \textsc{ad} 40–103), who, as he unsuccessfully tried to fit the Greek name Eiarinos into his Latin verses, mused:

\begin{quote}
And yet poets say \textit{Eiarinos}; {\textbar} but they are Greeks to whom nothing is denied, {\textbar} whom it beseems to chant \textit{Āres}, \textit{Ăres}. {\textbar} We, who cultivate more austere Muses, {\textbar} cannot be so clever.\footnote{Martial, \textit{Epigrammata} 9.12.10–14, here quoted in the English translation of the Loeb series. The original Latin verses are: “dicunt Eiarinon tamen poetae, {\textbar} sed Graeci, quibus est nihil negatum {\textbar} et quos Ἆρες Ἄρες decet sonare: {\textbar} nobis non licet esse tam disertis, {\textbar} qui Musas colimus severiores”. See \citet[*.ii\textsc{\textsuperscript{v}}]{Manutius1496} and e.g. also \citet[187 \textsc{\textsuperscript{r}}]{Enoch1555}.}
\end{quote}

Robertus \citet[146\textsc{\textsuperscript{v}}–148\textsc{\textsuperscript{v}}]{Vuidius1569} was, however, the first Hellenist to provide a systematic synopsis of the linguistic features of the poetical dialect, mainly consisting – as with the other dialects – in permutations of letters. Its inclusion among the dialects was encouraged by the fact that the problematic language of Greek poetry, apart from being dialectally mixed, also had certain formal characteristics of its own that could not be ascribed to the traditional four or five dialects. Yet Vuidius did not problematize the fact that the texts transmitted in the Aeolic and Doric dialects were almost exclusively poetical in nature. His vague definition of \textit{dialectus} as “particularity of tongue” allowed him to apply the term also to the manner of speaking characteristic of poets (\citealt{Vuidius1569}: 138\textsc{\textsuperscript{r}}–138\textsc{\textsuperscript{v}}). The Greek poetical dialect, an innovation of the 1560s that might strike modern readers as rather odd, became a highly successful construct and was present in numerous later classifications of the Greek dialects.\footnote{See e.g. \citet[\textsc{x}.1\textsc{\textsuperscript{v}}]{Dabercusius1577}; \citet[\textsc{i.1}\textsc{\textsuperscript{v}}]{Camden1595}; \citet[376--377]{Kober1701}; \citet[113]{Petisco1764}. Cf. also \sectref{sec:2.8} below.} The concept of \textsc{poetical} \textsc{dialect} was gradually applied to varieties of other languages too, as the case of Alexander Gill has already demonstrated. This was frequent especially in the eighteenth century, when philology was much less restricted to the classical languages than it had been in earlier times.\footnote{See e.g. \citet[101]{Hickes1705} on the \textit{dialectus poetica Dano-Saxonica}; \citet[*.3\textsc{\textsuperscript{r}}]{Verwer1707} for Dutch; \citet[24]{Wesley1736} and \citet{Vogel1764} for Hebrew; \citet[240, 241]{Beattie1778} for Latin and French; \citet[292]{Macnicol1779} for English.}

The existence of a poetical dialect was, however, not accepted by everyone, and some Hellenists discarded it in the later seventeenth and eighteenth centuries. Critical voices were primarily heard in the Holy Roman Empire, where Greek studies never ceased to flourish in the early modern period. A late seventeenth-century grammarian warned his readers not to forge any new dialects, arguing that it was better not to refer to poetical license in terms of “dialect” \citep[512]{Ursin1691}. Almost a century later, the existence of a poetical dialect was even rejected as absurd by the German Hellenist and pedagogue Friedrich Gedike: “The grammarians speak almost unanimously of an additional fifth dialect, namely a specific poetical dialect. Yet this division brings little honor to their judgment”.\footnote{\citet[21]{Gedike1782}: “Die Grammatiker reden fast insgesammt noch von einem 5ten Dialekt, nehmlich einem besondern poetischen. Allein diese Eintheilung macht ihrer Beurtheilungskraft wenig Ehre”.} Poets introduced linguistic particularities for metrical reasons, Gedike suggested, and it was rather the case that they mixed dialects than that they had one of their own. Another late eighteenth-century German Hellenist, Johann Friedrich Facius, supported his rebuttal of the poetical dialect by means of his definition of \textit{dialectus}:

\begin{quote}
Besides, the poetic dialect, usually added to these four dialects, cannot be called a dialect properly speaking, as it rather is a certain kind of speech not proper to a nation, but to a certain class of writers only.\footnote{\citet[]{Facius1782} (\textsc{v}): “Quae praeterea his quattuor dialectis uulgo additur \textit{poetica}, proprie dialectus dici nequit, cum dictionis potius sit quoddam genus, non genti, sed scriptorum tantum ordini cuidam proprium”.}
\end{quote}

Facius used \textit{dialectus}, understood here as a variety of a language particular to a certain people, to deny dialect status to the amalgam of linguistic particularities restricted to poets.\footnote{Cf. also \citet[67]{Haas1780}: “Manche machen die Freyheit, deren sich die Poeten in ihren Versen bedienen, zu einem Dialekt, und nennen ihn \textit{dialectum poet}. Ein solcher Dialekt aber setzet eine poetische Stadt oder Landschaft voraus”.} It was, he maintained, nevertheless justified to discuss poetical particularities together with the dialects, since poetical forms were so common that knowledge of them was indispensable for reading Greek poets.\footnote{\citet[98]{Facius1782}. Other critical voices include \citet[\textsc{a.3}\textsc{\textsuperscript{v}}]{Bolius1689}; \citet[\textsc{d.2}\textsc{\textsuperscript{v}}]{Thryllitsch1709}, where \citet[147]{Reyher1634} is reproached for adding the poetic dialect to his classification of the Greek dialects; \citet[136--167]{Walch1772}.} Likely the best-informed solution to the problem of the poetical dialect was proposed by the theologian Christian Siegmund Georgi (1702–1771), who was also active in the Holy Roman Empire and who specialized in the language of the Greek New Testament. Georgi pointed out that it depended on one’s interpretation of the polysemous term \textit{dialectus} whether one could speak of a “poetical dialect”. If one interpreted the word as “style”, as Aristotle had done, then this was justified. If one understood it as a “variety of one and the same language”, it was most certainly not \citep[169]{Georgi1733}. Remarkably enough, scholars dismissing the existence of the poetical dialect as a rule did not mention any colleagues by name. As a result, the debate on this topic was always indirect.

In conclusion, at the end of the early modern period, many scholars realized that the introduction of a poetical dialect into classifications of the Greek dialects had been a severe setback; it had no historical \textit{raison d’être}. In fact, the poetical dialect owed its existence largely to a process of simplification for philological and didactic purposes. It was designed as a unitary rubric for a diverse range of linguistic phenomena with which every would-be Hellenist had to deal in his study of Greek poetry. With the benefit of hindsight, we now know that these poetical particularities have multiple origins. Apart from various dialectal features, they also include archaic, metric, and stylistic properties. The invention of the poetical dialect and the subsequent discussion of its historical validity resulted from the ambiguous polysemy of the word \textit{dialect(us}), as it could signify in very general terms “style” and “manner of speaking”, but also “variety of a language and particular to a certain region and people”. The dismissal of the poetical dialect suggests that the latter meaning eventually prevailed in the eyes of many Hellenists of the period.

Another early modern solution to the problematic status of poetical Greek was more pragmatic. Several grammarians stated that Greek poets intermingled all the dialects, even though they usually made primary use of only one of them. This mixing was variously explained. Some grammarians suggested that the mixture was the result of the poets’ intense traveling across Greece or simply a conscious choice of the poets in order to have more possibilities in versification.\footnote{For a general reflection on the causes of the poets’ mixed usage, see e.g. \citet[*.3\textsc{\textsuperscript{r}}–*.4\textsc{\textsuperscript{r}}]{Gottleber1765}. The mixed poetic variety was sometimes explicitly identified with the poetic dialect; see e.g. \citet[111]{Bayly1756} and \citet[198]{Peternader1776}.} These ideas were inspired by dominant views on the nature of Homer’s Greek, which I will discuss in the next chapter.

\section{Adapting traditional classifications}\label{sec:2.8}

The invention of a poetical dialect necessarily led to the emergence of new classifications of the ancient Greek dialects, which were, in fact, as a rule adaptations and extensions of the two traditional Greek ones. \tabref{tab:2.1} offers by way of demonstration an overview of the most important new classifications, ordered chronologically according to their first appearance. It would be straying too far from the central topic of this book to tease out the details of all these early modern classifications. I will instead focus on the most noteworthy innovations. Looking at the table, one is immediately struck by the fact that early modern scholars augmented traditional classifications by adding newly created dialects – most importantly the poetical and Hebraizing dialects – as well as dialects that had already been recognized in antiquity, but had not yet been canonized despite their being employed in literature – principally Boeotian, the variety in which the enigmatic poetess Corinna composed her verses. In some cases, lesser-known tongues from the margins of Greece, such as Phrygian and Macedonian, were also included among the canonical dialects, either because the scholar had only little acquaintance with the Greek dialects or because he proposed a particular interpretation of the historical status of the koine.\footnote{See e.g. \citet[131]{Kircher1679} for Phrygian, which exemplifies the former reason, and \citet[\textsc{c.2}\textsc{\textsuperscript{v}}]{Schwartz1702} for Macedonian, an instance of the latter reason. Cf. also Chapter 5, \sectref{sec:5.4}} The total number of Greek dialects mentioned varied from author to author and was impressively high in the \textit{Polyglot thesaurus} of the Stuttgart-born scholar Hieronymus Megiser (ca. 1554/1555–1618/1619), who seems to have attributed a dialect to each Graecophone region or city he knew \citep[.7\textsc{\textsuperscript{r-v}}]{Megiser1603}. One Hellenist originating from Taranto in southern Italy even hyperbolically suggested 600 as the number of ancient Greek dialects, even though it is more likely that he used the Latin numeral \textit{sescenti} in the metonymic sense of “innumerable” rather than in its literal meaning of “six hundred” \citep[9]{Giovane1589}. There were indeed other scholars who emphasized the sheer innumerability of the Greek dialects (e.g. \citealt{Bischoff1708}: 127; \citealt{Ries1786} [1782]: 196).

\begin{table}
\caption{The principal new classifications of the ancient Greek dialects. T4 refers to the traditional four dialects Aeolic, Attic, Doric, and Ionic. T5 includes all of these and the koine.}\label{tab:2.1}
 
\begin{tabularx}{\textwidth}{lQQ}
\lsptoprule
\# & Classification & Example(s)\\\midrule
 5 & T4 \& poetical & \citet[137\textsc{\textsuperscript{v}}]{Vuidius1569}; \citet[193–198]{Peternader1776}\\
 6 & T5 \& poetical [+ secondary] & \citet[\textsc{x.1}\textsc{\textsuperscript{r}}\textsc{–x.1}\textsc{\textsuperscript{v}}]{Dabercusius1577}; \citet[334]{Alsted1630}; \citet[64]{Bregius1684}\\
 6 & 4 proper (= T4) \& 2 less proper (koine + poetical) & \citet[3\textsc{\textsuperscript{r}}]{Baile1588}; \citet[4]{Schmidt1604}\\
 7 & 5 (T4 + Boeotian) \& 2 (poetical + Hebraizing) & \citet[1--2]{Pasor1632}; \citet[3]{Wyss1650}\\
 5 & T4 \& poetical [+ less principal] & \citet[302]{Opitz1687}; \citet[100--101]{Giraudeau1739}\\
 6 & T4, Boeotian \& poetical & \citet[48]{Wright1691}; \citet[121]{Holmes1735}\\
 3 & Attic, Doric \& Ionic & \citet[66--67]{Busby1696}; \citet[i–ii]{Maittaire1706}\\
\lspbottomrule
\end{tabularx}
\end{table}

Some scholars reduced the number of principal dialects instead of adding new ones. The English Hellenist Richard Busby (1606–1695) listed only three principal dialects: Attic, Doric, and Ionic. Boeotian was subsumed under Doric, just like Aeolic, although \citet[66--67]{Busby1696} claimed Aeolic also shared features with Ionic. This inspired Michael Maittaire (1668–1747), a French-born pupil of Busby’s, to posit a tripartite division into Attic, Doric, and Ionic, which in turn influenced the views of, among others, Heinrich Ludolf Ahrens (1809–1881), generally regarded as the founding father of modern ancient Greek dialectology.\footnote{\citet[i–ii]{Maittaire1706}. See e.g. \citet[\textsc{viii}.177]{Brekle1992}; \citet[463]{Finkelberg2014}. Maittaire also influenced e.g. \citet[213]{Thompson1732}; \citet[162]{Gesner1774}; \citet[\textsc{xxviii}]{Harles1778}. \citet[(1884–1890): 92]{Pott1974} still praised Maittaire’s work.} \citet[\textsc{i.}1]{Ahrens1839} followed Maittaire, for instance, in leaving out the koine from his dialect classification. At the same time, however, he curiously misinterpreted his predecessor’s division as being quadripartite (Attic, Ionic, Doric, Aeolic) rather than threefold (Attic, Ionic, Doric). Ahrens’ dependency on Maittaire indicates that the so-called founding father of Greek dialectology relied on earlier scholarship for a key aspect of his work; this suggests that his contribution to ancient Greek dialectology needs to be revaluated from a historical perspective, a task which, however, lies outside the scope of this book. In the eighteenth century, Michael Maittaire’s tripartition of the Greek dialects was dismissed by the German Hellenist Johann Friedrich \citet[\textsc{iv–v}]{Facius1782}, who emphasized the peculiar character of the Aeolic dialect and clearly separated it again from Doric. Some scholars reduced the number of principal dialects even further, primarily for didactic reasons. The French classical scholar and pedagogue Tanneguy Le Fevre (1615–1672) posited two “dominant dialects” (\textit{dialectes dominantes}) only, on which Greek courses should focus: Doric and Ionic. The reason \citet[61]{Le1731} provided was that Aeolic was too obscure and very rare in extant literature, and that the Attic dialect was remarkably close to common grammar and therefore did not require separate treatment.

\tabref{tab:2.1} shows that the main early modern classifications of the ancient Greek dialects were generally much more detailed than their ancient and medieval sources of inspiration. The best example of this tendency is, however, found in an early eighteenth-century dissertation, presented in the city of Wittenberg on February 9, 1709, in which an idiosyncratic, geographically motivated division consisting of three hierarchical layers was proposed:

\begin{itemize}
\item there were four principal and primary dialects, spoken by an entire tribe – \textit{dialecti primariae}, \textit{principales}, or \textit{ethnikaí} (ἐθνικαί): Ionic, Attic, Doric, and Aeolic;

\item each primary dialect comprised several secondary, regional dialects – \textit{dialecti secundariae} or \textit{egkhṓrioi} (ἐγχώριoι) – which emerged as a result of the geographical dispersion of the four principal tribes;

\item each secondary dialect comprised several city or local dialects – \textit{dialecti urbicae} or \textit{topikaí} (τoπικαί; \citealt{Thryllitsch1709}: \textsc{d.3}\textsc{\textsuperscript{r}}).

\end{itemize}

This can be seen as a further elaboration of the “principal”/“less principal” dichotomy I have discussed in \sectref{sec:2.6} above; it was, however, also partly inspired by Greek scholarship, as the author drew inspiration for his hierarchical division from Byzantine observations on varieties of Doric (\citealt{Thryllitsch1709}: \textsc{d.2}\textsc{\textsuperscript{v}}\textsc{–d.4}\textsc{\textsuperscript{r}}).

Despite the abundance of different classifications in the early modern period, reflections on the differences among them were rare. Some scholars did, however, feel the need to discuss the correct sequence in which the dialects should be named. The German Hellenist Johann Friedrich \citet[\textsc{iv}]{Facius1782} held that the order should be determined by the antiquity of each dialect, which brought him to the following sequence: Doric, Aeolic, Ionic, and Attic (cf. Chapter 5, \sectref{sec:5.4}). The alleged close kinship between Attic and Ionic, on the one hand, and Aeolic and Doric, on the other, an idea current since Strabo, led an eighteenth-century Hellenist from the Dutch Republic to arrange the four principal Greek dialects as Attic, Ionic, Doric, and Aeolic (\citealt{Koen1766}: \textsc{xxix}).

All classifications were to different degrees indebted to Greek scholarship. The type of division proposed greatly depended on a scholar’s aims and the context in which he discussed the Greek dialects. For instance, the so-called Hebraizing dialect was only included by authors interested in New Testament Greek, as I will demonstrate in \chapref{chap:3}. In other words, early modern classifications were partly based on the ancient Greek and Byzantine tradition, and partly on the interests of scholars, who tended to focus on Biblical and poetical Greek, both highly problematic varieties. Linguistic principles were largely out of the picture in designing Greek dialect classifications. Indeed, early modern scholars did not engage in systematic historical-comparative research into the relationships among the Greek dialects exclusively or primarily based on linguistic data. Instead, they resorted to ancient authorities and adhered to received views, adjusting them to their scholarly programs on the basis of a priori arguments. In doing so, they only rarely invoked actual linguistic evidence.\footnote{See also \chapref{chap:5}, where this generalization will be nuanced.}

\section{The koine, an eternal problem?}\label{sec:2.9}

The sole difference between the two traditional Greek classifications was the absence or presence of the koine. This was a symptom of a larger problem, the inability of ancient and medieval Greek scholars to arrive at an adequate understanding of the historical position of the koine. How did early modern scholars conceive of the koine and its relationship to the Greek dialects? Did they likewise run into trouble when trying to grasp the precise status of this variety or were they more successful? As can be expected, traditional Greek insights persisted. \tabref{tab:2.2} shows early modern examples of the way in which traditional Greek views on the koine (outlined in \sectref{sec:2.3} above) were adopted and adapted, usually silently. In some early cases, the koine was not mentioned at all, which might be explained by its absence in the widely read account of Strabo (see e.g. \citealt{Stapleton1566}: 58\textsc{\textsuperscript{v}}–59\textsc{\textsuperscript{r}}).

\begin{sidewaystable}\footnotesize
\caption{Early modern uses of traditional Greek views on the koine. T4 refers to the traditional four dialects Aeolic, Attic, Doric, and Ionic.\label{tab:2.2}}
\begin{tabularx}{\textwidth}{l@{ }QQ}
\lsptoprule 
 & {Traditional Greek view} & Early modern example(s) of adoption and adaptation\\\midrule
 (1) & T4 derived from the koine. & \citet[335]{Borghini1971}: “[The koine] was common to all men of that nation and as it were the principal fundament of that language. Subsequently, […] it was divided into four other tongues, which in reality were not languages, but dialects”.\footnote{“fu comune a tutti di quella nazione, e come fondamento principale di quella lingua. Di poi, […] si divise in quattro altre, le quali in verità non furono lingue ma \textit{dialetti}”.} See e.g. also \citet[209]{Vergara1537}.\\
 (2) & The koine was mixed out of T4 and embraced them as a mother. It therefore consisted of the common properties of the dialects. & \citet[52]{Oecolampadius1518}: “Koine, i.e. common, is collected out of the other dialects and is commonly used by authors”.\footnote{“Koινή, id est communis, collecticia est ex ceteris, qua scriptores communiter utuntur”.} See e.g. also \citet[10\textsc{\textsuperscript{r}}\textsc{–10}\textsc{\textsuperscript{v}}]{Girard1541}. Henri \citet[28--34]{Estienne1581} was exceptional in attempting to substantiate this view with extensive linguistic evidence.\\
 (3) & The koine was the subject of grammar and characterized by rules, whereas T4 were variations on it. & The koine’s grammatical and analogical status was accepted as a given by most early modern grammarians (e.g. \citealt{Gaza1495}: α.1\textsc{\textsuperscript{v}}; \citealt{Schmidt1604}: 4; \citealt{Walch1772}: 137), who presented the dialects as deviations from it (cf. \citealt{Ciccolella2008}: 123), even though sometimes a special place was accorded to Attic, especially in the eighteenth century (e.g. \citealt{Luscinius1517}; \citealt{Hemsterhuis1721}: 68; \citealt{Jehne1782}: 288).\\
 (4) & The koine was used commonly by all. & \citet[a.i\textsc{\textsuperscript{v}}]{Melanchthon1518}: “The speech that is common to all is called common language”.\footnote{“Qui sermo communis omnibus est, lingua communis dicitur”.}\\
 (5) & The koine was a fifth dialect (without further explanation). & \citet[138\textsc{\textsuperscript{v}}]{Beroaldo1493}: “For the Greeks have five tongues: Ionic, Doric, Attic, Aeolic, and common”.\footnote{“nam linguas quinque habent: Ionicam, Doricam, Atthicam [\textit{sic}], Aeolicam et communem”.} See e.g. also \citet[85\textsc{\textsuperscript{r}}]{Perotti1489}.\\
\lspbottomrule
\end{tabularx}
\end{sidewaystable}

Did early modern scholars develop original solutions to the problematic position of the koine too? The answer to this question should be a clear yes. I argue for several reasons that the issue was problematized to a higher degree by early modern Hellenists than by their predecessors. This was for a large part due to evolutions in vernacular language studies during the Renaissance. Grammarians started to develop a norm for these tongues that could rival Latin as a valid and elegant means of communication (see e.g. \citealt{Giard1992}). This fostered the opposition of the normative variety, usually termed “common language”, to the other varieties, the “dialects”, a contrast becoming ever more emphatic in the course of the early modern period. This had enormous repercussions for conceptions of the Greek koine. For instance, a late seventeenth-century Hellenist and grammarian asked himself “whether the koine is likewise to be reckoned as a fifth dialect among those mentioned before”, to which he offered the following straightforward answer:

\begin{quote}
It is not, as this is not so much a dialect as the basis, and like a mother of, the dialects, to which these dialects belong as to a genus its species. It should therefore be designated with the term language rather than dialect.\footnote{\citet[495]{Ursin1691}: “\textit{Annon et quinta dialectus communis prioribus illis accensenda est?} Non est: quippe haec non tam dialectus, quam dialectorum basis et ueluti mater est, cui hae ut generi species suae accidunt; linguae igitur potius quam dialecti nomine appellanda”.}
\end{quote}

Furthermore, the relationship between the koine and the traditional four dialects was specified in much more detail by early modern scholars. The French grammarian Petrus Antesignanus was convinced that the koine was a variety consisting of the best features of the principal Greek dialects and mainly Attic.\footnote{\citet[12--13]{Antesignanus1554}, on which see \citet[130--131]{VanRooy2016c}.} A number of scholars considered the koine a common language based on Attic, thus coming close to the historical truth. This position is adopted, for example, by the Westphalian orientalist Hermann von der \citet[17–18, (1660–1746)]{Von1705}. Occasionally, Attic was even identified with the koine, in which case Attic was further specified as a more recent variety of Attic (\citealt{Georgi1733}: 3, 5). Others regarded the koine as the most important prose dialect next to Attic and as opposed to the other dialects principally employed in poetry, which were only to be tackled by brilliant minds and good students (\citealt{Vives1531}: 97\textsc{\textsuperscript{v}}; cf. \citealt{Vuidius1569}: 137\textsc{\textsuperscript{v}}).

There were contradictory opinions about the social status of the koine. Most scholars understood it as a literary variety, and some even explicitly associated it with the higher classes. For instance, the renowned English philologist Richard Bentley (1662–1742), known among other things for his restoration of the digamma in Homer’s epic poems, described the koine as “perfectly a language of the learned, almost as the Latin is now”, emphasizing that it “was never at any time or in any place the popular idiom” (1699: 406). Others advanced exactly the opposite view and associated the koine with lower social classes, among other reasons because it was much easier to learn than the literary dialects. One author, the seventeenth-century French lexicographer of medieval Greek Charles Du Cange (1610–1688), even claimed that reading too many koine books could defile one’s speech \citep[iv]{Du1688}. Du Cange mentioned this when discussing the causes of the corruption of the Greek language.

Claude de \citet[esp. 405--406]{Saumaise1643a} launched a highly influential interpretation of the koine, claiming that it started out as a dialect historically, specifically that it was a variety peculiar to the people of Thessaly (see also Chapter 5, \sectref{sec:5.4}). It was named Hellenic after Hellen, the forbear of the Greeks. Additionally, there was a city called Hellas in Thessaly. Saumaise was inspired to do so by the idiosyncratic classification propounded by Heraclides Criticus, which included a Hellenic dialect (see \sectref{sec:2.1} above). This Hellenic dialect, also called Thessalian and Macedonian, after giving birth to the four other Greek dialects, developed into a high-end common variety, employed by writers and no longer particular to a specific people, a criterion Saumaise deemed indispensable in order to speak of a “dialect”. A dissertation presented in Wittenberg in 1702, which elaborated on Saumaise’s framework, went as far as claiming that the original Hellenic language – Saumaise’s Thessalian-Hellenic-Macedonian dialect – was extinct, and that scholars later had collected common elements from the surviving dialects and had formed by means of analogy a new common language out of these (\citealt{Schwartz1702}: \textsc{c}.1\textsc{\textsuperscript{v}}; see \citet[]{VanRooyFce}).

The problematic status of the koine even led some scholars to doubt and indeed dismiss its very right to existence. This occurred in two Wittenberg dissertations of 1709 to which the obscure German classical scholar Georg Friedrich Thryllitsch (1688–1715) contributed, each time as \textit{respondens} and at least once as the sole author. In the first, a historiographical dissertation on the Greek dialects, \citet[\textsc{d.1}\textsc{\textsuperscript{v}}]{Thryllitsch1709} argued that there was no irrefutable proof that something like the koine actually existed. This view was expressed even more emphatically in the second dissertation, which was entirely devoted to the issue of the koine. In it, Thryllitsch, possibly together with the Wittenberg professor of Greek Georg Wilhelm Kirchmaier (1673–1759), tried to convince the reader

\begin{quote}
that there was no Alexandrian dialect, except for a secondary one which proceeded from Attic, that the Macedonian dialect was semi-barbarous and a daughter of Doric rather than Attic, that the Hellenic dialect had already become extinct before or certainly during Homer’s time, and that the common dialect, in truth, was a dream of learned men.\footnote{\citet[\textsc{c.2}\textsc{\textsuperscript{v}}]{Kirchmaier1709}: “quod Alexandrina non nisi secundaria eque Attica emanans dialectus, Macedonica semibarbara et Doricae potius quam Atticae filia, Hellenica iam ante, aut cum Homero certe abolita, communis uero doctorum hominum somnium fuerit”.}
\end{quote}

The Greek koine, it was argued, did not exist before grammars of Greek started to be composed around Plutarch’s time (\citealt{Kirchmaier1709}: \textsc{a.4}\textsc{\textsuperscript{v}}). In fact, the koine was forged so as to account for the presence of common forms in the Greek dialects and to distill grammatical analogy from them.

In conclusion, there was much uncertainty among early modern scholars about the position of the koine vis-à-vis other varieties of Greek and within the general history of the Greek language. This had a double cause. On the one hand, although proposing sometimes highly original answers to the question, scholars always expressed a priori views. In doing so, they were usually in some way or another inspired by Greek ideas, both common and unusual ones. On the other hand, scholars often lacked the necessary historical insight into the genesis of the koine and the Greek dialects. In fact, when trying to sketch the history of the Greek dialects – a theme further developed in Chapter 5 – they usually depended on Strabo’s account, in which, however, the koine did not figure. Still, setting the koine clearly apart from the dialects can be viewed as a major achievement of early modern scholarship, especially since earlier Greek conceptions of it were so blurred.

\section{Test case: Classifying vernacular Greek dialects}\label{sec:2.10}

Inspired by the Greek heritage, early modern scholars offered relatively rigid classifications of the ancient Greek dialects, even though they were somewhat more flexible than their predecessors. Yet how did they map out \textit{vernacular} Greek variation, which they usually distinguished clearly from the ancient dialects?\footnote{Some scholars did, however, regard vernacular Greek as a dialect of Greek. See e.g. \citet[.7\textsc{\textsuperscript{v}}]{Megiser1603} on “vulgar or new Greek, or Barbaric Greek [Graeca uulgaris, seu noua, uel Barbarograeca]”. Early modern attitudes toward the vernacular Greek language require a more thorough investigation. See, however, already \citet{Caratzas1952}, \citet{Rotolo1973}, and \citet{Toufexis2005}.} On what principles did they rely in doing so? Here, their approach was radically different, as interest in this issue was not primarily triggered by philological or historiographical concerns. Rather, early modern scholars only treated vernacular variation when they took a genuine interest in contemporary Greece and its inhabitants, an interest which grew rather slowly. Symptomatically, the earliest rudimentary estimation of the number of vernacular dialects came from the pen of a learned Greek correspondent of the German Philhellene Martin Crusius (1526–1607), Symeon Cabasilas (1546–after 1605), who stated that “there are many different dialects, more than seventy”, of which “that of the Athenians is the worst”.\footnote{Cabasilas in \citet[461]{Crusius1584}: “Περὶ δὲ τῶν διαλέκτων, τί ἂν καὶ ἔπoιμι [\textit{sic}]; Πoλλῶν oὐσῶν, καὶ διαφόρων, ὑπὲρ τῶν ἑβδoμήκoντα; Toύτων δ' ἁπασῶν, ἡ τῶν Ἀθηναίων χειρίστη”. In the Latin translation printed together with the original letter, the adverb \textit{fortassis}, ‘perhaps’, is added before “amplius septuaginta”. Perhaps Crusius wanted to mitigate Cabasilas’ claim.} A couple of decades earlier, the Swiss language cataloguer Conrad Gessner (1516–1565) had limited himself to mentioning some vernacular Greek dialects, for instance those of Crete, Cyprus, and the Peloponnese, without classifying them \citep[47\textsc{\textsuperscript{r}}]{Gessner1555}.

In the early eighteenth century, a number of Western scholars tried to make sense of vernacular Greek and its varieties. The most extensive classification of vernacular dialects was offered by the German academic Johann Tribbechow (1677–1712) in his dissertation on the emergence and nature of vernacular Greek, prefixed to his grammar of that language, published in Jena in 1705. In this interesting text, \citet[a.4\textsc{\textsuperscript{r}}\textsc{–}a.4\textsc{\textsuperscript{v}}]{Tribbechow1705} proposed a division into an insular and a continental class of dialects, to which he added, somewhat hesitatingly, the dialect of Constantinople as a third class. It is clear that his classification was principally inspired by geopolitical factors. The geographical contrast between the Greek mainland and the islands was transferred to the linguistic plane. The introduction of the Constantinopolitan dialect into the division was politically motivated, as the city was the seat of the patriarchate and the heart of the Ottoman empire at that time. Despite the presence of speakers of all vernacular Greek dialects in this city, Tribbechow still presented the speech of Constantinople as the purest and best variety. It allegedly shared its purity with the continental tongues of Thessaloniki, the Peloponnese, and the rest of mainland Greece, especially that of the city of Ioannina. There, vernacular speech had remained pure, because of the intensive cultivation of the erudite ancient language by its inhabitants and because of its geographical isolation. Interestingly, Tribbechow claimed to have verified his views with Greek youngsters studying at the Halle Oriental Theological College, with which he was associated.\footnote{On the Greek students at the \textit{Collegium Orientale Theologicum}, see \citet{Moennig1998} and \citet[283]{Makrides2006}.}

Tribbechow’s division into insular and continental dialects was apparently picked up by a Greek émigré born in Larissa and mainly active in England, Germany, and Russia during his adult life, Alexander Helladius (1686–after September 1715).\footnote{See \citet{Helladius1714}, \citet[315--317]{Moennig1998}, and \citet{VanRooyFca} for (auto)biographical information.} In a 1714 book published in Altdorf and devoted to the contemporary state of Greece and the Greek church, Helladius politicized the contrast between insular and continental dialects by stressing that the islands were Venetian-oc\-cu\-pied, whereas the mainland was Ottoman-occupied. Remarkably enough, he linked this to a linguistic criterion: lexical evidence.\footnote{Before Helladius, other Greek scholars had already tried to offer rudimentary groupings of vernacular Greek dialects based on linguistic evidence such as case variation resulting from the loss of the dative: see \citet[108]{Kritopoulos1924} and \citet[(writing ca. 1650) e.g. 1 \& 8]{Nikiforos1908} for some interesting but isolated observations.} The insular Italo-Greeks (\textit{insulani}/\textit{Italo-Graeci}), Helladius argued, had many words not used by mainlanders (\textit{continentem inhabitantes}). This must be read in close connection with the geopolitical opposition between the islands and the continent; mainlanders used more Turkish words, which they borrowed from their Ottoman occupiers, whereas the insular Greeks under Venetian rule introduced many Italian words into their speech (\citealt{Helladius1714}: 190–191, 194, 203). To exemplify the confusion caused by this vernacular variation, Helladius recounted several amusing anecdotes from his own life (see \citet{VanRooyFca}).

\section{Conclusion}\label{sec:2.11}

Contrary to ancient and medieval scholars, who roughly agreed on one classification – Attic, Ionic, Aeolic, and Doric with or without the koine – early modern Hellenists proposed a number of diverging classifications in their attempts at creating order in the chaos of Greek linguistic variation. This occurred most importantly in the context of philology and, to a lesser extent, historiography. The principles underlying their dialect classifications were almost without exception of a non-linguistic nature. Instead, they were informed by cultural, historical, and philological circumstances and by the authority of ancient, Byzantine, and early modern scholars. This shows that the study of the ancient Greek dialects was culturally embedded in various ways before the rise of ancient Greek dialectology in modern times, generally connected to the work of Ahrens. It can be stressed in this regard that nineteenth-century scholars did not create ancient Greek dialectology \textit{ex nihilo} (cf. \citealt{Colvin2007}: 22). Indeed, Ahrens and his colleagues elaborated on the achievements of their early modern predecessors, such as Michael Maittaire. For this reason, it is necessary to frame their contribution within earlier scholarship, a task awaiting completion. Major innovations of early modern scholarship include the philologically inspired distinction between primary and secondary dialects, the odd introduction and eventual dismissal of a poetical dialect, and the clear setting apart of the koine from the Greek dialects, even though the exact nature and history of the koine was often poorly understood.

The careful early modern attention for the classifications of the canonical ancient Greek dialects contrasts sharply with the lack of interest in grouping vernacular Greek varieties. In the early eighteenth century, however, certain scholars active in Germany did attempt to make general distinctions in vernacular Greek dialectal variation, most importantly by setting up the categories of island and mainland Greek. In doing so, they relied on geographical, political as well as linguistic principles, no doubt because in this case they were not as bound by an authoritative tradition as philologists had been when discussing the revered ancient dialects.

