\chapter{Picturing ancient Greece through the dialects}\label{chap:7}\largerpage

When in 1579 Franciscus Junius the Elder (1545–1602) held his \textit{Discourse on the antiquity and excellence of the Hebrew language} at the short-lived reformed academy of Neustadt (the \textit{Casimirianum}), he could not resist emphasizing the merits of this sacred tongue vis-à-vis the Greek language:

\begin{quote}
Indeed, as to individual words, fluency of expression is achieved by the fact that there are neither innumerous words nor so many dialects [in Hebrew] as among the verbose and mendacious Greeks, since almost every single author among them seems to have forged himself his own language because of a certain malicious rivalry.\footnote{\citet[\textsc{b.3}\textsc{\textsuperscript{v}}]{Junius1579}: “In uocibus enim singulis pertinet ad facilitatem istud, quod non habentur innumerae uoces neque dialecti tam multae, ut apud uerbosos et mendaces Graecos, quorum singuli paene auctores suam sibi linguam cacozelo quodam uidentur fabricasse”. This discourse was reprinted in Junius’s Hebrew grammar (\citealt{Junius1580}: ẽ.ii\textsc{\textsuperscript{v}}–ẽ.iii\textsc{\textsuperscript{r}}). The word \textit{cacozelus} (< Greek κακόζηλoς) can mean both ‘using a bad, affected style’ and – in the neuter (τὸ κακόζηλoν) – ‘unhappy imitation; rivalry’ (\citealt{LiddellScott1940}: \textit{s.v.}). Here, “cacozelo quodam” must be interpreted as an ablative of the substantivized adjective expressing a cause.}
\end{quote}

Junius’s observation on the uniformity of Hebrew, favorably compared to the endlessly affected variation of Greek, betrays his negative ideas about the countless differences existing among the Greek dialects. It moreover shows that he connected the Greek dialects to other aspects of Greekness, in this case the Greeks’ innate verbosity, mendacity, and malicious competitiveness. Junius was not the only scholar to do so. Numerous early modern thinkers related dialectal differences existing in Greek to language-external aspects of ancient Greece. How and why did they do so? And to what extent were they inspired by ancient and medieval sources?

\section{Texts and tribes}\label{sec:7.1}

As the Greek dialects were principally studied for philological reasons, scholars associated them closely with the literary texts composed in them (see Chapter \ref{chap:3}). As a matter of fact, ever since antiquity, it had been customary to link a dialect primarily to an author or a group of authors. Aeolic was written by authors such as Alcaeus and Sappho, Attic by Plato and Thucydides, Doric by Alcman and Theocritus, and Ionic by Herodotus and Hippocrates. Oddly enough, several Greek scholars mistook Pindar’s language for the Koine, a misconception definitively corrected only in the Renaissance. The Italian Hellenist Angelo Canini (1521–1557) was already able to rightly identify the poet’s speech as principally Doric (\citealt{Canini1555}: a.4\textsc{\textsuperscript{r}}). The dialects were moreover tied up with specific literary genres. Doric was, for example, the usual dialect of bucolic poetry and tragic choral odes. At the same time, the dialects were also associated with the homogeneously conceived tribes speaking them. Aeolic was the dialect of the Aeolian Greeks, Doric of the Dorians, Ionic of the Ionians, and Attic of the inhabitants of Attica. This coincidental close linking of the dialects with literature, on the one hand, and the people speaking them, on the other, made authors prone to transferring evaluative labels associated with literary genres and Greek tribes to the dialects themselves. In this and the following sections I will focus on such dialect attitudes.

Research into language and dialect attitudes in general is a recent, though well-established field of investigation (see e.g. \citealt{Edwards2009}: 73–98; \citealt{Garrett2010}: 19–29). It studies what qualities and vices are ascribed to specific speech forms, and how and why this happens. In other words, it endeavors to map out the impressions languages and dialects convey to speakers. Such impressions are often construed or reinforced by cultural stereotypes – i.e. assumptions about the alleged characteristics of specific regions and ethnic groups – so that the study of language and dialect attitudes may be considered a contribution to imagology as well (on imagology, see \citealt{BellerLeerssen2007}). Early modern attitudes to other languages and dialects have already received considerable attention. William J. \citet{Jones1999}, for instance, has studied the attitudes of early modern German scholars toward European languages. However, no systematic treatment of early modern attitudes toward the ancient Greek dialects exists, which is why I aim to offer a first exploration of the matter here, with a focus on attitudes toward the canonical four dialects: Aeolic, Attic, Doric, and Ionic.\footnote{For attitudes toward Attic in early modern German works, see the brief account of \citet[251--252]{Roelcke2014}.} Here, too, it is impossible to understand early modern views separately from ancient Greek and Byzantine ideas. For this reason, I will very briefly delve into Greek views first.

\section{Dialect attitudes from antiquity to early modernity}\label{sec:7.2}


Ancient and Byzantine Greek authors expressed their assessments of individual dialects at various occasions in their works, almost as a rule in passing. This occurred in diverging genres, including works of grammar, philosophy, history, geography, rhetoric, and even poetry. As most relevant comments are of a cursory nature, there was no canonical, generally accepted evaluation of the Greek dialects. Some ancient Roman authors also attributed labels to Greek dialects in the same sporadic fashion. \tabref{tab:7.1} offers a synoptic overview of the most important ancient and medieval attitudes toward the dialects. It suggests that negative labels were more numerous than positive ones. This does not indicate, however, that the canonical dialects were predominantly assessed in a negative way. Many of the unfavorable evaluations were only mentioned by one author, such as the label of “barbarian” in the case of (Lesbian) Aeolic, whereas some of the positive labels were widespread, in particular the eloquence and elegance of Attic.

\begin{longtable}{>{\raggedright\arraybackslash}p{3cm}>{\raggedright\arraybackslash}p{\textwidth - 3\tabcolsep - 3cm}}
\caption{Ancient and medieval attitudes toward the canonical Greek dialects}\label{tab:7.1}\\
\lsptoprule Label & Sources (\& early modern authors relying on them)\\\midrule\endfirsthead\midrule Label & Sources (\& early modern authors relying on them)\\\midrule\endhead\endfoot\lspbottomrule\endlastfoot
\multicolumn{2}{c}{Aeolic}\\\midrule
 barbarian & Plato, \textit{Protagoras} 341c, said specifically of Lesbian Aeolic.\\
 obscure & Dionysius of Halicarnassus, \textit{De imitatione} 31.2.8.\\
 unusual, affected, insolent & Apuleius, \textit{Apologia (Pro se de magia liber)} 9; Athenaeus, \textit{Deipnosophistae} 14.19. (See e.g. \citealt{Munthe1748}: 3.)\\
 old-fashioned, archaic & \textit{Scholia Vaticana} (ed. \citealt{Hilgard1901}: 117).\\
\midrule\multicolumn{2}{c}{Attic}\\\midrule
 mixed & Pseudo-Xenophon, \textit{Atheniensium respublica} 2.8; Athenaeus, \textit{Deipnosophistae} 3.94; Pseudo-Plutarch, \textit{De Homero 2}. (See e.g. \citealt{Schwartz1721}: 223; \citealt{Maittaire1706}: iii; \citealt{Saumaise1643a}: 437–438, respectively.)\\
 (too) elaborate & Heraclides Criticus, \textit{Descriptio Graeciae} 1.4. (See e.g. \citealt{Estienne1573}: ¶.ii\textsc{\textsuperscript{v}}–¶.iii\textsc{\textsuperscript{r}}, referring to “Artemidori geographiae fragmentum”.)\\
 concise, popular, fitting for pleasantries & Demetrius, \textit{De elocutione} 177; Cicero, \textit{Orator} 89. (See e.g. \citealt{Munthe1748}: 3.)\\
 excellent, charming, eloquent & Quintilian, \textit{Institutio oratoria} 6.3.107, 8.1.2 \& 10.1.100; Cicero, \textit{Orator} 25 \& 28 and \textit{Brutus} 172; Velleius Paterculus, \textit{Historiae Romanae} 1.18.1. (See e.g. \citealt{Duret1613}: 690; \citealt{Rollin1726}: 118–119.)\\
 artificial & \textit{Scholia Vaticana} (ed. \citealt{Hilgard1901}: 117).\\\midrule\multicolumn{2}{c}{Doric}\\\midrule
 broad, flat & Theocritus, \textit{Idyllia} 15.87–88 and \textit{Scholia in Theocritum} (\textit{scholia uetera}) on this passage; Hermogenes, Περὶ ἰδεῶν λόγoυ 1.6; Demetrius, \textit{De elocutione} 177. (See e.g. Caelius \citealt{Caelius1542}: 465; \citealt{Estienne1573}: ¶.ii\textsc{\textsuperscript{r}}-¶.ii{\textsc{\textsuperscript{v}}; \citealt{Saumaise1643a}: 77.)\\
 annoying, affected & Suetonius, \textit{De uita Caesarum}, \textit{Tiberius} 56.1.\\
 obscure & Porphyry, \textit{Vita Pythagorae} 53. (See e.g. \citealt{Bentley1699}: 317; \citealt{Mazzocchi1754}: 119 n.5.)\\
 rustic & Pseudo-Probus, \textit{Commentarius in Vergilii Bucolica et Georgica}, \textit{praefatio}. Marcus Manilius (\textit{Astronomica} 767) associated Dorians with rusticity in general terms. (See \citealt{Rapin1659}: 121; cf. \textit{infra}.)\\\relax
 [\textit{said of old Doric:}] rough, difficult & \textit{Scholia in Theocritum (scholia uetera)} F.a.–c. (For old Doric, see e.g. \citealt{Mazzocchi1754}: 118–119; for new Doric, see e.g. \citealt{Valckenaer1773}: 208.)\\\relax
 [\textit{said of new Doric:}] gentler, easier & \\
 magnificent & \textit{Scholia Vaticana} (ed. \citealt{Hilgard1901}: 117). (See e.g. \citealt{Estienne1581}: 15–16.)\\\midrule\multicolumn{2}{c}{Ionic}\\\midrule 
fluent, pleasant & Quintilian, \textit{Institutio oratoria} 9.14.18. (See e.g. \citealt{Munthe1748}: 9.)\\
 relaxed, frivolous & \textit{Scholia Vaticana} (ed. \citealt{Hilgard1901}: 117).\\
% % % \begin{tabularx}{\textwidth}{lQQ}
% % % \lsptoprule
% % % \textsc{Dialect} & \textsc{Label} & \textsc{Sources} (\& early modern authors relying on them)\\
% % % \multicolumn{1}{c}{Aeolic} & barbarian & Plato, \textit{Protagoras} 341c, said specifically of Lesbian Aeolic.\\
% % % & obscure & Dionysius of Halicarnassus, \textit{De imitatione} 31.2.8.\\
% % %  & unusual, affected, insolent & Apuleius, \textit{Apologia (Pro se de magia liber)} 9; Athenaeus, \textit{Deipnosophistae} 14.19. (See e.g. \citealt{Munthe1748}: 3.)\\
% % %  & old-fashioned, archaic & \textit{Scholia Vaticana} (ed. \citealt{Hilgard1901}: 117).\\
% % % \multicolumn{1}{c}{Attic} & mixed & Pseudo-Xenophon, \textit{Atheniensium respublica} 2.8; Athenaeus, \textit{Deipnosophistae} 3.94; Pseudo-Plutarch, \textit{De Homero 2}. (See e.g. \citealt{Schwartz1721}: 223; \citealt{Maittaire1706}: iii; \citealt{Saumaise1643a}: 437–438, respectively.)\\
% % % & (too) elaborate & Heraclides Criticus, \textit{Descriptio Graeciae} 1.4. (See e.g. \citealt{Estienne1573}: ¶.ii\textsc{\textsuperscript{v}}–¶.iii\textsc{\textsuperscript{r}}, referring to “Artemidori geographiae fragmentum”.)\\
% % %  & concise, popular, fitting for pleasantries & Demetrius, \textit{De elocutione} 177; Cicero, \textit{Orator} 89. (See e.g. \citealt{Munthe1748}: 3.)\\
% % %  & excellent, charming, eloquent & Quintilian, \textit{Institutio oratoria} 6.3.107, 8.1.2 \& 10.1.100; Cicero, \textit{Orator} 25 \& 28 and \textit{Brutus} 172; Velleius Paterculus, \textit{Historiae Romanae} 1.18.1. (See e.g. \citealt{Duret1613}: 690; \citealt{Rollin1726}: 118–119.)\\
% % %  & artificial & \textit{Scholia Vaticana} (ed. \citealt{Hilgard1901}: 117).\\
% % % \multicolumn{1}{c}{Doric} & broad, flat & Theocritus, \textit{Idyllia} 15.87–88 and \textit{Scholia in Theocritum} (\textit{scholia uetera}) on this passage; Hermogenes, Περὶ ἰδεῶν λόγoυ 1.6; Demetrius, \textit{De elocutione} 177. (See e.g. Caelius \citealt{Caelius1542}: 465; \citealt{Estienne1573}: ¶.ii\textsc{\textsuperscript{r-v}}; \citealt{Saumaise1643a}: 77.)\\
% % % & annoying, affected & Suetonius, \textit{De uita Caesarum}, \textit{Tiberius}, 56.1.\\
% % %  & obscure & Porphyry, \textit{Vita Pythagorae} 53. (See e.g. \citealt{Bentley1699}: 317; \citealt{Mazzocchi1754}: 119 n.5.)\\
% % %  & rustic & Pseudo-Probus, \textit{Commentarius in Vergilii Bucolica et Georgica}, \textit{praefatio}. Marcus Manilius (\textit{Astronomica} 767) associated Dorians with rusticity in general terms. (See \citealt{Rapin1659}: 121; cf. \textit{infra}.)\\
% % %  & [\textit{said of old Doric:}] rough, difficult & \textit{Scholia in Theocritum (scholia uetera)} F.a.–c. (For old Doric, see e.g. \citealt{Mazzocchi1754}: 118–119; for new Doric, see e.g. \citealt{Valckenaer1773}: 208.)\\
% % %  & [\textit{said of new Doric:}] gentler, easier & \\
% % %  & magnificent & \textit{Scholia Vaticana} (ed. \citealt{Hilgard1901}: 117). (See e.g. \citealt{Estienne1581}: 15–16.)\\
% % % \multicolumn{1}{c}{Ionic} & fluent, pleasant & Quintilian, \textit{Institutio oratoria} 9.14.18. (See e.g. \citealt{Munthe1748}: 9.)\\
% % % & relaxed, frivolous & \textit{Scholia Vaticana} (ed. \citealt{Hilgard1901}: 117).\\
% % % \lspbottomrule
% % % \end{tabularx}
\end{longtable}

It can be noted here that ancient scholars were prone to link the Greek tribes and their dialects to styles within certain arts as well. The Greek dialects were in other words not approached in isolation, but viewed as an undeniable characteristic of the Greek world, pervading numerous dimensions of it. Modes of music were called Doric and Aeolic because they were reminiscent of certain features of these dialects, and a similar association occurred in scholarship on architecture. It would lead me too far to treat this complex extrapolation of the traditional Greek tribal-dialectal scheme to music and architecture in detail here, all the more since its impact on early modern views was highly limited.\footnote{See \citet{Cassio1984}. \citet[118]{Mazzocchi1754} was exceptional in connecting the canonical dialects and architectural styles with the same evaluative properties. In the case of Doric, this was coarseness and roughness. In doing so, he no doubt relied on Vitruvius, \textit{De architectura} 4.1.6–8.} Yet it is important to keep in mind that the dialects were intertwined with other domains of knowledge, and that they were able to evoke strong sensual associations going beyond the level of language even in ancient and medieval times.

As can be expected, early modern scholars relied to a considerable degree on ancient and Byzantine sources when attributing evaluative labels to the canonical dialects; this can be gathered from \tabref{tab:7.1}, which offers a rudimentary chart of this dependence of early modern Hellenists on earlier sources. There are nonetheless three major differences between ancient and medieval texts, on the one hand, and early modern works, on the other. Firstly, scholars introduced numerous new assessments, as \tabref{tab:7.2} reveals. These were very often a direct consequence of the literary usage of the dialect in question. For instance, the frequent characterization of Doric as “boorish” or “rustic” seems to have largely been an early modern innovation. Pseudo-Probus already called Doric \textit{rusticus} in his commentary on Vergil’s \textit{Bucolics} and \textit{Georgics}, but this is an isolated instance, which barely influenced early modern authors. The early modern emphasis on Doric rusticity is likely to have been due to a stronger association of Doric with the bucolic poetry of authors such as Theocritus, a very popular poet among humanists and in their schools. This is in agreement with a broader tendency in language attitudes. Indeed, Brigitte \citet{Schlieben-lange1992} has shown that it is not uncommon for properties of texts to be transferred to the variety in which they are written. For example, in a letter dating to November 1511, a German student learning Greek in Paris characterized the Doric dialect as “scabrous” or “filthy” (\textit{scaber}) and “somewhat rustic” (\textit{subrusticus}). He complained that his teacher, the polyglot humanist and later cardinal Girolamo Aleandro (1480–1542), kept focusing on the Doric poetry of Theocritus instead of reading texts in the \textit{lingua communis}, the Greek Koine. The student did admit, however, that this dialect was very apt for rustic subject matter.\footnote{The letter, written by a certain Johannes Kierher, is cited in \citet[220 n.435; cf. also p. 103]{Botley2010}.} The idea of Doric roughness was also fostered by its close association with the rugged Peloponnese and the rather unrefined mores of its inhabitants, not in the least those of warlike Sparta. The Dutch philologist Isaac Vossius (1618–1698) linked harshness and rusticity to Doric in his 1673 treatise on ancient poetry, claiming that the Ionians laughed at the Dorians for this reason. The Dorians, in turn, allegedly mocked the Ionians for their effeminacy \citep[55]{Vossius1673}. Vossius was, in a sense, fictitiously reconstructing the mutual social behavior of two ancient Greek tribes by relying on widespread stereotypes about them.


\begin{longtable}{>{\raggedright\arraybackslash}p{3cm}>{\raggedright\arraybackslash}p{\textwidth - 3\tabcolsep - 3cm}}
\caption{Early modern attitudes toward the canonical ancient Greek dialects. The number of examples offered in the right column can be taken as an indication of the frequency of each label.\label{tab:7.2}}\\
\lsptoprule Label & Testimonies\\\midrule\endfirsthead\midrule Label & Testimonies\\\midrule\endhead\endfoot\lspbottomrule\endlastfoot
\multicolumn{2}{c}{Aeolic}\\\midrule
 sweet, adequate for lyric poems & \citet[a.4\textsc{\textsuperscript{r}}]{Canini1555} called Aeolic “melicis apta”. Cf. \citet[103]{Hoius1620}. \citet[106]{Giraudeau1739} regarded it as “pronuntiatu suauissima”.\\
 heavy, weighty, serious & \citet[16]{Estienne1581} believed it to display a certain \textit{grauitas}, ‘seriousness’, which is central to his discussion of the qualities of French vis-à-vis Italian as well \citep[71]{Swiggers2009}.\\
 rough, uncultivated, unpleasant & \citet[61]{Walper1589}; \citet[415]{Walper1590} labeled it together with the allegedly cognate Doric dialect \textit{incultior}, \textit{ingratus auribus}, \textit{minus politus}, and \textit{insuauis}. See e.g. also \citet[515, \textit{asper}]{Fabricius1711}; \citet[6, \textit{rudis}]{Georgi1729}; \citet[e.g. 28, \textit{inamoenus}]{Munthe1748}.\\
 broad, rather thick & By analogy with Doric, to which Aeolic was believed to be closely cognate, \citet[582]{Nibbe1725} called Aeolic \textit{breit}. See e.g. also \citet[\textsc{a.2}\textsc{\textsuperscript{v}}]{Hauptmann1776}, where the verb \textit{platustomé\={o}} (πλατυστoμέω), ‘to speak with a broad mouth’, is applied to Aeolic. \Citet[17]{Von1705} characterized Aeolic pronunciation as \textit{obtusior}.\\
\midrule\multicolumn{2}{c}{Attic}\\\midrule\relax
 (most) elegant, noble, polished, cultivated, tender, fine, pure, neat, honey-sweet, etc. & \citet[a.i\textsc{\textsuperscript{v}}]{Melanchthon1518} called Attic “elegantissima”. See e.g. also \citet[209]{Vergara1537}; \citet[5\textsc{\textsuperscript{r}}]{Baile1588}; \citet[334]{Alsted1630}. \citet[226]{Ruland1556} attributed \textit{concinnitas} to Attic and characterized it as beautiful and charming. See e.g. also \citet[\textsc{e}.iii\textsc{\textsuperscript{v}}]{Oreadini1525}; \citet[76, 112, 424]{Saumaise1643a}, who linked this label to a round-mouthed pronunciation. \citet[96]{Hoius1620} called Attic \textit{mellitus}.\\
 copious & \citet[a.3\textsc{\textsuperscript{v}}]{Canini1555} dubbed Attic \textit{copiosus}.\\
 manly, weighty & \citet[6]{Georgi1729} applied the adjectives \textit{uirilis} and \textit{grauis} to Attic. See e.g. also \citet[515]{Fabricius1711}.\\
\midrule\multicolumn{2}{c}{Doric}\\\midrule
 boorish, rustic & \citet[317]{Bentley1699} e.g. labeled Doric \textit{rustic}. This property led the translator of \citet[117]{Rapin1659} to call Doric “sometimes scarce true grammar” \citep[31]{Rapin1684}. See also the main text.\\
 pleasant, adequate for smoother poets & \citet[a.4\textsc{\textsuperscript{r}}]{Canini1555} dubbed it “suauissima” and “poetis mollioribus accommodatissima”. See e.g. also \citet[139\textsc{\textsuperscript{r}}]{Vuidius1569}.\\
 rough, uncultivated, unpleasant & See above on Aeolic and \citet[46\textsc{\textsuperscript{r}}]{Gessner1555}, labelling Doric \textit{crassissimus}. \citet[54]{Vossius1673} characterized Laconian, a variety of Doric, as rough, threatening, and “doglike”. This last property was linked to the frequency of the letter rho, the dog’s letter, at the end of many Laconian words. Cf. \citet[24]{Munthe1748}.\\
 short in speech & Attributed to Laconian Doric by Plato (\textit{Leges} 641e), it was extrapolated to Doric as a whole by \citet[393]{Saumaise1643a}. Cf. \citet[138\textsc{\textsuperscript{v}}]{Beroaldo1493}.\\
 magnificent, warlike, manly & \citet[55]{Vossius1673} described the Doric dialect as “magnifica et bellica, sed absque iracundia”. He also associated it with manliness.\\
 distinguished, flourishing & \citet[161]{Gesner1774} called Doric \textit{florentissimus}.\\
\midrule\multicolumn{2}{c}{Ionic}\\\midrule
 long in speech, slow, redundant & Caelius \citet[677]{Caelius1542} opposed Ionic lengthiness in speech to Laconian brevity (he called the Ionians “\textit{makrológoi} [μακρoλόγoι]”). \citet[75]{Saumaise1643a} spoke of Ionic slowness and redundancy.\\
 elegant, polished, neat, honey-sweet & \citet[\textsc{a.2}\textsc{\textsuperscript{r}}]{Hauptmann1776} ascribed \textit{mundities} to Ionic. \citet[290]{Verwey1684} spoke of the \textit{mel Ionicum}.\\
 faint, delicate, womanish & \citet[75]{Saumaise1643a} linked the \textit{genius} of Ionic to the mores and the “long and fluid” clothing style of the Ionians, which he characterized as both faint and womanish. He pointed to the migration to Asia as the cause of their effeminacy. See e.g. also \citet[139\textsc{\textsuperscript{r}}]{Vuidius1569}.\\
\end{longtable}
% % % \begin{table}
% % % \caption{Early modern attitudes toward the canonical ancient Greek dialects. The number of examples offered in the right column can be taken as an indication of the frequency of each label.}\label{tab:7.2}
% % % 
% % %  
% % % \begin{tabularx}{\textwidth}{XXX}
% % % \lsptoprule
% % % 
% % % \multicolumn{1}{c}{\textbf{\textsc{Dialect}}} & \textbf{\textsc{Label}} & \textbf{\textsc{Testimonies}}\\
% % % \multicolumn{1}{c}{Aeolic} & sweet, adequate for lyric poems & \citet[a.4\textsc{\textsuperscript{r}}]{Canini1555} called Aeolic “melicis apta”. Cf. \citet[103]{Hoius1620}. \citet[106]{Giraudeau1739} regarded it as “pronuntiatu suauissima”.\\
% % % & heavy, weighty, serious & \citet[16]{Estienne1581} believed it to display a certain \textit{grauitas}, ‘seriousness’, which is central to his discussion of the qualities of French vis-à-vis Italian as well \citep[71]{Swiggers2009}.\\
% % %  & rough, uncultivated, unpleasant & \citet[61]{Walper1589}; \citet[415]{Walper1590} labeled it together with the allegedly cognate Doric dialect \textit{incultior}, \textit{ingratus auribus}, \textit{minus politus}, and \textit{insuauis}. See e.g. also \citet[515 (\textit{asper})]{Fabricius1711}; \citet[6 (\textit{rudis})]{Georgi1729}; \citet[e.g. 28 (\textit{inamoenus})]{Munthe1748}.\\
% % %  & broad, rather thick & By analogy with Doric, to which Aeolic was believed to be closely cognate, \citet[582]{Nibbe1725} called Aeolic \textit{breit}. See e.g. also \citet[\textsc{a.2}\textsc{\textsuperscript{v}}]{Hauptmann1776}, where the verb \textit{platustoméo\={} } (πλατυστoμέω), ‘to speak with a broad mouth’, is applied to Aeolic. Von der \citet[17]{Von1705} characterized Aeolic pronunciation as \textit{obtusior}.\\
% % % \multicolumn{1}{c}{Attic} & (most) elegant, noble, polished, cultivated, tender, fine, pure, neat, honey-sweet, etc. & \citet[a.i\textsc{\textsuperscript{v}}]{Melanchthon1518} called Attic “elegantissima”. See e.g. also \citet[209]{Vergara1537}; \citet[5\textsc{\textsuperscript{r}}]{Baile1588}; \citet[334]{Alsted1630}. \citet[226]{Ruland1556} attributed \textit{concinnitas} to Attic and characterized it as beautiful and charming. See e.g. also \citet[\textsc{e}.iii\textsc{\textsuperscript{v}}]{Oreadini1525}; \citet[76, 112, 424]{Saumaise1643a}, who linked this label to a round-mouthed pronunciation. \citet[96]{Hoius1620} called Attic \textit{mellitus}.\\
% % % & copious & \citet[a.3\textsc{\textsuperscript{v}}]{Canini1555} dubbed Attic \textit{copiosus}.\\
% % %  & manly, weighty & \citet[6]{Georgi1729} applied the adjectives \textit{uirilis} and \textit{grauis} to Attic. See e.g. also \citet[515]{Fabricius1711}.\\
% % % \multicolumn{1}{c}{Doric} & boorish, rustic & \citet[317]{Bentley1699} e.g. labeled Doric \textit{rustic}. This property led the translator of \citet[117]{Rapin1659} to call Doric “sometimes scarce true grammar” \citep[31]{Rapin1684}. See also the main text below.\\
% % % & pleasant, adequate for smoother poets & \citet[a.4\textsc{\textsuperscript{r}}]{Canini1555} dubbed it “suauissima” and “poetis mollioribus accommodatissima”. See e.g. also \citet[139\textsc{\textsuperscript{r}}]{Vuidius1569}.\\
% % %  & rough, uncultivated, unpleasant & See above on Aeolic and \citet[46\textsc{\textsuperscript{r}}]{Gessner1555}, labelling Doric \textit{crassissimus}. \citet[54]{Vossius1673} characterized Laconian, a variety of Doric, as rough, threatening, and “doglike”. This last property was linked to the frequency of the letter rho, the dog’s letter, at the end of many Laconian words. Cf. \citet[24]{Munthe1748}.\\
% % %  & short in speech & Attributed to Laconian Doric by Plato (\textit{Leges} 641e), it was extrapolated to Doric as a whole by \citet[393]{Saumaise1643a}. Cf. \citet[138\textsc{\textsuperscript{v}}]{Beroaldo1493}.\\
% % %  & magnificent, warlike, manly & \citet[55]{Vossius1673} described the Doric dialect as “magnifica et bellica, sed absque iracundia”. He also associated it with manliness.\\
% % %  & distinguished, flourishing & \citet[161]{Gesner1774} called Doric \textit{florentissimus}.\\
% % % \multicolumn{1}{c}{Ionic} & long in speech, slow, redundant & Caelius \citet[677]{Caelius1542} opposed Ionic lengthiness in speech to Laconian brevity (he called the Ionians “\textit{makrológoi} [μακρoλόγoι]”). \citet[75]{Saumaise1643a} spoke of Ionic slowness and redundancy.\\
% % % & elegant, polished, neat, honey-sweet & \citet[\textsc{a.2}\textsc{\textsuperscript{r}}]{Hauptmann1776} ascribed \textit{mundities} to Ionic. \citet[290]{Verwey1684} spoke of the \textit{mel Ionicum}.\\
% % %  & faint, delicate, womanish & \citet[75]{Saumaise1643a} linked the \textit{genius} of Ionic to the mores and the “long and fluid” clothing style of the Ionians, which he characterized as both faint and womanish. He pointed to the migration to Asia as the cause of their effeminacy. See e.g. also \citet[139\textsc{\textsuperscript{r}}]{Vuidius1569}.\\
% % % 
% % % \lspbottomrule
% % % \end{tabularx}
% % % \end{table}

A second major difference is that the sources and motivations of early modern scholars to propose dialect evaluations are more transparent than those of their ancient and Byzantine predecessors. Early modern attitudes toward the Ionic dialect provide a good example of the various ways in which Hellenists supplemented the ancient and Byzantine sources. To start with, philologists introduced new properties by quoting ancient testimonies that did not so much concern the dialects as the tribes speaking them. These ancient text passages encouraged early modern scholars to construe a specific mental picture of these tribes, their customs, and their speech. Claude de Saumaise, for example, characterized Ionic as \textit{mollis}, ‘effeminate, delicate’, by referring to a verse of the Roman poet Martial (ca. \textsc{ad} 40–103): “nor let the \textit{delicate Ionians} be praised for their temple of Trivia”.\footnote{\citet[75]{Saumaise1643a}, citing Martial, \textit{Spectaculorum liber} 1.3: “nec Triuiae templo \textit{molles} laudentur \textit{Iones}” (my emphasis).} Saumaise moreover linked Ionic effeminacy to their clothing style and, more fundamentally, to their migration from Greece to Asia, thus presenting a classic case of an Orientalist attitude (cf. \citealt{Said2003}). The Danish philologist and professor Caspar Frederik Munthe (1704–1763) and his colleague Ludvig Heiberg (1723–1760), in turn, relied on the Byzantine scholia on Thucydides for their opposition of Ionic delicacy to Doric manliness.\footnote{See \citet[15]{Munthe1748}, relying on \textit{Scholia in Thucydidem} (\textit{Scholia uetera et recentiora}), commentary at 1.124.1. On the Doric–Ionic opposition in antiquity, see \citet{Cassio1984}.} Here, the alleged properties of the people speaking a dialect were transferred to the dialect itself, a procedure very common throughout history. Indeed, John \citet[66--68]{Edwards2009} has pointed out that there exists a clear causative link between stereotypes about certain social groups and the esthetic qualities attributed to the varieties they speak (see also \citealt{Silverstein2003}; \citealt{Preston2018}: 200). Some early modern philologists even argued that certain tribal characteristics manifested themselves in specific dialectal features. Isaac Vossius linked Ionic delicacy and effeminacy to concrete features of the dialect: the frequency of the letter eta ⟨η⟩ in it, its lack of contractions, its many diminutives, and other linguistic “flatteries”, such as the alleged usage of feminine articles with male objects and animals, even with the most “monstrous” ones.\footnote{\citet[55]{Vossius1673}: “Nihil hac mollius et effeminatius, siue ubique occurrentem litteram ἦτα, siue frequentes uocalium hiatus, siue etiam crebra diminutiua aliaque spectes blandimenta. Adeo huic populo terrori fuit, quidquid esset uirile, ut quibusque fere rebus masculis et beluis etiam quantumuis immanibus, sequioris sexus articulos praeposuerint”.} Before the early modern period, the link between linguistic features and evaluative properties was practically non-existent with one sole exception: the idea of Doric broadness was sometimes connected to the frequency of the letter alpha ⟨α⟩ in this dialect.\footnote{See \textit{Scholia in Theocritum} (\textit{scholia uetera}) at \textit{Idyllia} 15.87–88.} Finally, one scholar, Henri Estienne, created new authoritative documentation himself in order to establish the smooth character of Ionic. In his commentary on Attic of 1573, \citet[ii\textsc{\textsuperscript{r}}]{Estienne1573} quoted – somewhat pretentiously, one might say – a Greek epigram of his own invention to prove the historical primacy of Ionic as well as its sweet and delicate character. He had prefixed this poem to his edition of the Ionian historian Herodotus, published three years earlier:

\begin{quote}
The Ionic dialect is indeed sweet, far above all,
\end{quote}

\begin{quote}
and utters delicate noises, but certainly,
\end{quote}

\begin{quote}
as far as Ionic surpasses all, so far
\end{quote}

\begin{quote}
does Herodotus surpass those speaking Ionic.\footnote{\citet[8]{Estienne1570}: “Ἔστι μὲν ἔστιν Ἰὰς λιγυρὴ διάλεκτος ἁπασῶν / ἔξοχα, καὶ μαλακοὺς ἐξαφιεῖσα θρόους· / ἀλλὰ γὰρ ὅσσον Ἰὰς πασῶν προφερεστάτη ἐστί, / τόσσον ἰαζόντων Ἡρόδοτος προφέρει”.}
\end{quote}

A third difference is that early modern Hellenists tried to organize their evaluations in a much more systematic manner. In Greek scholarship, there had been only one isolated attempt at doing so. A Byzantine scholiast, commenting on the ancient grammar attributed to Dionysius Thrax, was exceptional in trying to systematize the characteristic properties of the Greek dialects, linking them to the customs of the individual Greek tribes:

\begin{quote}
The Greeks indeed differ from the barbarians with respect to customs, speech as well as ways of life. One has to know, however, that, among the Greeks, there are the Dorians, the Aeolians, the Ionians, and the Attics. And we are explaining qualities occurring among these, for even these [tribes] do differ from one another in their ways as well as their customs. In fact, the Doric tribe seems to be manlier in its ways of life, and magnificent in the sounds of its names and in the tone of its voice, whereas the Ionic is relaxed in all these aspects, since the Ionians are frivolous. The Attic tribe seems to differ as regards way of life and artificiality of speech, whereas the Aeolic is distinctive through the austerity of its way of life and the old fashion of its speech.\footnote{%
  \textit{Commentaria in Dionysii Thracis Artem Grammaticam}, \textit{Scholia Vaticana (partim excerpta ex Georgio Choerobosco, Georgio quodam, Porphyrio, Melampode, Stephano, Diomede)} (ed. \citealt{Hilgard1901}: 117): 
  “καὶ γὰρ ἤθεσι καὶ διαλέκτῳ καὶ ἀγωγαῖς διαφέρουσιν <οἱ> Ἕλληνες τῶν βαρβάρων. Γινώσκειν δὲ χρὴ ὅτι τῶν Ἑλλήνων οἱ μέν εἰσι Δωριεῖς, οἱ δὲ Αἰολεῖς, οἱ δὲ Ἴωνες, οἱ δὲ Ἀττικοί. συμβεβηκυίας δὲ διὰ τούτων δηλοῦμεν ποιότητας, καὶ γὰρ καὶ οὗτοι τρόποις καὶ ἤθεσι διαφέρουσιν ἀλλήλων· δοκεῖ γὰρ τὸ Δώριον ἀνδρωδέστερόν τε εἶναι τοῖς βίοις, καὶ μεγαλοπρεπὲς τοῖς φθόγγοις τῶν ὀνομάτων καὶ τῷ τῆς φωνῆς τόνῳ, τὸ δὲ Ἰωνικὸν ἐν πᾶσι τούτοις ἀνειμένον – χαῦνοι γὰρ οἱ Ἴωνες – τὸ δὲ Ἀττικὸν εἴς τε δίαιταν καὶ φωνῆς ἐπιτέχνησιν ἀεὶ διαφέρειν, τὸ δὲ Αἰολικὸν τῷ τ’ αὐστηρῷ τῆς διαίτης καὶ τῷ τῆς φωνῆς ἀρχαιοτρόπῳ”.}
\end{quote}

Such general accounts are as a rule absent from ancient and Byzantine treatises on the dialects. During the early modern period, however, dialect evaluations were frequently included in handbooks for the Greek dialects as a piece of standard information, especially from the seventeenth century onward. This is in keeping with a more general development in early modern discourse on stereotypes of ethnic groups, as, Joep \citet[17]{Leerssen2007} argues,

\begin{quote}
the cultural criticism of early-modern Europe […] began, in the tradition of Julius Caesar Scaliger (1484–1558), to sort European cultural and societal patterns into national categories, thereby formalizing an older, informal tradition of attributing essential characteristics to certain national or ethnic groups.
\end{quote}

In early modern formalized discussions of the Greek dialects and their properties, many of the same qualities and vices recurred, thus encouraging the canonization of a number of properties. Numerous instances of this tendency could be cited, but let me limit myself here to listing three representative examples from different centuries, which all have a clear link with philology:

\begin{quote}
Attic is the most elegant and copious of all and the cherisher of eloquence, which most of the noblest writers employed.
\end{quote}

\begin{quote}
Related to this is Ionic, which the oldest authors used, Democritus, Hippocrates, Herodotus; Homer also for a large part and Hesiod.
\end{quote}

\begin{quote}
Doric is the most pleasant and the most adequate for smoother poets, which the choruses of tragedians have also received so as to moderate the bitterness of the subject. This dialect was used by the Pythagoreans, Pindar, Epicharmus, Sophron, and Theocritus.
\end{quote}

\begin{quote}
Similar to this is Aeolic, adequate for lyric poems, which Alcaeus, Sappho, and many others expressed in their writings, of whom fortune has left nothing at all, except for those passages that are cited by others.\footnote{\citet[a.3\textsc{\textsuperscript{v}}–a.4\textsc{\textsuperscript{r}}]{Canini1555}: “Attica omnium elegantissima et copiosissima eloquentiaeque altrix, quam plurimi nobilissimi scriptores celebrarunt. Huic affinis Ionica, quam uetustissimi auctores usurparunt, Democritus, Hippocrates, Herodotus; Homerus etiam magna ex parte atque Hesiodus. Dorica suauissima est et poetis mollioribus accommodatissima, quam etiam tragicorum chori ad temperandam argumenti acerbitatem receperunt. Ea usi sunt Pythagorici, Pindarus, Epicharmus, Sophron et Theocritus. Huic similis Aeolica, melicis apta, quam scriptis expressere Alcaeus, Sappho aliique permulti, e quibus, praeter pauca quae ab aliis citantur, nihil omnino fortuna reliquum fecit”. For another sixteenth-century example, see \citet[138\textsc{\textsuperscript{v}}–139\textsc{\textsuperscript{r}}]{Vuidius1569}. Cf. also already \citet[12\textsc{\textsuperscript{v}}]{Lopad1536} and \citet[\textsc{a.6}\textsc{\textsuperscript{v}}\textsc{–a.7}\textsc{\textsuperscript{r}}]{Gessner1543}.}
\end{quote}

\begin{quote}
The first is Attic, which indeed must be preferred as the noblest above all others. It was mainly in this dialect that Thucydides, Demosthenes, Isocrates, and the majority of the historiographers wrote.
\end{quote}

\begin{quote}
The second is Ionic, which has a wonderful grace and charm, which mainly Herodotus, Hippocrates, and the poets, even Doric ones, used.
\end{quote}

\begin{quote}
The third is Doric, a little rougher and harder because of the pronunciation, as the Dorians are said “to pronounce broadly” (that is, to speak with a wide and open mouth). This dialect was employed by, among others, Theocritus and Pindar.
\end{quote}

\begin{quote}
The fourth, finally, is Aeolic, which no authors have followed avowedly, but the poets have interspersed it hither and thither in their writings, especially, however, Alcaeus, Sappho, what is more, Theocritus himself and Pindar (as it has many things in common with Doric), also Homer and therefore others.\footnote{\citet[2--3]{Merigon1621}: “Prima est Attica, quae quidem ut nobilior omnibus aliis praeponi debet; hac autem scripsere praecipue Thucydides, Demosthenes, Isocrates et maior pars historiographorum. Secunda, Ionica, quae mirificum habet leporem et uenustatem, qua usi sunt praecipue Herodotus, Hippocrates et poetae, etiam Dorici. Tertia Dorica, quae paulo asperior et durior propter pronuntiationem, quippe πλατυάζειν (hoc est lato et diducto ore loqui) dicuntur Dores; hanc autem dialectum celebrauit inter alios Theocritus et Pindarus. Quarta denique est Aeolica, quam nulli auctores ex professo sectati sunt, sed eam huc illuc in suis scriptis insperserunt poetae, praecipue uero Alcaeus, Sappho, immo Theocritus ipse et Pindarus (ut pote cum Dorica multa communia habentem) tum Homerus aliique ideo”. See \citet{Hoius1620} and \citet{Rhenius1626} for other seventeenth-century examples.}
\end{quote}

\begin{quote}
In this way, it happened that neither the Ionic nor the Doric nor any other dialect was similar to the Attic dialect, but that Attic surpassed all these dialects, as it is not too delicate, like the Ionic, nor too hard, like the Doric, nor too rude, like the Aeolic, but moderate, manly, weighty, and most shining of all.\footnote{\citet[6]{Georgi1729}: “[…] quo contigit, ut Atticae dialecto neque Ionica neque Dorica neque alia quaedam, similis fuerit, sed eas omnes superaret, cum neque nimis mollis sit, ut Ionica, neque nimis dura, ut Dorica, neque nimis rudis, ut Aeolica, sed temperata, uirilis, grauis atque omnium nitidissima”. Cf. also \citet[197--199]{Ries1786}.}
\end{quote}

The above three accounts also exhibit differences. In the first case, the emphasis is on the link between a dialect and its literary usage by different authors and in distinct genres. The primacy of Attic is also suggested, but this stands out much more clearly in the second account, which seems to construe a kind of evaluative ranking of the dialects: Attic first, Ionic second, Doric third, and Aeolic fourth. In the third passage, the superiority of Attic is likewise maintained, but it seems that the other three dialects were believed to be on more or less the same level.

Evaluative attitudes toward the ancient Greek dialects were, for the greater part, the product of post factum projections of virtues and vices on these literary varieties. Indeed, in the early modern period and even in antiquity, attitudes were usually based on an esthetic sensation during the act of reading. The ancient Roman rhetorician Quintilian experienced the fluency and pleasantness of Ionic in this fashion, as he added to his judgment the following reservation: “at least as I perceive it”.\footnote{Quintilian, \textit{Institutio oratoria} 9.14.18: “ut ego quidem sentio”.} To demonstrate the different impressions distinct Greek dialects conveyed, the French Hellenist Henri Estienne even transposed a Doric verse of the Hellenistic poet Callimachus (4th/3rd cent. \textsc{bc}) into Ionic as follows:

\begin{quote}
Original Doric: \textit{Tòn dè kholōsaménā per hómōs proséphēsen Athā́nā} [Tὸν δὲ χoλωσαμένα περ ὅμως πρoσέφασεν Ἀθάνα].
\end{quote}

\begin{quote}
Ionicized version: \textit{Tòn dè kholōsaménē per hómōs proséphēsen Athḗnē} [Tὸν δὲ χoλωσαμένη περ ὅμως πρoσέφησεν Ἀθήνη].\footnote{\citet[15--16]{Estienne1581}, with reference to Callimachus, \textit{In lauacrum Palladis} 5.79.}
\end{quote}

\begin{quote}
English translation according to the Loeb series: “And Athena was angered, yet said to him”.
\end{quote}

The Doric verse allegedly became, when transposed to Ionic, feeble and inadequate and lost its seriousness and majesty, and this solely through the replacement of the letter alpha by eta. An evaluative label could also result from a conscious critical review of the style in which a literary work was composed. The Doric texts known to the Neoplatonist philosopher Porphyry (ca. 234–305/310) seemed to be written in an obscure style, which is why in his biography of Pythagoras he labeled the dialect itself obscure (\textit{Vita Pythagorae} 53). In other words, it was not direct, oral contact with a dialect that triggered evaluative attitudes, but indirect confrontation through reading, either as an immediate sensation or as the result of a conscious assessment of the style of a text. This distinguishes premodern attitudes toward the ancient Greek dialects from those toward vernacular languages and dialects, which were usually at least partly informed by direct exposure to the variety in its spoken form.

Apart from encounters with literary texts, it was the link that scholars frequently made between the customs of a tribe and its language – \textit{lingua et mores} in Latin – which led them to conjure up evaluative labels for Greek dialects.\footnote{\citet{VanHal2013} offers a preliminary historical survey of the \textit{lingua et mores} link, while pointing out that it deserves further study.} Indeed, many attitudes were motivated by stereotypes about the four canonical Greek tribes, as I have shown throughout this section.\footnote{See e.g. the \textit{Scholia Vaticana} quotation above as well as the ideas of Saumaise and Vossius.} Early modern scholars took ancient and Byzantine attitudes as their starting point and complemented them in various ways. This materialized not only in the form of new evaluative statements and an increased emphasis on certain properties, especially Attic elegance and Doric rusticity, but – most notably – it also resulted in a tendency toward canonizing dialect attitudes. Even though there remained some variation in the early modern perception of the Greek dialects, it is nonetheless safe to state that the evaluation of the four traditional dialects became a canonized format. Indeed, it constituted an almost inherent part of the study of the Ancient Greek language and its literary dialects and was for this reason integrated into many Greek language manuals. Since scholars usually felt the Koine to be of a particular nature, they did not assign specific properties to it, either in antiquity and the Byzantine era or in the early modern period.

\section{Evaluative discourse between Greek and the vernacular}\label{sec:7.3}


The evaluative discourse on the Greek dialects must have been widely known in learned circles, as it apparently influenced attitudes toward vernacular speech forms to some extent. The terminology used to label vernacular tongues and their dialects sometimes resembled that found in evaluations of the Greek dialects. This emerges most clearly from cases in which scholars assigned labels to both Greek and vernacular speech forms in their works. Let me look at two noteworthy examples from the seventeenth and the eighteenth century, respectively: Isaac Vossius and Friedrich Gedike.

In his widely read treatise on ancient poetry and its original rhythm, published in Oxford in 1673, the Dutch philologist Isaac \citet[54--55]{Vossius1673} opposed the effeminate Ionic dialect to virile Doric, for which he may have relied on a Byzantine commentary on Thucydides.\footnote{Cf. \textit{Scholia in Thucydidem} (\textit{Scholia uetera et recentiora}), commentary at 1.124.1.} Directly after that, he provided a brief outline of the qualities of a number of vernacular tongues of his time. Especially relevant to my purposes is his characterization of English, with which he was very well acquainted, having moved to England in 1670. Vossius described the language as “delicate” (\textit{mollis}) and “effeminate” (\textit{muliebris}). To exemplify this linguistically, he referred to the English preference for the letter \textit{êta} (“ἦτα”) and its avoidance of the letter ⟨a⟩. Vossius’s views on the Ionic dialect, cited earlier in this chapter, irrefutably informed his assessment of English (see \sectref{sec:7.2} above). Ionic was also known for having the letter eta where the other dialects had a long alpha, and Vossius spoke of the Greek letter eta rather than the English letter ⟨e⟩ in characterizing this supposed property of English. He did add, however, that English “delicacy” (\textit{mollities}) was somewhat tempered by the harshness of its syllables and the frequency of consonants in this language (\citet[56]{Vossius1673}. After that, Vossius praised French for its strength and its many war-related words, which is reminiscent of his description of the Doric dialect.

Friedrich \citet[\textsc{xx}]{Gedike1779}, a German scholar from the late Enlightenment, drew a detailed comparison between the Greek and German dialect contexts in his \textit{Thoughts on purism and language enrichment}. Gedike modeled his threefold classification of Greek on his perception of vernacular German diversity, thus proceeding in a direction opposite to Vossius, who had moved from Greek to the vernacular. First, Gedike compared Ionic with Low German (\textit{Niederdeutsch} or \textit{Plattdeutsch}), both of which he described as being “smooth” (\textit{sanft}) and “delicate” (\textit{weich}). He associated this characteristic with the absence of aspirations and rough diphthongs, and emphasized the obviousness of the parallel he was pointing out. He proceeded by treating the similarity of Doric and Upper German (\textit{Oberdeutsch}), which was situated in the broadness with which they were pronounced. They moreover contained, Gedike argued, many hissing sounds, aspirations, and diphthongs. This gave them a “solemn” (\textit{feierlich}) and “splendid” (\textit{prunkvoll}) air. Gedike thus assessed Doric in distinctly positive terms. Finally, the “middle dialects” were discussed: Attic and High German (\textit{Hochdeutsch}). They were, however, not exactly in the middle, because both inclined toward the respective “solemn” varieties: Doric and Upper German. Gedike refrained from elaborating more extensively on the properties of Attic and High German in his 1779 work. However, three years later, in an article on the Greek dialects, he stated that Attic was less rough than Doric and less fluid, yet more consistent than Ionic. Something similar held true for High German, he suggested \citep[25]{Gedike1782}. \citet[\textsc{xx–xxi}]{Gedike1779} rounded off his comparison by stating that, just like the ancient Greek dialects, the three German dialects also used to be “book languages” (\textit{Büchersprachen}), until the High German speech of the Lutheran Reformation expelled the two others from writing. Gedike’s comparison of Greek and German dialects was applauded by several of his contemporaries, including the famed grammarian of German Johann Christoph Adelung (1732–1806; see \citealt{Adelung1781}: 56 and also \citealt{Moritz1781}: 20).

\section{Beyond the early modern era}\label{sec:7.4}


The evaluative discourse on the Greek dialects did not end with the arrival of modernity. On the contrary, it persisted until very late. In the nineteenth century, the distinguished German philologist Heymann Steinthal (1823–1899) noted the following on the Greek dialects in general and Attic in particular:

\begin{quote}
Each dialect counts as a phase in time and an interior moment of the spirit. In the Attic dialect, the Greek spirit manifested itself last, but also most perfectly, and, to be sure, in such an encompassing manner that one may rightly say that the other dialects have been neutralized in it. This is also why all Greek dialects have perished in and with it.\footnote{\citet[9]{Steinthal1891}: “Jeder Dialekt gilt als ein Abschnitt in der Zeit und ein inneres Moment des Geistes. Im attischen Dialekt offenbarte sich der griechische Geist am spätesten, aber auch am vollkommensten, und zwar in so umfassender Weise, dass man wol sagen darf, in ihm seien die andren Dialekte aufgehoben gewesen. Darum sind auch in und mit ihm alle griechischen Dialekte zu Grunde gegangen”.}
\end{quote}

Steinthal’s underlying assumptions were, however, different from those of early modern evaluative discourse. He presumed the existence of a Greek \textit{Volksgeist}, which has to be viewed against the background of his interest in the psychology of tribes and nations (\textit{Völkerpsychologie}), and he supposed that some Greek tribes represented that \textit{Geist} better than others. Still, it is telling that, as with his early modern predecessors, evaluating the Greek dialects came naturally to him. Today, the idea of Attic elegance and primacy is still latent in the sense that it is taught as the principal variety of Ancient Greek in most high school and university curricula. This is largely a modern innovation, as early modern grammars tended to describe “the Greek language”, usually a form of the Koine with typically Attic and Ionic elements interspersed, as Federica \citet[123]{Ciccolella2008} has rightly suggested. Be that as it may, literary Attic was generally valued most highly even by early modern Hellenists (cf. \citealt{Roelcke2014}: 251). In other words, the shift from the early modern to the modern period coincided with a shift in the prototypical form of Greek: from a hybrid form of Koine Greek to Attic Greek.\footnote{Differences in the prototypicalization of Greek throughout history require further study (\citealt{VanRooyFcb}).}

Early modern scholars approached and evaluated the Greek dialects principally against the backdrop of reading and understanding Greek literature, even though stereotypes about the traditional four Greek tribes likewise constituted an important trigger for dialect attitudes. The authors sometimes also assumed a connection between the dialects and certain other aspects of ancient Greece, albeit in a much looser way than with Greek literature and the Greek tribes. What are these other aspects?

\section{Geography, politics, and natural disposition}\label{sec:7.5}


First of all, in keeping with the idea, widespread in early modern times, that geography was responsible for dialectal diversification, the terrain of Greece was frequently appealed to in order to account for the existence of Greek dialects.\footnote{On the link between geography and dialectal diversity, see \citet[]{VanRooyFcd}.} The Protestant theologian and renowned Hellenist Philipp \citeauthor{Melanchthon1518} (\citeyear[a.1\textsc{\textsuperscript{v}}]{Melanchthon1518}; \citeyear[\textsc{a.}i\textsc{\textsuperscript{v}}]{Melanchthon1520}) described in his grammar of the language Greece as “spacious” (\textit{ampla}) and “wide” (\textit{lata}), while presenting dialectal diversity as a self-evident consequence of this aspect of Greek geography (cf. also \citealt{Ruland1556}: 1). The dialects were linked to the many islands of ancient Greece in particular, most notably by the Anglo-Welsh writer James Howell (ca. 1594–1666). Inspired by the prominent philologist Josephus Justus Scaliger, \citet[89]{Howell1650b} emphasized that “the cause why from the beginning ther wer so many differing dialects in the \textit{Greek} tongue was because it was slic’d into so many islands” (cf. \citealt{Howell1642}: 138–139; \citealt{Scaliger1610}: 121). Howell’s treatment of Greek diversity was actually triggered by a comment on Italian dialects, which he subsequently compared to their Greek counterparts. He claimed that, in the case of Italian, dialectal variation was caused by “multiplicity” or “diversity of governments” rather than geography. This brings me to a second major link made by early modern scholars, that between the dialects and the political diversity of ancient Greece, which, in turn, was often viewed as a consequence of the rugged geography of the area. The humanist Lorenzo Valla’s famous praise of the Latin language cannot be left unmentioned in this context:

\begin{quote}
Just as the Roman law is one law for many peoples, so is the Latin language one for many. The language of Greece, a single country, is shamefully not single, but as various as there are factions in the state.\footnote{Valla in \citet[122]{Regoliosi1993}: “multarum gentium, uelut una lex, una est lingua Romana: unius Graeciae, quod pudendum est, non una sed multae sunt, tamquam in republica factiones”. The translation is adopted from \citet[10]{Trapp1990}. On this passage, see e.g. Tavoni in \citet[90 n.55]{Benvoglienti1975} and \citet[212--213]{Trovato1984}.}
\end{quote}

The polyhistor Daniel Georg Morhof (1639–1691) made a similar point, emphasizing the inability of Athens to impose its dialect on neighboring city-states \citep[146]{Morhof1685}. The German classical scholar Johann Matthias Gesner (1691–1761) similarly suggested that Greek dialectal diversity was caused by the fact that “ancient Greece did not have a capital and dominant city, but several cities had the same and equal rights”.\footnote{\citet[160--161]{Gesner1774}: “Origo autem dialectorum uariarum haec est; quia Graecia antiqua non habuit caput et dominam urbem, sed plures urbes eadem habebant et paria iura”. Cf. \citet[395--396]{Rollin1731}; \citet[136--138]{Priestley1762}; \citet[204]{Ries1786}.} The poet Pierre de Ronsard (1524–1585), for his part, contrasted Greek diversity to his native French context, connecting at the same time Greek linguistic abundance to the fragmented political landscape of ancient Greece (\citeyear{Ronsard1565}: 5\textsc{\textsuperscript{r}}). Ronsard interestingly added that if there still were political diversity in France, each ruler would desire, for reasons of honor, that their subjects wrote in the language of their native country.\footnote{Cf. \citet[lxviii]{Court1778} for a similar observation. See also Chapter 8, \sectref{sec:8.1.3}.} An odd characterization of ancient Greece was proposed by the Bohemian Protestant scholar Christoph(orus) Crinesius (1584–1629). Operating within a biblical framework and deriving Greek from Hebrew, \citet[77]{Crinesius1629} held that the Greek dialects were the varieties spoken in the different provinces of the kingdom of Javan, a grandson of Noah and traditionally associated with the Ionians. In other words, he incorrectly claimed that the linguistic variation of ancient Greece coincided with the regional-administrative division of a politically unitary empire. Apart from political diversity, the dialects were often also connected to the many colonies established by the Greeks (see e.g. \citealt{Simonis1752}: 207). It is worthwhile recalling here that certain early eighteenth-century scholars believed the geopolitical diversity of early modern Greece to correlate with vernacular Greek dialectal variation as well (see Chapter 2, \sectref{sec:2.10}). In other words, ancient and vernacular dialects of Greek were thought to have emerged under similar circumstances.

Certain scholars associated the dialects with the Greeks’ natural disposition and innate character. This connection was, however, much rarer. In this case, the Greek dialects were taken as a symptom of a negative characteristic of the Greek people as a whole: their inconstancy. This emerges most clearly from the words of Franciscus Junius the Elder, quoted at the outset of this chapter and labeling the Greeks as “verbose” and “mendacious” because of a certain “malicious rivalry” that led them to forge so many different dialects. This view was silently copied by the Dutch biblical scholar Johannes Leusden (1624–1699).\footnote{\citet[a.4\textsc{\textsuperscript{r}}\textsc{–}a.4\textsc{\textsuperscript{v}}, 167]{Leusden1656}. Schultens (in \citet[§\textsc{xlix.}δ]{Eskhult_albert_nodate}) quoted Leusden, without realizing that Leusden relied on Junius. For the rivalry among speakers of different dialects, cf. also \citet[5\textsc{\textsuperscript{r}}]{Baile1588}; \citet[\textsc{b.3}\textsc{\textsuperscript{r}}]{Schorling1678}.} It was moreover implicit in Lorenzo Valla’s ridiculing of Greek multiplicity as opposed to Roman uniformity, quoted earlier in this section.

\section{Reconstructing ancient Greece: Antiquarians on the dialects}\label{sec:7.6}


As the previous section has shown, Renaissance Hellenists realized that the phenomenon of Greek linguistic diversity was not only relevant for the study of language and literature, but could also help a scholar shed light on other aspects of ancient Greece, especially the character of its tribes and its geopolitical constitution. This realization motivated many authors to devote attention to the Greek dialects outside of philological contexts in the strict sense, especially in the not always clearly distinguished fields of historiography, antiquarianism, and geography. How did scholars active in these branches fit dialectal diversity into their descriptions and reconstructions of ancient Greece and its regions and colonies? Let me provide a brief and necessarily eclectic answer to this question, which deserves further study.

In 1589, the obscure Taranto philologist and antiquarian Giovanni Giovane (Latinized: Johannes Juvenis) published his \textit{Eight books on the antiquity and changing fortune of the people of Taranto}. One of the first sections of this historiographical-antiquarian monograph comprised a short lexicon of the ancient Greek dialect spoken in the city of Taranto or \textit{Táras} (Tάρας), its original Greek name, situated in modern-day southern Italy (\citealt{Giovane1589}: 9–18). Giovane was, however, aware that not all words he included were specific to Taranto. Yet he still presented Tarentine as a distinct Greek dialect and recognized it as a variety of Doric. \citet[8--9]{Giovane1589} did so on the authority of Aristotle as well as by pointing out the Doric character of the extant fragments attributed to the Pythagorean philosopher Archytas of Tarentum (5th/4th cent. \textsc{bc}). What is more, Giovane believed it to be common knowledge that grammarians have reckoned Tarentine Greek among the countless dialects of the language (cf. Chapter 2, \sectref{sec:2.8}).

The Dutch antiquarian Johannes Meursius (1579–1639) inserted information on specific Greek dialects in a fashion similar to Giovane in two posthumously published works: firstly, a book on ancient Laconia, in which the Doric character and particularities of its dialect were outlined (\citealt{Meursius1661}: 216–233), and secondly, a treatise on ancient Crete and other Greek islands, in which the Doric Cretan dialect was described and Cretan words were listed (\citealt{Meursius1675}: 254–258). Apart from such antiquarian writings, the Greek dialects attracted attention in more general works on the history of ancient Greece and neighboring areas, especially in the eighteenth century, for instance in Charles Rollin’s (1661–1741) popular multivolume account of ancient history and in Nicolas Fréret’s (1688–1749) dissertation on the first inhabitants of ancient Greece. \citet[395--396]{Rollin1731} linked the dialects to the enormous geopolitical diversity of ancient Greece, whereas \citet[esp. 107--129]{Freret1809} framed Greek within a larger family of dialects anciently spoken over an area stretching from Celtic lands to those of the Syrians and Medes; against this background, he described the development of Greek and its dialects out of a now lost protolanguage.

Clearly written out of historiographical interest was the \textit{Brief dissertation on the settlements and colonies of the dialects of the Greek language} (\citeyear{Hoius1620}) of the Bruges humanist Andreas Hoius (1551–1635). A professor of Greek and history at the university of Douai, today in northern France, \citet[95]{Hoius1620} principally attempted to trace the history of the Greek tribes and their migrations, which he held responsible for the variation in the Greek tongue, as well as to map out the geography of Greece. The dialects themselves hovered in the background of this dissertation, and Hoius mentioned only some of their linguistic particularities explicitly. One of his main theses was that all the Greek dialects were originally spoken in Greece in the strict sense, from which he excluded Asia Minor, part of modern-day Turkey.\textsuperscript{} What is more, there were initially only two tribes in Greece, which Hoius asserted on the authority of Herodotus: the migratory “Pelasgians”, equated with the Aeolians, from whom the Romans derived, and the stationary “Hellenists” (\citealt{Hoius1620}: 102, referring to Herodotus 1.57–58).

The history of the Greek tribes and the historical status of the dialects also served as the principal focus of a dissertation defended by the Hellenist Georg Friedrich Thryllitsch in 1709 at the university of Wittenberg.\footnote{Cf. the dissertation presented (likewise at Wittenberg) by Georg Caspar Kirchmaier and Johannes Crusius (= \citealt{KirchmaierCrusius1684}), even though here the history of the Greek alphabet (chapters \textsc{i}{}-\textsc{iii}), the correct pronunciation of Greek (chapter \textsc{iv}), and the particularities of the Greek dialects (most of chapter \textsc{v}) were the main focus of attention.} Its title neatly summed up Thryllitsch’s goal, which consisted in presenting “some historical-technical suggestions about the Greek dialects collected on the basis of a consideration of the origins and migrations of the Greek tribes”.\footnote{“Suspiciones quasdam historico-technicas de dialectis Graecis ex consideratione originum migrationumque Graecarum nationum collectas”.} One of the main aims of this dissertation consisted in reconciling the biblical account with that of Greek historiographers, for which the traditional association of Javan with Ion – making Ionic the oldest Greek dialect – was invoked (\citealt{Thryllitsch1709}: \textsc{a.4}\textsc{\textsuperscript{r}}\textsc{–b.3}\textsc{\textsuperscript{r}}; cf. Chapter 5, \sectref{sec:5.4}). The French etymologist Gilles Ménage (1613–1692) apparently planned to compose seven books on the ancient Greek dialects, as \citet[252]{Leibniz1991} informs us; this might have been the culmination of the early modern historiographical interest in the dialects, since Ménage’s work was not only intended to include – like Hoius’s and Thryllitsch’s accounts – information on Greek geography, tribes, and colonies, but also an extensive description of the linguistic particularities of the dialects. It was, unfortunately, never realized.

Historiographers and especially antiquarians sometimes put their philological knowledge of the Greek dialects into practice when analyzing the language of Greek inscriptions discovered in the Mediterranean area. This application was, however, relatively rare, probably because the discipline of epigraphy was only nascent – the first collections containing Greek inscriptions were published in the late sixteenth century – and the Greek dialects remained predominantly tied up with the study of literary texts.\footnote{On Greek inscriptions in the early modern period, see \citet{Stenhouse_greekness_nodate}. \citet{Stenhouse2005} discusses the occasional usage of Greek inscriptions by sixteenth-century Italian historiographers. \citet{Liddel2014} briefly elaborates on the usage of the so-called Parian Marble in early modern chronology. A more comprehensive study of the early interest in Greek epigraphy remains a desideratum.} Yet the antiquarian editors of Greek inscriptions did their best to identify the dialect of the pieces they were publishing, with varying success. Thomas Lydiat relied on his knowledge of the Greek dialects and of Greek history to identify the language of the Parian Chronicle as a mixed Koine–Ionic variety; scholars now agree, however, that it is composed in Attic, even if on the island of Paros, where the chronicle was found, a variety of central Ionic was originally spoken.\footnote{See Lydiat in \citet[\textsc{ii}.116–117]{Prideaux1676}. On the dialect of Paros, see e.g. \citet[531]{Alonso2018}.} Lydiat’s observation featured in the epigraphic collection edited by Humphrey Prideaux (1648–1724) in 1676 and centered around the so-called Arundel marbles. These were named after the eager art and antiquities collector Thomas Howard (1586–1646), Earl of Arundel, who had acquired the marble sculptures and inscriptions through his contacts in the Ottoman empire, thus laying the foundation of the first major collection of Greek inscriptions in England, now principally preserved at the Ashmolean Museum in Oxford (on the eventful history of the marbles, see \citealt{Vickers2006}). \citet[\textsc{i.}a.1\textsc{\textsuperscript{v}}, 123]{Prideaux1676} himself drew attention to an inscription in the collection regarding a treaty between two Cretan cities because of its unusual dialect. Even though he cautiously pointed out some Doric features in his notes to this inscription, he did not feel confident enough to identify its language as Doric. In summary, Lydiat and Prideaux activated their philological knowledge of the Greek dialects for antiquarian-epigraphic purposes, but not always successfully so.

In the seventeenth century, inscriptional evidence was occasionally also invoked by scholars tackling typical philological questions such as the variety and history of the Greek language and the literary usage of the dialects. Claude de \citet[430]{Saumaise1643a} saw the Doric character of Cretan Greek confirmed by epigraphic data, whereas Richard \citet[311]{Bentley1699} combined his knowledge of the Greek dialects and inscriptional evidence to correctly identify the dialect of Sicily as Doric. Bentley did so in his well-known dismissal of the authenticity of a collection of letters written in Attic and attributed to Phalaris, the tyrant of Akragas on Sicily (modern-day Agrigento) in the sixth century \textsc{bc}. How could a Sicilian tyrant ever have written letters in Attic, especially considering that this dialect had not yet eclipsed all the others in Phalaris’s lifetime? If the letters were indeed authored by Phalaris, Bentley convincingly pointed out, they would have been written in a variety of Doric.

The eighteenth century witnessed an increasing interest in Greek inscriptions, especially among antiquarians who had enjoyed a decent philological education. Hellenists finally started to consider inscriptions to be a valuable source of dialectal data (cf. \citealt{Walch1772}: 87). This growing fascination with epigraphical documents also resulted in lengthier discussions of the dialectal identity of specific inscriptions or collections of inscriptions. Let me take a look here at two notable Italian examples. The priest and early archeologist Alessio Simmaco Mazzocchi (1684–1771) was the first to edit in their entirety the so-called Heraclean Tablets, two bronze plates discovered separately in 1732 and 1735 near the ancient city of Heraclea Lucania in the southernmost part of modern-day Italy and currently preserved in the archeological museum of Naples. One side of the tablets contains a Latin legal inscription from the first century \textsc{bc}, which Michael Maittaire had already published in 1735; the other has two Greek inscriptions from the late fourth or early third century \textsc{bc}.\footnote{See \citet{Uguzzoni1968} for a modern edition and discussion of the Heraclean Tablets. See also \citet{Weiss2016}, who argues that the dating of the tablets should be reconsidered.} Mazzocchi included an extensive commentary on the tablets in his edition, which appeared in 1754 at the Naples printing press of Benedetto Gessari, and which also touched on linguistic aspects of the inscriptions. Thanks to his excellent philological education, he was able to correctly identify the dialect of the Greek inscriptions as Doric, which he believed to be the oldest variety of Greek. However, misguided by the obscure ancient and medieval accounts on the Greek dialects as well as by the odd-looking alphabet of the inscriptions, \citet[118--120]{Mazzocchi1754} further specified the language as “Old Doric” as opposed to the “New Doric” dialect. This New Doric was allegedly introduced by Sicilian poets such as Epicharmus and Sophron in the fifth century \textsc{bc}. Mazzocchi contended, however, that New Doric did not spread to all regions at the same time, and some regions, like Magna Graecia in Italy, preserved Old Doric for a longer period. This complex argument allowed Mazzocchi to situate the two Greek inscriptions in approximately the correct time frame – i.e. around 250 \textsc{bc} – as well as to account for its unusual orthography. He even proposed a relative chronology for the two inscriptions, based on orthographic and linguistic data \citep[135]{Mazzocchi1754}. In conclusion, Mazzocchi’s philological schooling enabled him to formulate a detailed and well-founded assessment of the language of the Heraclean Tablets, even if his results were still firmly grounded in traditional ideas on the Greek dialects and his views have been surpassed by modern scholarship (see \citealt{Weiss2016} for a state of the art).

My second example is the Sicilian antiquarian and numismatist Gabriele Lancillotto Castelli (1727–1794), who relied on established dialectal features to prove that not only Doric, but also Attic and Ionic were spoken on his native island, contrary to what was commonly believed (\citeyear{Castelli1769}: \textsc{xv}). The language of inscriptions of various types, including coins, constituted one of Castelli’s principal pieces of evidence for his hypothesis (\citealt{Castelli1769}: \textsc{xv–xvi,} \textsc{xxi}). At the same time, however, he also made ample use of ancient authorities to substantiate his views. For example, inspired by the historian Thucydides, he claimed that a kind of intermediate Doric–Chalcidian Ionic variety was in use among the inhabitants of the Sicilian city of Himera (see \citealt{Castelli1769}: \textsc{xxxiii} for a neat overview of his theses). He still wavered, in other words, between evidence and authority as he was exploring the new inscriptional data available to him.

Moving beyond historiographical and antiquarian works focusing on ancient Greece, I cannot leave unmentioned here that the Greek dialects of antiquity were often the only ones discussed at some length in geographical descriptions of Europe or the world, especially before the eighteenth century. The English churchman Peter  (1599–1662) referred to them in his long description of Greece, included in his \textit{Microcosmus, or A little description of the great world} of \citeyear{Heylyn1621}: “The language they spake was the \textit{Greeke}, of which were five dialects, \textit{1 Atticke. 2 Doricke. 3 Aeolicke. 4 Beoticke. 5} The \textit{common} dialect or phrase of speech” (\citealt[205]{Heylyn1621}; see e.g. also \citealt[15, 60, 63]{Speed1676}). He claimed to be relying on Nicolaus Clenardus’s grammar of Greek, but Heylyn’s classification into Attic, Doric, Aeolic, Boeotian, and Koine does not feature in Clenardus’s work and has no parallels in the early modern period. In the revised edition of 1625, \citet[375]{Heylyn1625} replaced Boeotian by Ionic, most likely because he had realized his idiosyncrasy.

In summary, the dialects occupied an important place in a number of early modern historiographical and antiquarian works concentrating on parts of ancient Greece, and they were often discussed in close conjunction with the history, geography, and tribes of Greece. In the eighteenth century, antiquarians increasingly involved epigraphic dialect evidence in their attempts at providing encompassing descriptions of ancient Greece and its many different settlements, especially those in regions of modern-day Italy. The Greek dialect inscriptions from these areas were, after all, better accessible to Western scholars than the ones hidden away in Ottoman Greece. The dialects, finally, also figured in comprehensive geographical works covering more than Greece alone, albeit more marginally so. These accounts tended to be rather unoriginal in their information regarding the dialects, as Heylyn’s case demonstrates.

\section{Conclusion}\label{sec:7.7}


Before the modern period, scholars eagerly applied evaluative labels to the canonical ancient Greek dialects. Most of these attitudes must be understood against the background of the study of Greek literature and resulted in particular from the perceptions readers had of texts and their form. This holds for ancient and medieval times as well as for the early modern period, even though early modern philologists also relied to a considerable extent on the attitudes of their predecessors. Scholars linked the Greek dialects with other aspects of ancient Greece and Greek culture as well, and increasingly so from the Renaissance onward. Assumptions about the customs of individual Greek tribes triggered specific attitudes toward their respective dialects, and, on a more general level, the fickleness of the Greek people in its entirety was believed to have caused the vast dialectal diversity of its language. Put another way, early modern stereotypes about Greeks in general and the tribes of ancient Greece in particular played a pivotal role in evaluating Greek linguistic diversity. In addition, the authors perceived a close connection between the Greek dialects and the ethnic and geopolitical constitution of Greece. To sum up, early modern scholars attempted to fit the dialects into the larger picture of ancient Greece. Even though they principally had a philologically colored view of the matter, they frequently related the dialects to other, non-textual aspects of Greek culture and Greekness.

