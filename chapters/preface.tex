\addchap{\lsPrefaceTitle}\label{ch:preface}

In his twenty books on education, the renowned Spanish philologist and humanist pedagogue Juan Luis Vives (1492/1493–1540) warned students of the Ancient Greek language of its great difficulty and diversity:

\begin{quote}
In the Greek language, there are great labyrinths and enormously vast recesses, not only in the various dialects, but in every one of them. The Attic dialect and the common one, which is very close to Attic, are especially necessary, because they are also the most eloquent and cultivated. And whatever the Greeks have that is worthy of reading and knowing is recorded in these dialects. The remaining dialects are used by the authors of poems, but it is less important to understand these.\footnote{\citet[e3\textsuperscript{v}]{Vives1531}: “In Graeca magni sunt labyrinthi et uastissimi recessus, non solum in dialectis uariis, sed in unaquaque illarum. Attica et Atticae proxima communis maxime sunt necessariae, propterea quod et sunt facundissimae atque excultissimae, et quicquid Graeci habent legi ac cognosci dignum istis dialectis est consignatum. Reliquis utuntur auctores carminum, quos non tanti est intelligi”.}
\end{quote}

As a kind of Ariadne, Vives endeavored to guide the reader of his book, his Theseus, through the vast labyrinth of the Greek tongue. In order to make sure that prospective Hellenists learned the language as efficiently as possible, he suggested that they should focus on the Attic dialect and on Koine Greek, both for intellectual and esthetic reasons. Dialects such as Doric and Aeolic, primarily poetical media, were deemed to be of lesser importance.

Vives left no doubt as to the immense diversity within the Greek language, which posed an enormous challenge not only to students but also to scholars in the early modern era. Fascinated with the heritage of ancient Greece, early modern intellectuals cultivated a deep interest in its language, the primary gateway to this long-lost culture, rediscovered by Westerners during the Renaissance. The humanist battle cry “Ad fontes!” – Latin for “To the sources!” – forced them to take a detailed look at the Greek source texts in the original language and its different dialects. In doing so, they saw themselves confronted with several major linguistic questions. Is there any order in this great diversity? Can the Greek dialects be classified into larger groups? Is there a hierarchy among the dialects? Which dialect is the oldest? Where should problematic varieties such as Homeric and Biblical Greek be placed? How are the differences between the Greek dialects to be described, charted, and explained? What is the connection between the diversity of the Greek tongue and the Greek homeland? And, last but not least, are Greek dialects similar to the dialects of the vernacular tongues? Why (not)? In the present book, I discuss and analyze the often surprising and sometimes contradictory early modern answers to these questions.
