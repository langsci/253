\chapter{The Greek dialects in confrontation}\label{chap:8}

“It is common knowledge that there are nowhere better-known and more distinct dialects than in the Greek language”.\footnote{\citet[23]{Wesley1736}: “In propatulo est quod nullibi notiores aut distinctiores sint dialecti quam in lingua Graeca”.} This is how the English clergyman and poet Samuel Wesley (1662–1735) introduced his concern that it was difficult to formulate rules of dialectal change. Wesley did so when discussing the style and language of the Old Testament Book of Job, which he regarded as a kind of Hebrew that had features of related dialects. By dialects he mainly meant Arabic and Syriac. He attempted to discover a certain regularity in Oriental variation and referred in this context to the ancient Greek dialects. In fact, Wesley assumed that the letter mutations among the Greek dialects could be transposed to the Oriental context without any problem. This implies a presupposition on Wesley’s part that both linguistic contexts were comparable, which also emerges from his explicit connecting of specific Greek dialects to individual Oriental tongues. Indeed, \citet[24]{Wesley1736} attributed similar linguistic properties to Doric Greek and Syriac, on the one hand, and to Attic Greek and Arabic, on the other.

Samuel Wesley was not alone in comparing ancient Greek dialectal diversity to other contexts of dialectal or dialect-like variation. Indeed, it was common early modern practice to assert that the Greek dialects were either comparable with, or clearly different from, diversity within other languages or language families, especially the Western European vernaculars, Latin, and the close-knit group of the so-called Oriental tongues, now known as the Semitic language family. What arguments did early modern scholars invoke when claiming comparability or lack thereof? And how do their views relate to the intellectual and linguistic context in which they operated? It is these two major questions I want to address in the final chapter of this book.

\section{The vernaculars of Western Europe and the Greek reflex}\label{sec:8.1}

It comes as no surprise that scholars from early modern Western Europe compared the ancient Greek dialects most frequently to their native vernaculars. The confrontation with Greek triggered a reflex among Western European scholars to relate Greek variation to the regional diversity which they encountered in their mother tongues. It is, however, remarkable that they did so in various ways and for various purposes. What were their most significant incentives to emphasize or dismiss the comparability of ancient Greek with vernacular dialects?

\subsection{Explanation: The Greek dialects in need of clarification}\label{sec:8.1.1}


When Greek studies started to develop on the Italian peninsula from the end of the Trecento onward, Renaissance Hellenists were initially compelled to focus primarily on one principal form of the language, consisting in the Koine interspersed with some occasional features typical of Attic and Ionic. Toward the end of the Quattrocento, however, Hellenists developed an ever-growing interest in the Greek dialects per se and their individual features (see also Chapter 1, \sectref{sec:1.2}). In this process, the dialects obtained a more clearly defined position in the teaching of the Greek language, being usually reserved for more advanced students, often in connection with the study of poetry and its dialectally diversified genres. Grammarians soon realized that if they wanted to efficiently explain the nature of the ancient Greek dialects to their students, they needed to appeal to a situation more familiar to their audience, in particular the regional diversity in their native vernacular tongue. As Greek studies boomed first in the states of northern Italy, it is not hard to see why vernacular dialects were first invoked by Italian grammarians to explain the existence of different forms of Ancient Greek. For instance, in his updated commentary on Guarino’s abridgement of Manuel Chrysoloras’s Greek grammar, published in Ferrara in 1509, the professor of Greek Ludovico da Ponte noted that there were five principal tongues among the Greeks: the Koine, Doric, Aeolic, Ionic, and Attic, the most pre-eminent among them. \Citet[20\textsc{\textsuperscript{v}}–21\textsc{\textsuperscript{r}}, 46\textsc{\textsuperscript{v}}–47\textsc{\textsuperscript{r}}]{Da1509} compared these dialects at two different occasions to the varieties of Italian spoken by the Venetians, the Bergamasques, the Florentines, etc. (on da Ponte, see also Chapter 2, \sectref{sec:2.6}). Originally from the city of Belluno in the Veneto region, he drew a comparison between his native Venetian and elegant Attic speech, even claiming that Venetian was “the most beautiful and learned speech of all, scented with the entire majesty of the Greek language”.\footnote{\Citet[47\textsc{\textsuperscript{r}}]{Da1509}: “pulcherrimus et doctissimus omnium sermo, in quo redolet tota linguae Graecae maiestas”.} Such explanatory comparisons, in this case with a distinctly patriotic touch, occurred very frequently from the early sixteenth century onward, usually in a didactic context.

The procedure was quickly picked up by grammarians outside of the Renaissance heartland of Italy. It happened particularly early in Philipp Melanchthon’s successful Greek grammar, first published in 1518, in which the Protestant Hellenist assumed the existence of a certain south-western High German common language in Bavaria and Swabia. Melanchthon might have been thinking of the southern German print language, one of the three regional print languages emerging after 1500 (see \citealt{Mattheier2003}: 216), or some other form of a regional koine. The reference to his native German context served to explain the status of the Greek Koine to his readership of prospective Hellenists (\citealt{Melanchthon1518}: a.i\textsc{\textsuperscript{v}}). The first Greek grammar composed by a Spanish scholar, Francisco de Vergara, adopted the same technique; a brief description of native regional varieties was offered to help the Spanish reader understand ancient Greek diversity (\citealt{Vergara1537}: 209–210). Revealing in this context is the 1561 edition of the Greek grammar composed by the German pedagogue Michael Neander (1525–1595), who silently copied the bulk of Vergara’s discussion of the Greek dialects. In doing so, however, \citet[340--343]{Neander1561} left out the reference to Spanish variation, as this would not have been very helpful to a reader with a German background.\footnote{The first edition of Neander’s work (i.e. \citealt{Neander1553}) did not yet contain the passage in question.}

The explanatory use of German dialects in Greek handbooks occurred extremely frequently.\footnote{See e.g. \citet[3--4]{Schmidt1604}; \citet[83]{Rhenius1626}; \citet[\textsc{b.4}\textsc{\textsuperscript{r}}]{Schorling1678}; \citet[\textsc{b.2}\textsc{\textsuperscript{v}}]{KirchmaierCrusius1684}; \citet[376]{Kober1701}; \citet[\textsc{c.2}\textsc{\textsuperscript{v}}]{Thryllitsch1709}; \citet[b.2\textsc{\textsuperscript{v}}\textsc{–}b.3\textsc{\textsuperscript{r}}]{Nibbe1725}; \citet[141]{Georgi1733}; \citet[13]{Schuster1737}; \citet[207--209]{Simonis1752}; \citet[191--192]{Peternader1776}; \citet[\textsc{xxvi}]{Harles1778}.} It is summed up neatly by the renowned Saxon lexicographer of Latin Immanuel Johann Gerhard Scheller (1735–1803), who, though not a grammarian of Greek, briefly discussed the Greek dialects in his reflections on the properties of the German \textit{Schriftsprache}. In this context, Scheller remarked:

\begin{quote}
I want to adduce only a few examples that demonstrate the similarity of the German and Greek dialects, so that in this manner a young person, if he knows it in German, will not be so astonished at it in Greek.\footnote{\citet[229]{Scheller1772}: “Ich will nur wenige Beyspiele anführen, die die Aehnlichkeit der deutschen und griechischen Dialecte beweisen: daß also ein junger Mensch, wenn er es im Deutschen wüste, im Griechischen nicht sich so verwundern würde”.}
\end{quote}

The intensive Greek–German comparison seems to be related to two main historical circumstances: the continuous early modern interest in the history, language, and literature of ancient Greece in German-speaking areas and the flourishing of regional dialects there, which from the end of the seventeenth century onward received monograph-length studies, with a focus on lexical particularities (see \citealt{Hasler2009}: 877). Clarifying the Greek dialects by referring to native vernacular diversity also occurred in grammars by native speakers of French and English, albeit much less frequently.\footnote{For French, see e.g. \citet[11--12]{Antesignanus1554}, on which see \citet{VanRooy2016c}. For English, see e.g. \citet[191--192]{Milner1734} and \citet[121]{Holmes1735}.} This might be related to the fact that in these politically unified areas grammarians more easily reached a consensus on the vernacular standard to be adopted. As a result, Hellenists in these regions might have sensed that French and English dialects, conceived as corrupt deviations from the revered standard, could not be so easily compared with the highly valued literary dialects of Ancient Greek.

Early modern Hellenists did not only fall back on their native context when Greek dialectal variation needed to be explained as a general phenomenon. It was also employed as a point of reference for clarifying the different evaluative attitudes toward the Greek dialects (cf. Chapter 7, \sectref{sec:7.3}). Notably, in his monograph on the Greek dialects, the German professor Otto Walper presented Attic and Ionic as more polished and smooth, whereas he claimed Aeolic and Doric to be less cultivated and not as pleasant to the ears. This, Walper explained, was not very different in “our German language”, which “some provinces speak more smoothly, elegantly, and neatly than others”.\footnote{\citet[61]{Walper1589}: “Vt autem superiores dialecti politiores et suauiores fuere; ita hae duae (Dorica et Aeolica) incultiores et auribus ingratiores existimantur, haud secus atque in lingua nostra Germanica prouinciae aliae aliis loquuntur suauius, concinnius atque politius”.} Also a specialist of Hebrew, Walper went on to suggest that Hebrew resembled Attic and Ionic, whereas Syriac and Aramaic had properties similar to Aeolic and Doric.

Hellenists addressing a more international audience referred to various vernacular contexts when explaining features of the sociolinguistic situation of ancient Greece. In his comprehensive Greek grammar, destined for Jesuit schools in various parts of Europe, the Jesuit Jakob \citet[20]{Gretser1593} referred to the German, Italian, and French “common languages”, the allegedly geographically neutral standard languages that were being developed, to explain the status of the Greek Koine to his student readers (for his intended audience, see \citealt{Gretser1593}: )(.4\textsc{\textsuperscript{r}}). The French Hellenist Petrus \citet[11--12]{Antesignanus1554}, one of Gretser’s main sources of methodological inspiration, also clarified the status of the Greek Koine using a more familiar situation, his native French context. Antesignanus’s case is revealing in that it shows that the explanation did not occur in an entirely unidirectional manner from the vernacular to the ancient Greek context. Instead, certain aspects of the French context seem to have been forced into the Greek straitjacket, as, for instance, the idea that the French common language could be adorned by features of certain approved French dialects. Not all grammarians of French would have agreed with this rather bold claim by Antesignanus. Something similar happened when the eighteenth-century Frisian Hellenist Tiberius Hemsterhuis (1685–1766) took the comparability of Greek and Dutch for granted, using his native context to clarify the status of Greek variation and the Greek common language. In order to explain what the Koine was, “I will use”, Hemsterhuis said, “the example of our fatherland”.\footnote{\citet[102]{Hemsterhuis2015}: “Mirabitur quis quae sit illa κoινή. Exemplo utar nostrae patriae, ut id possim explicare”.} This led him to boldly present both the Greek and the Dutch common languages as the standard speech of high society composed out of different dialects and not bound to a specific region (\citealt{Hemsterhuis2015}: 102–104). In doing so, he neglected the fact that the Greek Koine and the Dutch standard were based principally on specific dialects: Attic in the case of the Koine, and Brabantian and Hollandic in the case of Dutch.

In summary, Hellenists widely assumed that it was possible to explain and clarify the foreign as well as ancient phenomenon of Greek dialectal diversity by means of a more familiar context. This usually coincided with the dialects of the native language of the early modern Hellenist grammarian and – more importantly – of his intended readership. Needless to say, this practice emerged out of didactic concerns. As such, it was a neat realization of Juan Luis Vives’s pedagogical insight that a teacher was better equipped to give instruction in Latin and Greek if he also possessed a thorough knowledge of his mother tongue and that of his audience (see \citealt{Padley1985}: 146).

The explanatory usage also appeared outside of strictly grammaticographic and didactic contexts, in which case no thorough knowledge of the Greek dialects was required on the part of the author. For example, in his Latin–Polish dictionary of 1564, Jan Mączyński (ca. 1520–ca. 1587) invoked variation among Slavic tongues alongside the Greek dialects to explain the Latin term \textit{dialectus}, without mentioning, however, any Greek dialect by name:

\begin{quote}
The Greeks call \textit{dialects} species of languages, \textit{A property of languages, like in our Slavic language, the Pole speaks differently, the Russian differently, the Czech differently, the Illyrian differently, but it is nevertheless still one language. Only every region has its own property, and likewise it was in the Greek language}.\footnote{\citet[ \textit{s.v.} “dialectus”]{Maczynski1564}: “Dialectos Graeci uocant linguarum species, Vlasność yęzyków yáko w nászim yęzyku Slawáckim ynáczey mowi Polak ynáczey Ruśyn, ynáczey Czech ynaczey Ilyrak, á wzdy yednak yeden yęzyk yest. Tylko ysz każda ziemiá ma swę wlasność, y tákże też w Greckim yęzyku bylo”. The form “wzdy” should be “wżdy”, but the diacritic dot above the ⟨z⟩ does not appear in the original text. I kindly thank Herman Seldeslachts for this information and for helping me translate this early modern Polish passage.}
\end{quote}

Before Mączyński, Thomas \citet[\textsc{xxxiii}\textsc{\textsuperscript{v}}]{Eliot1538} had likewise defined \textit{dialectus} with reference to his native context. However, unlike Mączyński, Eliot made no mention at all of Greek variation. This suggests that Eliot preferred to explain the Latin word \textit{dialectus} by means of a familiar situation instead of troubling his reader with the diversity of Ancient Greek, far distant in time and space from the sixteenth-century English audience of his dictionary. Later dictionaries focusing on English did, however, include references to both English and Greek dialects.\footnote{See e.g. \citet[\textit{s.v.} “dialect”]{Bullokar1616} and \citet[\textit{s.v.} “dialect”]{Blount1656}. See \citet[7]{Blank1996}.}

In a sixteenth-century English controversy on early Church practices, including the language used during Mass in the east of the Roman Empire, a more familiar linguistic situation was invoked to make claims about ancient Greek diversity. In this so-called Challenge controversy – so named because it started out as a challenge mounted by the Protestant John Jewell (1522–1571) – the English recusant John Rastell (1530/1532–1577) assumed a certain degree of comparability between Greek and English variation, claiming that in both cases there was no mutual intelligibility. He did so as he was trying to demonstrate that not all speakers of Greek would have understood the learned Greek used in Mass.\footnote{On the controversy, see e.g. \citet[115--154]{Jenkins2006}. On the use of the English word \textit{dialect} in this context, see \citet[647--651]{VanRooyConsidine2016}.} Since an Englishman could not understand a Scotsman, there was no reason to stipulate that speakers of different Greek dialects were able to comprehend each other, \citet[68\textsc{\textsuperscript{r}}]{Rastell1566} argued. Rastell’s native English context thus clearly informed his views on the lack of mutual intelligibility among the ancient Greek dialects to make a point in a theological controversy.

The explanatory comparison of vernacular with ancient Greek dialectal diversity occurred in various genres other than Greek grammars, dictionaries, and theological invectives, too. These ranged from philological commentaries on classical works and monographs on New Testament Greek to geographical publications, prefaces to lexica, and various historiographical works.\footnote{For a philological commentary, see e.g. \citet[68; French–Greek comparison]{Casaubon1587}. For a monograph on New Testament Greek, see e.g. \citet[212--213; also French-Greek]{Cottiere1646}. For a geographical publication, see e.g. \citet[60; English–Greek comparison]{Speed1676}. For a preface to a lexicon, see e.g. \citet[(b.3)\textsuperscript{v}\textsc{;} also English–Greek]{Phillips1658}. For a historiographical work, see e.g. \citet[108, 117; French/Italian–Greek comparison]{Freret1809}.} A particular case in point is John Williams (?1636–1709), who, in his discourse on the language of church service, mentioned English and Greek variation alongside each other when explaining the concept of \textsc{dialect} to his readership, interestingly adding that the Greek dialects were “well known to the learned” \citep[5]{Williams1685}. Does this imply that Williams was providing a reference to the readers’ native context for those who were not as learned? Whatever the case, Williams drew a direct parallel between the Greek Koine “standard” – he used this exact term – and court English, projecting along the way his conception of the English standard back onto the Greek Koine.

Before proceeding to the next early modern trend in comparing ancient Greek with vernacular dialectal diversity, I want to point out briefly here that the explanatory function did not come into being with the Renaissance revival of Greek studies. As a matter of fact, explaining one linguistic context of variation by means of another occasionally occurred in the late Middle Ages as well, for instance in exegetical works on biblical passages alluding to regional linguistic differences, in particular the shibboleth incident in Judges 12.\footnote{See \citet[199--200]{VanRooy2018b} for the views of Nicholas of Lyra (1265–1349).} This was especially frequent in travel writings. Chinese variation was compared with Gallo-Romance diversity in the \textit{Book of the marvels of the world}; this work constitutes the written version of what the famous Venetian traveler Marco Polo (1254–1324) dictated to his cell mate in Genoa, Rustichello da Pisa, in 1298–1299. Rustichello, drawing up Marco Polo’s words in Old French, wanted to explain Chinese diversity by referring to the native context of his intended readership. Interestingly, an Italian translator substituted the allusion to diversity in France by referring to Italo-Romance variation, clearly adapting the text to his Italian audience.\footnote{See \citet[157]{Polo1938}, where both the French original and the Italian rendering are offered in an English translation. Cf. \citet[855]{Borst1959}.} Mutual intelligibility was explicitly posited for Chinese variation as well as the Italo-Romance dialects, but not for the Gallo-Romance context. This kind of comparison, omitting any reference to Ancient Greek, continued to be drawn throughout the early modern period, even though these comparisons were far less frequent than those between Ancient Greek and the vernacular. The French explorer and diplomat Pierre Belon (1517–1564), for example, employed his native linguistic context to explain to his readers that inhabitants of Constantinople mocked the Vernacular Greek spoken by outsiders. Just as the French laughed at Picard speech and any other Gallo-Romance variety that was not true French, residents of Constantinople jibed at other varieties of Vernacular Greek, \citet[5\textsc{\textsuperscript{v}}]{Belon1553} remarked.\footnote{On Belon as a traveler in Greece, see \citet[esp. 122]{Vingopoulou2004}.} Such comparisons of different contexts of variation excluding Ancient Greek also occurred outside of travel writings. The Spanish Dominican Domingo de Santo Tomás (1499–1570) explained Quechua diversity by referring to Romance differences in pronouncing Latin in his grammar of the South-American language.\footnote{See \citet[1\textsc{\textsuperscript{v}}]{Santo1560}. On his eye for variation, see \citet[140]{Calvo2005}.} Another noteworthy example stems from the correspondence of the Parisian humanist Claude Dupuy (1545–1594). \citet[274]{Dupuy2001} clarified Provençal diversity to his Neapolitan colleague Gian Vincenzo Pinelli (1535–1601) by comparing it to the variation in his addressee’s native Italian language in a letter dated December 12, 1579.

\subsection{Justification and description: Greek as a polyvalent model}\label{sec:8.1.2}

It came as a relief to many humanists that, unlike Latin, the revered Ancient Greek language was not a monolithic linguistic whole. This reminded them of the situation in their native vernaculars and, at the same time, made them aware of the fact that dialectal variation was not necessarily an insurmountable obstacle to the regulation and grammatical codification of their mother tongue. An observation in the first printed grammar of Dutch is revealing in this regard. This language, its authors argued, could be regarded as one entity, even though there were regional differences in pronunciation, “but not in such a manner that they do not understand each other very well”. Interestingly, they added that “in like manner the Greek language, which enjoys such high esteem, also had its different ‘dialects’”.\footnote{\citet[110]{[spieghel]1584}: “Ick spreeck […] int ghemeen vande Duytse taal, die zelve voor een taal houdende, […] wel iet wat inde uytspraack verschelende, maar zó niet óf elck verstaat ander zeer wel, tis kenlyck dat de Griexe taal, die zó waard gheacht is, óóck haar verscheyden \textit{dialectos} had”. For the authorship of this grammar, see \citet{Peeters1982}.} The addition of the relative clause “which enjoys such high esteem” clearly points to a justificatory use of ancient Greek diversity. This suggests that an acquaintance with the Greek literary dialects, however slight, catalyzed the emancipation of the vernaculars from Latin, which, certainly in the fourteenth and fifteenth centuries, was often conceived of as a highly uniform language synonymous with grammar.\footnote{On the catalyzing effect, see e.g. already \citet[688]{Bonfante1953}; \citet[9]{Trapp1990}; \citet[67]{Rhodes2015}. On Latin as an allegedly uniform tongue, see \sectref{sec:8.2} below.} The catalyzing effect seems confirmed by the fact that early comparisons of Greek with Italian diversity sometimes included an explicit contrast with the unity of Latin.\footnote{See e.g. \citet[\textsc{ii}.41]{Landino1974} and \citet[*.ii\textsc{\textsuperscript{v}}]{Manutius1496Aldus}. See \citet[172--173]{Alinei1984}; \citet[209--210, 215]{Trovato1984}. For the justificatory use of the Greek model in Italy, see \citet[46, 50]{Tavoni1998}.} It comes as no surprise then that the diversified linguistic patchwork of ancient Greece widely functioned as a model for scholars engaged in elevating, standardizing, and describing their native vernacular language. This intriguing tendency manifested itself in various ways.

To begin with, many sixteenth-century scholars saw in the Greek dialects a literary model, which must be framed in the tradition of claiming a close link between Ancient Greek and one’s native vernacular.\footnote{On the Greek–vernacular link, see \citet{Demaiziere1982}, with a focus on the French context; \citet{Trapp1990}; \citet{Dini2004}, with reference to Prussian. For a late example, see \citet[435--436]{VanHal2016}, who concentrates on \citet[119--132]{Reitz1730} and his linking of Dutch to Greek.} The most telling example of this use of the Greek dialects can be found in the work of the renowned French printer and Hellenist Henri Estienne.\footnote{On Estienne’s comparison of French and Greek diversity, see already \citet[70]{Demaiziere1988}.} In his \textit{Treatise on the conformity of the French language with the Greek}, Estienne defended the usage of dialect words in French literary works, adding that dialect words needed to be adapted to the common French tongue, just like meat imported from elsewhere must be prepared in the French manner and not as it was cooked in the land of origin.\footnote{\citet[¶¶.ii\textsc{\textsuperscript{v}}]{Estienne1565}. Cf. \citet[\texttt{\char"2720}\textsc{\textsuperscript{r}}]{Ronsard1550}, on which see \citet[170]{Alinei1984}; \citet[24]{Barbier-mueller1990}; \citet[14]{Trapp1990}. Cf. also \citet[456, 458]{Mambrun1661}. Similar views were expressed by scholars from other areas: see e.g. \citet[\textsc{e.}iii\textsc{\textsuperscript{v}}–\textsc{e.}iv\textsc{\textsuperscript{r}}]{Oreadini1525} for Italian and \citet[\textsc{a}.vi\textsc{\textsuperscript{r}}]{Craige1606} for English.} Estienne (\citet[133]{Estienne1579} and \citet[*.iii\textsc{\textsuperscript{v}}–*.iiii\textsc{\textsuperscript{r}}]{Estienne1582}) propagated the usage of the ancient Greek satirical author Lucian as a model for this practice. Inspired by the Greek heritage, he regarded French dialectal diversity as a source of richness that could adorn the French language (\citealt{Estienne1582}: *.iii\textsc{\textsuperscript{v}}; see \citealt{Auroux1992}: 366–367). The fact that \citet[143]{Estienne1579} allowed for dialect words and even dialect endings in French implies that to his mind “the pure and native French language” (\textit{le pur et nayf langage françois}) did not entirely correspond to Parisian speech, the variety on which the French norm was primarily based. He explained this by drawing a comparison with the Attic dialect, in which not every Athenian feature was allegedly approved. \citet[133--134]{Estienne1579} denied the same flexibility to Italian, since its Tuscan-based standard was much less prone to adopt features from other dialects. To sum up, Estienne, inspired by the Greek dialects he knew so well, viewed the French dialects as a source of richness that could embellish French language and literature. He perceived esthetic and typological similarities between French and Greek dialects, even though he did not go so far as to make any claims about the genealogical dependency of French on Greek (\citealt{Droixhe1978}: 99; \citealt{Considine2008a}: 62).

Such ideas also appeared outside of France. The great German grammarian Justus Georg Schottel (1612–1676), for instance, argued that not everything outside of the selected dialect – in particular Attic Greek and the German of Meissen – was faulty (\citealt{Schottel1663}: 176; cf. \citealt{Roelcke2014}: 250). What is more, not all dialect words must be avoided, since some could be current in certain technical jargons. These considerations led Schottel to conclude that frequent and important dialect words needed to be included in a dictionary. The value he attached to dialectal material clashes somewhat with his view, expressed only some pages earlier, that dialects were inherently incorrect and unregulated \citep[174]{Schottel1663}. William J. \citet[1110]{Jones2001} has summed up this contradiction nicely:

\begin{quote}
Himself a native speaker of Low German, Schottelius was caught between admiration for a[n] […] etymologically valuable dialect, and an awareness that prestige and currency precluded any choice but High German.
\end{quote}

Other German scholars stressed the richness of vernacular dialects as well, often with reference to the ancient Greek context.\footnote{See e.g. \citet[\textsc{a.3}\textsc{\textsuperscript{r}}\textsc{–a.3}\textsc{\textsuperscript{v}}]{Chytraeus1582}; \citet[\textsc{c.1}\textsc{\textsuperscript{r}}]{Meisner1705}; \citet[73]{Hertling1708}.}

Occasionally, patriotic sentiments tempted scholars to accord a special status to the dialects of their native vernacular tongue. This happened in Manuel de Larramendi’s (1690–1766) Basque grammar, which contains a section “On the dialects of the Basque language” (“De los dialectos del bascuenze”; \citealt{Larramendi1729}: 12–15). Larramendi’s views were clearly informed by early modern scholarship on the Greek dialects. He emphasized that, much like Greek, Basque had a common language, a “body of language common and universal to all its dialects”.\footnote{\citet[12--13]{Larramendi1729}: “cuerpo de lengua, comun y universal à todos sus dialectos”.} Further, he seems to have projected the distinction between principal and minor dialects from early modern grammars of Greek onto the Basque context (\citealt{Larramendi1729}: 12; see Chapter 2, \sectref{sec:2.6}). Greek and Basque diversity was, however, not comparable on every level, claimed \citet[12]{Larramendi1729}:

\begin{quote}
The difference is that the dialects of the Basque language are very regulated and consistent, as if they were invented with devotion, discretion, and expediency, which the Greek dialects did not have and others in many other languages do not have.\footnote{“La diferencia está que los dialectos del bascuenze son muy arreglados y consiguientes, como inventados con estudio, discrecion y oportunidad: lo que no tenian, ni tienen los dialectos griegos, y otros en otras muchas lenguas”. On this passage, see also \citet[876]{Hasler2009}.}
\end{quote}

In other words, the Greek dialects served as a model for Larramendi in several respects, but were at the same time valued less highly than their Basque counterparts, an idea quite unusual in the early modern period. In a work published a year earlier, however, \citet[142]{Larramendi1728} had presented the Greek dialects as also being regulated. It is unclear exactly why he had this change of heart, but patriotic sentiment no doubt played a role.

Not all scholars associated the Greek dialects with spoken varieties of the vernaculars. The Dutch grammarian Adriaen Verwer (ca. 1655–1717) was aware of the literary character of the Greek dialects and compared them with different written registers of his native vernacular rather than with spoken regional dialects. \citet[53--54]{Verwer1707} divided written Dutch into three main forms: (1) the common language (\textit{lingua communis}), (2) the dialect used in government (\textit{dialectus curiae senatuique familiaris}), and (3) the poetical dialect (\textit{dialectus poetis familiaris}). Verwer also mentioned a court dialect (\textit{dialectus forensis}), a variety closely cognate to the common language, from which it only differed in rhetorical – and not in grammatical – terms. The focus on register variation is also apparent from his definition of the Latin term \textit{dialectus}; dialects were “various particular speech forms in our written language”.\footnote{\citet[53]{Verwer1707}: “dese ende gene, bysondere spraekvormen in onse schrijftaele”.}

The situation of ancient Greece also functioned as a model for selecting a variety to be codified as the vernacular norm. A very straightforward example of such an approach can be found in Nathan Chytraeus’s (1543–1598) preface to his Latin–Low Saxon lexicon of 1582. In it, \citet[\textsc{a.3}\textsc{\textsuperscript{r}}\textsc{–a.3}\textsc{\textsuperscript{v}}]{Chytraeus1582} described the constitution and elevation of a German common language as a process awaiting completion and stressed the model function of the Greek Koine in this context. He moreover saw a key role for the dialects, which could beautify the common language. More theoretical still were the proposals by certain early Cinquecento Italian scholars to create a mixed common language after the example of the Greek Koine as an artificial solution to the \textit{questione della lingua}.\footnote{See Vincenzo Colli’s ideas as quoted by Pietro \citet[\textsc{xii}\textsc{\textsuperscript{v}}\textsc{–xiii}\textsc{\textsuperscript{r}}]{Bembo1525}. See \citet[119]{Melzi1966}; \citet[215--218]{Trovato1984}; \citet[12]{Trapp1990}.} Not all humanists limited themselves to mere reflection. The Dutch scholar and priest Pontus de Heuiter (1535–1602) put the active creation of a vernacular common language through mixture to actual practice in his \textit{Dutch orthography}. De Heuiter explicitly mentioned his debt to the ancient Greek model for his initiative:

\begin{quote}
I have taken the Greeks as an example, who, having the four good tongues of the country in usage, namely \textit{Ionic}, \textit{Attic}, \textit{Doric}, and \textit{Aeolic}, have created a fifth one out of them, which they called the \textit{common language}. Thus I have created my Dutch over a period of twenty-five years out of Brabantian, Flemish, Hollandic, Guelderish, and Kleverlandish.\footnote{\Citet[93]{De1581}: “[…] heb ic exempel ande Grieken genomen, die vier lants goude talen in ufenijng hebbende, te weten: \textit{Ionica, Attica, Dorica, Aeolica}, die vijfste noh daer uit gesmeet hebben, die zij nommen \textit{gemeen tale}: aldus heb ic mijn Nederlants over vijf en twintih jaren gesmeet uit Brabants, Flaems, Hollants, Gelders en Cleefs”. See also \citet[110]{Dibbets2008} and \citet[13--14]{De1917}. The latter has linked this passage to Hieronymus Wolf’s reference to Greek in his discussion of German dialects. However, Wolf did not explicitly take the Greek context as a model and seems to have stressed, instead, the incomparability of both contexts. See \sectref{sec:8.1.3} below.}
\end{quote}

Not all scholars using the Greek Koine as a model for their vernacular norm believed the Koine to be created out of the different dialects. The grammarian Kaspar von Stieler (1632–1707) held that the Greek Koine, which he saw as a model for his High German norm, was exempt from dialectal elements \citep[2]{Stieler1691}. Interestingly, later authors emphasized the frequently drawn parallel between the Greek Koine and the German norm by referring to the former as “High Greek” (\textit{Hoch-Griechisch}) by analogy to “High German” (\textit{Hochdeutsch}, \citet[13]{Schuster1737}).

Not everybody regarded the Greek Koine as the model for the selected, normative variety of their vernacular tongue. Almost equally often, scholars put forward the Attic dialect as the main form of Greek and the principal model after which one’s mother tongue should be developed. This holds especially true in cases where scholars emphasized the literary function of the selected variety. A telling example is Henri \citet[*.iii\textsc{\textsuperscript{v}}]{Estienne1582}, who put French in the capital city of the kingdom; just as Athens was the “Greece of Greece” in terms of speech, Paris was the “France of France”. Estienne added, however, that this was the case not because the French capital was frequently visited by the royal court, but because it had a parliament – he was perhaps inspired here by the example of Athenian democracy. He was thus comparing the French language to Attic rather than to the Greek Koine. This was surely prompted by his emphasis on the codification of French as a respected literary norm similar to Attic rather than a language understood by all inhabitants of the kingdom. In fact, \citet[*.iii\textsc{\textsuperscript{r}}]{Estienne1582} seems to have regarded pure French as a social privilege which the lower classes could never attain.\footnote{Cf. \citet[\textsc{xxxiii}\textsc{\textsuperscript{v}}]{Marineo1497} for an early comparison of Castilian Spanish with Attic Greek.}

Taking Attic and especially the Greek Koine as the model for selection had far-going glottonymic consequences. Indeed, the designation “common language” was widely used to refer to the selected variety of a vernacular language in imitation of the Greek Koine, usually termed \textit{lingua communis} in Latin. What is more, some even referred to the vernacular norm, by the procedure of antonomasia, as “Attic”. The Greek scholar Alexander Helladius (1686–after mid-1714) attributed the label of “Attic” to what he called the “High German par excellence” (“κατ’ ἐξoχὴν \textit{das Hochteutsche}”; \citealt{Helladius1714}: 187). Attic or Koine Greek were not, however, the only speech forms that could serve as the model for selecting a vernacular norm. In cases where a vernacular variety was described that was not or not yet fully established as the selected norm but which an author wanted to see established, it was sometimes compared to varieties of languages other than Greek that were widely accepted as the standard form. One scholar writing in 1595 wanted to promote his native Croatian dialect as the Slavic norm, for which Tuscan Italian constituted his model (\citealt{Veranzio1595}: *.3\textsc{\textsuperscript{v}}; cf. also \citealt{Schoppe1636}: 46).

Apart from selection, Greek could also be the model for another key standardization process in vernacular tongues: codification in spite of the presence of dialectal variation. Early in the sixteenth century, the French humanist Geoffroy Tory (ca. 1480–before late 1533) commented as follows on the regulation and grammatical codification of French, which he regarded more as a set of varieties rather than a unitary language with a single norm:

\begin{quote}
Our language is as easy to regulate and put in good order as the Greek language once was, in which there are five speech varieties, which are the Attic, Doric, Aeolic, Ionic, and common language. These have certain mutual differences in their noun declensions, verb conjugations, orthography, accents, and pronunciation.\footnote{\citet[\textsc{iv}\textsc{\textsuperscript{v}}\textsc{–v}\textsc{\textsuperscript{r}}]{Tory1529}: “Nostre langue est aussi facile a reigler et mettre en bon ordre, que fut jadis la langue grecque, en la quelle ya cinq diversites de langage, qui sont la langue attique, la dorique, la aeolique, la ionique et la commune, qui ont certaines differences entre elles en declinaisons de noms, en conjugations de verbes, en orthographe, en accentz et en pronunciation”. See \citet[466--467]{Trudeau1983} for Tory’s “pandialectal” conception of French. Cf. \citet[19--20]{Defaux2003}, where the passage is contextualized within the French grammatical tradition; \citet[23]{Cordier2006}, who frames it in Tory’s general reception of antiquity.}
\end{quote}

Tory proceeded by mentioning a number of French speech forms: the court variety, Parisian (which he seems to have associated closely with the court variety), Picard, Lyonnais, Limousin, and Provençal. Inspired by the Greek model, he did not view dialectal variation as a negative property hindering the regulation of the vernacular. Other scholars were not as optimistic about the codification of dialect-ridden tongues. The Hellenist Erasmus \citet[239]{Schmidt1615} emphasized the impossibility of reducing the dialects of both Greek and his native German to a norm. It goes without saying that not only Greek was used as a model for the selection and codification of a norm. Latin or other vernacular contexts were a major source of inspiration as well. The renowned grammarian Johann Christoph Gottsched (1700–1766), for instance, was inspired by the example of the Latin tongue in declaring it necessary to ban dialectal features from the German norm \citep[334]{Gottsched1748}.

The ancient Greek dialect context also served as a descriptive model, taken here in a very broad sense and therefore encompassing a range of approaches. To start with, the Greek prototype was projected onto the linguistic situation on the Iberian peninsula by the Spanish humanist Bartolomé Jiménez Patón (1569–1640). More particularly, Jiménez Patón relied on the traditional classification of Greek into five dialects to map out variation in his native land:

\begin{quote}
And thus we say that among the Greeks there are five manners of tongue with different dialects, which are the Attic, Ionic, Doric, Aeolic, and common tongue. And in Spain there are five others, which are the Valencian, Asturian, Galician, and Portuguese. All of these derive from this fifth, or principal and first Original Spanish of ours, different from the Cantabrian.\footnote{\citet[10\textsc{\textsuperscript{r}}\textsc{–10}\textsc{\textsuperscript{v}}]{Jimenez1604}:“Y asi entre los Griegos decimos aver cinco maneras de lengua con differentes dialectos que son la lengua attica, ionica, dorica, aeolica y comun. Y en España ay otros cinco, que son la valenciana, asturiana, gallega, portuguesa. Las quales todas se an derivado de esta nuestra, quinta o principal y primera, originaria española differente de la cantabria”.}
\end{quote}

Jiménez Patón’s circumscription of the historical position of “Original Spanish” vis-à-vis the four other dialects may suggest that he envisioned the relationship of the Koine to the Greek dialects in much the same terms. If so, the projection did not happen solely from Greek to Spanish, but partly also vice versa. In other cases, the Greek dialects were unmistakably forced into a vernacular straitjacket, reversing the directionality of the comparison. For example, Friedrich Gedike’s analysis and classification of the Greek dialects were modeled on his tripartite conception of the German dialects (see Chapter 7, \sectref{sec:7.3}).

The Greek dialects were also eagerly used as a descriptive point of reference by scholars wanting to sketch the degree of kinship among certain vernacular varieties, even among varieties that today are usually considered to be distinct but related languages. The preacher from Dordrecht Abraham Mylius (1563–1637) compared in his \textit{Belgian language} the superficial variation among some of the languages now known as Germanic to differences between Aeolic and Ionic, stressing that, in both cases, the root and character of speech had remained the same (\citealt{Mylius1612}: 90; cf. e.g. also \citealt{Boxhorn1647}: 75–76). This also occurred on a lower level, as in Sven Hof’s (1703–1786) pioneering monograph on the dialect of Västergötland, a province in the west of modern-day Sweden. In this work, \citet[esp. 12–13, 23]{Hof1772} relied on his familiarity with the Greek context in seeing dialects as classifiable entities and in describing individual dialect features. For some scholars, using the Greek dialects as a model context had glottonymic consequences. The Italian humanist Claudio Tolomei (ca. 1492–1556), writing around 1525, contended that in much the same way as it was justified to group the Greek dialects together and designate them with one and the same label, the varieties of Italian should be seen as one linguistic class and should be called by one and the same name (\citealt{Tolomei1555}: 14; see \citealt{Trovato1984}: 216).

Individual Greek dialects were frequently proposed as a point of comparison for clarifying the status and position of a vernacular dialect in its broader linguistic landscape. Attic was said to be similar to Misnian – the German of Meissen – often presented as the standard variety of German (see e.g. \citealt{Borner1705}: \textsc{b.4}\textsc{\textsuperscript{v}}; \citealt{Simonis1752}: 214–215). Henri Estienne perceived parallel features in individual French and Greek dialects. For instance, \citet[3--4]{Estienne1582} compared the broadness of Franco-Provençal speech – \textit{sermo Romantius} he termed it in Latin – to that of Doric Greek, pointing out that both varieties were characterized by the prominence of the vowel [a]; examples he cited were Franco-Provençal \textit{cla} and Doric \textit{kláks} (κλάξ), both words meaning ‘key’. In a similar vein, the Enlightenment scholar Ferdinando Galiani (1728–1787), in his monograph on his native Neapolitan dialect, stressed its archaism and contended that it had phonetic properties – open vowels, a great expressivity of words, and strong consonants – similar to Doric, the Greek dialect spoken by the ancient inhabitants of Naples and surroundings. In sum, Galiani claimed, “Neapolitan could well be called the Doric dialect of the Italian tongue”.\footnote{\citet[16]{Galiani1779}: “il napoletano potrebbe ben dirsi il dorico della favella italiana”.} His glottonymic suggestion did not, however, enjoy any success.

Things are very different with an early modern comparison of a Greek with an English dialect. As a matter of fact, a development with consequences that resonate today began around the mid-seventeenth century, when the church historian Thomas Fuller (1608–1661) linked Scots with Doric Greek. According to Fuller, “the speech of the modern Southern-\textit{Scot} [was] onely a \textit{Dorick} dialect of, no distinct language from \textit{English}” \citep[81]{Fuller1655}. Forty years later, Patrick \citet[20]{Hume1695}, a commentator of John Milton’s \textit{Paradise Lost}, remarked on Milton’s use of the verb \textit{to rouse} that it signified ‘to get up’, being “a more northern pronunciation of rise, like the Dorick dialect”. Around the same time, the writer John Dryden (1631–1700) characterized the English poet Edmund Spenser’s (1552/1553–1599) language as follows:

\begin{quote}
But Spencer, being master of our Northern dialect and skill’d in Chaucer’s English, has so exactly imitated the Doric of Theocritus, that his love is a perfect image of that passion which God infus’d into both sexes, before it was corrupted with the knowledge of arts and the ceremonies of what we call good manners. (Dryden in \citealt{Virgil1697}: \textsc{a.2}\textsc{\textsuperscript{r}})
\end{quote}

Why was there such a close association between Doric and Scots? This parallel seems to have been informed not only by certain shared linguistic features, such as the frequency of [a] and a presumed broad pronunciation, but also – and probably primarily – by the alleged rustic nature and status of both dialects as well as their being used in bucolic poetry. This practice continued into the modern period (\citealt{Colvin1999}: v). A vestige of this early modern tradition is reflected in current glottonymic practice; the variety of Scots spoken in the Aberdeen area, now known as Mid-Northern or North-East Scots among linguists, is still labeled \textit{Doric} in popular usage to this day.\footnote{See \citet[116]{Mccoll2007}: “In the course of the twentieth century, the North-East variety became known as The Doric, a term previously applied to all Scots varieties”.} The history of the association of Scots with Doric, which I have shown to go back at least to the seventeenth century, deserves a closer investigation, but this lies outside the scope of this book.

Yet another important manner in which Greek diversity was used as a descriptive point of reference was the extrapolation of letter permutations closely and prototypically associated with Greek to the diversity among the tongues of Western Europe. Greek letter changes were already around the turn of the sixteenth century a source of inspiration to describe similar variations in Italo-Romance.\footnote{See e.g. \citet[*.ii\textsc{\textsuperscript{v}}]{Manutius1496Aldus} and \citet[97\textsc{\textsuperscript{r}}]{Da1509}. See also Chapter 6, \sectref{sec:6.2}.} Especially in West Germanic-speaking Europe, this was a prominent phenomenon; there, the sigma–tau alternation present in, for instance, Koine \textit{glôssa} (γλῶσσα) and Attic \textit{glôtta} (γλῶττα), meaning ‘tongue’, was very often understood as somehow cognate to the ⟨s⟩–⟨t⟩ alternation among varieties of West Germanic, as in High German \textit{Wasser} vs. Dutch \textit{Water}.\footnote{See e.g. \citet[21]{Mylius1612}. Cf. also \citet[\textsc{m}.ii\textsc{\textsuperscript{r}}]{Althamer1536}; \citet[\textsc{a.3}\textsc{\textsuperscript{r}}]{Chytraeus1582}; \citet[119--132]{Reitz1730}; \citet[61--62]{Ruhig1745}; \citet[23--24]{Hof1772}.}

A final and somehow peculiar use of Greek diversity as a model can be found in the work of the Enlightenment pedagogue Friedrich \citet[7]{Gedike1782}, who assumed that the Greek context could assist in predicting dialectal evolution in other languages. Gedike’s knowledge of the history of Greek colonization and its impact on dialect formation led him to prophesize the emergence of a new English dialect in the United States, which at his time of writing in 1782 had just recently declared independence from Great Britain 1776, even though this was officially recognized by Great Britain only in September 1783 through the Treaty of Paris. Gedike was, however, probably not very familiar with the linguistic situation in the US; otherwise he would have realized that his prediction was, in fact, already becoming a reality at his time of writing.

In summary, Greek variation was eagerly used as a model by early modern scholars engaged in the elevation, standardization, and description of the vernacular tongues of Western Europe, usually their native ones. This happened in various ways, which can be placed under three main, not always easily distinguishable headings; the Greek linguistic context with its characteristic dialectal diversity was employed as (1) a literary \textit{exemplum}, (2) a model for standardization, and (3) a descriptive point of reference, this in very broad terms. The fascination with the Greek model was sometimes so intense that one could speak of a true Hellenomania, as with the printer-philologist Henri Estienne. An intimate acquaintance with the Greek language and its dialects was not always an indispensable prerequisite, even though it usually stimulated the exemplary use of the Greek language strongly, as again in Estienne’s case.

\subsection{Dissociation: The particularity of the Greek dialects foregrounded}\label{sec:8.1.3}

At first, humanist scholars seem to have largely agreed upon the comparability of Greek and vernacular dialectal variation, which for them seems to have been a kind of uncontested assumption. Gradually, however, different voices were heard, especially from the end of the sixteenth century onward, when the selection of the linguistic norm was more or less settled for many Western European vernaculars, even though this process was completed at different moments for each language.\footnote{See e.g. \citet[217--222]{Mattheier2003}, who points out that Luther’s German and so-called general German (an East Upper German koine) competed for most of the early modern period, even though the former eventually gained the upper hand.} Two early scholars with a particularly outspoken opinion on the issue were Benedetto Varchi (1503–1565) and Vincenzo Borghini (1515–1580), both Italian humanists involved with the \textit{questione della lingua}.

Benedetto \citet[95]{Varchi1570} regarded the Greek dialects as “equal” (\textit{eguali}) – they were of the same noblesse and dignity – whereas there was inequality among Italian varieties, since Florentine speech was elevated above the rest. This seems to be reflected in Varchi’s usage of the term \textit{dialetto}, which he restricted to varieties of the Greek language. He nevertheless reserved a particular place for Attic, which he claimed to be similar to Italian, by which he meant Tuscan \citep[141]{Varchi1570}. Siding with Pietro Bembo (1470–1547) against Baldassare Castiglione (1478–1529) and Gian Giorgio Trissino (1478–1550), Varchi was fiercely opposed to the use of the Greek Koine as a model for a common Italian language.\footnote{\citet[269--271]{Varchi1570}, with reference to \citet{Bembo1525}, \citet{Castiglione1528}, and \citet{Trissino1529}.} Varchi argued that there were only four Greek dialects, out of which the Greeks easily created a common tongue, but the varieties in Rome were innumerable, making it impossible to produce an Italian koine out of them.

Like Varchi, the Italian monk and exceptional Hellenist Vincenzo Borghini was convinced that Greek and Italo-Romance variation were incomparable, a train of thought he developed in a manuscript treatise entirely devoted to this problem – it bears the title \textit{Whether the diversity of the Greek language is the same as the Italian} and was likely composed in the first half of the 1570s (edition in \citealt{Borghini1971}; see \citealt{Alinei1984}: 171, 191). \citet[335]{Borghini1971} argued instead that if the Greek context really needed to be compared with variation on the Italian peninsula, it should be with variation in the Tuscan subgroup rather than with Italian as a whole. After all, Italo-Romance tongues differed from each other to a far greater extent than the Greek dialects did. The Tuscan–Greek comparison was all the more preferable, Borghini continued, since the varieties of both linguistic groups were approved speech forms, in contrast to other Italian varieties such as Lombard. \citet[338--340]{Borghini1971} dismissed the comparison of Italian and Greek also for historical reasons. Speakers of Italian did not have a common tongue because, unlike the ancient Greeks, there was originally no unitary Italian people speaking a common language. In fact, Italian emerged out of the mixture and corruption of the tongues of several different peoples. This was why constructing a common Italian language was a bad idea. What is more, much like Varchi, Borghini contrasted the approved and written Greek dialects, which only showed slight mutual differences, with the innumerable Italo-Romance varieties, which could not be reduced to writing and which exhibited substantial divergences.\footnote{\citet[341]{Borghini1971}. See \citet[171]{Alinei1984}; \citet[210]{Trovato1984}; \citet[32--37]{Beninca1988}. Cf. \citet[253--254]{Salviati1588} for an argument similar to Borghini’s.} During the sixteenth century, voices similar to Varchi’s and Borghini’s were heard outside of Italy as well.\footnote{See e.g. \citet[595--596]{Wolf1578}, on whom see \Citet{Von1856}, \citet[58–59]{Jellinek1898, Jellinek1913}, and \citet[esp. 214--218]{Mattheier2003}. Cf. also \citet[xiii.\textsc{\textsuperscript{v}}]{Palsgrave1530}.} This continued throughout the seventeenth century and reached its peak in the eighteenth century, especially in France, to which I turn now.\footnote{For seventeenth-century examples, see \citet[458--459]{Mambrun1661} and \citet[146--147]{Morhof1685}.}

The stress on incomparability was particularly prominent in the widely read works of the French historian and classical scholar Charles Rollin, who distinguished between the dialects of the Greek language, termed \textit{idiomes} and \textit{dialectes}, and the patois of the different provinces of France, called \textit{jargons}. Rollin characterized these latter as vulgar and corrupted manners of speaking not deserving the label of \textit{language} (\textit{langage}). A dialect, in contrast, was “a language perfect in its own right”, apt for literary use, having its own rules and elegant features.\footnote{\citet[117]{Rollin1726}: “Chaque dialecte étoit un langage parfait dans son genre”. See also \citet[395]{Rollin1731}.} In a later work, \citet[395]{Rollin1731} connected this to the political fragmentation of Greece as opposed to the high degree of centralization in France (cf. Chapter 7, \sectref{sec:7.5}). The comparability was subsequently denied in Greek grammars composed by French scholars, as in the 1752 edition of a lengthy \textit{Introduction to the Greek language} by the French Jesuit Bonaventure Giraudeau (1697–1774). This grammar, composed in Latin, was first published in Rome thirteen years earlier, but that edition lacked a reference to the French dialects, as it would not have been useful to its Italian audience. Only when it was published in French-speaking territory – the edition of 1752 appeared in La Rochelle and was sold in Paris – did a comment about French linguistic diversity become relevant \citep[117]{Giraudeau1752}.

The criticism of the comparability of French and ancient Greek regional diversity reached an apogee in the “Langue” article included in the ninth volume of Diderot and d’Alembert’s \textit{Encyclopédie}, published in 1765. The author of the entry was the French grammarian Nicolas Beauzée (1717–1789). In his lengthy article, \citet[249]{Beauzee1765} elaborated on two types of regional language variation, correlating with political differences. He contrasted Latin and French diversity with variation in ancient Greece, Italy, and Germany. Greeks, Italians, and Germans were made up of “several equal and mutually independent peoples” (“plusieurs peuples égaux et indépendans les uns des autres”), which was why their dialects were “equally legitimate” (“également légitimes”) forms of their respective national language. The situation was different for Latin, which was the language of a politically unified empire. It therefore had only “one legitimate usage” (“un usage légitime”), while everything deviating from it did not deserve the label “dialect of the national language” (“dialecte de la langue nationale”). Instead, it should be circumscribed as “a patois abandoned to the populace of the provinces” (“un patois abandonné à la populace des provinces”).\footnote{Cf. \citet[135--136]{Priestley1762}, who expressed a view similar to Beauzée’s in the English context.} The same held true for his contemporary French context, claimed Beauzée. Yet not every contributor to the \textit{Encyclopédie} seems to have been convinced of the differences between French and Greek diversity. The anonymous author of the “Patois” entry asked himself: “What are the different dialects of the Greek language other than the patois of the different areas of Greece?”\footnote{\citet[174]{Anon.1765}: “Qu’est-ce que les différens dialectes de la langue greque, sinon les patois des différentes contrées de la Grece?”}

The emphasis on the incomparability of vernacular and Greek variation also occurred outside of France, especially in German-speaking territories.\footnote{See e.g. \citet[b.2\textsc{\textsuperscript{v}}\textsc{–}b.3\textsc{\textsuperscript{r}}]{Nibbe1725}, who stressed differences in literary usage; \citet[1131--1132]{[frisch]1730}, who opposed the literary Greek dialects to the German dialects of the lower social classes (\textit{Pöbel-Sprach}); \citet[6--8]{[frederick1780}; \citet[203--204]{Ries1786}. For an example from England, see \citet[13--14]{Bayly1756}.} Of particular interest is the work of the eighteenth-century German classical scholar Johann Matthias Gesner, who provided an insightful account of the comparability of German and ancient Greek diversity. In the past, Gesner argued, they were comparable. The absence of a centralized government and capital caused dialectal variation in both areas.\footnote{\citet[160--161]{Gesner1774}. Cf. \citet[lxviii]{Court1778}, who limited the comparability to the period before France had a centralized government.} Moreover, Greek as well as German dialects were initially used in writing. Starting with the Lutheran era, the German dialects lost their prominence and social prestige, leading them to be ridiculed and to attain a status different from the ancient Greek dialects. \citet[162]{Gesner1774} likewise considered it unacceptable to compose dialectally mixed poetry in German, arguing at the same time that this was equally inappropriate for Greek authors writing in or after late antiquity.

In conclusion, scholars frequently stressed the incomparability of Greek and vernacular dialects, especially toward the end of the early modern period, when most vernacular dialects had slipped into the shadows of their overarching standard varieties and the comparison must have appeared less convincing. In assessing this lack of comparability, authors were generally inspired by language-external circumstances, usually geopolitical and sociocultural. On some occasions, however, incomparability was maintained on a more strictly linguistic basis, for instance, when attempting to map out different degrees of linguistic kinship. This is what happened when certain eighteenth-century Scottish scholars compared the Greek dialects with the relationship among a number of tongues known today as Celtic. The early eighteenth-century Scottish antiquarian David Malcolm stressed the incomparability of both contexts, leading him to propose a different terminology for each situation:

\begin{quote}
Many indeed say that the \textit{Welsh} and \textit{Irish} are but different dialects of the same language, but those who have enquired into them will easily see that they differ more widely than the dialects of the \textit{Greeks}. Perhaps it may not be amiss to call them sister languages. (\citealt{Malcolm1738}: 46–47; cf. \citealt{Macnicol1779}: 311)
\end{quote}

The Greek dialects were not always directly involved when scholars emphasized the incomparability of two dialect contexts. Comparisons of different Western European vernaculars sometimes served to devalue the dialects of one language in favor of the dialects of another. Henri \citet[133--134]{Estienne1579}, for example, praised the richness and utility of French dialectal diversity, both properties he denied to Italian (see \citealt{Swiggers1997}: 306; \citeyear{Swiggers2009}: 73). Also, when comparing two or more vernacular dialect contexts, scholars noticed different degrees of mutual intelligibility and variation.\footnote{For mutual intelligibility, see e.g. \citet[158\textsc{\textsuperscript{r}}\textsc{–158}\textsc{\textsuperscript{v}}]{Hosius1560}; \citet[77 – I refer to the German translation of the Swedish original, published in 1746/1747]{Hogstrom1748}. For different degrees of variation, see e.g. \citet[27, 57]{Sajnovics1770}.}

\subsection{Synthesis}

Vernacular diversity was very often compared to the ancient Greek dialects during the early modern period. This happened for various purposes, most importantly, (1) to explain the nature of Greek dialectal diversity, mainly to would-be Hellenists or to an intended readership unacquainted with the Greek language, (2) to justify and describe (certain uses of) dialectal variation in the Western European vernaculars, and (3) to emphasize differences between Greek and vernacular variation, especially in literary and sociopolitical terms. I have visualized the directionality of the comparisons in \tabref{tab:8.1} below.

\begin{figure}
\caption{Directionality of comparison of ancient Greek with vernacular dialects\label{tab:8.1}}
\begin{tikzpicture}[baseline]
\matrix (directionality) [anchor=base,baseline,matrix of nodes] {
(1) & ancient Greek &[5cm] vernacular\\
(2) & ancient Greek & vernacular\\
(3) & ancient Greek & vernacular\\};
\draw[-{Triangle[]}] (directionality-1-3) -- (directionality-1-2);
\draw[{Triangle[]}-] (directionality-2-3) -- (directionality-2-2);
\draw[{Triangle[]}-{Triangle[]}] (directionality-3-3) -- (directionality-3-2) node[midway,circle,draw,fill=white] (circle) {}; \draw [] (circle.north east) -- (circle.south west);
\end{tikzpicture}
\end{figure}
In the cases of (1) and (2), the figure suggests a strictly unidirectional movement. However, as I have argued, especially in \sectref{sec:8.1.1} above, this is too simple a picture. Scholars often suppressed, usually silently, the differences between both dialect contexts in order to underline the similarities, and they sometimes even forced one situation into the straitjacket of the other. This could happen either consciously or subconsciously. It is, however, difficult to tell the degree of consciousness from the actual evidence, as the suppressing of the differences was nearly always left unmentioned. The reason for this is obvious; mentioning differences would invalidate the scholar’s claim of comparability.

The enumeration above may be taken to carry chronological implications as well. At first, the tendency to explain the phenomenon of ancient Greek dialectal diversity prevailed, soon after which the directionality was reversed with the Greek linguistic context functioning as a model for justifying and describing vernacular variation. The third element, dissociation, came about as a reaction against this latter use of the Greek dialects in the second half of the sixteenth century and culminated in the eighteenth century. This occurred especially in France, where the devalued patois were emphatically opposed to the literary Greek dialects. Even though it is possible to distinguish certain tendencies throughout the early modern period, one must be aware that, once the three main approaches toward Greek vis-à-vis vernacular diversity were established, they often coexisted. What is more, one scholar could reflect and reunite different approaches in their writings, even as seemingly contradictory attitudes as (2) and (3). For example, in Henri Estienne’s work, the model function of Greek took center stage, as I have established above in \sectref{sec:8.1.2} Elsewhere in his work, however, \citet[93--94]{Estienne1587} granted that the literary use of dialects was much more restricted in French than it had been in Ancient Greek, thus displaying an awareness of differences between both dialect contexts. He noticed that Homer was allowed to mix different dialects in his epic poems, but in French this primarily happened in comic pieces and was uncommon in more serious writings, with the exception of certain dialect words in the poetry of Pierre de Ronsard and Joachim du Bellay.

What vernacular varieties were compared most intensively to the ancient Greek dialects? It should come as no surprise that Italian humanists were the first to compare ancient Greek diversity with their vernacular context, as they were at the cradle of Renaissance Greek studies.\footnote{On the comparison of the Greek and Italian contexts, see also \citet[2–3, 51]{Dionisotti1968}, \citet[179]{Alinei1984}, \citet[215]{Trovato1984}, and \citet[36--37]{Lepschy2002}.} Indeed, Italian diversity was frequently compared to the Greek dialects, primarily in the sixteenth century. After the selection of the normative variety was more or less settled, comparisons of Greek and Italian variation became less frequent. It seems to have occurred only occasionally in the seventeenth and eighteenth century, mainly to stress the similarities both contexts displayed (e.g. Salvini in \citealt{Muratori1724}: 99–100). Almost immediately after the revival of Greek studies reached the other side of the Alps, intuitive comparisons of the Greek and German dialect contexts started to appear. Soon, they occurred in the work of Frenchmen, too, in which it seems to have been related to the patriotic claim that French derived from Greek and not from Latin. Paradoxically, it turned out to be French scholars who stressed most strongly the incomparability of Greek and French variation in the eighteenth century. This was no doubt related to the purist and prescriptivist attitudes current in French linguistic thought at the time as well as to a reverence for the literary dialects of Greek.\footnote{On French purism in the eighteenth century, with specific reference to the \textit{Académie française}, see \citet[]{Francois1905}.} In England, comparisons were frequent, too, albeit less so than in Italy, Germany, and France, and the comparability of Greek and English variation was usually taken for granted. It was somewhat less customary to compare the ancient Greek dialects with variation in Dutch, Spanish, and North Germanic, and much less so with varieties of Baltic, Basque, Celtic, Portuguese, and Slavic. This is not really astonishing; intense comparisons of Greek with vernacular variation were principally conducted by scholars active in areas and cities that were centers of Greek studies, including most importantly Italy, Germany, and France. Comparative approaches toward ancient and vernacular Greek variation were exceptional, most likely because Western European scholars did not feel the need to justify or describe the dialectal variation of a foreign language they considered barbarous and because they approached the matter largely in terms of discontinuity rather than incomparability (see Chapter 2, \sectref{sec:2.10}; Chapter 5, \sectref{sec:5.5}). A notable exception was the Italian Jesuit missionary Girolamo Germano (1568–1632), who tried to justify his focus on the dialect of Chios in his Vernacular Greek grammar by referring to the central status of Attic among the ancient dialects.\footnote{\citet[10]{Germano1622}. Cf. \citet[vi-vii]{Du1688}, who reminded his readers of ancient Greek dialectal diversity in order to explain vernacular Greek variation.}

Early modern scholars positioned the ancient Greek dialects in various ways vis-à-vis those of the Western European vernaculars. Yet how did they relate the Greek dialects to other languages they eagerly studied, primarily Latin and the so-called Oriental tongues, including Hebrew and Arabic?

\section{Latin: Uniquely uniform or diversified like Greek?}\label{sec:8.2}

In the early stages of the Renaissance, there was a common belief that, in contrast to Ancient Greek, Latin was uniform and therefore exempt from dialectal variation. This view was most famously championed by the Italian humanist Lorenzo Valla. For Valla, as I have shown, the unifying power of Latin was a great advantage, in sharp contrast to the internal linguistic discord among the Greeks. Later humanist scholars such as Aldus Manutius and Juan Luis Vives also adhered to the idea of Latin uniformity, which lived on throughout the early modern period.\footnote{See \citet[*.ii\textsc{\textsuperscript{v}}]{Manutius1496Aldus}; \citet[\textsc{x}.iii\textsc{\textsuperscript{v}}]{Vives1533}: “Romana dialectos non habet, unica est et simplex”. See \citet[11]{Trapp1990}. Cf. \citet[34--35]{Erasmus1528}; \citet[121]{Rapin1659}; \citet[29]{Wesley1736}; \citet[113--114]{Primatt1764}.} Unlike Valla, however, Manutius regarded it as a cause of poetical poverty. Vives, on the other hand, denied the existence of diversity in classical Latin, but at the same time felt compelled to grant that Latin had clearly changed over time – he was no doubt thinking of the traditional four-stage periodization offered by the Early Christian author Isidore of Seville (ca. 560–636).\footnote{See \citet[229--232]{Denecker2017} on Isidore’s division of the history of Latin into ancient, Latin, Roman, and mixed.} Valla, Manutius, and Vives all opposed Greek diversity directly to Latin uniformity. The illusion of Latin internal harmony seems to have obstructed an early recognition of the universality of dialectal variation and perhaps also a more avid interest in language-internal diversity in general. Regional variation in Latin was nevertheless gradually recognized in the sixteenth century.\footnote{For a modern linguistic study of regional variation within Latin, see the detailed account of \citet{Adams2007}.} A telling early example is the Flemish nobleman Georgius Haloinus’s (ca. 1470–1536/1537) \textit{Restauration of the Latin language}, a strong plea for usage and against grammar in learning correct Latin; this work was first published in 1533, but Haloinus had already composed it several decades earlier, around 1508. \citet[55]{Haloinus1978} stressed that Latin, too, was internally diversified and pointed to the alleged Paduan touch to Livy’s speech, his so-called “Patavinity” (\textit{Patauinitas}), to prove this. Livy’s Patavinity became a prototype and leitmotiv in demonstrating the existence of Latin dialects.\footnote{See e.g. also \citet[b.viii\textsc{\textsuperscript{v}}]{Castiglione1528}; \citet[*.iii\textsc{\textsuperscript{r}}]{Estienne1582}; \citet[174, 176]{Schottel1663}; \citet[311]{Rice1765}; \citet[: \textsc{lix}]{Mazzarella-farao1779}; \citet[203--204]{Ries1786}. See \citet{VanRooy2018a} for a more extensive discussion of sixteenth and seventeenth-century ideas about Livy’s Patavinity.} Some scholars even posited the existence of several other Latin varieties by analogy with Patavinity. In an eighteenth-century dissertation presented in Copenhagen, reference was made to Vergil’s alleged Mantuan dialect, his “Mantuanity” (\textit{Mantuanitas}; \citet[22]{Munthe1748}). Scholars went further than simply varying on the Patavinity theme, however. The Dutch scholar and politician Ernst Brinck (1582/1583–1649) even made a list of Latin dialects in his manuscript catalogue of linguistic specimens. Brinck referred to “dialects” (\textit{dialecti}) specific to a certain social or gender group – peasants or women, for instance – as well as to “dialects” characteristic of a certain locality, including Praeneste and Tusculum, noting some particular words for each variety.\footnote{\citet[56\textsc{\textsuperscript{v}}]{Brinck1615}. Cf. also \citet[43]{Stubbe1657}, where a list of Latin dialects is provided, albeit mixed up with Isidore of Seville’s four-stage periodization of Latin.}

Once it had been established that Latin also must have had its dialects, seven\-teenth-century scholars began to compare the Latin dialect context with its ancient Greek counterpart, always resulting in the a priori affirmation that they showed great differences. In his monograph on Livy’s Patavinity, Daniel Georg \citet[146]{Morhof1685} emphasized that the Greek language had greater dialectal variation and license than Latin because of the political diversity of ancient Greece, which he opposed to the highly centralized Roman Empire. This did not mean, however, that Latin did not have any dialects at all, and \citet[148--149]{Morhof1685} indeed listed several dialects of the language. About a decade later, the Hebraist Louis Thomassin (1619–1695) stressed that Latin, in comparison to Ancient Greek, “had few or no dialects”, with the exception of “a number of native and vernacular tongues of certain cities”. Thomassin attributed this to the Roman desire for unity and simplicity.\footnote{\citet[xix]{Thomassin1697}: “Graeca rursus lingua dialectis etiam statim ab initio luxuriata est. Quattuor quidem ex iis eminent; sed plurium supersunt uestigia. Porro singulae dialecti de iure mutandi uetera nouaque superstruendi uocabula cum suis dicendi modis, inter se quasi certatim contenderunt. Latina uero suae tum unitatis tum simplicitatis tenacior, paucas aut nullas habuit dialectos, si aliquot excipias quarundam ciuitatum patrios uernaculosque sermones”.}

It was, however, only in the eighteenth century that Latin dialects were described in explicitly negative terms in comparison to the ancient Greek dialects. The German theologian (Johannes) Nicolaus Hertling (1666–1710) contrasted Greek dialects with Latin varieties in esthetic terms. Greek had various dialects pleasant to the ears, which Latin and most other languages lacked, as they only contained corrupt dialects \citep[73]{Hertling1708}. The English grammarian Joseph Priestley (1733–1804) provided a more neutral and down-to-earth account. \citet[138]{Priestley1762} stressed that, in Latin, “dialects are unknown”, since these were not introduced into writings. “The \textit{Patavinity} of Livy is not to be perceived”. Put differently, “the \textit{Romans}, having one seat of power and of arts, allowed of no dialects”.\footnote{\citet[280]{Priestley1762}. See \citet[52]{Amsler1993}. Cf. \citet[49]{Galiani1779}; \citet[203--205]{Ries1786} for similar views.} In sum, Priestley did not deny that Latin dialects existed, but pointed out that they were no longer knowable, since, unlike the Greek dialects, they had not received written codification.

The diversity of the Romance languages that developed out of Latin was sometimes compared to the Greek dialects. The sixteenth-century Hellenist and orientalist Angelo Canini even forced the Romance tongues into the straitjacket of Greek as well as Oriental diversity. This involved \citet[\textsc{a}.iii\textsc{\textsuperscript{r}}]{Canini1554} interpreting both Greek and Latin as linguistic tetrads, the former consisting of Attic, Ionic, Doric, and Aeolic, and the latter encompassing Latin, Italian, French, and Spanish (cf. also \citealt{Canini1555}: a.3\textsc{\textsuperscript{v}}). Oddly enough, he did not elaborate on the precise relationship of Latin to the three Romance tongues he mentioned. Together with the Hebrew tetrad, consisting of Hebrew, Syriac, Arabic, and Ethiopian, the Greek and Latin tetrads constituted a linguistic triad, Canini suggested. This makes it clear that Canini’s scheme, into which Latin and three Romance tongues descending from it were forced in an ahistorical way, was very much numerologically inspired and not based on much linguistic evidence.

In a nutshell, Latin was regarded as uniform by many scholars throughout the entire early modern period. However, an alternative view emerged in the early sixteenth century, attributing regional variation to Latin, a realization which paved the way for the insight that regional variation was a universal phenomenon. In the seventeenth century, some scholars even attempted to list Latin dialects despite the scarcity of the evidence available to them. At the same time, they started to intuitively compare Latin to Greek variation with a focus on language-external, sociopolitical differences. In the eighteenth century, the superiority of Greek over Latin dialects was explicitly stressed on account of the literary value of the former. In other words, the main aim of the comparison was dissociation (cf. \sectref{sec:8.1.3} above). Exceptionally, Greek dialectal variation was put forward as a descriptive model for Romance diversity (cf. \sectref{sec:8.1.2} above).

\section{The Oriental language family and the Greek dialects}\label{sec:8.3}
\subsection{The Oriental dialects}\label{sec:8.3.1}

Early modern scholars compared the ancient Greek dialects very frequently to the Oriental tongues, up to the point that it seems to have become a refrain. Why was this the case? A large part of the answer can be found by looking at what the Swiss humanist Theodore Bibliander (1504/1509–1564) had to say about the interrelationship of a number of Oriental languages:

\begin{quote}
By means of a diligent investigation one knows that the Chaldean, Assyrian, Arabic, and Syriac tongues are so cognate that some take them to be one, which is true if the matter would be understood in terms of all dialects of the Greeks, which are called one Greek language.\footnote{\citet[58]{Bibliander1542}: “diligentique inquisitione cognitum est Chaldaeum, Assyrium, Arabicum, Syriacum sermonem ita finitimos, ut pro uno quidam accipiant, quod uerum est, si, ut omnes Graecorum dialecti una lingua Graeca dicuntur, ea res intelligatur”.}
\end{quote}

\citet[58--59]{Bibliander1542} proceeded by elaborating on the close connection between these Oriental languages and the primeval Hebrew tongue. It is obvious that he employed the example of the Greek context to justify the idea that these Oriental tongues actually constituted one language (see also \citealt{Metcalf2013}: 61). Bibliander used Greek dialectal variation as a touchstone and a descriptive point of reference to analyze and approach Oriental diversity, a method omnipresent in early modern descriptions of this language family.\footnote{Semitic variation was often also explained by referring to one’s native or another more familiar linguistic context. See e.g. \citet[41]{Purchas1613}; \citet[197]{Kircher1679}; \citet[b.1\textsc{\textsuperscript{v}}]{Le1696}; \citet[\textsc{i.}230, 4th sequence of pagination]{Chambers1728}; \citet[57--58]{Kals1752}.} Consider, for instance, how Bibliander’s pupil Conrad Gessner described Aramaic and its relationship to Hebrew:

\begin{quote}
Today, the more erudite men use the Chaldean language in Egypt and Ethiopia, as far as I hear. It is close to Hebrew and, perhaps, does not differ much more from it than Doric from the common Greek.\footnote{\citet[15\textsc{\textsuperscript{r}}]{Gessner1555}: “Chaldaica lingua hodie eruditiores in Aegypto et Aethiopia utuntur, ut audio. Hebraicae confinis est, nec forte multo amplius differt quam Dorica a Graeca communi”. See \citet[43]{Peters1970}. Cf. e.g. also \citet[325]{Rocca1591}, silently adopting Gessner’s phrase; \citet[459]{Saumaise1643a}; \citet[88]{Bagnati1732}; \citet[24]{Wesley1736}; \citet[22]{Eichhorn1780}.}
\end{quote}

The comparability of the Greek and Oriental contexts was especially prominent in the work of the Dutch orientalist Albert Schultens (1686–1750), who held that the four Oriental tongues Hebrew, Aramaic, Syriac, and Arabic derived from a now lost primeval tongue just like the four Greek dialects descended from a common Greek, “Pelasgian” mother language (\citealt{Schultens1739}: 234–235). \citet[\textsc{xcvi}]{Schultens1748} also believed that Attic and Hebrew were similar because of their tendency toward contractions, whereas Ionic and Arabic shared the property of being conservative varieties (see \citealt{Eskhult2015}: 85). Other scholars likewise paired a Greek dialect with an Oriental tongue. Like Schultens, some perceived similarities between Attic and Hebrew, whereas others connected Doric to Syriac because of their alleged broadness.\footnote{See \citet[425--432]{Lakemacher1730} for the Attic–Hebrew comparison. For Doric and Syriac broadness, see Chapter 5, \sectref{sec:5.7}.} Comparing Greek to Oriental variation is truly a topos throughout Schultens’s work, in which Greek diversity always served as a point of reference for understanding Oriental variation.\footnote{See e.g. \citet[490]{Schultens1769}, \citet[4]{Schultens1732}, \citet[5]{Schultens1737}, \citet[19--21]{Schultens1738a}, \citet[106--107, stressing that the Oriental and the Germanic contexts were less comparable]{Schultens1738b}; \citet[187]{Schultens1739}, \citet[\textsc{xcvi}]{Schultens1748}; Schultens in Eskhult (fc.) [ca. 1748–1750]: §\textsc{xxvii}. On this topos in Schultens’s work, see also \citet[105]{Fuck1955}; \citet[707]{Covington1979}; Eskhult (fc.). Cf. in Schultens’s tracks \citet[5]{Polier1739}; \citet{Groddeck1747}.} This procedure occurred in the work of other scholars as well, whether or not in combination with a reference to vernacular variation (see e.g. \citealt{Bochart1646}: 778; \citealt{Blount1680}: 71–72).

Some scholars even claimed that Greek dialects differed more from each other than the Oriental tongues, thus dissociating both linguistic contexts (cf. \sectref{sec:8.1.3} above). Angelo \citet[34]{Canini1554} already did so when discussing verb conjugations in his 1554 comparative grammar of a number of Oriental tongues (see \citealt{Contini1994}: 50; \citealt{Kessler-mesguich2013}: 211). The idea was expressed more clearly still by the orientalist Christian Ravis (Raue/Ravius; 1613–1677).\footnote{\citet[*.2\textsc{\textsuperscript{r}}]{Ravis1646}. See e.g. also \citet[51--52]{Hunt1739}; \citet[\textsc{xxvi}]{Groddeck1747}.} \citet[48]{Ravis1650} also emphasized that even though there were separate chairs for each Oriental language at universities, but not for the Greek dialects, this institutional fact should not lead to the conclusion that Hebrew, Syriac, Arabic, and so on were truly “divers tongues”. In fact, just like the Greek language, they were “only one”. The practice of comparing Oriental to Greek diversity was criticized by Johann Heinrich Hottinger (1620–1667), who explicitly reacted against his colleague Christian Ravis’s views on the matter. \citeauthor{Hottinger1661}'s two main points were that Hebrew was not an Oriental dialect, but the primeval language, and that the differences among the Oriental tongues were much greater than those among the Greek dialects (\citeyear[a.3\textsc{\textsuperscript{v}}–a.4\textsc{\textsuperscript{r}}]{Hottinger1661}) . The Dutch orientalist Sebald Rau (Sebaldus Ravius; ca. 1725–1818) adopted a similar perspective. \citet[20--21]{Rau1770} argued that the Greek dialects were spoken by one nation, whereas the “Oriental dialects” (\textit{dialecti Orientales}) were current among different nations, living in various climates and having diverging ways of living, customs, and rites. This resulted in greater linguistic differences, he argued.

In rare instances, the Oriental context served as a reference point to understand developments in the history of the Greek language (cf. \sectref{sec:8.1.1} above). A late seventeenth-century Hellenist active in Leipzig used the alleged decay and dialectal diversification of the Hebrew language during the Babylonian captivity in the sixth century \textsc{bc} to clarify the decline of the Greek language (\citealt{Eling1691}: 318–319). In a sixteenth-century handbook on the Greek literary dialects, the Oriental context was cited as an additional example, next to the grammarian’s native one, to explain differences in elegance among the ancient Greek varieties (\citealt{Walper1589}: 61–62).

\subsection{Hebrew dialects}

As to variation within Hebrew, identified by many authors as the primeval language spoken by Adam and Eve and confused at the Tower of Babel, early modern opinions differed greatly.\footnote{In the present section I discuss views on variation within Hebrew, thus excluding cases in which Semitic tongues such as Arabic were dubbed “dialects” of Hebrew (see e.g. \citealt{Bochart1646}: 56; \citealt{Martin1737}: 134–135).} Some scholars were eager to claim that Hebrew did not have any dialects. The Leipzig theologian Bartholomaeus Mayer (1598–1631) did so while citing Lorenzo Valla’s comparison of Latin and Ancient Greek \citep[10]{Mayer1629}. \citet[\textsc{b.3}\textsc{\textsuperscript{v}}]{Junius1579} took a more moderate stance, as he contrasted the immense variability of Greek to the relatively uniform Hebrew tongue, claiming that the latter did not have as many dialects as Ancient Greek (cf. Chapter \ref{chap:7}). The English orientalist Thomas Greaves (1612–1676) also attributed dialects to both Hebrew and Ancient Greek, while praising Arabic for lacking them in his \textit{Oration on the utility and preeminence of the Arabic language}, held at Oxford in 1637 and published there in 1639.\footnote{See \citet[19--20]{Greaves1639}, who inspired \citet[60]{Leigh1656} and \citet[73]{Blount1680}.}

Scholars often found it sufficient to prove regional variation within Hebrew by simply referring to the shibboleth incident in the Old Testament at Judges 12.5–6 or to the supposed Galilean character of St Peter’s speech, alluded to in the New Testament at Matthew 26.73.\footnote{See e.g. \citet[6]{Bovelles1533}; \citet[\textsc{b.3}\textsc{\textsuperscript{v}}]{Bachmann1625}; \citet[102]{Weemes1632}; \citet[2]{Wyss1650}; \citet[18]{Walton1657}; \citet[180]{Webb1669}; \citet[6]{Kiesling1712}; Salvini in \citet[103]{Muratori1724}; \citet[30]{Hauptmann1751}; \citet[13--14]{Hof1772}. For the relevant biblical passages, see also \citet[199--200]{VanRooy2018b}.} Gradually, however, philologists focusing on the Bible started to recognize that St Peter was more likely to have spoken a variety of Aramaic or – in early modern terms – of (Chaldeo-)Syriac (e.g. \citealt{Pfeiffer1663}), whereas others denied that the shibboleth incident was evidence of variation within Hebrew (e.g. \citealt{Mayer1629}: 10–11). Sometimes, they developed historically nuanced answers to the question of whether Hebrew was dialectally diversified. In a dissertation entirely devoted to the question of St Peter’s speech and presented in Wittenberg, a periodization of Hebrew was designed in order to show the development of the language. The authors of the dissertation argued, among other things, that Hebrew was originally a unitary language like Latin, but underwent dialectal diversification after the Babylonian captivity (\citealt{Pfeiffer1663}: \textsc{a.4}\textsc{\textsuperscript{v}}). In these more focused investigations into the question of whether Hebrew had dialects, the Greek dialects occupied a marginal position at best.

\subsection{Summary}

Briefly put, the Greek dialects were frequently used as a point of reference to map out the close genealogical relationship among the Oriental tongues, which aroused great philological interest in the early modern era. Scholars were struck by the close kinship between these languages and tried to find an adequate way to express it. Since most orientalists were also trained as Hellenists, many of them thought of the Greek dialects as a revealing parallel. These, too, were closely cognate, despite their many formal differences. What is more, the Greek dialects had received written codification, just like the Oriental tongues. These two similarities made ancient Greek diversity a helpful reference point for early modern orientalists. Some of them went a step further and claimed that the Oriental tongues were even more alike than the Greek dialects. Such an exaggerated conception was usually rejected by orientalists in the seventeenth and eighteenth century, who preferred to maintain the comparability of both contexts. This stance culminated in the work of the Dutch philologist Albert Schultens, who formulated the Greek–Oriental simile in nearly every one of his publications. Finally, as with Latin, scholars struggled to assert Hebrew uniformity, even though from the sixteenth century onward there were voices admitting that Hebrew, too, that sacred tongue often identified with the language of Adam, had its dialects just like Greek, Latin, and the vernaculars.

\section{Conclusion: Between exemplarity and particularity}\label{sec:8.4}

In the present chapter, I have attempted to demonstrate that early modern scholars compared and contrasted the linguistic diversity of Ancient Greek to dialectal or dialect-like variation in a wide range of other languages and language families. This occurred most frequently with reference to Oriental diversity and dialectal variation in Western European vernaculars, especially Italian, German, French, and English. Modern scholars have often emphasized the exemplarity of the Greek context to grasp or ennoble vernacular diversity. For example, Peter \citet[35--36]{Burke2004} states that, for the early modern awareness of dialectal variation, “the model situation was that of Ancient Greece with its Ionic, Doric, Attic, and other varieties of speech”.\footnote{Cf. \citet[923]{Haugen1966}; \citet[216]{Giard1992}: “la question des dialectes portée au passif des vernaculaires est considérée autrement dès lors qu’on remarque la signification et l’usage positifs qu’ils avaient en grec”.} In selecting the variety to be adopted as the literary standard in the so-called language questions during the Renaissance, the Greek context indeed seems to have functioned as a paradigmatic touchstone and was taken as a noble and close parallel to vernacular dialectal diversity (\citealt{Alinei1984}; \citealt{Trovato1984}; \citealt{Trapp1990}). Moreover, the Greek example with its allegedly dialectally mixed Koine suggested that vernacular dialects, too, could contribute to the literary standard language under construction. However, as I have endeavored to demonstrate in this chapter, this is only part of the picture, albeit a very important one. The situation was very different in early modern manuals for Ancient Greek. As a matter of fact, there, the Greek context did not serve as a model at all. Instead, the grammarians needed to explain it by referring to the native vernacular context of their intended readership. In other words, ancient Greek diversity constituted a phenomenon that very often required elucidation. This was especially common in works published in German-speaking areas, where Greek studies flourished throughout the entire early modern period and vernacular dialectal diversity was not easily transcended by an established standard language. In order to maintain the comparability of Greek with other dialect contexts, early modern scholars could tone down some of the differences between them so as to emphasize their similarities. To this end, they projected certain characteristics of one context onto the other, a process of which they were not always fully aware (cf. \citealt{Alinei1980}: 20).

Even though most early modern scholars seem to have assumed that ancient Greek dialectal diversity was highly similar to variation within other languages, the point of comparison being the close kinship among the dialects, there were nonetheless also a considerable number of authors who emphasized the particular place of ancient Greek diversity, certainly during later stages of the early modern period and particularly in eighteenth-century France. In the large majority of cases, the incomparability of Ancient Greek with another dialect context was mainly motivated by language-external circumstances. This included, most importantly, the political diversity of ancient Greece and the literary and codified status of the canonical Greek dialects. Several scholars contrasted this to cases of political centralization, as in France, or to the existence of a sole written standard, as in the case of German. Authors emphasizing comparability likewise concentrated on language-external circumstances, but less exclusively so. The relative lack of reference to specific linguistic features in this discussion may seem remarkable at first sight, but this should be seen in connection with the main goal of the early modern discourse on comparability; this consisted in making a statement – either explanatory, justificatory, descriptive, or dissociating – about the precise status of a specific dialect situation in its broader sociolinguistic and cultural context rather than about the actual linguistic forms of the dialects.

A scholar’s emphasis on comparability or lack thereof depended to a large extent on his discursive intentions as well as his underlying language ideology. For instance, when explaining ancient Greek diversity in a grammar, comparability was usually stressed, since the grammarian hoped to help his readers understand the status of the Greek dialects by referring to a similar and more familiar context. Early modern literary critics, however, tended to deny comparability, as they emphasized the literary insignificance of vernacular dialects, which stood in glaring contrast to the high esteem of the ancient Greek dialects. This lack of comparability made it to the mind of certain scholars impossible to apply the term “dialect” to any linguistic context other than Ancient Greek. Put differently, early modern scholars vacillated between exemplarity and particularity. On the one hand, the ancient Greek linguistic situation was used as a model to approach variation within other languages or language families or turned out to be the situation in need of clarification by means of a more familiar vernacular example. On the other hand, scholars could stress, whenever it suited them, the extreme idiosyncrasy of the Greek dialects and the exceptional historical coincidence that these speech forms have been eagerly used as literary media.

The level of competence in Ancient Greek was also of relevance for the discourse on comparability. It seems that the better a scholar’s competence was, the more detailed their comparison tended to be and the more likely it was that their ideas were picked up by later scholars, as in the cases of Henri Estienne, Charles Rollin, and Albert Schultens. Inspired by their thorough knowledge of Greek – Estienne even claimed it to be his second language before Latin – they put forward various ideas on the (in)comparability of Greek with vernacular or Oriental diversity, all with considerable influence.

As a final point, I want to add that not all comparisons of different dialect contexts involved the ancient Greek dialects, even though this Greek-free approach occurred with a noticeably lower frequency. The relative rarity of such instances demonstrates the tremendous importance of ancient Greek diversity in triggering early modern interest in dialectal variation as a general phenomenon affecting every language. It also clearly indicates that the widespread comparison of dialect contexts was largely an early modern development, catalyzed by the Renaissance revival of Greek studies, all the more since the procedure was so exceptional in the Middle Ages. In sum, the well-chosen words of the Austrian Germanist Max Hermann Jellinek (1868–1938), which pertained specifically to German grammarians, may well be generalized: early modern scholars “cannot speak of dialects and written language without calling in Attic, Ionic, Doric, and the Koine”.\footnote{\citet[21]{Jellinek1913}: “Diese Männer können nicht von Dialekten und Schriftsprache reden, ohne das Attische, Jonische, Dorische, Aeolische und die κoινή aufmarschieren zu lassen”.}

