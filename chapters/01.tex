\chapter{Introduction}\label{chap:1}

Early modern scholarship on the Greek dialects has thus far attracted almost no attention at all (cf. \citealt{Ben-Tov2009}: 157–158). It is the subject of only a handful of case studies (e.g. \citealt{VanRooy2016c}), with a small number of scholars making some cursory comments on the matter (e.g. \citealt{Botley2010}; \citealt{Roelcke2014}: esp. 246–254, 352). The neglect is glaringly apparent from the entry “Classification of dialects” in Brill’s \textit{Encyclopedia of \ili{Ancient Greek} language and linguistics} \citep{Finkelberg2014}, where a discussion of ancient \isi{dialectology} is immediately followed by an account of modern classifications of the dialects (see already \citealt{VanRooy2016a}). It is ancient and medieval ideas on the Greek dialects which have taken center stage in historiographical studies (see \citealt{VanRooy2018b} for a recent state of the art). As the subject is largely unexplored for the early modern period, it represents an untapped vein of precious information for the reader interested in language and history. But it also harbors dangers. For instance, the fact that this book deals with a topic that has never been systematically studied before makes it not only impossible but also simply undesirable to attempt to say the last word on the topic. Instead, the reader should regard this book as a first exploration of the subject matter.

In this introduction, I want to achieve two things. To start with, I will offer the reader a concise history of the Greek language, so that they understand the central place of dialectal variation in it. In a second step, I briefly outline why early modern intellectuals developed an interest in the Greek dialects and how this manifested itself in their scholarly production.

\section{The history of the Greek language in a nutshell}\label{sec:1.1}

An Indo-European language, Greek was anciently spoken in and around the present-day country of Greece and in Greek colonies scattered across the islands and coasts of the Mediterranean and the Black Sea. Before it was united under Macedonian rule in the fourth century \textsc{bc}, the region constituted a patchwork of polities, lacking a central government and a \isi{common language}. This allowed for the official and \is{literary usage}literary use of local varieties of \ili{Ancient Greek}, which most scholars today divide into six main dialect branches: \ili{Aeolic}, \ili{Arcado-Cypriot}, \ili{Attic}–\ili{Ionic}, \ili{Doric}, \il{Ancient Greek!North-West}North-West Greek, and \ili{Pamphylian} (see e.g. \citealt{Colvin2010}; \citealt{Finkelberg2014}). However, extant dialect literature and \is{epigraphy}inscriptions reflect the actual spoken varieties only to a limited extent (\citealt{Colvin2010}: 201–202). According to Stephen \citet[300, 303]{Colvin1999}, there was \isi{mutual intelligibility} among most dialects, but Albio Cesare \citet[4--5]{Cassio2016} convincingly casts doubt on this assumption. Over time, certain dialects became tied up with literary genres rather than locality. \ili{Aeolic} became established as the dialect of lyric poety; \ili{Attic} of, among other things, \isi{rhetoric} and the dialogic parts of tragedy and comedy; \ili{Doric} of \isi{bucolic poetry} and the choral odes in tragedy; and \ili{Ionic} of science and \isi{historiography}. The \isi{literary usage} of dialects was, however, not straightforward. Even though most prose authors opted for one dialect – \iai{Herodotus} (ca. 485–424 \textsc{bc}), for example, used \ili{Ionic} – poets frequently combined features from different dialects in their work. For this practice, Homer’s epic poems (8th cent. \textsc{bc}) were the main model. Even though the \textit{Iliad} and \textit{Odyssey} principally exhibit \ili{Ionic} properties, they also have \ili{Aeolic} and \ili{Attic} features and contain archaisms, traces of earlier phases of the Greek language (\citealt{Hackstein2010}: 401–408; \citealt{Tribulato2010}: 390). As a result, one verse could display features of various dialects. Homer’s artificial language constituted, so to speak, the first literary Greek koine, thus enhancing the Greek feeling of linguistic unity (\citealt{Morpurgo1987}: 15–19; \citealt{Colvin2010}: 200).

In the Hellenistic era, the Greek \ili{Koine} – short for \textit{hē koinḕ diálektos} (ἡ κoινὴ διάλεκτoς), ‘the common speech’ – developed out of Great \ili{Attic}. The latter was a written variety of \ili{Attic} used in administration and influenced by \ili{Ionic} (\citealt{Horrocks2010}: 73–77, 80–83). The \ili{Koine} normally lacked \ili{Attic} features that were considered too local or too complex. These included the \ili{Attic} double tau which corresponded to the less regionally marked double sigma in, for instance, \textit{thálatta} (θάλαττα) versus \textit{thálassa} (θάλασσα), ‘sea’. Moreover, some properties of \ili{Attic}, including its complicated verbal morphology, were adopted in the \ili{Koine} in a simplified or regularized form (\citealt{Brixhe2010}: 230; \citealt{Horrocks2010}: 75, 82).

The reliance on \ili{Attic} as the basis for the \isi{common language} was a consequence of the important political status of Athens in the fifth century \textsc{bc}, the Golden Age of Pericles, and of its immense literary prestige, which is still recognized today; \ili{Attic} is the variety through which students usually start to learn \ili{Ancient Greek} today. The \ili{Koine} rapidly spread across the Greek-speaking world, vastly enlarged by Alexander the Great’s (356–323 \textsc{bc}) conquests. Over time, local koines came into being, some of which eventually developed into different \isi{vernacular} Greek dialects (\citealt{Brixhe2010}: 244–249). As a matter of fact, with the exception of \ili{Tsakonian}, a form of Greek descending from an ancient \ili{Laconian} \ili{Doric} variety – although much influenced by the \ili{Koine} \citep[88]{Horrocks2010} – all modern dialects derive from forms of \ili{Koine} Greek. Indeed, the \ili{Koine} had made the ancient dialects virtually \is{dialect extinction}extinct by late antiquity (\citealt{Horrocks2010}: 84, 88). The \ili{Koine} itself was diversified, too, not only regionally, but also in terms of social strata and registers as well as diachronically. Its conception as a unitary linguistic entity was therefore largely an ideal constructed by grammarians of the period, much like the \is{standard (language)}standard languages of today (\citealt{Brixhe2010}: 230–231; cf. also \citealt{VanRooy2016b}).

In the Hellenistic and Roman periods, most literary Greek texts were still written in the ancient dialects, with the so-called \is{Atticism}Atticistic movement flourishing in the second century \textsc{ad}. In this movement, authors consciously took Classical \ili{Attic} texts as their stylistic, literary, and linguistic models \citep[42]{Whitmarsh2005}. Varieties of the \ili{Koine} were, however, also very popular, as evidenced by the Greek of the \isi{New Testament} and that of the countless Egyptian papyri (\citealt{Evans2010}). This gradually led to a diaglossic situation, with pure \ili{Attic} at the highest end of the prestige spectrum and \isi{vernacular} varieties of the common people at the lowest. Note that I employ the concept of \textsc{diaglossia}, referring to a linguistic context in which there is a spectrum of varieties between the low \isi{vernacular} dialects, on the one hand, and the high varieties, on the other (see e.g. \citealt{Auer2005,Rutten2016}). This \isi{diaglossia} continued throughout the Byzantine and early modern era and well into modernity, during which it polarized more radically as a \textit{di}glossia between the low \isi{vernacular} variety, \il{Early Modern Greek@(Early) Modern Greek!\textit{Dimotikí}}\textit{Dimotikí} (Δημoτική), and the \il{Early Modern Greek@(Early) Modern Greek!\textit{Katharévousa}}\textit{Katharévousa} (Kαθαρεύoυσα) tongue, reserved for high registers. The \textit{Katharévousa}, ‘the pure tongue’, was a mixed learned language created out of Vernacular and \ili{Ancient Greek}, whereas the \textit{Dimotikí} referred to popular varieties of Greek, strongly influenced by centuries of \ili{Venetian} and especially Ottoman rule.

The \isi{diglossia} was largely resolved with the replacement of the \textit{Katharévousa} tongue by \il{Early Modern Greek@(Early) Modern Greek!Standard Modern}Standard Modern Greek as the official language of Greece in 1976, two years after the military junta had fallen and the Third Hellenic Republic was installed. This new \is{standard (language)}standard variety had its base in Demotic Greek, but was elaborated by many features of the \textit{Katharévousa}. In the meantime, \isi{vernacular} Greek dialects of various kinds continue to be spoken all across Greece, whereas the Greek \isi{Orthodox Church} still makes use of the \textit{Katharévousa}.

In conclusion, dialects have played a major role in the long history of the Greek language, especially in antiquity, when they were eagerly used for \is{epigraphy}epigraphic, administrative, and especially literary purposes.

\section[The dialects of ancient Greece in premodern scholarship]{The dialects of ancient Greece in premodern scholarship: A typology of sources}\label{sec:1.2}

It was because of their literary importance that the dialects of the \ili{Ancient Greek} language were primarily studied by scholars of the premodern era. We know that there was a lively tradition of ancient studies on the matter, likely initiated by the first-century \textsc{bc} grammarian \iai{Tryphon}, active in Alexandria, Egypt. Yet only a distorted picture can be reconstructed of this early history, largely filtered through Byzantine treatises that are not all of the highest quality, to put it mildly. Greek scholarship on the dialects was very much characterized by a hands-on approach. Grammarians devoted their efforts in the first place to describing the features of the canonical literary dialects \ili{Attic}, \ili{Ionic}, \ili{Doric}, \ili{Aeolic}, and to a lesser extent the \ili{Koine}. Out of these data \iai{Tryphon} and his successors distilled a framework of word modifications perceivable across different varieties of Greek. One might expect this to have given rise to a comparative approach toward the dialects, but nothing could be farther from the truth. Extant source texts show that the Greeks did not do much more than sum up the features of individual dialects. The main focus was on how they differed from common Greek. This was not any prehistoric \ili{Proto-Greek} language, but could mean only one of two things: either what (most) Greek dialects had in common or the Greek \ili{Koine}. The former view was typical of ancient grammarians, whereas the latter conception seems to have prevailed principally among Byzantine scholars. It is, however, not always an easy task to distinguish between both conceptions.

Treatises on the dialects were indispensable instruments for students of the linguistically diverse literature of ancient Greece, and their appearance coincided more or less with the \is{dialect extinction}near-extinction of the Greek dialects of antiquity. Yet when fourteenth- and fifteenth-century Italian humanists started to direct their attention to Greek language and literature, knowledge of which had largely vanished in medieval Western Europe, these instruments were inaccessible for decades. They remained in manuscripts within the confines of the crumbling Byzantine empire. Even when these manuscripts gradually made their way to Italy, they did not make popular reading material. They were too difficult for Italian students, who did not have a variety of Greek as their \isi{mother tongue} as Byzantine students did. Instead, the Italians relied on the teachings of Byzantine teachers who traveled to the West from the end of the fourteenth century onward (see e.g. \citealt{Harris1995,Botley2010,Wilson2016}). These teachers soon realized that the Byzantine language manuals were too complex for their new audience. They met their students halfway and produced simplified grammars of Greek, tailored to the needs of their Italian audience; these described a more or less unitary form of Greek, in fact a mixture of \ili{Koine}, \ili{Attic}, and \ili{Ionic} elements \citep[123]{Ciccolella2008}. Strange dialectal features were kept to a minimum in these introductory handbooks, as can be gathered from the concise overview of early Renaissance grammars by Paul \citet{Botley2010}. However, they were not entirely absent. For example, the first Byzantine scholar to successfully teach Greek in Italy, Manuel Chrysoloras (ca. 1355–1406), explicitly noted in his grammar that in the \ili{Attic} dialect the \isi{nominative} and the \isi{vocative} cases are formally identical. This misinformation he took over from Greek tradition, and in particular from a popular Byzantine treatise on the dialects by \iai{Gregory of Corinth}, which I discuss at greater length below.\footnote{See \textcite[166 n.70]{Botley2010}. \citet[20]{Chrysoloras1512}: “καθόλoυ μὲν oἱ Ἀττικoὶ τὰς αὐτὰς ἔχoυσιν ὀρθὰς καὶ κλητικάς”. I quote from a Renaissance edition, as the grammar has not yet been critically edited to modern standards (see \citealt{Nuti2013}: 241 n.8).} While it is certainly true that Byzantine scholars simplified Greek grammar and reduced dialect information in their handbooks for the benefit of their Italian audience, it seems that this was not motivated by didactic concerns alone. In fact, these Greek teachers were unlikely to have been experts in the matter of the dialects themselves. It is revealing in this regard that the Italian humanist Francesco Filelfo (1398–1481) lamented in 1441 that even in Constantinople no \ili{Aeolic} was taught (\citealt[88 n.4]{Rotolo1973}; cf. \citealt{Botley2010}: 71–114).

In the second half of the fifteenth century, a change was underway, and it is remarkable that Western students of Greek played a major role in it rather than their Byzantine teachers. The topic of the dialects, marginally present at best in the early Renaissance, drew more and more attention from the 1460s onward. Three events of the final decades of the fifteenth century mark this change. In 1460, the Constantinopolitan grammarian Constantine Lascaris (1434–1501), active in Italy, published a brief work in which he treated the Greek pronoun from a new angle. Writing for advanced students interested in poetry, he described the way in which the pronominal system varied across different dialects, a matter that must have given many students a headache. The work first circulated in manuscript and was printed only after some forty years (see \citealt{Botley2010}: 26, 124, 175 n.272). Among the Byzantine teachers active in the West, Lascaris was exceptional in providing his students with a treatise related to the thorny issue of the dialects.

Western humanists, however, turned \isi{dialectology} – though the term was not coined before 1650 – into a subfield of Greek philology. But the first step was taken not in Italy, but in Paris, which experienced an extended first flourishing of Greek studies mainly thanks to the émigré George Hermonymus of Sparta (ca. 1430–ca. 1509). In the winter of 1477/1478, \ia{Reuchlin, Johann|(}Johann Reuchlin (1455–1522), Hermonymus’s promising student from Pforzheim, compiled a \textit{Booklet on the four differences of the Greek language} (ed. \citealt{VanRooy2014}). Although presenting it as an original work of his own which he based on Byzantine sources, Reuchlin did in fact not much more than translate a Greek manuscript treatise of questionable quality into \ili{Latin}. \ia{Reuchlin, Johann|)}Reuchlin’s booklet did not circulate widely and survives in two manuscripts only, and with good reason, since it is hard to see how a student of Greek would have benefitted from it.

An event of much greater significance occurred in August 1496, when Aldus Manutius (ca. 1449/1451–1515) issued in his \ili{Venetian} office a large collection of Greek grammatical treatises, deservedly called a “treasure” (\textit{thesaurus}). Toward the end of this collection, the reader could find three treatises on the Greek dialects. The first consisted, in fact, of two abbreviated redactions of a work entitled \textit{Tekhniká} (Tεχνικά), which can be translated as \textit{Matters relating to the art of grammar}. It is usually attributed to \iai{John the Grammarian}, likely to be identified with the early Byzantine philosopher \ia{John the Grammarian}John Philoponus (ca. 490–575), and still awaits a critical edition.\footnote{This is also why I quote this text from the first edition of 1496 in the present book.} The second treatise was an excerpt from an anonymous biography of Homer usually ascribed to \iai{Plutarch}; it was less extended in scope than John the Grammarian’s work, since it focused on the dialects as they are used by Homer.\footnote{See the edition in \citet{Kindstrand1990}. A Greek–English edition is available in \citet{Keaney1996}. See also \citet{VanRooy2018c} on the Renaissance fate of the treatise.} The third treatise, entitled \textit{On the dialects}, was the longest; its author, the Byzantine grammarian and theologian \iai{Gregory of Corinth} (11th/12th century), drew on the work of \iai{Tryphon} and \ia{John the Grammarian}John Philoponus as well as on his personal reading of the classics, as he explained in his proem.\footnote{For a modern critical edition, see the unfortunately unpublished PhD dissertation of Didier \citet{Xhardez1991}.} Manutius cannot be credited with being the first to have published these works – the second and third had been separately printed some years earlier – but he was the first to print them together. In fact, these three texts became something of a dialectological canon in the early modern period (see especially the appendix to \citealt{Trovato1984}). Manutius himself further contributed to their canonization by republishing them in 1512 with an accompanying \ili{Latin} translation. In no time, these texts enjoyed numerous reissues, frequently in \ili{Latin} translation and often appended to other helpful instruments for students of Greek, such as dictionaries.

The success of these treatises indicates that there was a market for handbooks discussing the Greek dialects in the early sixteenth century. Indeed, not only did the grammars of Western humanists, including the \isi{Protestant} leader Philipp Melanchthon (1497–1560), increasingly treat the subject matter, but the sixteenth century also witnessed the appearance of the first original monographs on this topic. The earliest of its kind was a popular booklet entitled \textit{On the diverse dialects of the Greek inflections in verbs as well as in nouns, drawn from Corinth, \iai{John the Grammarian}, \iai{Plutarch}, \ia{John the Grammarian}John Philoponus, and others of the same order} \citep{Amerot1530}. Originally part of a 1520 Greek grammar printed in Leuven, the work was published independently for the first time in 1530 with Gérard Morrhe in Paris. Its author was Adrien Amerot (ca. 1495–1560), a young Hellenist from Soissons and professor of Greek in Leuven for the greater part of his life (see \citealt{Hummel1999}; \citealt{VanRooyFcb}). Soon other Hellenists followed suit, and a tradition of \ili{Latin} handbooks on the Greek dialects quickly emerged in Europe. These include, but are not limited to:

{\sloppy\begin{itemize}
\item 
Martin Ruland’s (1532–1602) voluminous \textit{Five books on the Greek language and all of its dialects} (Zurich, \citeyear{Ruland1556});

\item 
the \textit{Booklet on the Greek dialects} (Paris, \citeyear{Vuidius1569}) by the further unknown Frenchman Robertus Vuidius;

\item 
the French Jesuit Guillaume Baile’s (1557–1620) \textit{Booklet on the Greek dialects} (Paris, \citeyear{Baile1588});

\item 
the Marburg professor Otto Walper’s (1543–1624) successful \textit{On the principal dialects of the Greek language} (Frankfurt am Main, \citeyear{Walper1589});

\item 
Jakob Zwinger’s (1569–1610) \textit{Outline of the Greek dialects} (Basel, \citeyear{Zwinger1605});

\item 
the \textit{Easy and compendious treatise of the dialects of the Greek language} (Paris, \citeyear{Merigon1621}) of the somewhat enigmatic figure Pierre Bertrand Mérigon;

\item 
Caspar Wyss’s (1604/1605–1659) \textit{Sacred dialectology} (Zurich, \citeyear{Wyss1650});

\item 
the \textit{Greek prosody, with dialectology} (Tübingen, \citeyear{Bregius1684}) by the obscure Hellenist Johannes Bregius;

\item 
Michael Maittaire’s (1668–1747) influential \textit{Dialects of the Greek language} (London, \citeyear{Maittaire1706});

\item 
the \textit{Dialectology in paradigms} (Neubrandenburg, \citeyear{Nibbe1755}) by the poorly known scholar Johann Barthold Nibbe;

\item 
Johann Friedrich Facius’s (1750–1825) \textit{Compendium of Greek dialects} (Nuremberg, \citeyear{Facius1782}).
\end{itemize}}

Such handbooks constitute the core corpus of the present study, alongside Greek grammars which devote substantial chapters to the matter of the dialects. An early example of the latter category is the first Greek grammar by a \ili{Spanish} Hellenist: Francisco de Vergara’s († 1545) \textit{Five books on the grammar of the Greek language}, originally published in 1537 in Alcalá de Henares, with several reprints in Paris (1545, 1550, 1554, 1557) and Cologne (1552, 1588). The fifth book of the work, covering 27 pages, is entirely devoted to the dialects (\citealt{Vergara1537}: 209–235). Even though most of these manuals were principally aimed at familiarizing the prospective philologist with the different literary forms of Greek, they also contain many revealing observations that made a lasting impact on Western linguistic thought, well before the study of language was institutionalized in the nineteenth century.

Apart from handbooks, there are also a number of scholarly, mainly philological works elaborating at length on the Greek dialects, including Claude de \citeauthor{Saumaise1643a}’s (1588–1653) voluminous \textit{Commentary on the \ili{Hellenistic} tongue, deciding the controversy on the \ili{Hellenistic} tongue and thoroughly treating the origin and dialects of the Greek language} (Leiden, \citeyear{Saumaise1643a}), a result of his dispute with his rival Daniel Heinsius (1580–1655), and Friedrich Gedike’s (1754–1803) German essay \textit{On the dialects, especially the Greek} (Berlin, \citeyear{Gedike1782}). Other texts of various genres in which the Greek dialects occupy a prominent or revealing place are likewise involved in the analysis; these include, most importantly, grammars and lexicons of other languages (e.g. \citealt{Gill1619}), philological studies of non-Greek texts (e.g. \citealt{Schultens1748}), and antiquarian works on Greece or related areas (e.g. \citealt{Castelli1769}).

\section{Content in context}\label{sec:1.3}

When I analyzed the source texts, a number of themes immediately caught my attention, and for this reason it seemed wise to adopt a thematic rather than a strictly chronological approach in presenting the results of my research. This subject-based structure will, I hope, enhance the coherence and readability of the book. Each thematic chapter will, however, have a diachronic dimension, in that where possible I will first sketch ancient and medieval ideas, which were usually the starting point for early modern Hellenists. For the pre-Renaissance era, I will principally draw on existing scholarship. Significant evolutions and major points of disagreement in early modern thought will also be charted.

Which themes have I selected? I have opted to concentrate on the issues that took center stage in early modern discussions; all of these can be framed within broader intellectual currents, either scientific, philological, \is{historiography}historiographical, eth\-no-geographical, or religious, as linguistics – let alone \isi{dialectology} – was not yet an autonomous research field. The seven main chapters (Chapters~\ref{chap:2}–\ref{chap:8}) of this study are devoted to early modern ideas about, and approaches to, the Greek dialects, which are contextualized throughout and especially in Chapters~\ref{chap:3}, \ref{chap:4}, \ref{chap:7}, and~\ref{chap:8}. In \chapref{chap:2}, the many different attempts at classifying the Greek dialects are treated. I discuss the main motivation behind the early modern interest in this topic, philology, in the third chapter. \chapref{chap:4} analyzes views on two specific varieties of Greek that posed problems to early modern scholars: the Greek of Homer’s epic poems and that of the Bible, both speech forms still debated by specialists today. Chapter~\ref{chap:5} concerns the early modern attempts at writing the \isi{linguistic history} of the Greek language, which constituted a great challenge due to its many dialects. In \chapref{chap:6}, I demonstrate how early modern Hellenists tried to make sense of the great linguistic variation among the Greek dialects in describing their particularities. \chapref{chap:7} treats the way in which the Greek dialects were related to other aspects of ancient Greece, including its literary tradition as well as its diversified geopolitical and ethnic constitution. Finally, in \chapref{chap:8}, I zoom out in order to chart the different and often contradictory usages of \ili{Ancient Greek} dialect diversity as a point of reference for understanding the dialects of other languages, especially the \isi{vernacular} tongues that were being emancipated in the early modern era.

My content-in-context approach is indispensable, since the book intends to contribute not only to the history of linguistic thought, but also to intellectual history and especially to the growing subfield of the history of Hellenism. In doing so, it aims to appeal to intellectual historians as well as to linguists and classicists interested in the long and understudied history of their disciplines.
