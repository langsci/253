\chapter{Using words like wax: The many mutations of the Greek dialects}\label{chap:6}

\begin{quote}
“They add, subtract, transmute, invert. What do they not do? In short, they use words like wax”.\footnote{{For the original Latin quotation, see Chapter 2, \sectref{sec:2.6}.}} 
\end{quote}

The Venice-based printer Aldus Manutius, the first great publisher of Greek texts, left no doubt about it: the Greeks, especially their poets, could do almost anything with their language thanks to the dialects, a liberty not granted to their Latin colleagues. The sheer endlessness of linguistic variation did not, however, scare early modern Hellenists, who did their best to accurately chart it in their handbooks for the Greek language and its dialects. How did they tackle this thorny issue? And on what sources did they rely? Let me start by briefly sketching the way in which ancient and Byzantine scholars described the Greek language and its diversity.

\section{Dialects and the pathology of words}\label{sec:6.1}

In ancient and medieval treatises on the Greek dialects, linguistic variation was almost as a rule discussed per dialect and its characteristic letter mutations, and not per linguistic category such as phonology or the verbal system.\footnote{{A rare exception is Herodian’s Περὶ παραγωγῶν γενικῶν ἀπὸ διαλέκτων, which discusses dialectal deviations within one category: the genitive case.}} This method of presentation was a consequence of the widespread assumption among Greek grammarians that individual dialects showed an inclination toward certain types of letter mutations; this made it self-evident for them to describe these changes per dialect and not per morphological feature. These letter mutations were framed within a methodological framework now known as \isi{pathology}, because the mutations were usually called “modifications of the word”, \textit{páthē tês léxeōs} (πάθη τῆς λέξεως) in Greek.\footnote{{See \citet{Wackernagel1876}, \citet[150]{Siebenborn1976}, and \citet[esp. 118]{Lallot1995} for this framework of \isi{pathology} and its link with ancient dialect studies.}} A letter was, in this context, conceived in its traditional ancient meaning, conjoining its form and the sound it represented, but the emphasis seems to have been on its formal appearance, as the treatises were conceived as an aid for philologists to understand literary texts written in different dialects. 


One of the many kinds of word modifications was, for instance, \textit{pleonasmós} (πλεoνασμός), meaning the insertion of an additional letter into a word. In Aeolic, the letter upsilon ⟨υ⟩ was usually inserted before a vowel or a rho ⟨ρ⟩, according to one ancient grammarian.\footnote{{See Περὶ κλίσεως ὀνoμάτων (ed. \citealt{Lentz1870}: 640), a work ascribed to Herodian.}} A systematic discussion of \isi{pathology} was provided by Tryphon in his treatise “On modifications”, \textit{Perì pathôn} (Περὶ παθῶν), extant in several different versions and eagerly read by early modern Hellenists. The work offered a classification of the various word modifications and exemplified some of them by referring to features of certain Greek dialects. Apart from the four canonical dialects, Tryphon also mentioned Boeotian and Laconian. Sometimes a certain word modification was assigned exclusively to one specific dialect. \textit{Parénthesis} (παρένθεσις), which designated the insertion of a vowel in the middle of a word without creating a new syllable, was supposedly typical of the Ionic dialect (Tryphon, Περὶ παθῶν 1.16). Other word modifications were attributed to several different dialects. Both Ionic and Aeolic were, for example, said to exhibit \textit{diplasiasmós} (διπλασιασμός), the doubling of a consonant in the middle of a word without causing an additional syllable to emerge (Περὶ παθῶν 1.17). The point of reference for the modifications was always common Greek, usually identified with the Koine or also with what was believed to be common to most or all dialects. These two interpretations of common Greek were not clearly distinguished by Greek grammarians, many of whom viewed the Koine as an amalgam of the different dialects (see Chapter 2, \sectref{sec:2.3}).



Pathology also constituted the background against which the early Byzantine author John the Grammarian developed his three levels of variation to describe the Greek dialects. They differed from each other, John claimed, on the level of entire words, parts of words, and word accidents such as accent and spiritus (see \citealt{Manutius1496Thesaurus}: 237\textsc{\textsuperscript{r}}). Greek dialectal variation, in other words, manifested itself in the lexicon and in small modifications within words, either in terms of letters or superficial features, to which John referred using the Aristotelian concept of “accidents”.\footnote{The Greek word John used was τὸ συμβεβηκός, “the accident”, as opposed to ἡ oὐσία, “the essence”.} John focused in his treatise on the latter category, describing in the first place letter mutations in the tradition of \isi{pathology} as well as accent and spiritus deviations. The description of lexical variation was usually reserved for separate works. These focused either on rare dialectal words – so-called glosses – attributed to regions or cities rather than to the canonical dialects, as in the case of Hesychius, or on Attic words, like the wordlists of Phrynichus (2nd cent. \textsc{ad}), Moeris (?2nd/3rd cent. \textsc{ad}), and Thomas Magister (?1275–1350/1351).\footnote{{For Magister’s life dates, see \citet[417]{Baloglou1998}.}} The latter should be explained by the \is{Atticism}Atticistic movement, which emerged during the \isi{Second Sophistic} in the first centuries \textsc{ad}; the lexica were intended to cater to the needs of those aspiring to write pure Attic Greek, which functioned as a kind of high-end \isi{shibboleth}. It served to distinguish true scholars from would-be intellectuals.\footnote{On the \isi{Second Sophistic} and its fascination with Attic, see \citet{Whitmarsh2005}.}



Pathology is reminiscent of the Roman framework known as \is{permutation of letters}\textit{permutatio litterarum}, ‘permutation of letters’, designed for etymological purposes by the polymath Marcus Terentius Varro (116–27 \textsc{bc}) and also outlined by the rhetorician Quintilian (ca. \textsc{ad} 35–100).\footnote{{On the} {\textit{permutatio litterarum} }{and its link with \isi{pathology}, see \citet[esp. 25–28, 37]{Ax1987}. On Varro’s \isi{etymological method}, see e.g. \citet{Pfaffel1981} and \citet[7--10, {\textit{passim}}]{Taylor1996}.}} The number of letter permutations was, however, much more limited than the word modifications in \isi{pathology} and amounted to four: addition (\textit{adiectio}), omission (\textit{detractio}), transposition (\textit{metathesis}), and permutation (\textit{permutatio}). The precise relationship between both frameworks remains, however, obscure, and further research is required to cast light on it (cf. \citealt{Ax1987}: 25), which lies outside the scope of the present book. Latin authors were, in any case, not very interested in Greek dialect variation, with the exception of Priscian, who worked in early sixth-century Byzantium and tried to demonstrate the close connection between Latin and Greek, also through the dialects. He, too, believed in the Greek (Aeolic) origin of Latin and moreover pointed out the importance of Attic syntax for Latin.\footnote{{See in particular \citet[]{Conduche_latin_nodate} and Chapter 2, \sectref{sec:2.5}.}}


\section{The heritage of pathology}\label{sec:6.2}

In the early Renaissance, the description of Greek dialectal features was initially restricted to occasional cursory remarks in grammatical handbooks and limited to the bare necessities for two reasons. On the one hand, these manuals were intended for beginners. The dialect particularities which a grammarian deemed indispensable knowledge for this specific audience were included; all else was wisely omitted. After all, why scare off students by overemphasizing a major difficulty of this language, which in itself was suspect enough because of, among other things, its association with the \isi{Orthodox Church}? On the other hand, Greek philology was not yet so advanced as to enable scholars to produce adequate descriptions of the Greek dialects. From the second half of the Quattrocento onward, however, scholars became increasingly acquainted with the Greek language and its manifold forms. This was made possible by the wide availability of the triad of dialectological works attributed to John the Grammarian, Plutarch, and Gregory of Corinth (see Chapter 1, \sectref{sec:1.2}). The relative difficulty of these Greek texts, which were inconveniently arranged from a didactic perspective, soon motivated scholars to produce treatments of their own, which were more transparent and instructive. In doing so, early modern grammarians frequently opted to discuss dialectal particularities per linguistic feature – usually per part of speech – and not per dialect as was normally the case in the Greek tradition. In other words, these Hellenists adopted a more comparative-contrastive approach, as they treated the different dialectal realizations of a feature in one and the same paragraph and not scattered throughout their work, as their predecessors had done. An early example is Adrien Amerot’s successful and pioneering booklet on the Greek dialects, which focused on variation in nominal and verbal morphology.\footnote{{\citet{Amerot1530}. See \citet[1--19]{Hoven1985} on this treatise and its success. See also \citet{Hummel1999}; Chapter 1, \sectref{sec:1.2}.}} This new structure developed naturally out of Renaissance grammars of Greek; these works increasingly contained dialectal information, which was inserted into the sections discussing the relevant part of speech.\footnote{{On Renaissance Greek grammars before 1530, see \citet{Botley2010}.}} In fact, it is telling that Amerot’s booklet was actually a separately published extract from his Greek grammar, printed ten years earlier.

Renaissance grammarians of Greek attempted to discover a certain regularity in the dialectal variation they were describing, which led them to formulate rules of change and exceptions to them. The French Hellenist Petrus Antesignanus made the following comment on Attic in his \textit{Appendix on the dialects}, included among his remarks on Nicolaus Clenardus’s Greek grammar:

\begin{quote}
Attic puts tau instead of sigma, as in \textit{glôtta} instead of \textit{glôssa}, `tongue'. This is always observed when there is a double \textit{ss} and sometimes when there is a simple, as in \textit{tḗmeron} instead of \textit{sḗmeron}, `today'.\footnote{{\citet[13]{Antesignanus1554}: “Attica ponit τ pro σ, γλῶττα, pro γλῶσσα, lingua; hoc semper obseruatur, ubi est duplex σσ; atque interdum ubi est simplex, ut τήμερoν, pro σήμερoν, hodie”.}}
\end{quote}

Antesignanus added that Attic “rejoices” (“gaudet”) in vowel contractions. In the margin, he called such rules “general precepts about the dialects” (“generalia praecepta de dialectis”). These were understood as guiding principles for understanding and recognizing the Greek dialects, mainly through letter changes, and should not be taken as strict grammatical rules without exceptions, let alone as precursors of later \is{sound law}sound laws. Antesignanus’s description of the dialects served, in other words, as a philological tool allowing its readers to reach a better understanding of the variability of the Greek tongue rather than as a scientific linguistic account. Especially revealing is what he stated toward the end of his \textit{Appendix}:

\begin{quote}
But do not believe that these things we have said here are observed everywhere in all words. In fact, these do not take place, except in certain words and in certain cases of the parts of speech, which are inflected through cases, and in certain persons and tenses of verbs. We will also add some other, less general rules, if the context allows it.\footnote{{\citet[15]{Antesignanus1554}: “Ne uero credas ea quae hic diximus passim in omnibus dictionibus obseruari. Non enim ista locum habent, nisi in certis quibusdam uocibus et certis casibus partium orationis, quae per casus inflectuntur, atque in certis quibusdam personis et temporibus uerborum [...]. Adiciemus quoque nonnullas alias regulas minus generales iuxta locorum opportunitatem” (translation adapted from \citealt{VanRooy2016c}: 129).}}
\end{quote}

Antesignanus thus stressed the lack of regularity in the precepts he was offering. Some decades later, Henri \citet[46--47]{Estienne1581} stressed the limitations of such dialect rules in much the same manner, even though he suggested that one ought to look for rules that are as general as possible. The fallibility of dialect rules was widely acknowledged by Hellenists.\footnote{{Cf. e.g. also \citet[38--39]{Schmidt1604}; \citet[2]{Heupel1712}; \citet[1136, 1139]{[frisch]1730};\ia{Frisch, Johann Leonhard@Frisch, Johann Leonhard} \citet[299]{Jehne1782}.}} The frequent use of adverbs meaning ‘sometimes’, ‘occasionally’, or ‘frequently’ in the formulation of such rules, inherited from the Greek tradition, is therefore not surprising.\footnote{{See \citet[53]{Forstel1999}; \citet[516]{VanRooy2014}. Cf. the adverbs} {\textit{aliquando}} {(e.g. \citealt{Walper1589}: 41),} {\textit{frequenter}} {(e.g. \citealt{Walper1589}: 42),} {\textit{interdum}} {(e.g. \citealt{Antesignanus1554}: 13),}{ }{and} {\textit{quandoque}} {(e.g. \citealt{Walper1589}: 64).}} Some scholars went so far as to deny the possibility of formulating rules altogether (e.g. \citealt{Camden1595}: \textsc{i}.1\textsc{\textsuperscript{v}}). The Jena academic Johann Andreas Grosch (1717–1796) explicitly contrasted grammatical rules to dialects, which he associated with \isi{anomaly} and irregularity (\citealt{Grosch1753}: 17–18, 24–25). This was no doubt a consequence of the fact that dialects were increasingly seen as anomalous deviations of the \is{analogy}analogical standard from the seventeenth century onward. This interpretation of \textsc{dialect} as opposed to \textsc{language} was fostered by ideas on \isi{vernacular} dialects, which could not boast such a rich literary tradition as the Greek dialects had and which scholars associated with the lower classes and their allegedly depraved speech.\footnote{{See \citet[]{VanRooyFcd} for the history of this interpretation of the} {\textsc{dialect}} {concept.}}

A number of Hellenists debated the validity of individual dialect rules at some length. Some even raised fundamental objections against the methods and approaches of their predecessors. The German scholar Georg Heinrich \citet[512]{Ursin1691} deplored the fact that the Greek dialects were a source of discord among grammarians up to the point that some of them even contradicted themselves. Ursin moreover warned his readers not to forge new dialects – he was no doubt thinking of the so-called \isi{poetical dialect} (see Chapter 2, \sectref{sec:2.7}) – stressing the importance of considering actual usage in ancient literary texts. This was part of a broader tendency, as many scholars claimed to rely on their own reading of Greek texts when formulating and exemplifying dialect rules, even though some of them were simply drawing on their predecessors for the greater part.\footnote{{See e.g. \citet[†.7\textsc{\textsuperscript{r}}]{Walper1589}; \citet[{\footnotesize{)(}}.4\textsc{\textsuperscript{r}}]{Portus1603}; \citet[5\textsc{\textsuperscript{[a]}}]{Merigon1621}; \citet[b.4\textsc{\textsuperscript{v}}–b.5\textsc{\textsuperscript{r}}, 432]{Nibbe1725}.}}  An in-depth analysis of Greek grammatical works, outside the scope of this book, could shed more light on the innovativeness of each scholar and his exact method in describing Greek dialectal variation.

Early modern Hellenists conducted their analyses of dialectal variation principally in the spirit of their ancient Greek and Byzantine predecessors, whose methodology they largely appropriated, even if they opted to present the matter differently. As a matter of fact, the Greek framework of \isi{pathology} proved to be keenly used by early modern scholars to account for dialectal differences, not only by specialists of Greek, but also by grammarians of \isi{vernacular} tongues such as German and English who had mastered Greek.\footnote{{For Greek, see e.g. \citet[b.iv\textsc{\textsuperscript{v}}]{Melanchthon1518}; \citet[{7}{\textsc{\textsuperscript{v}}}{–11}{\textsc{\textsuperscript{v}}}]{Baile1588}; \citet[11]{Schmidt1604}; \citet[2–8, 20–22]{Hill1658}. For German, see e.g. \citet{Wolf1578}. For English, see e.g. \citet[130--133]{Gill1619}.}} Early on, a cross-fertilization with the Roman letter permutation framework seems to have taken place, which surfaces in the terminology used by certain grammarians. The Marburg Hellenist Otto Walper, for instance, discussed the “permutation of vowel letters” of the Doric dialect.\footnote{{\citet[62]{Walper1589}:} {\textit{permutatio litterarum uocalium}}.} Walper noted, among other things, the omission of the letter jota and the addition of the very same letter in other contexts.\footnote{{\citet[63]{Walper1589}: “Deinde iota frequenter detrahunt” \& “Rursum iota ad o apponunt”.}} He moreover mentioned letter transpositions as well as a considerable number of letter changes.\footnote{{See e.g. \citet[63]{Walper1589}, “per metathesin litterarum” (i.e. in ῥέζω becoming ἔρζω), and \citet[64]{Walper1589}, “Item θ quandoque mutatur in χ, ut ὄρνιχα, pro ὄρνιθα”, respectively.}} These corresponded to the four main letter change processes of the \textit{permutatio litterarum}, which were all present in the framework of \isi{pathology}, too.\footnote{{See Tryphon’s Περὶ παθῶν and especially \citeauthor{Amerot1520}'s adaptation of Tryphon’s  treatise  (\citeyear[\textsc{p.}{iv}{\textsc{\textsuperscript{v}}}]{Amerot1520}).}}

Many of the letter changes in the Greek dialects were so widely known that they were sometimes used to formulate generally fictitious etymologies of words in various languages. Scholars assumed that if a letter change could occur among Greek dialects, the very same change could take place in other linguistic contexts, distinct in place and time, as well. So it became possible for Philipp Clüver (1580–1622), a geographer and historian from Danzig, to derive the English toponym \textit{Thetford} from Celtic \textit{Sitomagus} by appropriating a letter change known from Greek: “\textit{Sit}, moreover, could have been as easily changed into \textit{Thet} by a variation of dialect as the Greeks’ \textit{Theós} [`god'] into \textit{Siós}”.\footnote{{\citet[64]{Cluver1616}: “}{\textit{Sit}} {autem tam facile, uariatione dialecti, mutari potuit in} {\textit{Thet}}{, quam Graecorum Θεὸς in Σιὸς”. See \citet[114--115]{Metcalf2013}.}} What is more, knowledge of Greek letter changes seems to have heightened scholars’ awareness of similar variations in their own vernaculars. In fact, they sometimes tried to justify \isi{vernacular} differences by stressing parallel changes in Greek.\footnote{{For more details on Greek–\isi{vernacular} parallels, see Chapter 8, especially \sectref{sec:8.1.2}.}} Comparisons in the other direction could serve to help the reader understand the nature of Greek dialect changes. In his well-known dialogue on the ancient \isi{pronunciation} of Latin and Greek, Desiderius Erasmus mentioned the change of the letter ⟨r⟩ into ⟨s⟩ in the French of Parisian women as a way of clarifying a similar phenomenon in Greek.\footnote{{\citet[52]{Erasmus1528}: “Hanc asperitatem quidam mitigant supposito σ, ut θαρσεῖν pro θαρρεῖν. Idem faciunt hodie mulierculae Parisinae, pro} {\textit{Maria}} {sonantes} {\textit{Masia}}{, pro} {\textit{ma mere ma mese}}{”. Cf. Chapter 8, especially \sectref{sec:8.1.1}.}} Similarities between letter changes were also adduced to support the link between a contemporary people – e.g. the Venetians – and an ancient Greek tribe – e.g. the Ionians.\footnote{{See \Citet[97{\textsc{\textsuperscript{r}}}]{Da1509}. Cf. \citeauthor{Reitz1730}'s (\citeyear[e.g., 122, 125, 126–127]{Reitz1730}) efforts to establish kinship between Germanic and Ancient Greek (for which, see \citealt{VanHal2016}).}} Early modern Greeks also relied on Ancient Greek letter changes to account for properties of the Vernacular Greek language. Around 1650, a Greek grammarian even oddly claimed that the Turks Doricized the prepositional phrase \textit{stḕn pólin} (στὴν πόλιν), ‘to the city’, at that time pronounced as \textit{stimbolin}, into \textit{stampól}/\textit{stamból} (σταμπόλ), which allegedly resulted in the toponym of the well-known city of \is{Istanbul, etymology of}Istanbul.\footnote{{\citet[14]{Nikiforos1908}. See also \citet[35]{Nikiforos1908}, referring to Ionic and Attic letter particularities. Cf.} also \citet[\textsc{a.4}\textsc{\textsuperscript{r}}]{Rodigast1685}.}

More at the margins, the intense early modern debate over the correct \isi{pronunciation} of Ancient Greek also provoked analyses of specific letter changes across the dialects. There were two main camps in this discussion: those defending an itacist \isi{pronunciation}, largely corresponding to Vernacular Greek \isi{pronunciation} and connected with the Pforzheim Hellenist Johann Reuchlin, and those propounding a reconstructed etacist \isi{pronunciation}, resembling that of fifth-century \textsc{bc} Attic and closely associated with the proposals of Desiderius Erasmus.\footnote{See \citet[130]{Sandys1908}. See \citet{Bywater1908} on the Erasmian \isi{pronunciation} and its precursors.} Already in Erasmus’s dialogue on Greek \isi{pronunciation} of 1528, frequent allusion was made to Greek dialectal features (e.g. \citealt{Erasmus1528}: 52, 106). He did so for various reasons, among other things to prove the cognate nature of certain letters, such as alpha and eta, which often changed across dialects \citep[62]{Erasmus1528}. Early modern, especially seventeenth and eighteenth-century, views on Ancient Greek \isi{pronunciation} deserve closer attention, as does the role of dialectal evidence in this context.

\section{Beyond letter changes}\label{sec:6.3}

Letter changes, although vastly important, are not the entire story. Early modern Hellenists often included other types of dialectal differences in their descriptions, ranging from accent and spiritus through alphabet, morphology, and lexicon to syntax and even style. The view that dialectal variation affected style – often \is{syntaxis figurata}termed \textit{syntaxis figurata} – was, much like ideas about other levels of variation, to a large extent inherited from the Greek tradition, in which certain \is{rhetoric}rhetorical figures were claimed to be specific to a dialect.\footnote{{See, most importantly, Lesbonax’s} {\textit{De figuris}}{, the source of, among others, \citet[145--146]{Saumaise1643a}.}} This link between dialects and stylistic peculiarities was related to the literary status of the canonical dialects. As to the alphabet, scholars were aware of certain peculiarities across the dialects, most notably the \isi{digamma}. This letter was generally seen as exclusive to Aeolic, in spite of the fact that it was also used in \is{standardization!codification}codifications of non-Aeolic varieties of Greek. This can be explained by the fact that the philological focus was mainly on literary dialects, and ancient scholars noticed that only texts in Aeolic contained this letter. The grammarian Tryphon, however, already referred to the wider application of the \isi{digamma} (Περὶ παθῶν 1.11), later confirmed by \is{epigraphy}inscriptional evidence. The phonetic value of this Greek letter was correctly explained by such ancient authors as Dionysius of Halicarnassus, who identified it with a [u̯] sound, the voiced labiovelar approximant (\textit{Antiquitates Romanae} 1.20.3). Yet early modern scholars experienced great difficulty in trying to discover the value of this letter with the limited evidence available to them. \citet[68–69, 108]{Erasmus1528}, in his aforementioned dialogue on the correct \isi{pronunciation} of Latin and Greek, first accorded the \isi{digamma} a value between [u̯] and [p\textsuperscript{h}], but then claimed that it stood for a [u̯] sound only (see Kramer in \citealt{Erasmus1978}: 177 n.361). Other scholars took it to express a [f] or a [v] sound or presumed that it had several different values.\footnote{{For [f], see e.g. \citet[4]{Sylvius1531} and \citet[5]{Rhenius1626}. For [v], see \citet[108-109]{Freret1809}, who noted the presence of the \isi{digamma} in \is{epigraphy}inscriptions on ancient medallions of Aeolic cities. For the idea that the \isi{digamma} had different values, see e.g. \citet[107-108]{Canini1555} and \citet[b.1{\textsc{\textsuperscript{r}}}{–b.2}{\textsc{\textsuperscript{v}}}]{ThryllitschBrunner1709}.}} Still others correctly recognized its [u̯] value, sometimes inspired by ancient authors such as Dionysius.\footnote{{See e.g. \citet[b.4{\textsc{\textsuperscript{r}}}]{KirchmaierCrusius1684} and \citet[19]{Reynolds1752}.}} In the eighteenth century, \is{epigraphy}epigraphic evidence made a number of philologists and antiquarians realize that there were other graphemes, such as ⟨$⊏$⟩, which could express the [u̯] sound in varieties of Ancient Greek.\footnote{{See e.g. \citet[128-130]{Mazzocchi1754}, where, besides, the \isi{digamma} was interpreted as having multiple values.}}


\section{Debating dialectal features}\label{sec:6.4}

The case of the \isi{digamma} raises the question as to whether it became a more frequent occurrence that scholars discussed a specific dialect feature at greater length and offered various interpretations of it. Looking at early modern views on letter changes believed to occur among the Greek dialects and at ideas about other dialectal particularities – on the level of syntax, for instance – one is left to conclude that the case of the \isi{digamma} was relatively exceptional. Indeed, early modern discussions of dialectal features were usually of a highly rigid nature, in that they generally complied with Greek tradition even if the dialect rule in question could not be supported by actual evidence found in extant Greek texts. For example, according to the traditional view, it was peculiar to people from Attica to use the \isi{vocative} where one would expect a nominative and vice versa.\footnote{See e.g. Apollonius Dyscolus, \textit{De constructione} 3, Uhlig page 301; Gregory of Corinth, \textit{De dialectis} 2.41 \& 2.53, where only examples from poetry – Homer and tragedy – are offered. Cf. also Priscian, \textit{Institutiones grammaticae}, book 17 (ed. Martin Hertz in \citealt[208]{Keil1859}).} Less frequently, the feature was attributed to Macedonian and Thessalian, too.\footnote{Apollonius Dyscolus, \textit{De constructione} 3, Uhlig page 301. Cf. also Priscian, \textit{Institutiones grammaticae}, book 17 (ed. Martin Hertz in \citealt[208]{Keil1859}).} This idea was adopted unquestioningly by most early modern scholars. The Swiss Hellenist and physician Martin Ruland the Elder (1532–1602) dubbed it a “rule” (\textit{regula}) in his handbook for the Greek language and its dialects, while trying to demonstrate it not only with examples from pagan literature, but also by means of passages from the \isi{Septuagint} and the \isi{New Testament} (\citealt{Ruland1556}: 251; cf. Chapter 4, \sectref{sec:4.5}). \citet[302]{Ruland1556} extrapolated the feature to Thessalian, for which he was likely inspired by Apollonius Dyscolus or Priscian.\footnote{Cf. e.g. also \Citet[36\textsc{\textsuperscript{r}}]{Da1509}; \citet[216]{Vergara1537}; \citet[50\textsc{\textsuperscript{v}}]{Nunez1555}; \citet[\textsc{x.1}\textsc{\textsuperscript{v}}]{Dabercusius1577}; \citet[5, second pagination sequence]{Rhenius1626}; \citet[8--9]{Pasor1632}; \citet[85-87]{Wyss1650}; \citet[88]{Leusden1670}. \citet[\textsc{b.3}\textsc{\textsuperscript{r}}]{Kirchmaier1709} regarded it as a Macedonian feature, referring to Priscian.} Other scholars took it to be particular to poetry or even to the Koine, which was claimed to imitate Attic.\footnote{For poetry, see e.g. \citet[\textsc{q.}i\textsc{\textsuperscript{v}}]{Amerot1520}; \citet[129]{Antesignanus1554}; \citet[34]{Gretser1593}; \citet[157]{Schmidt1604}. For the Koine, see \citet[54]{Lancelot1655}.} The syntactic particularity nevertheless remained closely associated with Attic and even led the influential grammarian Nicolaus Clenardus to posit that, in Attic, the nominative and \isi{vocative} were morphologically identical, a misconception prominent in early modern Greek grammars.\footnote{\citet[7 (misprint for 6)]{Clenardus1530}. See e.g. also \citet[534]{Crusius1558}; \citet[12\textsc{\textsuperscript{r}}]{Baile1588}; \citet[11, 37]{Walper1589}; \citet[32, 34]{Gretser1593}; \citet[53, 453]{Lancelot1655}; \citet[101]{Giraudeau1739}; \citet[20]{Facius1782}.}

The renowned printer and Hellenist Henri Estienne fiercely refuted the widespread idea that Attic authors used the nominative instead of the \isi{vocative} in his \textit{Comments on the particularities of the Attic language or dialect}, for which he relied on the actual usage of Attic authors such as Thucydides and other writers. Instead, Estienne ascribed this feature to the Boeotian and Aeolic dialects, a view he oddly supported by referring to the Byzantine grammarian Eustathius of Thessalonica rather than by actual usage in Boeotian and Aeolic texts.\footnote{\citet[15]{Estienne1573}. For Aeolic, see also \citet[{\scriptsize{)(}}.4\textsc{\textsuperscript{v}}]{Schmidt1604}. For Boeotian, see also \citet[71]{Merigon1621}, who believed it to be a Doric feature as well (cf. \citealt{Maittaire1706}: 257–258).} Clearly, Estienne was not making progress here. Although correctly refuting on the basis of empirical evidence the faulty idea that the nominative was used instead of the \isi{vocative} in Attic, he at the same time attributed the feature to other dialects by invoking only the authority of a Byzantine scholar. What is more, a little further on, Estienne argued, in agreement with Clenardus and others, that, in Attic, the nominative and \isi{vocative} were morphologically identical, while dismissing the idea that the \isi{vocative} was replaced by the nominative on the syntactic level.\footnote{\citet[17]{Estienne1573}. See also \citet[29, 42–43, 150]{Estienne1573}, where his views are recapitulated.} Estienne thus showed himself to be more critical toward the canonical dialectal features transmitted by Greek tradition, which were blindly adopted by many early modern Hellenists, even though this inquisitive attitude still failed to bring him to correct insights.

Estienne himself was reproached in the early eighteenth century by the Hellenist Georg Friedrich \citet[d.3\textsuperscript{v}]{Thryllitsch1709} for failing to be consistent in his attitude toward dialectal particularities. Thryllitsch noticed a contradiction in Estienne’s work. In his commentary on Attic, \citet[13]{Estienne1573} contended that Thucydides wrote \textit{thálassa} (θάλασσα), even though he had printed \textit{thálatta} (θάλαττα) in his edition of the Greek historiographer’s work, thus failing to put his views into actual practice. This critique seems somewhat unfair in view of the fact that Estienne published his commentary on Attic (\citeyear{Estienne1573}) nine years after his edition of Thucydides (1564), in which period of time he might have studied Thucydides’s language more closely.

Even though Estienne’s case reveals that there could be detailed discussions of the validity of certain dialect rules, this remained relatively rare and one could say that early modern analyses of the Greek dialects lacked the empirical focus and dialogic interaction necessary for achieving considerable scholarly progress. However, there were a number of critical voices other than Estienne’s. Some scholars were cautious about attributing certain properties to a specific dialect. As a matter of fact, the authors of a dissertation on \isi{Atticism} presented at Leipzig in 1737, Johann Gottfried Hauptmann (1712–1782) and Christian Ernst Schmid (1715–1786), even formulated a kind of methodological precept stipulating that a feature was not to be labeled Attic straight away if it was used by only one author writing in that dialect (\citealt{Hauptmann1737}: 16). Moreover, if only one example of a certain linguistic feature could be found in an author – however eminent – or if it was typical of poets, it should be considered an idiomatic peculiarity or a poetic feature rather than an \isi{Atticism}. Some decades earlier, a German Hellenist had suggested a similar methodological precept, albeit from the reverse perspective: determining the identity of a Greek author’s dialect must be based on the general appearance of his language and not on one or two particular words and their features.\footnote{\citet[495--496]{Ursin1691}. See e.g. also the critical approach toward earlier sources of \citet{Walper1589}; \citet[{\scriptsize{)(}}.3\textsc{\textsuperscript{r}})]{Portus1603}; \citet[10-12]{Gedike1782}.} In both cases, a thorough knowledge of the Greek literary dialects was presupposed, and one can once again see right away how crucial the philological incentive was in studying Greek linguistic diversity (see Chapter~\ref{chap:3}).

\is{epigraphy|(}
When scholars described Greek dialectal particularities, the same examples tended to recur, since they were often taken from traditional Greek treatises on the subject and from the grammatical work of their early modern predecessors. They could be supplemented by a scholar’s own reading of Greek literary texts, including the \isi{Septuagint} and the \isi{New Testament}, and – at a later stage and much less frequently so – inscriptions.\footnote{For pagan literary texts, see e.g. \citet[]{Amerot1520, Amerot1530}. For the Greek Bible, see e.g. \citet{Pasor1632}.} The French-born Hellenist Michael Maittaire (1668–1747), who worked as a teacher in England, relied on inscriptional evidence from steles and coins to describe and exemplify certain Doric particularities as well as to reconstruct the ancient orthography of Greek, which according to him was close to that of Latin.\footnote{\citet[e.g. 161–167, 170, 184, 205–206, 211–212, 221, 240, 243]{Maittaire1706}.} On rare occasions, scholars tried to introduce new dialectal features into the canon on the basis of inscriptional evidence. The English clergyman Thomas Lydiat assumed that the change of [n] into [m] at the end of a word before [b], [m], [p], or [p\textsuperscript{h}] was a particularity of Ionic, since he had found this feature in an Ionic inscription (Lydiat in \citealt{Prideaux1676}: \textsc{ii}.116). This was, in fact, nothing more than a somewhat clumsy solution to account for what we would today consider a straightforward case of phonological assimilation in front of a labial sound.
\is{epigraphy|)}

In sum, discussions of Greek dialectal features were normally not very animated. Some Hellenists did put forward more innovative views, even though these usually remained at the margins of early modern scholarship. Let me round off by citing two final intriguing examples from the sixteenth century. Firstly, Henri Estienne innovatively tried to map out, in some detail, currents of interdialectal influence (i.e. the introduction of certain dialectal features of one dialect into another); \citet[22--28]{Estienne1581} did so with specific attention to Attic and its alleged adoption of Ionic, Doric, and Aeolic elements. It shows that he did not consider the Greek dialects, including the revered Attic dialect, to be stable closed systems but forms of speech susceptible to external influence. Secondly, the first Spanish grammarian of Greek, Francisco de Vergara, made an interesting idiosyncratic remark on what is now called the “\isi{deictic iota}”, for instance, in \textit{toutí} (τoυτί), ‘this here’, instead of \textit{toûto} (τoῦτo), ‘this’. Although following the traditional faulty idea that the \isi{deictic iota} was an exclusively Attic feature, Vergara was at the same time uniquely aware of its pragmatic function, as he revealingly suggested that it was used “to indicate an object more clearly and as if with a certain gesture”.\footnote{\citet[218]{Vergara1537}: “clarius ac ueluti gestu quodam rem indicent”.}

\section{Conclusion}\label{sec:6.5}

There seems to have been a consensus among early modern Hellenists that the Greek dialects exhibited certain regular variations and that rules could be formulated to grasp dialectal changes even if these were, as scholars widely agreed, by no means without exception. Letter changes constituted the focus of attention, a tendency enhanced by the seeming merger of two ancient, letter-centered frameworks: Greek \isi{pathology} and Roman letter permutation, the interplay between which deserves further study. Despite the focus on letter variation, the Greek dialects were believed to differ from each other on every possible linguistic level, including accent, lexicon, syntax, and even style. Dialect rules – often dubbed \textit{regulae} or \textit{leges} in Latin – were largely adopted from Greek scholarship, with relatively limited room for adaptation and innovation. The rigidity of Greek dialect descriptions can be at least partly explained by the fact that the Greek language and its many forms were not studied in and for themselves, but nearly always with reference to reading and understanding ancient Greek literature (see Chapter~\ref{chap:3}). In other words, early modern scholars continued Greek tradition in the sense that they also primarily associated the Greek dialects with literature. Critical voices such as Henri Estienne’s were exceptional and did not always bring about a change for the better. To put it differently, the contribution of early modern scholars to Ancient Greek \isi{dialectology} seems to have been rather modest on the micro-level of linguistic description. However, as I have stressed before, the mere fact that modern scholars like Ahrens relied in part on early modern scholarship when they were laying the foundations of modern forms of Ancient Greek \isi{dialectology} calls for a more systematic and comprehensive investigation of early modern approaches toward Greek dialectal features. This holds especially true if one reckons that scholars like Thomas Lydiat and Michael Maittaire, both active in England, where one of the first extensive collections of Greek inscriptions was published (i.e. \citealt{Prideaux1676}), increasingly included data from non-literary sources in their discussions of the dialects. Even though they usually tried to understand such data within the traditional framework as designed by Greek and maintained by early modern scholarship, this evolution was of paramount importance for the development of Ancient Greek \isi{dialectology} in the nineteenth century, as it helped to break up the absolute monopoly of literary texts in the corpus of dialectal source material.

