\chapter{A true man of letters: Greek dialects and philology}\label{chap:3}
\begin{quote}
The Greek language ruled and held great sway through its four principal dialects, which complete each other in such a manner that no one should rightly be reckoned to be versed in any of them if he has not mastered them all.\footnote{\citet[53--54]{Schroeder1748}: “Lingua Graeca per quattuor praecipuas dialectos regnauit et amplissimam habuit ditionem, quarum una alteram ita perficit, ut in nulla recte callere censendus sit, qui non omnes fuerit complexus”.}
\end{quote}

The Marburg-born professor Nicolaus Wilhelm Schröder (1721–1798) made this point in 1748 when pronouncing an oration on how to acquire a thorough knowledge of the Hebrew language. Schröder did so in his capacity as professor of Oriental tongues and Greek in Groningen, in the Dutch Republic, while arguing that a comparative approach to Hebrew, Aramaic, Syriac, and Arabic was warranted since they were as closely cognate as the Greek dialects. However noteworthy this methodological consideration on these so-called Oriental tongues may be, I am more interested here in Schröder’s suggestion that mastering the main dialects of Greek appears to have been a requirement to be considered a true Hellenist in the early modern period. Schröder implied that it was otherwise impossible to correctly understand Greek language, literature, and culture, a view with which most of his early modern colleagues no doubt agreed. The primary motivation to study the dialects was in other words philological in nature, much as it had been in ancient and Byzantine Greece. What main philological needs should knowledge of the dialects fulfill in the eyes of early modern scholars?

\section{The basic motivation: Reading Greek poetry}\label{sec:3.1}

Enabling students to read difficult literary texts from Greek antiquity was the basic motivation for Hellenists to reflect on the dialects and their linguistic features, to which the countless early modern manuals for the Greek dialects bear witness. Sometimes the authors of these handbooks made their goals and readership explicit. The Swiss doctor and Hellenist Martin Ruland (1532–1602) believed that his manual was to be of great use to students of good literature such as Godfrid Seiler, one of the two people to which his handbook was dedicated, and “to other youngsters who likewise just now engage in Greek or also Roman learning”.\footnote{\citet[α.4\textsc{\textsuperscript{v}}]{Ruland1556}: “Tibi itaque mi carissime ac bonarum litterarum studiosissime Godfrid καὶ ἄλλoις τoῖς μειρακίoις καὶ νεωστὶ τoῦ μαθήματoς τoῦ Ἑλληνικoῦ ἢ καὶ τoῦ Ῥωμαϊκoῦ ἁπτoμένoις profuturum hoc tibi magno labore elaboratum opusculum puto”.} Ruland moreover alluded to the widespread humanist idea that knowledge of Greek language and literature was indispensable to understand ancient Latin literature.\footnote{Cf. \citet[139]{Ben-Tov2009} for Melanchthon’s expression of this idea. See also Chapter 2, \sectref{sec:2.6}.} The intended readership of manuals for the Greek dialects was frequently a specific group of students, showing that they often catered to very local markets and highly specific audiences. One French handbook of 1588 was, for instance, directed to the youth of Aquitaine (\citealt{Baile1588}: \textsc{a.2}\textsc{\textsuperscript{v}}), whereas a German one published a year later was aimed at the students of the Academy of Marburg \citep{Walper1589}.

The fact that such handbooks started to appear might suggest that these works were considered the primary gateway to mastery of the Greek dialects. Was this, however, really the case? A powerful voice in this debate was that of the German Jesuit Jakob Gretser (1562–1625), the author of the standard Greek grammar for Jesuit colleges. Gretser believed that a grammatical work on the Greek dialects was not so much a handbook to be studied in isolation as it was a didactic instrument to which a teacher should refer the student for more information when reading poets \citep[.5\textsc{\textsuperscript{v}}–.6\textsc{\textsuperscript{r}}]{Gretser1593}. As a matter of fact, Gretser even contended that “the dialects [can]not be learned better and more easily than by reading poets and others who have inserted in their written works idioms of this kind”.\footnote{\citet[.5\textsc{\textsuperscript{v}}]{Gretser1593}: “pro certo habendum sit, dialectos melius et expeditius non disci, quam lectione poetarum aliorumque, qui suis monumentis huiuscemodi idiomata inseruerunt”.} Gretser’s case thus also demonstrates that the main focus of attention of early modern scholars interested in the Greek dialects was on the reading of poetry. The invention of a poetical dialect should also be viewed in this context.\footnote{See Chapter 2, \sectref{sec:2.6}, especially with regard to Manutius, and \sectref{sec:2.7}.} There were, in fact, only a few Hellenists concentrating on the dialects as they appeared in other literary genres such as orations (see e.g. \citealt{Labbe1639}: 5–7 for an exception).

\section{The dialects for the advanced philologist}\label{sec:3.2}

There were other incentives to pay extensive attention to the Greek dialects, which, however, principally belonged to the domain of the advanced and well-trained philologist. These ranged from etymology through to textual criticism and Neo-Paleo-Greek poetry composition. In this book I can only touch very briefly on this matter, mainly looking at it through the evidence found in manuals for the Greek dialects, as research on these often complex issues is still in its infancy. Interest in Neo-Paleo-Greek texts has, however, started to grow in recent years, so that this may well be considered a subfield of classical scholarship and reception studies (see e.g. \citealt{Pall2018}). With the term Neo-Paleo-Greek, which I have Anglicized from German \textit{Neualtgriechisch}, I refer to texts written in varieties of Ancient Greek by scholars from the Renaissance and later. I prefer this designation to alternatives such as \textit{Humanist Greek} for two main reasons. On the one hand, \textit{Neo-Paleo-Greek} does not carry any ideological connotations. On the other, it captures well the somewhat paradoxical nature of this exceptional type of writings, which served, among other things, to show off one’s erudition.

\subsection{Etymology}

Dialect forms were viewed as a useful tool to arrive at the correct etymology of a word, an idea only marginally present in the Greek tradition.\footnote{For exceptions, see e.g. Proclus, \textit{In Platonis Cratylum commentaria} 85, and Michael Psellos, \textit{Poemata} 6.187.} The renowned French Hellenist Henri Estienne (1528/1531–1598) demonstrated this by correctly deriving the Ionic noun \textit{apódexis} (ἀπόδεξις), ‘demonstration, exhibition’, from \textit{apodeíknumi} (ἀπoδείκνυμι), ‘to show, to demonstrate’, rather than from \textit{apodékhomai} (ἀπoδέχoμαι), ‘to accept, to receive’ (\citealt{Estienne1581}: 42–43). Typically for early modern scholarship, Estienne supported this view by invoking a letter change process; Ionic could drop the jota ⟨ι⟩ from the diphthong \textit{ei} ⟨ει⟩ of the Koine and the Attic dialect. A couple of decades after Estienne, his German colleague Erasmus Schmidt (1570–1637), professor of Greek in Wittenberg, emphasized the importance of being skilled in the dialects in order to comprehend certain morphological particularities of the Greek language in the dedicatory letter prefixed to his \textit{Treatise on the principal dialects of the Greeks} of 1604 (\citealt{Schmidt1604}: ):(.3\textsc{\textsuperscript{v}}–):(.4\textsc{\textsuperscript{r}}). Schmidt exemplified this by means of the Koine verbs \textit{klaíō} (κλαίω, ‘to cry, to lament’) and \textit{kaíō} (καίω, ‘to kindle, to burn’) and their respective future indicatives \textit{klaúsō} (κλαύσω) and \textit{kaúsō} (καύσω). He explained the presence of the letter upsilon ⟨υ⟩ in both forms by means of two dialect rules. First of all, Attic dropped the jota in both verbs, and has \textit{kláō} (κλάω) and \textit{káō} (κάω). Secondly, Aeolic changed alpha ⟨α⟩ into the diphthong alpha upsilon ⟨αυ⟩, resulting in the verb forms \textit{klaúō} (κλαύω) and \textit{kaúō} (καύω). These two Aeolic forms had a regular future ending in -\textit{aúsō} (-αύσω) and, though originally Aeolic, they were received into the Koine. Needless to say, such etymological experimentation is not always corroborated by modern linguistics, as in Schmidt’s case.{} In fact, the diphthong in Attic and Koine \textit{k(l)aúsō} (κ(λ)αύσω) reflects the original presence of a [u̯] sound at the end of the verbal root, normally lost in Attic and thus the Koine. It was, in other words, not the result of Aeolic influence, as Schmidt suggested.

\subsection{Textual criticism}

Mastering the dialects was vital not only for reading Greek literature and gaining better insight into the Greek language, but also for arriving at the correct version of Greek texts, transmitted for centuries by means of manuscript copies that left ample room for mistakes. Moreover, as \citet[47--48]{Reynolds1991} have pointed out, Byzantine scribes were often inclined to replace odd dialect forms by more familiar Attic or Koine forms, making it impossible for later editors to ever establish a dialect text closely approaching the ancient original. The editorial utility of dialect knowledge was summed up neatly by the zealous editor of Greek texts Henri Estienne in the preface to his extensive commentary on the Attic dialect. There, Estienne drew the following conclusion, after refuting several textual corrections conjectured by various philologists, including the renowned Italian humanist Lorenzo Valla (ca. 1407–1457): “By all means, there is nobody who cannot observe from these examples how dangerous ignorance of the dialects is”.\footnote{\citet{Estienne1573}: ¶.iii\textsc{\textsuperscript{v}}): “Equidem uel ex his, quam periculosa sit dialectorum ignoratio, nemo est qui perspicere non possit”.} In this context, Estienne boasted of a correction of his in his edition of Plutarch’s \textit{Lycurgus} (20.2):

\begin{quote}
For in the \textit{Lycurgus}, Demaratus, asked by a certain vile man who was the best among the Spartans, answers “\textit{hóti anomoiótatos}” [ὅτι ἀνoμoιότατoς], as the editions prior to mine read, even though it does not make any sense. In fact, “\textit{ho tìn anomoiótatos}” [ὁ τὶν ἀνoμoιότατoς] should be read (as is now written in my edition) with a very evident and suitable meaning, since Demaratus is answering “He who differs most from you”.\footnote{\citet[¶.iii\textsc{\textsuperscript{v}}]{Estienne1573}: “In Lycurgo enim Demaratus a quodam improbo homine interrogatus quis esset Spartiatarum optimus, respondet, ὅτι ἀνoμoιότατoς, ut in editionibus mea prioribus legitur, quamuis nullo sensu; cum legendum sit, ὁ τὶν ἀνoμoιότατoς (ut nunc in mea scriptum est) sensu manifestissimo et conuenientissimo; cum respondeat Demaratus, Qui tibi est dissimillimus”. Cf. \citet[36, 43–44]{Estienne1581}.}
\end{quote}

Estienne’s correction of \textit{hóti} (ὅτι), a common complementizer in Attic and the Koine, into \textit{ho tìn} (ὁ τὶν), the Greek article in the nominative singular followed by the Doric dative singular of \textit{sú} (σύ), ‘you’, is still accepted today. Interestingly, Estienne intended to devote an entire treatise to the causes of textual mistakes and the importance of the Greek dialects in this context. This work, to which he referred as his \textit{Work on the origin of errors}, does not seem to have materialized, unfortunately.\footnote{Estienne’s original Latin title for this planned work was \textit{De mendorum origine opus}.}

In Estienne’s wake, several other philologists relied on their knowledge of Greek dialect rules and particularities to correct ancient Greek texts, with varying degrees of success. The Bavarian classical scholar Gottlieb Christoph Harles (Harleß/Harless; 1738–1815), for instance, tried to do so for the works of the bucolic poet Theocritus. In the process, Harles criticized the changes made to Theocritus’s Doric dialect by Estienne and others, adding, however, that it was difficult to decide when and where to opt for the dialect form and even impossible to know for sure.\footnote{\citet[\textsc{xxii–xxiv}]{Harles1780}.} Harles moreover believed that it was dangerous to Doricize a word form against the testimony of all manuscripts, all the more since Theocritus’s fatherland Sicily was home not only to Doric varieties but to different Greek dialects (\citealt{Harles1780}: \textsc{xxxi–xxxii}).

\subsection{Writing Greek poetry}

Competence in the ancient Greek dialects was likewise indispensable for those early modern Hellenists wanting to show off their philological skills by composing ancient Greek texts themselves, especially poetry. This is why the early manual by the Swiss doctor Martin Ruland included several letters in different versions: Latin, Aeolic, Attic, Doric, and Ionic (\citealt{Ruland1556}: 328–335). Ruland composed these texts as examples for students with the ambition of writing in the Greek dialects. The Jesuit grammarian Jakob \citet[35]{Gretser1593} shared Ruland’s concerns but limited himself to emphasizing that Greek dialectal variation was to be carefully noted by students, not only in order to understand ancient Greek poets, but also to compose poems in Greek.

One of the best early modern handbooks for writing poetry in different Greek dialects was a 1610 work entitled \textit{On the method of producing Greek poems in an easy and skillful manner} by Christoph Helwig (1581–1617), professor of Greek and Hebrew at the then recently established university of Giessen.\footnote{The work was republished posthumously in a slightly augmented edition in 1623 in Nuremberg.} \citet[19]{Helwig1610} regarded Greek dialectal diversity as furnishing great abundance, specifying that it was not permitted to use dialectal diversity in prose, but it was necessary and befitting for poetry. There was something like a “legitimate usage of the dialects” in poetry, Helwig explained to his readers.\footnote{\citet[21–24]{Helwig1610}, where he speaks of the \textit{legitimus usus dialectorum}.} Dialects were not to be mixed without any distinction, as this would result in a cento rather than an actual poem. Instead, one should observe certain restrictions. To this end, Helwig distinguished two principal kinds of Greek dialect poetry: Ionic and Doric, the latter also comprising Aeolic. One was not allowed to randomly jump from one to the other, even though there was considerable overlap between both dialects. To enable students to capture this as conveniently as possible, Helwig composed extensive comparative tables of dialectal particularities, which took up the core of his handbook and in which deviations from the Koine were noted. In six columns, Koine, Attic, Ionic, Doric, Aeolic, and poetical forms were placed next to each other. Helwig was writing mainly for students wanting to improve their understanding of Greek poetry by composing themselves, and this was indeed considered the principal goal of this activity throughout the entire early modern period. More than a century after Helwig, the German classical scholar Johann Matthias Gesner (1691–1761), too, regarded it mainly as a student exercise. Gesner argued that students should write in Greek not to show off their erudition but in order to understand the rules of Greek poetry and the mixture of dialects in it. He nonetheless regarded this mixing as a foolish undertaking, which he compared to an imaginary case of a German poet mixing Swiss, Austrian, Low Saxon, and Dutch in his compositions \citep[162]{Gesner1774}.

\section{Labyrinths and enigmas}\label{sec:3.3}

Mastering the Greek dialects is not an easy endeavor, not even today with so many tools available to the student of Ancient Greek. In the early modern period, the complexity of the matter was so frequently stressed that it may well be called a topos. I have already mentioned how Juan Luis Vives warned his readers of the “vast labyrinths” in the Greek dialects. A German poet compared the phenomenon of Greek dialectal diversity to the enigma of the sphinx, which required a new Oedipus in order to be solved.\footnote{See the poem by Georg Meisner in \citet[††.2\textsc{\textsuperscript{r}}]{Walper1589}, where the Hellenist Otto Walper is dubbed “Oedipus alter”. On Walper, see also Van Rooy (fc. c).} In the early eighteenth century, a French classical scholar characterized the Greek dialects as “difficult nonsense” (\textit{difficiles nugae}), the analysis of which constituted a task unappealing to a scholar of standing (\citealt{Maittaire1706}: \textsc{a.4}\textsc{\textsuperscript{r}}). At the same time, he regarded dialectal diversity as a boring topic and – with reference to Juvenal’s \textit{Satires} 7.154 – as “reheated cabbage”, i.e. a topic discussed over and over again by scholars before him (\citealt{Maittaire1706}: \textsc{a.4}\textsc{\textsuperscript{r–v}}). The dialects also troubled early modern interpreters and translators, who often failed to arrive at a correct understanding of Greek texts because they had not mastered the dialectal diversity of the language (see \citealt{Facius1782}: \textsc{iii–iv}). What is more, even Ancient and Byzantine Greek scholars had great difficulties with them, which is why they had composed treatises on the subject. That was at least the claim of a late eighteenth-century Dutch Hellenist, who edited several Byzantine works on the dialects (see \citealt{Koen1766}: \textsc{xvii–xviii}).

Only few Hellenists argued that the dialects were easy to learn. The German scholar Erasmus \citet[):(.2\textsc{\textsuperscript{r}}]{Schmidt1604}, professor of Greek, Hebrew, and mathematics in Wittenberg stressed that Greek dialectal diversity, if taught well, caused no difficulties. Before Schmidt, it had been put forward that, after mastering the basics of grammar, achieving competence in a dialect would take one or two hours only (\citealt{Caselius1560}: \textsc{e.6}\textsc{\textsuperscript{v}}). This view was shared by an eighteenth-century French Jesuit, who added that it sufficed for a student to know that a certain feature was dialectal, without being able to tell to what dialect it belonged exactly. In fact, the same mutations could pertain to various dialects or could even be transferred to the common language \citep[101]{Giraudeau1739}.

Most scholars did, however, agree on the difficulty of the Greek dialects, which was reflected in their presentation of this subject matter in their handbooks for the language. In fact, grammarians of Greek adopted several strategies in dealing with the issue. The first and most important one was to clearly separate the dialects from the Koine, since scholars usually assumed that Koine forms were sufficient for beginning students.\footnote{See e.g. \citet[(i)]{Da1501} and \citet[223]{Tavoni1986}. Cf. \citet[aa.ii\textsc{\textsuperscript{v}}]{Glarean1524}; \citet[\textsc{a}.ii\textsc{\textsuperscript{v}}]{Metzler1529}; \citet[105]{Rollin1726}. One eighteenth-century German grammarian was exceptional in holding that the dialects were also to be tackled by beginning students, since they were omnipresent (\citealt{Trendelenburg1782}: 174–175).} The renowned French Hellenist Henri Estienne emphasized that knowledge of the dialects was not necessary to correctly decline and conjugate Greek nouns and verbs. The Koine/dialect separation could occur in different manners. Some grammars omitted dialectal information altogether in order to avoid overcomplication, whereas others reserved an entirely separate booklet for the issue.\footnote{\citet[\textsc{b.}iv\textsc{\textsuperscript{r}}]{Caselius1560} omitted the dialects, whereas \citet[†.6\textsc{\textsuperscript{v}}–†.7\textsc{\textsuperscript{r}}]{Walper1589} treated them in a separate booklet.} In other cases, dialect forms were discussed after the Koine had been described, which was Philipp Melanchthon’s \textit{modus operandi} in his Greek grammar (see e.g. \citealt{Melanchthon1518}: g.iv\textsc{\textsuperscript{v}}). Scholars often opted to typographically distinguish dialect from Koine forms, usually by employing fonts of different sizes. As can be expected, dialect forms were as a rule in smaller print than Koine forms.\footnote{See the method of presentation in e.g. \citet{Gretser1593}, \citet{Anon.1613}, and \citet{Lancelot1655}.}

Another way to separate Koine from dialect information was to postpone the matter to later sections of the grammar. A particular case in point is Urbano Bolzanio (1442–1524). In his book on Renaissance grammars of Greek, Paul \citet[36--40]{Botley2010} has described how this Italian Hellenist revised his successful Greek grammar several times, while trying to find a more adequate manner to include dialect forms in his manual. The first edition of his grammar contained information on the dialects throughout (cf. \citealt{Bolzanio1497}: e.vi\textsc{\textsuperscript{v}}). Aware of the difficulties this raised for students, Bolzanio relegated dialect forms to a second part in his revised text of 1512. This was intended for more advanced students who had successfully studied the first introductory part (\citealt{Bolzanio1512}: \textsc{h}.iii\textsc{\textsuperscript{r}}). In the posthumously published second revision, he elaborated further upon this bipartition. He excluded information on the dialects from the first three books, intended for beginners. Advanced students could direct themselves to the six subsequent books. These contained a description of dialect forms, especially book four, “On the variety of tongues” (“De linguarum uarietate”; \citealt{Bolzanio1545}: 60\textsc{\textsuperscript{v}}).

A second strategy consisted in presenting the dialects and their particularities in a way that was as accessible and didactically effective as possible. For instance, in an attempt at facilitating the study of dialect rules, the French Hellenist and Port-Royal professor Claude Lancelot (ca. 1615–1695) composed mnemonic verses describing the most important features (\citealt{Lancelot1655}: \textsc{xiv-xv;} cf. also \citet[]{Anon.1725}. On the Attic dialect, Lancelot mused in absurd French verses, which I refrain from translating:

\begin{quote}
1 Contracter l’Attique aime, 2 et des voix le meslange:
\end{quote}

\begin{quote}
3 Son ς en ξῖ, ῥῶ, ταῦ, assez souvent il change:
\end{quote}

\begin{quote}
4 Oste ι d’αϊ, εϊ; 5 d’\textit{omicron} fait ω grand,
\end{quote}

\begin{quote}
6 oὖν à la fin des mots, 7 aux adverbes ι prend.\footnote{\citet[558]{Lancelot1655}. For Ionic, see p. 560, for Doric p. 561, and for Aeolic p. 563.}
\end{quote}

Despite Lancelot’s good intentions, one might wonder how a student of Greek would benefit from these dense and enigmatic verses.

More often, grammarians introduced synoptic and systematized overviews in a schematic form, a method of presentation absent from the Ancient and Byzantine Greek tradition but widespread in early modern grammatical and typographical practice. In this approach, dialectal data were usually presented per linguistic feature – either in separate booklets or scattered throughout grammars – rather than per dialect, as their ancient and medieval predecessors had done.\footnote{For an example of a separately published booklet, see \citet{Amerot1530}, originally part of a grammar \citep{Amerot1520}. For an instance of schematized presentation of dialectal features throughout a grammar, see \citet{Gretser1593}.} This no doubt stimulated a contrastive comparison of the dialects and perhaps also of other languages, as not long after this method of presentation appeared scholars started to compare different languages while trying to assess their degree of kinship.\footnote{On this comparative turn, see e.g. \citet{Considine2010b}, with further references.} A number of scholars combined the per dialect and the per linguistic feature approach, discussing each dialect in separate chapters but structuring every chapter by means of grammatical properties (see e.g. \citealt{Zwinger1605}; \citealt{Merigon1621}). Grammarians usually did not comment on their motivation in adopting a specific method of presentation. An exception is an early eighteenth-century Hellenist who criticized the per linguistic feature structure and preferred the traditional approach per dialect because he regarded it as more transparent (\citealt{Heupel1712}: ):(.3\textsc{\textsuperscript{r}–}):(.3\textsc{\textsuperscript{v}}).

The variety of strategies adopted in presenting dialect forms triggered some debate. For instance, the schematic presentations in table form could become rather complex, thus losing their didactic perspicuity. This seems to have been one of the reasons for the grammarian Johann Friedrich \citet[\textsc{viii}]{Facius1782} to criticize the manuals of his predecessors. To remedy this, Facius composed his own handbook, granting, however, that practical considerations also motivated him to write it, as earlier manuals were difficult to find in bookshops. Around the same time, Friedrich Gedike blamed both ancient and recent grammarians of Greek for having obscured the study of the dialects. Specifically, Gedike criticized existing handbooks, because they “are all a dark chaos of piled up examples” lacking “a philosophical view on the entire matter”.\footnote{\citet[4]{Gedike1782}: “alle sind ein dunkles Chaos aufgehäufter Exempel, nirgends ist ein philosophischer Blik über das Ganze”.}

\section{Conclusion: Dialectology as an ancillary subfield of Greek philology}\label{sec:3.4}

In conclusion, the ancient Greek dialects principally attracted philological interest in the early modern period, albeit not so much as a topic in and of itself. Dialect studies were as a rule subsidiary to philological goals and skills that were considered more important. These principally included the ability to read Greek literature, to achieve more accurate etymological insights into the Greek language, to edit Greek literary texts, and to compose texts in the Greek language and its different dialects. The dialects were widely perceived as difficult subject matter, which grammarians presented in various ways in order to make it as clear as possible to their readership of would-be Hellenists. This led in some cases to a critique of the approach of others but more often to a struggle with presenting the dialects in a didactically effective manner. The opportunities offered by the printing press greatly facilitated this endeavor, as this new technique allowed for convenient schematic visualizations of Greek linguistic diversity.

Even though manuals for the Greek dialects were conceived primarily as auxiliary tools, Hellenists began to regard the study of the ancient Greek dialects as a separate subfield within Greek philology, especially in the eighteenth century. The Bible scholar Christian Siegmund Georgi (1702–1771) explicitly interpreted \textit{dialectologia} in the sense of “analysis and description of the ancient Greek dialects” as a distinct scholarly activity.\footnote{\citet[16]{Georgi1733}. Cf. also \citet[b.1\textsc{\textsuperscript{v}}]{Nibbe1725}; \citet[15]{Hauptmann1737}. On the history of the term \textit{dialectologia}, see \citet[]{VanRooyFcd}.} It is remarkable that this happened at a time when Greek studies were in crisis according to contemporary sources.\footnote{See \citet[86--90]{Reinhard1724}, where four causes of this crisis are offered.} There are indications, however, that even before this time scholars considered the study of the Greek dialects to be a separate and specialized branch of learning. Many Hellenists provided a state of the art of Greek dialect studies at the outset of their discussion (cf. \citealt{VanRooy2014}: 519–520). Initially, only Ancient and Byzantine Greek scholars were mentioned in these accounts, as is to be expected.\footnote{See e.g. Reuchlin (1477/1478 in \citealt{VanRooy2014}: 509–510); \citet[\textsc{r.}iii\textsc{\textsuperscript{r}}]{Amerot1520} \&  \citet[title]{Amerot1530}; \citet[α.3\textsc{\textsuperscript{r}}]{Ruland1556}.
\citet[a.3\textsc{\textsuperscript{v}}]{Canini1555}, however, complained that his predecessors had neglected the problem of Greek dialectal diversity.} From the end of the sixteenth century onward, early modern scholars were also included.\footnote{See e.g. \citet[†.7\textsc{\textsuperscript{r}}]{Walper1589} and \citet[1]{Schmidt1604}, referring to, among others, Bolzanio, Clenardus, Antesignanus, and Henri Estienne.} Some scholars, such as the Basel-born physician and Hellenist Jakob Zwinger (1569–1610), carefully indicated their sources, both Greek and early modern, in their handbooks on the dialects \citep{Zwinger1605}. This likely demonstrates that early modern Hellenists widely regarded the study of the dialects as a well-defined subfield of Greek philology, and that they considered knowledge of the Greek dialects to be indispensable for becoming a true man of letters.

