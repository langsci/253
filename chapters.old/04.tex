\chapter{Dialects in the mixer: Homeric and Biblical Greek}\label{chap:4}
\begin{quote}
For the entire Greek tongue is divided into five tongues, into the common, the Ionic, the Doric, the Aeolic, and the Attic. And they say that that sublime genius of Homer has inserted these into his own works in such a manner that each tribe of Greece can recognize its particularities in his work. For his divine genius could not be confined and restrained within the limits of sound and utterly pure Attic speech.\footnote{\citet[{\textsc{E}}.iii\textsc{\textsuperscript{v}}]{Oreadini1525}: “Graeca namque uniuersalis lingua in quinque linguas diuiditur, in communem, Ionicam, Doricam, Aeolicam et Acticam, quas ferunt sublime illud Homeri ingenium ita suis opibus inseruisse, ut unaquaeque gens Graeciae sua apud illum idiomata recognoscant. Diuinum enim ingenium non potuit coangustari retinerique intra limites sinceri purissimique Attici sermonis”.}
\end{quote}

The obscure early sixteenth-century Italian humanist Vincenzo Oreadini was concerned about Italian orthography and the problems dialectal diversity caused in this regard. More specifically, Oreadini published a book in 1525 on the question of whether new letters should be added to the Italian alphabet. In this work, published in Perugia, he compared the variation in his native tongue with the ancient Greek dialects. In passing, he praised the linguistic genius of Homer, which transcended Greek tribal divisions and allowed his \textit{Iliad} and \textit{Odyssey} to be enjoyed by all Greeks. Oreadini’s was only one of many premodern explanations of the peculiar nature of Homeric Greek. Most scholars did, however, agree that his language was a mixture of different dialects. A similar idea was put forward to account for the peculiar character of the Greek of the Bible. What is the history behind these mixed conceptions of Homeric and Biblical Greek, both of which still cause problems to present-day linguists?

\section{Homeric Greek: Puzzling scholars since antiquity}\label{sec:4.1}

In ancient and medieval scholarship, the mixed nature of Homeric Greek was widely accepted. One of the oldest testimonies dates back to the late first century \textsc{ad}. In the orations of Dio Chrysostom (ca. \textsc{ad} 40–after 112), a long passage was devoted to Homer’s Greek, treated as part of a larger argument on the function of poetry and other arts. Dio’s ideas deserve to be quoted in full here:

\begin{quote}
Very great indeed is the ability and power of man to express in words any idea that comes into his mind. But the poets’ art is exceedingly bold and not to be censured therefor; this was especially true of Homer, who practiced the greatest frankness and freedom of language; and he did not choose just one variety of diction, but mingled together every Hellenic dialect which before his time were separate – that of the Dorians and Ionians, and also that of the Athenians – mixing them together much more thoroughly than dyers do their colors – and not only the languages of his own day but also those of former generations; if perchance there survived any expression of theirs taking up this ancient coinage, as it were, out of some ownerless treasure-store, because of his love of language; and he also used many barbarian words as well, sparing none that he believed to have in it anything of charm or vividness. Furthermore, he drew not only from things which lie next door or near at hand, but also from those quite remote, in order that he might charm the hearer by bewitching and amazing him; and even these metaphors he did not leave as he first used them, but sometimes expanded and sometimes condensed them, or changing them in some other way. And, last of all, he showed himself not only a maker of verses but also of words, giving utterance to those of his own invention, in some cases by simply giving his own names to the things and in others adding his new ones to those current, putting, as it were, a bright and more expressive seal upon a seal. He avoided no sound, but in short imitated the voices of rivers and forests, of winds and fire and sea, and also of bronze and of stone, and, in short, of all animals and instruments without exception, whether of wild beasts, birds, or pipes and reeds. […] As a result of this epic art of his, he was able to implant in the soul any emotion he wished.\footnote{Dio Chrysostom, \textit{Orationes} 12.66–69: “πλείστη μὲν oὖν ἐξoυσία καὶ δύναμις ἀνθρώπῳ περὶ λόγoν ἐνδείξασθαι τὸ παραστάν. ἡ δὲ τῶν πoιητῶν τέχνη μάλα αὐθάδης καὶ ἀνεπίληπτoς, ἄλλως τε Ὁμήρoυ, τoῦ πλείστην ἄγoντoς παρρησίαν, ὃς oὐχ ἕνα εἵλετo χαρακτῆρα λέξεως, ἀλλὰ πᾶσαν τὴν Ἑλληνικὴν γλῶτταν διῃρημένην τέως ἀνέμιξε, Δωριέων τε καὶ Ἰώνων, ἔτι δὲ τὴν Ἀθηναίων, εἰς ταὐτὸ κεράσας πoλλῷ μᾶλλoν ἢ τὰ χρώματα oἱ βαφεῖς, oὐ μόνoν τῶν καθ' αὑτόν, ἀλλὰ καὶ τῶν πρότερoν, εἴ πoύ τι ῥῆμα ἐκλελoιπός, καὶ τoῦτo ἀναλαβὼν ὥσπερ νόμισμα ἀρχαῖoν ἐκ θησαυρoῦ πoθεν ἀδεσπότoυ διὰ φιλoρρηματίαν, πoλλὰ δὲ καὶ βαρβάρων ὀνόματα, φειδόμενoς oὐδενὸς ὅ τι μόνoν ἡδoνὴν ἢ σφoδρότητα ἔδoξεν αὐτῷ ῥῆμα ἔχειν\text{\textgreek{;}} πρὸς δὲ τoύτoις μεταφέρων oὐ τὰ γειτνιῶντα μόνoν oὐδὲ ἀπὸ τῶν ἐγγύθεν, ἀλλὰ τὰ πλεῖστoν ἀπέχoντα, ὅπως κηλήσῃ τὸν ἀκρoατὴν μετ' ἐκπλήξεως καταγoητεύσας, καὶ oὐδὲ ταῦτα κατὰ χώραν ἐῶν, ἀλλὰ τὰ μὲν μηκύνων, τὰ δὲ συναιρῶν, τὰ δὲ ἄλλως παρατρέπων. Tελευτῶν δὲ αὑτὸν ἀπέφαινεν oὐ μόνoν μέτρων πoιητήν, ἀλλὰ καὶ ῥημάτων, παρ' αὑτoῦ φθεγγόμενoς, τὰ μὲν ἁπλῶς τιθέμενoς ὀνόματα τoῖς πράγμασι, τὰ δ' ἐπὶ τoῖς κυρίoις ἐπoνoμάζων, oἷoν σφραγῖδα σφραγῖδι ἐπιβάλλων ἐναργῆ καὶ μᾶλλoν εὔδηλoν, oὐδενὸς φθόγγoυ ἀπεχόμενoς, ἀλλὰ ἔμβραχυ πoταμῶν τε μιμoύμενoς φωνὰς καὶ ὕλης καὶ ἀνέμων καὶ πυρὸς καὶ θαλάττης, ἔτι δὲ χαλκoῦ καὶ λίθoυ καὶ ξυμπάντων ἁπλῶς ζῴων καὶ ὀργάνων, τoῦτo μὲν θηρίων, τoῦτo δὲ ὀρνίθων, τoῦτo δὲ αὐλῶν τε καὶ συρίγγων […]. ὑφ' ἧς ἐπoπoιίας δυνατὸς ἦν ὁπoῖoν ἐβoύλετo ἐμπoιῆσαι τῇ ψυχῇ πάθoς”. The English translation is taken over from the Loeb series.}
\end{quote}

According to Dio, Homer’s intricate mix of dialects was more perfect than the way in which dyers dyed clothes in various colors. Apart from different dialects, Homer’s speech was also marked by archaisms, barbarian words, and neologisms. Using these various linguistic devices, Homer was able to evoke whatever emotion he wanted. Dio, in other words, believed Homer to have resorted to linguistic mixing for psychological effect rather than to transcend Greek tribal divisions, as Vincenzo Oreadini later suggested in the early sixteenth century.

Dio’s characterization of Homeric Greek roughly matches present-day views, even though the fundamental assumptions of Dio and modern research are quite different.\footnote{For modern views, see e.g. \citet{Hackstein2010} and \citet{Ruijgh2011}.} Much like Dio, scholars today assume that Homeric Greek is a mixed, multilayered, and artificial literary koine, but they do so against the background of historical-comparative linguistics rather than that of artistic functionality, as Dio had done. Contemporary linguists have demonstrated that Homer’s language is principally Ionic in nature, which likely shows that an important phase of redaction took place in central Ionic territory. There are, however, many Aeolic features, too, perhaps because there was an earlier or parallel epic tradition in this dialect on which the poet(s) behind the \textit{Iliad} and the \textit{Odyssey} elaborated. In the sixth century \textsc{bc}, the Athenian ruler Peisistratus commissioned the production of a definitive version of Homer’s poems, which likely explains the presence of some distinctively Attic features. Like Dio, modern scholars have also identified archaisms in Homer’s speech, such as the absence of the definite article.

The rhetorician Hermogenes of Tarsus, active during the reign of Marcus Aurelius (reigned \textsc{ad} 161–180), agreed with Dio that Homer’s speech was mixed, but added, much like modern linguists now, that Ionic predominated, as this dialect was both poetic and sweet in nature. Hermogenes did so in his treatise on style, while commenting on the language of the Ionic historian Herodotus rather than that of Homer.\footnote{Hermogenes of Tarsus, Περὶ ἰδεῶν λόγoυ 2.4: “ἡ γὰρ Ἰὰς oὖσα πoιητικὴ φύσει ἐστὶν ἡδεῖα. εἰ δὲ καὶ ἄλλων διαλέκτων ἐχρήσατό τισι λέξεσιν, oὐδὲν τoῦτo, ἐπεὶ καὶ Ὅμηρoς καὶ Ἡσίoδoς καὶ ἄλλoι oὐκ ὀλίγoι τῶν πoιητῶν ἐχρήσαντo μὲν καὶ ἄλλαις τισὶ λέξεσιν ἑτέρων διαλέκτων, τὸ πλεῖστoν μὴν ἰάζoυσι, καὶ ἔστιν ἡ Ἰὰς ὅπερ ἔφην πoιητική πως, διὰ τoῦτo δὲ καὶ ἡδεῖα”.} In fact, Hermogenes did not only regard Homer’s speech as mixed, but also that of, among others, the ancient didactic poet Hesiod and, oddly, that of Herodotus himself, whose language scholars today regard as straightforwardly Ionic.

However, not all ancient authors agreed that Homer’s language was mixed. The orator Aelius Aristides (\textsc{ad} 117–after 177), active in the decades between Dio and Hermogenes, conceived of Homeric speech as essentially Attic. Aristides expressed this idea in his \textit{Panathenaicus}, a speech held on the occasion of the Panathenaeic games of the year 155. As can be expected, the city of Athens was extravagantly praised in this oration, declaimed in Atticizing diction – one of his main models was the ancient rhetorician Demosthenes. Aristides maintained that Athens could stake a claim to Homer’s poetry as well, for two reasons. Not only did the great poet originate from Smyrna, a colony of Athens, but his language, too, was Attic, a statement not further substantiated by linguistic evidence.\footnote{Aelius Aristides, Παναθηναϊκός Jebb page 181: “εἰ δὲ δεῖ καὶ τῆς Ὁμήρoυ μνησθῆναι, μετέχει καὶ ταύτης τῆς φιλoτιμίας ἡ πόλις, oὐ μόνoν διὰ τῆς ἀπoίκoυ πόλεως, ἀλλ’ ὅτι καὶ ἡ φωνὴ σαφῶς ἐνθένδε”.}

The most important ancient source on the language of Homer was, however, a double biography of the poet, for a long time incorrectly attributed to the prolific writer and biographer Plutarch of Chaeronea (ca. \textsc{ad} 45–before 125). The work likely dates to the Roman period, but this question is complicated by the fact that it received later additions. In a part of this biography, Homer’s mixed use of the dialects was treated. Having visited each tribe of Greece, the poet inserted forms of every dialect into his compositions, according to the biographer. Indeed, Pseudo-Plutarch imagined Homer’s speech as a kind of linguistic potpourri: “In using a variegated diction, he mingled together the distinctive forms of each of the Greek dialects, out of which it is clear that he has visited the whole of Greece and each tribe”.\footnote{Pseudo-Plutarch, \textit{De Homero 2} 8 (ed. \citealt{Kindstrand1990}: 9–10): “λέξει δὲ πoικίλῃ κεχρημένoς τoὺς ἀπὸ πάσης διαλέκτoυ τῶν Ἑλληνίδων χαρακτῆρας ἐγκατέμιξεν, ἐξ ὧν δῆλός ἐστιν πᾶσαν [μὲν] Ἑλλάδα ἐπελθὼν καὶ πᾶν ἔθνoς”.} In contrast to his ancient colleagues, who did not do much more than briefly comment on Homer’s Greek, Pseudo-Plutarch tried to substantiate his claims by means of linguistic evidence. He mentioned actual dialect features in his treatment of Homer’s mixed speech. Three Doric features were discussed, among which was the shortening of words, claimed to be typical of that dialect. Six Aeolic, nine Ionic, and twelve Attic characteristics were likewise described (see \citealt{VanRooy2018c} for a more detailed overview). In keeping with these numbers, Attic was claimed to be the principal dialect of Homer, which, Pseudo-Plutarch argued on unclear grounds, was not unsurprising since that dialect had a mixed nature itself (see also Chapter 7, \sectref{sec:7.2}, \tabref{tab:7.1}). An account of two syntactic particularities in Homer’s speech, one from Attic and the other from Doric, rounded off the linguistic analysis of Homer’s language. Pseudo-Plutarch subsequently concluded:

\begin{quote}
It is clear, then, how, in mustering the sounds of all Greeks, he creates a richly varied discourse and sometimes employs unusual utterances, as the aforementioned are, and sometimes ancient ones, as for example each time he says \textit{áor} [‘sword’] and \textit{sákos} [‘shield’], and sometimes common and usual ones, such as each time he says \textit{ksíphos} [‘sword’] and \textit{aspís} [‘shield’]. And one might wonder that even common words preserve with him the elevation of his style.\footnote{Pseudo-Plutarch, \textit{De Homero 2} 14 (ed. \citealt{Kindstrand1990}: 14–15): “ὅπως μὲν oὖν τὰς πάντων Ἑλλήνων φωνὰς ἀθρoίζων πoικίλoν ἀπεργάζεται τὸν λόγoν καὶ χρῆται πoτὲ μὲν ταῖς ξέναις, ὥσπερ εἰσὶν αἱ πρoειρημέναι, πoτὲ δὲ ταῖς ἀρχαίαις, ὡς ὅταν λέγῃ ἄoρ καὶ σάκoς, πoτὲ δὲ ταῖς κoιναῖς καὶ συνήθεσιν, ὡς ὅταν λέγῃ ξίφoς καὶ ἀσπίδα, δῆλoν. καὶ θαυμάσειέ τις ὅτι καὶ κoιναὶ λέξεις παρ’ αὐτῷ σῴζoυσι τὸ σεμνὸν τoῦ λόγoυ”.}
\end{quote}

In his final paragraph on Homer’s language, the author stressed once more the composite nature of Homer’s speech, but he did not only point to dialectal features, here somewhat oddly termed “foreign, unusual” (\textit{ksénos}/ξένoς). He also noticed archaisms and the use of common words. Pseudo-Plutarch, in other words, seems to have suggested that Homer’s speech also contained Koine elements. If so, this would betray an ahistorical conception of the Koine, for Homer is today usually placed in the eighth century \textsc{bc}, whereas the Koine only emerged in the wake of Alexander the Great’s (356–323 \textsc{bc}) conquests. Such anachronistic ideas about the Greek language were, however, not unusual in premodern scholarship.

Pseudo-Plutarch’s discussion of the dialects turned out to be very welcome to Renaissance humanists, who eagerly read it in their attempts at deciphering Homer’s difficult poems. What is more, they excerpted it from the biography and printed it separately from the original work, most often together with two Byzantine treatises on the Greek dialects by John the Grammarian and Gregory of Corinth (see \citealt{VanRooy2018c}). The latter two works did not comment extensively on Homer’s Greek, even though Gregory suggested that the poet composed in Ionic (\textit{De dialectis} 1), without elaborating on this statement.

\section{In Plutarch’s footsteps: Renaissance ideas on Homer’s speech}\label{sec:4.2}

All in all, ancient and Byzantine scholars were not too much concerned by the peculiar nature of Homeric Greek. In the Renaissance, however, scholars problematized the matter to a far greater extent. The speech of Homer was part of the larger issue of the language of Greek poets in general, for which the concept of \textsc{poetical} \textsc{dialect} was devised, as I have demonstrated earlier (Chapter 2, \sectref{sec:2.7}). Still, ideas on Homer’s Greek deserve a separate treatment here, all the more so since his work occupied a prominent position in early modern teaching of, and scholarship on, Greek language and literature (see e.g. \citealt{Botley2010}: 81–85). What is more, scholars tended to linger on Homer’s speech at greater length than on the speech of other poets.

When Renaissance scholars were able to move beyond the basics of the Greek tongue and started to become interested in the diversity of this language, Pseudo-Plutarch’s widely known analysis of Homeric Greek was one of their primary starting points. The case of Vincenzo Oreadini, cited at the outset of this chapter, may stand as an example of this, especially since, much like Pseudo-Plutarch, he seems to have assumed that Homer was a traveling poet, who through his mixed speech neutralized Greek tribal divisions. This should come as no surprise, as Plutarch was considered a trustworthy ancient authority, and only few scholars disputed his authorship of the treatise (\citealt{VanRooy2018c}). In fact, there seems to have been a consensus among humanists, in Plutarch’s alleged tracks, that Homer mixed the four canonical dialects with common words in his epic poems. An early example can be retrieved from the \textit{Oration in the course of explaining Homer}, held in 1486/1487 by the pioneering Hellenist Angelo Poliziano (1454–1494) in Florence, the primary crib of Greek studies in Italy. A professor of Greek, Poliziano read Homer with his students, to whom he explained that

\begin{quote}
both [the \textit{Iliad} and the \textit{Odyssey}] were produced from all the tongues the Greeks call “dialects”, so that every tribe of Greece could discover its own proper features with him. Yet he does not reject common words either.\footnote{See Poliziano’s \textit{Oratio in expositione Homeri} (= \citealt{Poliziano1553}: 479): “utraque […] linguis [...] ex omnibus quas διαλέκτoυς Graeci uocant, conflata est, sic ut unaquaeque Graeciae gens sua apud illum idiomata deprehendat. Neque tamen communia respuit uerba”.}
\end{quote}

Poliziano’s words obviously echoed Pseudo-Plutarch’s comments, even though he did not explicitly refer to the ancient author. In fact, it is not inconceivable that Poliziano had a hand in excerpting the section on the Greek dialects in Homer from the biography associated with Plutarch. For the extant manuscripts suggest that the extraction occurred toward the end of the fifteenth century in northern Italy in the humanist circles to which Poliziano belonged (\citealt{VanRooy2018c}). What is more, Poliziano contributed to editing the collection of grammatical treatises issued in Venice by Aldus Manutius, which contained the first separate edition of the excerpt, where it was, however, attributed to the Byzantine commentator of Homer Eustathius of Thessalonica. In a later edition, Manutius changed this to Plutarch. Like Poliziano, the Venetian printer-scholar claimed that Homer’s speech was mixed, stating hyperbolically that he used all dialects and not only the principal ones (\citealt{Manutius1496Aldus}: *.ii\textsc{\textsuperscript{v}}).

The idea that Homer mixed the principal dialects in his poetry remained a common opinion throughout the entire early modern period; these principal dialects could be either four or five in number, depending on whether the Koine was included.\footnote{See e.g. \citet[\textsc{aa.}i\textsc{\textsuperscript{r}}]{Simler1512}; \citet[\textsc{a}.viii\textsc{\textsuperscript{r}}]{Liburnio1546}; \citet[xxxiv]{Lancelot1655}; \citet[20--23]{Grosch1753}; \citet[196]{Ries1786}.} Some scholars brought the Pseudo-Plutarchan view to a head by claiming that Homer could speak five dialects in one single verse.\footnote{See e.g. \citet[\textit{s.v.} “dialecte”]{Furetiere1701}; \citet[\textsc{i.}203, 3rd sequence of pagination]{Chambers1728}; \citet[934]{Dumarsais1754}.} The dialect mixture was often explained by means of Pseudo-Plutarch’s image of Homer as a traveling poet who had visited the whole of Hellas in order to be understood by all Greeks, which was also why he had introduced features common to all dialects into his speech. Poliziano was very explicit about this. Some humanists provided additional explanations. The quotation from Oreadini heading this chapter, for example, might reflect Poliziano’s insistence on the idea that Homer wanted his diverse Greek audience to recognize features of their own dialects in his work. Oreadini was, however, idiosyncratic in arguing that Homer’s genius could not be contained within the limits of the elegant Attic dialect. The Spanish scholar in exile Juan Luis \citet[\textsc{x}.iiii\textsc{\textsuperscript{r}}]{Vives1533} offered a different but vague explanation, as he claimed that Homer’s mixed language was the result of the fact that he considered all dialects to be one and could therefore draw no boundaries between them. Amalgamating the dialects was, in other words, natural for Homer, Vives suggested. Later scholars simply classified Homer as one of the poets using the mixed poetical dialect which they had introduced into their dialect divisions.\footnote{See e.g. \citet[6\textsc{\textsuperscript{r}}-6\textsc{\textsuperscript{v}}]{Baile1588}; \citet[333]{Alsted1630}; \citet[161]{Gesner1774}.} The great classical scholar Joseph Justus Scaliger, for instance, claimed that he had learned Greek in twenty-one days by studying Homer, during which time he had composed for himself a grammar of the poetical dialect based on his reading of this author (\citealt{Scaliger1594}: 56; see also Chapter 2, \sectref{sec:2.7}).

The idea that Homer mixed different dialects was by far the most popular explanation for the odd appearance of his speech. Like some of their ancient predecessors, a number of humanists claimed that one specific dialect predominated in Homer’s poems. In Hermogenes’s tracks, a large number of scholars assumed that Homer favored his allegedly native Ionic in mixing the dialects, whereas others followed Pseudo-Plutarch in proposing that he mainly used Attic.\footnote{For Ionic, see e.g. \Citet[47\textsc{\textsuperscript{r}}]{Da1509}; \citet[215]{Ringelbergh1541}; \citet[167]{Labbe1639}; \citet[\textsc{b.3}\textsc{\textsuperscript{v}}]{KirchmaierCrusius1684}; \citet[b.2\textsc{\textsuperscript{r}}, 334]{Nibbe1725}; \citet[161]{Gesner1774}. For Attic, see e.g. \citet[\textsc{f.}v\textsc{\textsuperscript{r}}]{Codro1502}; \citet[96\textsc{\textsuperscript{r}}]{Waser1610}; \citet[514]{Fabricius1711}.} Still others suggested a combined solution, stating that Homer mainly used both Attic and Ionic (e.g. \citealt{Schmidt1604}: 2; \citealt{Rhenius1626}: 84). One eighteenth-century author assumed that Homer principally mixed Ionic and Aeolic because he was born in the Ionian city of Smyrna to an Aeolic family and lived for many years on Chios, an Ionic island close to Aeolia (\citealt{Reynolds1752}: vi). Their suggestions were, however, usually not backed by linguistic evidence but by speculation, in which respect they differed from Pseudo-Plutarch’s account.

\section{Toward a historical solution}\label{sec:4.3}

Not all early modern scholars were convinced that Homer purposely mixed different dialects in his speech. Especially in the eighteenth century, scholars sought more convincing alternatives. Why and in which context did they do so? In the eighteenth century, much progress was achieved in Homeric scholarship, especially in Britain and German-speaking areas, where Greek philology still flourished, unlike in many other regions of Europe. In this period, the so-called Homeric question emerged: who was Homer? Scholars increasingly tried to contextualize this mystified poet in historical terms.\footnote{Primary sources central to the genesis of the Homeric question include \ia{Blackwell, Thomas@Blackwell, Thomas}\citet[]{[blackwell]1735}, \citet{Wood1775}, and \citet{Wolf1795}. On Blackwell and Wood, see \citet[90--108]{Bauman2003}. On Wolf, see \citet[55–57]{Sandys_history_1908-1}.} Concomitantly, they gained ever better insight into many aspects of Homer’s language. Most notably, the English philologist Richard Bentley (1662–1742) solved a major metrical problem by restoring the \textit{digamma} in the Homeric text. This ancient letter, representing a [u̯] sound, had been lost in most canonical dialects of Greek, but not in Aeolic, which is why it was often called “the Aeolic digamma” in premodern scholarship. The sound must have also been present in the original Homeric text, but was lost in one of the redactions the poems underwent. As a matter of fact, in many cases, one should presuppose the presence of a digamma in order to have a metrically correct verse.\footnote{On Richard Bentley and the digamma, see e.g. \citet[407]{Sandys1908} and especially \citet[182--186]{Haugen2011}.}

A heightened sense of the historicity of Homer’s epic poems and their language stimulated the idea that Homeric speech was an archaic variety of the Greek language. This view came in different guises. For instance, in early November 1709, a disputation on the Greek Koine was presented in Wittenberg by the young Hellenist Georg Friedrich Thryllitsch under the supervision of the professor of Greek Georg Wilhelm Kirchmaier. It is not known who exactly authored the disputation – Thryllitsch, Kirchmaier, or both of them together – but nine months earlier Thryllitsch had presented another disputation on the Greek dialects from a historical perspective. This text was certainly authored by Thryllitsch alone, and since its content shows some similarities with that of the later dissertation, one might argue that Thryllitsch and not Kirchmaier composed the later one as well (see Chapter 2, \sectref{sec:2.9}). Whatever the case, the text made an interesting, historically informed suggestion about the nature of Homeric Greek. There was, it claimed, a very ancient variety of Greek, which was called “Hellenic” or also “Ancient Attic” and was taken to be a kind of Proto-Greek language, to use a term from modern linguistics. This form of Greek was extinct by Homer’s time at the latest, and the specificity of the poet’s language should be partly explained by the fact that “residues” (\textit{rudera}) of this no longer extant Hellenic variety were still noticeable in his work (\citealt{Kirchmaier1709}: \textsc{b.4}\textsc{\textsuperscript{v}}).

Others identified Homer’s tongue with a variety of ancient Ionic. The German Hellenist Friedrich Gedike elaborated most extensively on this idea in his article on the Greek dialects of 1782. \citet[22--23]{Gedike1782} argued that Homer wrote in an ancient Ionic variety that had not yet been differentiated clearly from Attic, its mother dialect. It was moreover influenced by the speech of Dorians and Aeolians who roamed in Attica before migrating to other regions around the same time as the Ionians did. In doing so, Gedike made Attica the heartland of the Greek people and its language and rejected the idea attributed to Pseudo-Plutarch that Homer, as a traveling poet, purposely mixed different dialects in his language.\footnote{For ideas similar to Gedike’s, see e.g. \citet[115--116]{Freret1809}; \citet[202]{Beattie1778}; \citet[175--176]{Trendelenburg1782}.}

A final ingenuous solution was proposed by the English classical scholar Robert Wood (1717–1771), a major figure in the history of the Homeric question. Wood claimed that in Homer’s time the dialects had not yet been clearly distinguished, as there was not yet a cultivated state of language, a “standard” \is{standard (language)}in his terms. This made it to Wood’s mind anachronistic to state that Homer mixed various dialects, as it was impossible for him to use one clearly demarcated form of speech.\footnote{\citet[238]{Wood1775}. See also \citet[\textsc{xxiiii–xxv}]{Harles1778}, who elaborated upon this view; \citet[\textsc{v}]{Facius1782}.} Wood thus suggested that at that time Greek was more or less a dialect continuum, to use modern terminology.

In sum, several eighteenth-century scholars broke away from the traditional ideas of Pseudo-Plutarch and others. Instead, they viewed Homer’s Greek as representing an early stage of the Greek language rather than as an artificially mixed entity, which was, however, not really a step forward. In their attempt at understanding Homeric Greek in historical terms, they did not consider the idea, now widely accepted, that it was a \textit{Kunstsprache} that was never a native tongue. The fact that the historization of Homeric Greek occurred only in the eighteenth century suggests that the widespread early modern interest in language change and diversification, with roots in the sixteenth century, penetrated discussions of the language of Homer rather late.

\section{The struggle with Biblical Greek}\label{sec:4.4}

\is{Semitism|(}
For the peculiar nature of Biblical Greek, early modern scholars had no real precedent to follow, as ancient, Byzantine, and early Renaissance scholars had expressed limited interest in this issue.\footnote{Pre-early modern ideas about Biblical Greek deserve, however, a closer analysis.} Yet before moving to premodern ideas, I should clarify what exactly is meant by Biblical Greek here. Scholars today usually distinguish between the Greek of the \isi{Septuagint}, a translation of the Hebrew Old Testament produced in Ptolemaic Egypt in the third and second centuries \textsc{bc} for Greek-speaking Jews, and the Greek of the New Testament, which was originally composed in this language in the first two centuries \textsc{ad}. Both varieties are, however, usually considered to have more or less the same nature, in that they represent lower, substandard \is{standard (language)}varieties of the Greek Koine, into which Semitisms have been introduced, primarily in vocabulary, syntax, and idiom. In the case of Septuagint Greek, these are principally due to the influence of the original Hebrew text, whereas the Semitic character of New Testament Greek remains somewhat of a mystery.\footnote{On Septuagint Greek, see e.g. \citet[106--108]{Horrocks2010}. On New Testament Greek, see e.g. \citet{Janse2007} and \citet{Porter2013}.} A very likely explanation is that Semitisms were introduced in imitation of Septuagint Greek. Additionally, there might have been interference from Biblical and Mishnaic Hebrew as well as from Aramaic, like Greek an important \textit{lingua franca} in Palestine and elsewhere during the first centuries \textsc{ad} \citep{Janse2007}.\is{Semitism|)}

Research on the history of ideas on Biblical Greek has been limited, and existing scholarship has largely restricted itself to some passing comments on the matter. For this reason, it is difficult to provide a satisfying answer here to the question as to how this variety of Greek has been perceived throughout the ages. There was, however, a vague awareness that New Testament Greek was “somewhat peculiar” and simpler than classical literary Greek. In fact, it was characterized as the language of fishermen or sailors by some Latin and Greek Early Christian authors \citep[647]{Janse2007}. No author writing before the early modern period, however, seems to have argued that Biblical Greek was a mixed variety consisting of different dialects. This idea was an early modern innovation, on which I will concentrate in this section and the next. For reasons of space and focus, I will not provide here a discussion of all interpretations suggested in this period. A thorough, comprehensive study of this matter therefore has to remain a desideratum for now, even though some scholars have already analyzed certain episodes of this history, usually from a theological or historical point of view. Henk J. de Jonge, for instance, has treated the study of the New Testament at early modern Dutch universities. One of the debates in this context concerned the purity of New Testament Greek. A number of scholars regarded it as impure, which caused serious theological problems. After all, how could the linguistic medium of the divine message be void of purity (\citeauthor{De1980} \citeyear[35--38]{De1980}, \citeyear[117--118]{De1981})?

I will focus here on the ways in which early modern scholars employed the dialectal reality of ancient Greece to account for the problematic nature of Biblical and especially New Testament Greek. The study of this variety of Greek was fostered by the interest of several leading humanists in the earliest Christians and their desire to return to a purer form of Christianity, fueled by their disappointment in contemporary religion. It was made possible by the return to the original Greek text advocated most sedulously by Desiderius Erasmus, inspired in his endeavor by his rediscovery of Lorenzo Valla’s notes on the Greek New Testament.\footnote{See e.g. \citet[esp. 31]{Bentley1983} for the innovativeness of the \textit{ad fontes} approach closely associated with Erasmus.} This innovative approach gained ground mainly in Protestant areas, where scholars were eager to reach a correct vernacular translation of the Bible by all means possible, including the study of the original Greek New Testament. In Catholic areas, however, the sanctity of the Latin Vulgate seems to have largely obstructed the systematic study of the original Greek New Testament and its language. It therefore comes as no surprise that the first systematic dialectological solution to New Testament Greek was proposed by a Calvinist scholar, Georg Pasor (1570–1637), a German philologist and theologian mainly active in the Dutch Republic who compiled the first lexicon and grammar of New Testament Greek. Pasor’s activity must be viewed in connection with the creation of a course “New Testament Greek for theologians” at different Dutch universities, particularly in the Frisian city of Franeker, where he held the Greek chair.\footnote{\Citet[29--31]{De1980}, where Pasor’s work is discussed in its Dutch context.}

\section{New Testament Greek as a dialect mixture}\label{sec:4.5}

In his \textit{Form of the Greek dialects of the New Testament} of 1632, Pasor outlined his interpretation of New Testament Greek as follows: “There are without doubt seven dialects of the New Testament […], i.e. Attic, Ionic, Doric, Aeolic, Boeotian, Poetic, and the Hebraizing”.\footnote{\citet[1--2]{Pasor1632}: “\textit{Sunt uero dialecti Noui Testamenti} […], nempe \textit{Attica}, \textit{Ionica}, \textit{Dorica}, \textit{Aeolica}, \textit{Boeotica}, \textit{Poetica} et ἡ Ἑβραΐζoυσα”.} Pasor thus posited the four canonical dialects to be present in the Greek New Testament, to which he added the Boeotian and poetical dialects; these were frequently listed in early modern classifications of Greek dialects, also outside of discussions of Biblical Greek (see Chapter 2, \sectref{sec:2.8}). The Hebraizing dialect, however, was introduced by Pasor himself to account for the many \is{Semitism}Semitisms present in the New Testament. This means that Pasor must have presupposed the existence of a kind of Hebraizing or Jewish Greek nation, since he had defined \textit{dialectus} as “speech peculiar to whatever people it may be, and that in the same language”.\footnote{\citet[1]{Pasor1632}: “Διάλεκτoς \textit{est sermo cuique populo peculiaris idque in eadem lingua}”.} Pasor did not, however, further comment on this, and in the remainder of his treatise he frequently spoke of “Hebraisms” (\textit{Hebraismi}) rather than of a Hebraizing dialect. In another debate, which started around the time Pasor published his work, such a typically Jewish Greek dialect was indeed explicitly posited by one of the parties involved, as I will demonstrate in the next section.

The mixed nature of New Testament Greek did not imply, however, that all dialects were equally represented in it. In fact, after discussing Attic features, Pasor added that, compared to Attic, “the remaining Greek dialects are by far rarer in the New Testament”.\footnote{\citet[24]{Pasor1632}: “Ceterae dialecti Graecae in N. T. sunt longe rariores”.} He claimed that communicative reasons were behind this dialectal diversity in New Testament Greek. The apostles wanted to announce the gospel not only to Jews, who read the \isi{Septuagint}, but also to the remaining peoples speaking a variety of Greek dialects \citep[143]{Pasor1650}. This resembles premodern ideas on Homeric Greek in several ways. Firstly, both were considered to constitute a mixture of dialects. Secondly, many authors believed one dialect, often identified with Attic, to predominate in both varieties. Thirdly, intelligibility across ethnic divisions was frequently perceived as the main motivation behind the mixed nature of both Homeric and Biblical Greek. Yet, unlike some of his contemporaries, Pasor did not realize that by the time the New Testament was being composed, most Greek dialects had become extinct due to pressure from the Koine.

Pasor’s treatise discussed the supposed linguistic features of New Testament Greek per dialect and with extensive exemplification. The typically Attic double tau instead of double sigma, for instance, was frequently found in the New Testament, he claimed (\citealt{Pasor1632}: 11–12). Some particularities he compared to dialectal variation in contemporary German. The ⟨s⟩/⟨t⟩ alternation had a parallel in High German \textit{Wasser} as opposed to Low German \textit{Water}, ‘water’. On the frequency of the letter alpha in Doric and the peculiar Dorian pronunciation of this letter, Pasor remarked:

\begin{quote}
\emph{\textup{Besides,} \emph{much} \emph{as} \emph{the} \emph{Ionians} \emph{love} \emph{the} \emph{eta,} \emph{the} \emph{Dorians} \emph{love} \emph{the} \emph{alpha} \emph{and} \emph{the} \emph{Attics} \emph{the} \emph{omega.} \emph{The} \emph{Dorians} \emph{pronounce} \emph{the} \emph{alpha} \emph{with} \emph{an} \emph{open} \emph{mouth,} \emph{just} \emph{as} \emph{among} \emph{the} \emph{Germans,} \emph{too,} \emph{there} \emph{are} \emph{some,} \emph{primarily} \emph{the} \emph{Bavarians} \emph{and} \emph{the} \emph{Austrians,} \emph{who} \emph{usually} \emph{do} \emph{this,} \emph{as} \emph{is} \emph{well-known.}}\footnote{\citet[28]{Pasor1632}: “Ceterum sicut Iones amant τὸ η, ita Dores τὸ α et Attici τὸ ω. Dores τὸ α ore diducto pronuntiant, uti apud Germanos quoque quosdam, imprimis Bauaros et Austriacos, factitare notum est”. Cf. Chapter 8, \sectref{sec:8.1.1}.}
\end{quote}

Pasor drew on sixteenth-century scholarship to retrieve dialectal features in the Greek of the New Testament. For example, he repeatedly referred to Joachim Camerarius’s (1500–1574) 1541 notes on Herodotus’s Ionic dialect, whereas for the so-called Hebraizing dialect he made use of Santes Pagnino’s (1470–1541) \textit{Treasure of the holy language}, first published in 1529 in Lyon.\footnote{For Camerarius, see \citet[24--25, 27--28]{Pasor1632}. For Pagnino, see \citet[36]{Pasor1632}.}

Pasor’s \textit{Form} was frequently reprinted. It moreover gave rise to the emergence of a philological subdiscipline which could be dubbed “biblical dialectology” for two main reasons. On the one hand, the alleged multidialectal nature of New Testament Greek stimulated a considerable number of writings entirely dedicated to this theory, flourishing especially in Protestant scholarship.\footnote{\citet{Wyss1650}, \citet{Olearius1668}, \citet{Leusden1670}, and \citet{Nibbe1755}. Cf. \citet[347]{Parr1686}; \citet[18--19]{Von1705}; \citet[9--10]{Florinus1707}; \citet[\textsc{d.2}\textsc{\textsuperscript{r}}, \textsc{d.5}\textsc{\textsuperscript{v}}]{Thryllitsch1709}; \citet[18]{Reinhard1724}; \citet[121--122]{Holmes1735}, a school grammar, suggesting that the theory was also taught; \citet[136--137]{Walch1772}.} The idea that “without knowledge of the dialects, the New Testament cannot be accurately understood” was indeed a commonplace.\footnote{\citet[\textsc{d.5}\textsc{\textsuperscript{v}}]{Thryllitsch1709}: “Sine cognitione dialectorum Nouum Testamentum accurate intelligi non potest”.} On the other hand, the term \textit{dialectologia} was coined in 1650 by the Zurich Hellenist Caspar Wyss (1605–1659) to designate the study of the Greek dialects as they figured in the Greek New Testament. The word featured prominently in the title of his work on this matter: \textit{Sacred dialectology}, or \textit{Dialectologia sacra} in the original Latin. Wyss’s book is of interest for other reasons as well. He added the Koine as a geographically neutral variety to the varieties of Greek which Pasor had recognized in the New Testament \citep[3]{Wyss1650}. The Koine was opposed to the five principal and regional Greek dialects, Attic, Ionic, Doric, Aeolic, and Boeotian, to which, on Pasor’s authority, the poetic and the Hebraizing dialects needed to be added. However, these latter two were, Wyss explained, less important, since they were not tied to any province and exhibit idiomatically non-Greek properties (\citealt{Wyss1650}: 289–290, 295). Wyss was, in other words, trying to formulate a solution for a mismatch he had discovered in Pasor’s work. Wyss’s predecessor had provided a definition of \textit{dialectus} in ethnic terms which was difficult to reconcile with the Hebraizing and poetical dialects which he had posited for New Testament Greek. Interestingly, Wyss consciously arranged his discussion of the New Testament dialects in terms of frequency, an approach implicit at best in Pasor’s book. Attic, the most prominent dialect in the New Testament, was described first; Boeotian, the least prominent, last \citep[4]{Wyss1650}.

Pasor’s theory that New Testament Greek was an amalgam of different Greek dialects did not come out of nowhere. More or less simultaneously to Pasor, a Scottish exegete observed in passing: “So in the New Testament there are sundry dialects as \textit{Ionick}, \textit{Dorick}, \textit{Attick}, \textit{etc}.” \citep[102]{Weemes1632}. It appears that Pasor was systematizing a tradition already found in earlier work. Desiderius \citet[270]{Erasmus1519} relied as early as 1519 on his knowledge of the Greek dialects to refute a judgment of St Jerome’s about an alleged syntactic irregularity – a so-called \isi{solecism} – in the New Testament. This rebuttal by Erasmus presupposed the idea that certain features of the language of the New Testament could be explained by appealing to a Greek dialect. Such an assumption became even more clearly apparent in the second half of the sixteenth century. The French-Swiss Protestant theologian Theodore Beza (1519–1605), for example, explained several linguistic particularities of the New Testament by referring to the Attic dialect.\footnote{See e.g. \citet[\textsc{i.}226, \textsc{ii}.355]{Beza1594}, where Hebrew influence was also mentioned.} Grammars of Greek and manuals for the dialects, too, started to contain occasional references to the Greek New Testament to exemplify certain dialectal particularities. In 1589, the Marburg professor of Greek Otto Walper illustrated the alleged Attic feature of using comparatives and superlatives interchangeably by citing 1~Corinthians 13.13.\footnote{\citet[32]{Walper1589}: “Superlatiuis pro comparatiuis utuntur frequenter, et contra.1.Cor.13. μείζων δὲ τoύτων ἡ ἀγάπη, id est, μεγίστη”. See also already \citet[251]{Ruland1556}, explaining an Attic particularity with reference to, among other texts, the Bible.} The idea that the Greek New Testament exhibited dialectal features may moreover have been enhanced by poetical adaptations of the Greek Bible appearing in the second half of the sixteenth century, such as, for instance, Johannes Posselius the Younger’s (1565–1623) Greek versification of parts of the New Testament, which contained many dialect forms that were occasionally explained in the margins.\footnote{\citet{Posselius1599}. Cf. also \citet{Jamot1593} and \citet{Keimann1649}.}

The multidialectal interpretation of New Testament Greek was criticized early on, most notably by the prominent French humanist-printer Henri \citet[32–33, 138]{Estienne1581}. Some even saw the misuse of the Greek dialects as a danger for the vernacular translator of the New Testament, who might be able to distort the sense of a word by referring to the Greek dialects, thus introducing heresies into the text.\footnote{See \citet[429]{Rainolds1583}, on which see \citet[654--655]{VanRooyConsidine2016}.} Much later, biblical dialectology came to be rebutted by the eighteenth-century German professor of theology and philology Christian Siegmund Georgi. Georgi emphasized that the authors of the New Testament wrote in pure Attic, a medium befitting the divine message. He explained the presence of non-Attic elements by claiming that they had become Attic in the course of history (\citealt{Georgi1733}: 6–7). He moreover reacted against scholars inventing “pseudo-dialects” to account for the particularity of New Testament Greek, no doubt thinking of the so-called Hebraizing dialect as well as the Hellenistic dialect proposed by Daniel Heinsius, which I will treat in the next section.\footnote{Georgi coined the term “ψευδoδιάλεκτoι” to make his point. For New Testament Greek as Attic, see e.g. also \citet[3, 10--12]{Georgi1729} and \citet[b.7\textsc{\textsuperscript{r}}–b.8\textsc{\textsuperscript{v}}]{Fischer1754}.} Georgi’s ideas must be viewed in the context of the eighteenth-century debate between Hebraists and Purists which took place primarily in the Northern Low Countries (the modern Netherlands), Germany, and England. The Purists, including Georgi, argued that the New Testament was written in pure Greek, whereas the Hebraists contended that Hebrew elements were unmistakably present, which, however, were not barbarisms but adornments (\citealt{De1980}: 35). A thesis similar to Georgi’s had already been proposed for public discussion in 1702 in Wittenberg by Georg Wilhelm Kirchmaier and Christian Gottlieb Schwartz (see \citealt{Kirchmaier1702}: [2], thesis \textsc{i}). It intelligently suggested that a mixed use of dialects was highly unlikely, as this would have made the New Testament unintelligible to the populace. However, as late as 1765, the framework of biblical dialectology was still presented as canonical knowledge by several scholars (e.g. \citealt{Gottleber1765}: *.2\textsc{\textsuperscript{r}}), which indicates that the efforts of Georgi and others had not yet displaced the awkward idea of a dialectally mixed New Testament Greek.

\section{Biblical Greek, a Hellenistic dialect?}\label{sec:4.6}

A different dialectal solution to the problematic status of Biblical Greek was proposed by the philologist Daniel Heinsius (1580–1655), born in Ghent but mainly active at Leiden university. Heinsius assumed the existence of a clearly distinct dialect used in translating the Hebrew Old Testament into Greek and writing down the Greek New Testament. This dialect, he claimed, was spoken by the so-called Hellenists, a Greek nation separate from the others. The Burgundian classical scholar Claude de Saumaise (1588–1653) reacted sharply against this Hellenistic tongue, which he dismissed as an invention of Heinsius. Their fierce controversy was as much a matter of personal rivalry as it was a scholarly disagreement.\footnote{For this well-known and much-studied controversy, see \citet[32--34]{De1980}; \citet[391--392]{Muller1984}; \citet{Considine2010}; \citet[350--351]{VanHal2010a}. Daniel Georg \citet[\textsc{ii.}74–77]{Morhof1708} already summarized the controversy.} Heinsius and Saumaise were in fact arch-enemies because of their competing ambitions to succeed Joseph Justus Scaliger at Leiden university. Saumaise formulated his main attack of Heinsius and his Hellenistic dialect in two books, which he wrote in France but sent to Leiden to be printed.\footnote{\citet[]{Saumaise1643a, Saumaise1643}. The preface of \citet{Saumaise1639} already touched on the issue, too.} He did so in order to avoid his absence endangering his position at the university. Both works appeared in 1643 and were centered around the argument that there was no such thing as a Hellenistic people, let alone a Hellenistic dialect.

In Saumaise’s rebuttal of Heinsius’s Hellenistic dialect, the correct interpretation of the Greek term \textit{diálektos} played a pivotal role. Saumaise emphasized – repeatedly and ad nauseam – that in order to speak of a Hellenistic dialect, the existence of a Hellenistic people was required, which was not corroborated by historical evidence. This seems to indicate that Heinsius and Saumaise, their personal differences and their insistence on terminology aside, “were arguing over a serious scientific problem”, as Henk J. \citet[117]{De1981} has put it (see also \citealt{De1980}: 34–35). They had different views on the linguistic history of the Greek language, and their debate was, in consequence, not simply a matter of word choice, as has been maintained (see \citealt{Simon1689}: 318--319; cf. \citealt{Considine2012}: 298). The alternative \citet[98–99, 240–266]{Saumaise1643a} suggested for Heinsius’s Hellenistic dialect indeed seems to support this interpretation. He argued that instead of a Hellenistic dialect, the authors of the New Testament used the uneducated vulgar variety of Koine Greek – the \textit{stylus idioticus} – of their times.\footnote{See \citet[34--35]{De1980}. In the early sixteenth century, Erasmus had already proposed a similar solution to the issue \citep[181]{Bentley1983}.} Saumaise thus stressed the link with contemporary non-Biblical Greek and proposed an answer that, from a modern perspective, seems more correct. This vulgar variety, Saumaise proceeded, was influenced by Aramaic, thus obtaining a “translational” (\textit{hermēneutikós}/ἑρμηνευτικός) character. The language of the \isi{Septuagint} was likewise, and more understandably, said to be of a translational nature, but it was also claimed to have Macedonian characteristics due to Alexander the Great’s heritage \citep[264]{Saumaise1643a}. The notorious Heinsius–Saumaise controversy stirred up many subsequent discussions of the matter.\footnote{The “notoriousness” of the issue was already underlined by \citet[\textsc{ii.}74]{Morhof1708}, who used the Greek adjective \textit{poluthrúl<l>ētos}  (πoλυθρύλ<λ>ητoς) in this context.} Some scholars preferred to speak of “Biblical Graecism” (\textit{Biblicus Graecismus}) rather than of a dialect peculiar to this text, whereas others rejected the Hellenistic dialect as a product of Heinsius’s imagination.\footnote{For \textit{Biblicus Graecismus}, see \citet[\textsc{b.3}\textsc{\textsuperscript{v}}]{Bolius1689}. For Heinsius’s \textit{dialectus Hellenistica} as a “dream” (\textit{somnium}), see \citet{De1644}.} It does not, however, lie within the scope of this book to provide an exposition of the extensive early modern debate on the issue in its entirety, as this would require a separate study of its own.

In summary, both the term ‘dialect’ and the fact that the Greek language had different dialects were exploited by early modern Hellenists to make sense of the peculiar form of Greek they encountered in reading the \isi{Septuagint} and especially the New Testament. Whereas the framework of biblical dialectology, according to which New Testament Greek was a mixture of dialects, strikes the modern reader as highly artificial and even clumsy, the debate between Heinsius and Saumaise led to a relatively accurate hypothesis about the nature of Septuagint Greek. Biblical dialectologists were eager to attribute peculiar forms which they encountered in the New Testament to specific dialects, and the description of these particularities was their main concern. In the controversy about Hellenistic Greek, linguistic features were confined to the margins of the main argument. Saumaise focused on the interpretation of the technical term \textit{diálektos} as well as on the linguistic history of the Greek language and its speakers to dismiss Heinsius’s Hellenistic tongue. He did, however, claim that it was impossible for one and the same word to have different meanings in one and the same dialect (\citealt{Saumaise1643a}: 41–42 of the dedicatory letter). This impossibility of polysemy was employed by Saumaise as a supporting argument to refute the existence of a Hellenistic dialect, in which according to Heinsius and his followers certain words could have several interpretations.

\section{Conclusion}\label{sec:4.7}

In classifying the Greek dialects, early modern scholars encountered two major difficulties: the speech of Homer and the Greek Bible. Incidentally, these were also among the Greek texts that were read most eagerly in the early modern period. For Homer’s peculiar Greek, Renaissance Hellenists first followed the idea that it was a dialectally mixed variety, which was backed by a text attributed to the authoritative ancient polymath Plutarch. An increased awareness of the historicity of the figure of Homer and the conditions under which the epic poems associated with him emerged led eighteenth-century scholars to a re-evaluation of his language in historical terms. Even though their solutions were certainly not wholly satisfying, they paved the way for later interpretations of Homer’s language that took into account its historical evolution more fully.

Perhaps by analogy with Homer’s Greek, scholars from Protestant areas developed the idea that the language of that other great Greek textual corpus, the New Testament, was also dialectally mixed. In this case, philologists were not backed by ancient scholarship, and it is hard to understand why this seemed such an appealing solution. They appear to have believed that mixing dialects implied reaching a larger audience, an assumption that seems counterintuitive to most modern readers. The discussion about the right of a Hellenistic dialect to exist, which was initiated by the rival colleagues Heinsius and Saumaise, did, by contrast, lead to a better understanding of Biblical Greek, grounded, like eighteenth-century views on Homeric Greek, in an appreciation of the historical conditions under which Koine Greek emerged.

